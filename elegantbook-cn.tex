% !TeX spellcheck = en_US
% !TeX encoding = UTF-8
% !TeX program = xelatex

%\PassOptionsToPackage{unicode}{hyperref}
%\PassOptionsToPackage{naturalnames}{hyperref}

\documentclass[lang=cn,newtx,10pt,scheme=chinese,device=pad]{elegantbook}

\title{数学笔记}
\subtitle{偏微分方程--趋化模型}

\author{毛宣}
\institute{东南大学数学学院}
\date{2024/06/11}
\version{0.1.0}
\bioinfo{测试}{信息}

\extrainfo{注意:本模板自 2023 年 1 月 1 日开始,不再更新和维护!}

\setcounter{tocdepth}{3}

\logo{logo-blue.png}
\cover{cover.jpg}

% 本文档命令
\usepackage{array}
\newcommand{\ccr}[1]{\makecell{{\color{#1}\rule{1cm}{1cm}}}}


% 修改标题页的橙色带
\definecolor{customcolor}{RGB}{32,178,170}
\colorlet{coverlinecolor}{customcolor}
\usepackage{cprotect}

\usepackage{mathtools} %provided \shortintertext

\addbibresource[location=local]{mynote.bib} % 参考文献,不要删除

\allowdisplaybreaks[4]

%\includeonly{mathnote,real-analysis}
%\includeonly{functional-analysis}

\newcommand{\set}[1]{\left\{#1\right\}}

\newcommand{\Lref}[1]{Lemma~\ref{#1}}

\newcommand{\dd}{\;\mathrm{d}}

\DeclareMathOperator*{\esssup}{ess\,sup}
\DeclareMathOperator*{\loc}{loc}
\DeclareMathOperator*{\spt}{spt}
\DeclareMathOperator*{\diam}{diam}

\begin{document}

\maketitle
\frontmatter

\tableofcontents

\mainmatter

%\part{mathematics analysis}
\chapter{数学分析}


%-=-=-=-=-=-=-=-=-=-=-=-=-=-=-=-=-=-=-=-=-=-=-=-=
%	SECTION:
%-=-=-=-=-=-=-=-=-=-=-=-=-=-=-=-=-=-=-=-=-=-=-=-=

\section{集合与映射}\index{Set ! function}

\begin{example}\label{20151012-190708}
\hfill \\
设$f(x)$对任意的$x\in R$有$f(x)=f(x^2)$,且$f(x)$在$x=0$和$x=1$ 处连续。证明$f(x)$在$R$上为常数。(上海交通大学,2003)


steps
  $f(x)$在$x=0$处连续$\Leftrightarrow\forall\varepsilon>0,\exists\delta>0,\forall x\in(-\delta,\delta),|f(x)-f(0)|<\varepsilon$。
  
  $\forall x\in(-1,1),|f(x)-f(0)|=|f(x^2)-f(0)|\cdots=|f(x^{2n}-f(0)|$
  即对上述的$\varepsilon>0$,$\exists N=\max\{1,[\log_x\frac{\delta}2]\}$,$\forall n>N,x^{2n}<\delta$,从而$|f(x)-f(0)|=|f(x^{2n}-f(0)|<\varepsilon$成立。
  
  由$\varepsilon$的任意性,有$\forall x\in(-1,1),f(x)\equiv f(0).(|f(x)-f(0)|\leq 0)$
  
  又$f(x)$在$x=1$处连续$\Leftrightarrow\forall\varepsilon>0,\exists\delta>0,\forall x\in(1-\delta,1+\delta),|f(x)-f(1)|<\varepsilon$成立。
  而$\forall x>1,|f(x)-f(1)|=|f(x^{\frac12})-f(1)|=\cdots|f(x^{\frac1{2n}})-f(1)|$,
  即对上述的$\varepsilon>0,\exists N=\max\{1,[\frac1{2\log_x(1+\delta)}]\},\forall n>N,0<x^{\frac 1{2n}}-1<1+\delta,|f(x)-f(1)|=|f(x^{\frac1{2n}})-f(1)|<\varepsilon$。
  
  由$\varepsilon$的任意性,$0\leq|f(x)-f(1)|\leq 0$,从而$\forall x>1,f(x)\equiv f(1)$。
  
  易知$\forall x<-1,f(x)=f(x^2)=f(1)$,又$f(0)=\lim_{x\rightarrow1^-}f(x)=f(1)=f(-1)$,从而$\forall x\in R,f(x)\equiv f(0)$。

%\qdepend

%\qdependlist
\end{example}
\hfill\\

\section{数列极限}
  \begin{example}
  \hfill\\
   Cauchy收敛准则,叙述并证明。(浙江大学,2003)

  Cauchy收敛准则:数列$\{a_n\}$收敛的充分必要条件是$$\forall\varepsilon>0,\exists N,\forall m,n>N,|a_m-a_n|<\varepsilon.$$
  
  必要性:数列$\{a_n\}$收敛,则$\exists a$使得$\forall\varepsilon>0,\exists N,\forall n>N$,有
  $|a_n-a|<\frac{\varepsilon}2$,从而$\forall m,n>N,|a_m-a_n|\leq|a_m-a|+|a_n-a|<\varepsilon.$
  
  充分性:若$\forall\varepsilon>0,\exists N,\forall m,n>N,|a_m-a_n|<\varepsilon.$则当$\varepsilon=1$时,$\exists N_1$,固定$m+N_1+1,\forall n>N_1,|a_n-a_m|<\varepsilon$。从而数列
  $\{a_n\}$必有界,由抽子列定理,存在子列$\{a_{n_j}\}$收敛到$a$。
  
  于是$\forall\varepsilon>0,\exists N_2,\forall n_j>N_2,|a_{n_j}-a|<\varepsilon.$
  
  $|a_n-a|=|a_n-a_{n_j}+a_{n_j}-a|\leq|a_n-a_{n_j}|+|a_{n_j}-a|\leq\varepsilon+\varepsilon=2\varepsilon.$
  故$\{a_n\}$收敛。

\end{example}

\begin{example}
\hfill\\

 任意给定$x\in R$,令$x_1=\cos x,x_{n+1}=\cos x_n$,证明数列收敛。
 

  $\because x_1\in[-1,1],\therefore x_2\in[-1,1]$归纳易知,$\forall n\in N^+,0\leq|x_n|\leq1$又
  \[
  \begin{aligned}
  |x_{n+1}-x_n|&=|\cos x_n-\cos x_{n-1}|\\
  &=2|\sin^2\frac{x_n}2-\sin^2\frac{x_{n-1}}2|\\
  &=2|\sin\frac{x_n}2-\sin\frac{x_{n-1}}2||\sin\frac{x_n}2+\sin\frac{x_{n-1}}2|\\
  &\leq2|\sin\frac{x_n}2-\sin\frac{x_{n-1}}2|(|\sin\frac{x_n}2|+|\sin\frac{x_{n-1}}2|)\\
  &<2|\sin\frac{x_n}2-\sin\frac{x_{n-1}}2|(\sin\frac12+\sin\frac12)\\
  &=2|\sin\frac{x_n}2-\sin\frac{x_{n-1}}2|\cdot\delta,\verb+其中+\delta=2\sin\frac12\in(0,1)
  \end{aligned}
\]
由拉格朗日中值定理可证:$|\sin\frac{x_n}2-\sin\frac{x_{n-1}}2|\leq\frac12|x_n-x_{n-1}|$。于是$|x_{n+1}-x_n|<\delta|x_n-x_{n-1}|$其中$\delta\in(0,1)$。易推:$|x_{n+1}-x_n|<\delta^n|x_1-x|$于是对$0<n<m$有
\[
\begin{aligned}
|x_m-x_n|&=|x_m-x_{m-1}+x_{m-1}\cdots+x_{n+1}-x_n|\\
&\leq\sum_{i=n+1}^m|x_i-x_{i-1}|\\
&<|x_1-x|\sum_{i=n+1}^m\delta^{i-1}\\
&=|x_1-x|\frac{\delta^n(1-\delta^{m-n}}{1-\delta}\\
&<|x_1-x|\frac{\delta^n}{1-\delta}\rightarrow0,n\rightarrow\infty.
\end{aligned}
\]
即$\forall\epsilon>0,\exists N\in N^+,\forall m,n>N,\verb+有+|x_m-x_n|<\epsilon$,由Cauchy收敛准则知:$\{x_n\}$收敛。

\end{example}
\begin{example}
\hfill\\
设$\phi(x)$与$f(x)$是区间$[a,b]$上的正值连续函数,求证:$$\displaystyle\lim_{n\rightarrow\infty}\frac{\int_a^b\phi(x)f^{n+1}(x)dx}{\int_a^b\phi(x)f^n(x)dx}=\max_{a\leq x\leq b}f(x).$$

设$I_n=\int_a^b\phi(x)f^n(x)\mathrm{d}x$。应用Schwarz不等式,得
$$I_n^2\leq\int_a^b\phi(x)f^{n-1}(x)\mathrm{d}x\int_a^b\phi(x)f^{n+1}(x)\mathrm{d}x=I_{n-1}I_{n+1}.$$
故$$\frac{I_{n+1}}{I_n}\geq\frac{I_n}{I_{n-1}}.$$
因此数列$\{\frac{I_{n+1}}{I_n}\}$是递增的,又
$$\frac{I_{n+1}}{I_n}\leq\max_{a\leq x\leq b}f(x)\frac{I_n}{I_n}=\max_{a\leq x\leq b}f(x),$$
所以$\{\frac{I_{n+1}}{I_n}\}$有界。于是$\lim_{n\rightarrow\infty}\frac{I_{n+1}}{I_n}$存在,且
$$\lim_{n\rightarrow\infty}\frac{I_{n+1}}{I_n}=\lim_{n\rightarrow\infty}\sqrt[n]{I_n}.$$
而$$\lim_{n\rightarrow\infty}\sqrt[n]{I_n}=\max_{a\leq x\leq b}f(x).$$
这就完成了证明。
\end{example}

\begin{example}
\hfill\\
试证$\sum1/p$发散,其中$p$遍历一切质数。

提示:给定$N$,设$p_1,p_2,\cdots,p_k$是至少能整除一个不大于$N$的正整数。那么$$\sum_{n=1}^N\frac{1}{n}\leq\prod_{i=1}^k(1+\frac{1}{p_i}+\frac{1}{p_i^2}+\cdots)=\prod^k(1-\frac{1}{p_i})^{-1}\leq\exp\sum_{i=1}^k\frac{2}{p_i}.$$
最后这个不等式能成立,是因为当$0\leq x\leq \frac{1}{2}$时,$(1-x)^{-1}\leq e^{2x}$。显然当$N\rightarrow+\infty$时,$k\to\infty$,从而
$$\lim_{N\to\infty}\exp\sum_{i=1}^{N(k)}\frac{2}{p_i}\geq\lim_{n\to\infty}\sum_{n=1}^N\frac{1}{n}=+\infty.$$
于是$\sum\frac{2}{p_i}\to\infty$,$k\to\infty$。得证。
\end{example}

 \section{函数极限与连续函数}

\begin{theorem}[Stone-Weierstrass theorem]
  Suppose $f\in C([0,1])$. Then there exists a sequence $\{P_n\}$ of polynomials such that 
  $P_n$ converges to $f$ uniformly $x\in[0,1]$ as $n\to\infty$.  
\end{theorem}

\begin{proof}
  Without loss of generality, we may assume that $f(0) = f(1) = 0$ and extend $f$ to $\mathbb{R}$ such that 
  $f(x)=0$ for $x\not\in[0,1]$.
  Let $g_n(x) = c_n(1-x^2)^n$ with 
  \[
  c_n = \left(\int_0^1(1-x^2)^n\dd x\right)^{-1}.
  \]
  Let 
  \[
  P_n(x) = \int_0^1f(t)g_n(x-t)\dd t.
  \]
  It is clear that $P_n$ is a polynomial, as we need.
\end{proof}

 \begin{example}
  \hfill\\
  
  若函数$f(x)$在$[0,+\infty)$上连续,$\displaystyle\lim_{x\rightarrow\infty}f(x)=A$存在,则$f(x)$在$[0,+\infty)$上一致连续。(中国人民大学,2001)
 
  $\lim_{x\rightarrow\infty}f(x)=A\Leftrightarrow$对任意的$\varepsilon>0,\exists M>0,\forall x_1,x_2>M,|f(x_1)-f(x_2)|<\varepsilon$。
  
  因为$f(x)$在$[0,M+1]$上连续,从而在$[0,M+1]$上一致连续,即存在$0<\delta<1$,对任意的$x_1,x_2\in[0,M+1]$,当$|x_1-x_2|<\delta$时,有$|f(x_1)-f(x_2)|<\varepsilon$。因此,对任意的$x_1,x_2\in[0,+\infty)$,当$|x_1-x_2|<\delta$时,有$|f(x_1)-f(x_2)|<\varepsilon$。
  
  即$f(x)$在$[0,+\infty)|<\varepsilon$。
  \end{example}
  \begin{example}
  \hfill\\
  证明:函数$f(x)=\sqrt{x}lnx$在$[1,+\infty)$上一致连续。(北京大学,2001)
  
  steps
  对$f(x)$求导可得
  \[f'(x)=\frac{\ln x+2}{2\sqrt x},\]
  \[f''(x)=-\frac{\ln x}{4x\sqrt x.}\]
  当$x\in[1,+\infty)$时,$f'(x)>0,f''(x)<0$,因此,$f'(x)$在$[1,+\infty)$上单调递减趋于0,又$\lim_{x\rightarrow1}f'(x)=1$,意味着$f'(x)$在$[1,+\infty)$有界,不妨假设$|f'(x)|\leq M$。
  
  对任意的$\varepsilon>0,\exists\delta=\frac{\delta}M,$对任意的$x_1,x_2\in[1,+\infty)$,当$|x_1,x_2|<\delta$时,有$|f(x_1)-f(x_2)|=|f'(\xi)(x_1-x_2)|<M\cdot\frac{\varepsilon}{M}=\varepsilon.$其中$\xi\in[x_1,x_2]$,即函数$f(x)=\sqrt{x}\ln x$在$[1,+\infty)$上一致收敛。
  \end{example}

\begin{remark}
  札记:可以用Cauchy收敛准则和微分中值定理证明函数一致连续性
\end{remark}

\begin{example}
\hfill\\
  设$f(x)$在$[a,b]$上连续,对$x\in[a,b]$定义$\displaystyle m(x)=\inf_{a\leq t\leq x}f(t)$。证明:$m(x)$在$[a,b]$上连续。(大连理工大学,2004)
  
  
  首先,因为闭区间上的连续函数必有最大值和最小值,故$m(x)$是良定义的。
  
  其次,$\forall a\leq x_1<x_2\leq b$,
  \[
  \begin{aligned}
  m(x_2)&=\inf_{a\leq t\leq x_1}f(t)\\
  &=\min\{\inf_{a\leq t\leq x_1}f(t)+\inf_{x_1\leq t\leq x_2}f(t)\}\\
  &\leq\min_{a\leq t\leq x_1}f(t)\\
  &=m(x_1)\\
  \end{aligned}
  \]
  故$m(x)$在$[a,b]$上单调递减,$\forall x_1,x_2\in[a,b],|x_1-x_2|<2\delta$,因为$f(x)$在$[a,b]$上一致连续,故$\forall\varepsilon>0,\exists\delta<0,|f(x_1)-f(x_2)|<\varepsilon.$从而$\forall x\in[a,b]$,
  \[
  \begin{aligned}
  m(x+)&=\lim_{\delta\rightarrow0+}m(x+\delta)\\
  &=\lim_{\delta\rightarrow0+}\frac{m(x)+\inf_{x\leq t\leq x+\delta}f(t)-|m(x)-\inf_{x\leq t\leq x+\delta}f(\delta)|}{2}\\
  &=\frac{m(x)+f(x)-|f(x)-m(x)|}{2}\\
  &=m(x)\\
  \end{aligned}
  \]
  对$m(x-)$,若$m(x)<m(x-\delta)$,则$f(\eta)=m(x)\leq m(x-\delta)\leq f(x-\delta),(\eta\in(x-\delta,x))$
  因为$f(x)$在$[a,b]$上一致连续,故$\forall\varepsilon>0,\exists\delta>0,x_1,x_2\in[a,b],|x_1-x_2|<\delta,|f(x_1)-f(x_2)|<\varepsilon$。从而对上述$\varepsilon>0$,有上述$\delta>0$,$$|m(x)-m(x-\delta)|\leq|f(\eta)-f(x-\delta)|<\varepsilon,(\eta\in(x-\delta,x).$$
  由此$m(x)$在$x$处左连续,右连续。故$m(x)$在$[a,b]$上连续。
  \end{example}
  
  \begin{example}
  \hfill\\
  如果$f$是$(0,\infty)$上的正值函数,合于
  \begin{enumerate}
  \item[a] $f(x+1)=xf(x)$;
  \item[b] $f(1)=1$;
  \item[c] $\log f$是凸的;
  \end{enumerate}
  那么$f(x)=\Gamma(x)$。其中$$\Gamma(x)=\int_0^{\infty}t^{x-1}e^{-t}\mathrm{d}t.$$
  
  因为$\Gamma$函数满足以上三点,那么只需证明$x>0$时$f(x)$是由以上三点决定的函数就行了。根据$a$只需考虑$x\in(0,1).$
  
  令$\varphi =\log f$,那么$\varphi(x+1)=\varphi(x)+\log x(0<x<\infty).$ $\varphi(1)=0$,并且$\varphi$是凸的,假定$0<x<1$,而$n$是正整数,由上式$\varphi(n+1)=\log (n!).$现在考虑一下$\varphi$在$[n,n+1]$,$[n+1,n+1+x]$,$[n+1,n+2]$三个区间上的差商,既然$\varphi$是凸的,那么$$\log n\leq\frac{\varphi(n+1+x)-\varphi(n+1)}{x}\leq\log(n+1).$$
  $$\varphi(n+1+x)=\varphi(x)+\log[x(x+1)\cdots(x+n)].$$
  所以$$0\leq\varphi(x)-\log[\frac{n!n^x}{x(n+1)\cdots(x+n)}]\leq x\log(1+\frac{1}{n}).$$
  最末的式子当$n\to\infty$时趋于零。从而确定了$\varphi$。
  \end{example}
 
\begin{example}
定义区间$(a,b)$上的实连续函数满足:
$$\forall x,y \in(a,b),f(\frac{x+y}{2})\leq\frac{1}{2}(f(x)+f(y)).$$
求证:$f$是凸函数。其中,所谓凸函数是指
$$f(\lambda x+(1-\lambda)y)\leq\lambda f(x)+(1-\lambda)f(y),\forall x,y\in(a,b),\lambda\in(0,1).$$

\textbf{引理:}若非空集合$S\subset(0,1)\cap\mathit{Q}$满足:若$\frac{1}{2}\in S$;若$p\in S$,则$\frac{p}{1+p},\frac{1}{1+p}\in S.$则$S=(0,1)\cap\mathit{Q}$。

引理的证明只需注意到,$(0,1)$的有理数可用真分数表示,而全体真分数具有$\frac{q}{p}$形式,其中$p<q\in\mathrm{N}^*$。于是我们可以关于分母$p=N$作第二数学归纳法证明上述引理。

我们想用上述引理说明,当$\lambda\in S$时,凸函数定义中的不等式成立。又因为$S$在$(0,1)$中稠密,以及$f$的连续性,可以说明$\forall\lambda\in(0,1)$,凸函数定义中的不等式成立。从而证毕。

现在我们只需说明,若$p\in S$,则$\frac{p}{1+p},\frac{1}{1+p}\in S$即可。

\begin{align*}
f(\frac{p}{1+p}x+\frac{1}{p+1}y)&=f(p(\frac{1}{1+p}x+\frac{p}{1+p}y)+(1-p)y)\\
&\leq pf(\frac{1}{1+p}x+\frac{p}{1+p}y)+(1-p)f(y)\\
&\leq p^2f(\frac{p}{1+p}x+\frac{1}{p+1}y)+p(1-p)f(x)+(1-p)f(y)\\
\end{align*}
于是
$$f(\frac{p}{1+p}x+\frac{1}{p+1}y)\leq \frac{p}{1+p}f(x)+\frac{1}{p+1}f(y),$$
即$\frac{p}{1+p},\frac{1}{1+p}\in S.$证毕。
\end{example} 
 
\begin{proposition}
	Let $I\in\mathbb{R}$ be an open interval. Suppose that $f: I\mapsto\mathbb{R}$ is a convex function,
	i.e.,
	\[
		f(\lambda a + (1-\lambda)b) \leq \lambda f(a) + (1-\lambda)f(b),\quad a,b\in I, \quad \lambda\in(0,1).
	\]
	Then $f\in C^{0,1}_{\mathrm{loc}}(I,\mathbb R)$.
\end{proposition}

\begin{proof}
	Fix $a\in I$.
	If $a<b$, then we have by putting $\lambda\nearrow1$
	\[
		\limsup_{\lambda\nearrow1}f(\lambda a + (1-\lambda)b) = \limsup_{c\searrow a}f(c)\leq f(a),
	\]
	if $a>b$, then we have similarly
	\[
		\limsup_{\lambda\nearrow1}f(\lambda a + (1-\lambda)b) = \limsup_{c\nearrow a}f(c)\leq f(a).
	\]
	Particularly, using
	\begin{align*}
		f(a) &= f(\lambda(a-\varepsilon(1-\lambda)) + (1-\lambda)(a+\varepsilon\lambda))\\
			&\leq \lambda f(a-\varepsilon(1-\lambda)) + (1-\lambda)f(a+\varepsilon\lambda),
			\quad 0<\varepsilon\ll 1,
	\end{align*}
	we have by putting $\lambda\nearrow1$,
	\[
		f(a)\leq \liminf_{\lambda\nearrow1}\lambda f(a-\varepsilon(1-\lambda)) 
			+ \limsup_{\lambda\nearrow1}(1-\lambda)f(a+\varepsilon\lambda)
			= \liminf_{c\nearrow a}f(c),
	\]
	and by putting $\lambda\searrow0$,
	\[
		f(a)\leq \limsup_{\lambda\searrow0}\lambda f(a-\varepsilon(1-\lambda)) 
			+ \liminf_{\lambda\searrow0}(1-\lambda)f(a+\varepsilon\lambda)
			= \liminf_{c\searrow a}f(c).
	\]	
	Four estimates above imply the continuity of $f$ at $a$.

	Using 
	\[
	\frac{f(\lambda a + (1-\lambda)b) - f(a)}{\lambda a + (1-\lambda)b - a} \leq \frac{f(b) - f(a)}{b-a},\quad a < b,
	\]
	we have
	\[
	\limsup_{c\searrow a}\frac{f(c)-f(a)}{c-a} 
	= \limsup_{\lambda\nearrow 1}\frac{f(\lambda a + (1-\lambda)b) - f(a)}{\lambda a + (1-\lambda)b - a} 
	\leq \frac{f(b) - f(a)}{b-a},
	\]
	and hence 
	\[
	\limsup_{c\searrow a}\frac{f(c)-f(a)}{c-a} 
	\leq \liminf_{b\searrow a}\frac{f(b)-f(a)}{b-a},
	\]
	which implies $f'(a+)$ exists and satisfies 
	\[
	f'(a+)\leq \frac{f(b) - f(a)}{b-a},\quad b>a.
	\]
	If $a>c=b$, we similarly obtain $f'(a-)$ exists and satisfies 
	\[
	f'(a-)\geq \frac{f(c) - f(a)}{c-a},\quad c<a.
	\]
	Noting 
	\begin{align*}
	\frac{f(b) - f(a)}{b-a} - \frac{f(a) - f(c)}{a-c}
	&= \frac{f(b)(a-c) + f(c)(b-a) - f(a)(b-c)}{(b-a)(a-c)}\\
	&= \frac{(b-c)}{(b-a)(a-c)}\left(f(b)\frac{a-c}{b-c} + f(c)\frac{b-a}{b-c} - f(a) \right)\geq 0,
	\end{align*}
	we have $f'(a-)\leq f'(a+)$.
	Using 
	\[
	\frac{f(b) - f(\lambda a + (1-\lambda)b)}{b - \lambda a - (1-\lambda)b} \geq \frac{f(b) - f(a)}{b-a},\quad b>a,
	\]
	we obtain
	\[
	f'(b-) = \lim_{\lambda\searrow0}\frac{f(b) - f(\lambda a + (1-\lambda)b)}{b - \lambda a - (1-\lambda)b}\geq \frac{f(b) - f(a)}{b-a}\geq f'(a+).
	\]
	So $g(a) := f'(a-)$, $a\in I$ is a nondecreasing functions 
	which has at most countably many  discontinuous points.
	At its continuous point $p\in I$,
	\begin{align*}
		f'(p-) = g(p) = \lim_{a\searrow p}g(a) 
		= \lim_{a\searrow p} f'(a-) \geq \lim_{a\searrow p}f'(p+) = f'(p+),
	\end{align*} 
	which implies that $f'$ exists almost everywhere.
	Moreover, for any compact set $K\subset I$, 
	there exists an open interval $(c,b)$ such that $K\subset(c,b)$ and $\overline{(c,b)}\subset I$. 
	For any $a\in K$, if $f'(a)$ exists, 
	then  
	\[
	\frac{f(a) - f(c)}{a - c} 
	\leq f'(a) \leq \frac{f(b) - f(a)}{b - a}.
	\] 
	Therefore $f\in W^{1,\infty}_{\mathrm{loc}}(I) = C^{0,1}_{\mathrm{loc}}(I)$.
\end{proof}



\section{微分}
  \begin{example}
   已知$f(x)$可导,$f(x)+f'(x)\rightarrow a,x\rightarrow+\infty$,则$x\rightarrow+\infty$有$f(x)\rightarrow a,f'(x)\rightarrow0$. (清华大学, 2006)
  \end{example} 

  \begin{proof}
    Without loss of generality, we may assume that $a=0$, otherwise we consider $g=f-a$.
  For $\varepsilon>0$, there exists $R>0$ such that 
  \[
  |f(x)+f'(x)| < \varepsilon,\quad x>R.
  \]
  Then 
  \[
  -\varepsilon e^x < \left(e^xf\right)' < \varepsilon e^x,\quad x>R,
  \]
  which implies by integrating over $(R, x)$,
  \[
  -\varepsilon (e^x - e^R) < e^xf(x) - e^Rf(R) < \varepsilon(e^x-e^R),\quad x>R,
  \]
  thus,
  \[
  e^{R-x}f(R) - \varepsilon(1-e^{R-x}) < f(x) < e^{R-x}f(R) + \varepsilon(1-e^{R-x}), \quad x>R,
  \]
  and we get 
  \[
  \limsup_{x\to\infty}|f(x)| < \varepsilon.
  \]
  This concludes the proof.
  \end{proof}



  \begin{example}
  \hfill\\
   设$f(x)$,$g(x)$在$[a,b]$上连续,$(a,b)$内可导,$g'(x)\neq0$。证明:$\exists\xi\in(a,b)$,使$\frac{f(a)-f(\xi)}{g(\xi)-g(b)}=\frac{f'(\xi)}{g(\xi)}$(武汉大学,2004)
  
  记$F(x)=f(a)g(x)+g(b)f(x)-f(x)g(x)$,
  则$F(x)$在$[a,b]$上连续,在$(a,b)$内可导,又$F(a)=f(a)g(b)=F(b)$,从而由Largange中值定理:
  $\exists\xi\in(a,b),$$$\frac{F(a)-F(b)}{a-b}=F'(\xi)=0$$,即$$f(a)g'(\xi)+g(b)f'(\xi)-f'(\xi)g(\xi)-f(\xi)g'(\xi)=0.$$
  
  即$\frac{f(a)-f(\xi)}{g(\xi)-g(b)}=\frac{f'(\xi)}{g'(\xi)}$。
  
  考虑到$g'(x)\neq 0,$从而$g(\xi)\neq g(b)$,因为不然,若$g(\xi)=g(b),\xi\in(a,b)$,由微分中值定理,$\exists\eta\in(\xi,b),g'(\eta)=0$与$g'(x)\neq0$矛盾,故上式是有意义的。
  \end{example}
  \begin{example}
  \hfill\\
   设$f(x)$在$(a,b)$上可微,$f'(x)$在$(a,b)$上单调。求证:$f'(x)$在$(a,b)$上连续。(北京航空航天大学,2004)
  
  不妨设$f(x)$单调递增,从而$\forall c\in(a,b)$,$f'(x)$在$(a,c)$上单调递增有上界,$f(x)\leq f(c)$。由确界原理,$\exists S_c=\sup_{x\in(a,c)}f'(x)$,由确界定义与左极限定义易知:$f(c-)=S_c$。同理$f(c+)=I_c=\inf_{x\in(c,b)}f'(x)$。
  故单调的导函数不存在第二类间断点。
  
  又$f'(c-)\leq f'(c+)$,由达布引理,导函数具有介值性,若$f'(c-)<f'(c+)$,$\forall\eta\in[f'(c-),f'(c+)],\exists\xi\in[c-\varepsilon,c+\varepsilon],f'(\xi)=\eta.$而这样的$\xi$时不存在的,因为$f'(x)$单调。从而$f'(c-)=f'(c+)$,即导函数连续。
  \end{example}
  \begin{exercise}
  \hfill\\
 设$f(x)$在$[a,b](a<b)$上处处有正的二阶导数,证明存在常数$0<m\leq M$,使$\frac m4(x-a)^2\leq f(x)-2f(\frac{x+a}2)+f(a)\leq \frac M4(x-a)^2,x\in(a,b]$。(南京航空航天大学,2003) 
  
  设$$F(x)=
  \begin{cases}
  \frac{f(x)-2f(\frac{x+a}{2})+f(a)}{(x-a)^2},&x>a,\\
  \frac{f''(a)}{4},&x=a,\\
  \end{cases}
  $$
  则
  \begin{align*}
  F(a+)&=\lim_{x\rightarrow a+}\frac{f(x)-2f(\frac{x+a}{2})+f(a)}{(x-a)^2}\\
  &=\lim_{x\rightarrow a+}\frac{f'(x)-f'(\frac{x+a}{2}}{2(x-a)}\\
  &=\lim_{x\rightarrow a+}\frac{f''(x)}{2}-\frac{f''(\frac{x+a}{2})}{4}\\
  &=\frac{f''(a)}{4}\\
  &=F(a),\\
  \end{align*}
  从而$F(x)$在$[a,b]$上连续。
  
  下面证明$F(x)$在$[a,b]$上恒正。
 因为
 \begin{align*}
 &f(x)-2f(\frac{x+a}{2})+f(a)\\
 &=f'(\xi)(x-\frac{x+a}{2})-f'(\eta)(\frac{x+a}{2}-a)\\
 &=[f'(\xi)-f'(\eta)]\frac{x-a}{2}\\
 &=\frac{f''(\phi)}2(x-a)(\xi-\eta)\quad(x>a),\\
 \end{align*}
 其中$\xi\in(\frac{x+a}{2},x)$,$\eta\in(a,\frac{x+a}{2})$,$\phi\in(\eta,\xi)$,又$f''(x)$在$[a,b]$上恒正。
 故$F(x)$在$[a,b]$上恒正。$F(x)$在$[a,b]$上连续故闭存在最大值$\frac{M}{4}$和最小值$\frac{m}{4}$。这就完成了证明。  
  \end{exercise}  
  
  \hfill\\

  \section{微分中值定理及其应用}
  \begin{example}
  \hfill\\
  设$f(x)\in[0,1]$上可导,$f(1)=2\int xf(x)dx$。求证:存在$\xi\in(0,1)$,使得$f'(\xi)=-\frac{f(\xi)}{\xi}$。(清华大学,2003)
  
  
  反设$\forall\xi\in(0,1),f'(\xi)+\frac{f(\xi)}{\xi}=\frac{\xi f'(\xi)+f(\xi)}{\xi}=0$均不成立。
  
  从而$\xi f'(\xi)+f(\xi)$恒正或恒负(Darboux引理,导函数介值性)。不妨设
  $$\forall\xi\in(0,1),\xi f'(\xi)+f(\xi)>0.$$
  从而$f(x)x$在$(0,1)$上严格单调递增。从而
  \[
  \begin{aligned}
  f(1)&=2\int_0^{1/2}xf(x)dx\\
  &=\int_0^{1/2}xf(x)dx+\int_0^{1/2}xf(x)dx\\
  &<\int_0^{1/2}xf(x)dx+\int_{1/2}^1xf(x)dx\\
  &=\int_0^1xf(x)dx\\
  &<\int_0^1f(1)dx\\
  &=f(1).
  \end{aligned}
  \]
  矛盾。
  
  故$\exists\xi\xi(0,1),f'(\xi)\xi+f(\xi)=0$。
  \end{example}
  \begin{example}
  \hfill\\
  $x^y+y^x>1,x,y>0$.(中国科学院,2007)
  
  
  考虑$f(x)=(1+x)^r-rx-1,(x>0,r>1)$,
  
  $$f'(x)=r(1+x)^{r-1}-r>r(1+x)^0-r=0.$$
  故$f(x)>f(0)=0,$,得$(1+x)^r>rx+1.$
  $$(1+\frac{x}{y})^{\frac1x}>\frac1y+1>\frac1y,x=\frac xy,r=\frac1x.$$
  $$\Longrightarrow 1+\frac xy>\frac{1}{y^x}$$
  $$\Longrightarrow y^x>\frac{y}{x+y}$$
  同理$x^y>\frac{x}{x+y},$
  
  故$x^y+y^x>1,x,y\in(0,1).$ 
  \end{example}
  \begin{example}
  \hfill\\
   若$-\infty<a<b<c<+\infty,f(x)$在$[a,c]$上连续,且$f(x)$在$(a,c)$内二阶可导,求证:存在$\xi\in(a,c)$,使得(四川大学,2005)$$\frac{f(a)}{(a-b)(a-c)}+\frac{f(b)}{(b-c)(b-a)}+\frac{f(c)}{(c-a)(c-b)}=\frac12f^{''}(\xi)$$。
\end{example}
\begin{example}
\hfill\\
  已知$f(x)$在$[a,b]$上单调增加,$f(a)\geq a,f(b)\leq b$,求证:$\exists\xi\in[a,b]$,使得$f(\xi)=\xi$。(武汉大学,2006)
  
  
  反设$\forall\xi\in[a,b],f(\xi)\neq\xi,$则$\forall\xi\in[a,b],f(\xi)<\xi$或$f(\xi)>\xi$。
  记$A=\{\xi:f(\xi)>\xi\},B=\{\xi:f(\xi)>\xi\},$则由$f(a)>a,f(b)<b$,有$a\in A,b\in B$。于是$A,B$不空且$A\cup B=[a,b],A\cap B=\emptyset$。因为$B$有界,故$\eta=\inf B$存在且$\eta\in[a,b]$。
  
  若$\eta=a$,则存在子列$\{x_n\}\subset[a,b]$,且当$x_n\rightarrow a_+,n\rightarrow\infty$时有$f(x_n)<x_n$,从而对$\varepsilon=\frac{f(a)-a}{2},\exists N,\forall n>N,f(x_n)<x_n<a+\varepsilon<f(a)$与$f(x)$在$[a,b]$上单调增加矛盾,从而$\eta>a$,于是$\forall x\in[a,\eta),f(x)>x.$
  
  若$\eta\in A$即$f(\eta)>\eta$,同上矛盾。
  
  则$\eta\in B$,即$f(\eta)<\eta$,而$f(\eta-)=\lim_{x\rightarrow\eta-}f(x)\geq\eta>f(\eta)$,与$f(x)$单调增加矛盾。故$\eta\not\in B,\eta\not\in A$,则$f(\eta)=\eta$。
  \end{example}
  \begin{example}
  \hfill\\
   设$f(x)$在$[0,1]$上连续可微,且$f'(0)\neq0$,若对$\forall x\in(0,1),\theta(x)$满足$\int_0^xf(t)dt=f(\theta(x))x$,求证:$\displaystyle \lim_{x\rightarrow 0^+}\frac{\theta(x)}x=\frac12$.
   
     令$$F(x)=\int_0^xf(t)\mathrm{d}t,$$
  则有$F''(0)=f(0)\neq0$,用泰勒公式得
  $$F(x)=F(0)+f'(0)x+\frac{1}{2}F''(0)x^2+o(x^2)$$
  及
  $$F'(\theta)=F'(0)+F''(0)+o(\theta),$$
  由$F(0)=0$,$F(x)=f(\theta(x))x=F'(\theta)x$联立得
  $$F'(0)x+\frac{1}{2}F''(0)x^2+o(x^2)=x(F'(0)+F''(0)\theta+o(\theta)),$$
  消去$F'(0)x$并同时除$F''(0)x^2$得:
  $$\frac{1}{2}+o(1)=\frac{\theta}{x}+o(\frac{\theta}{x}=\frac{\theta}{x}(1+o(1)),x\rightarrow0+.$$
  即
  $$\frac{\theta}{x}=\frac{\frac{1}{2}+o(1)}{1+o(1)},x\rightarrow0+.$$
  即
  $$\lim_{x\rightarrow0+}\frac{\theta}{x}=\frac{1}{2}.$$
  \end{example}
  
  
 \hfill\\
 
  
  \section{定积分}
  \begin{example}
  \hfill\\
  设在任意的有限区间$[0,A]$上$f(x)$Riemann可积,且$\displaystyle\lim_{x\rightarrow+\infty}f(x)=0$,求证:$\displaystyle\lim_{t\rightarrow+\infty}\frac 1t\int_0^t|f(x)|dx=0$(清华大学,2006)
  
  
  因为$\lim_{x\rightarrow\infty}f(x)=0,$所以$$\forall\varepsilon>0,\exists A>0,\forall x>A,|f(x)|<\frac{\varepsilon}{2},$$
  又$f(x)$在$[0,A]$上Riemann可积,从而$f(x)$在$[0,A]$上有界,即$$\exists M>0,\forall x\in[0,A],|f(x)|<M.$$
  于是对上述$\varepsilon>0,\exists T=\max\{\frac{2MA}{\varepsilon},A\},\forall t>T,$
  $$
  \begin{aligned}
  \frac{1}{t}\int_0^t|f(x)|dx&=\frac1t\int_0^A|f(x)|dx+\frac1t\int_A^t|f(x)|dx\\
  &<\frac{MA}{t}+\frac{t-A}{t}\cdot\frac{\varepsilon}{2}\\
  &<\frac{\varepsilon}{2}+\frac{\varepsilon}{2}=\varepsilon\\
  \end{aligned}
  $$
  从而$$\lim_{t\rightarrow\infty}\frac{1}{t}\int_0^t|f(x)|dx=0.$$
  \end{example}
  \begin{example}
  \hfill\\
  $\int_0^{2\pi}\frac x{1+cos^2x}dx$(浙江大学,2004)
  
  
  \begin{equation}
  \begin{aligned}
    &\int_0^{2\pi }\frac{x dx}{1+\cos^2x}\\&=\int_0^{\pi}\frac{x dx}{1+\cos^2x}+\int_0^{\pi}\frac{x+\pi dx}{1+\cos^(x+\pi)}\\
    &=2\int_0^{\pi}\frac{xdx}{1+\cos^2x}+\int_0^{\pi}\frac{\pi dx}{1+\cos^2x}\\
    &=2[\int_0^{\frac{\pi}{2}}\frac{xdx}{1+\cos^2x}+\int_0^{\frac{\pi}{2}}\frac{x+\frac{\pi}{2}dx}{1+\cos^2(x+\frac{\pi}{2})}]+\\
    &\int_0^{\frac{\pi}{2}}\frac{\pi dx}{1+\cos^2x}+\int_0^{\frac{\pi}{2}}\frac{\pi dx}{1+\cos^2}\\
    &=2[\int_{-\frac{\pi}{2}}^0\frac{x+\frac{\pi}2dx}{1+\cos^2(x+\frac{\pi}{2})}+\int_0^{\frac{\pi}{2}}\frac{xdx}{1+\sin^x}+\frac{\pi}{2}\int_0^{\frac{\pi}{2}}\frac{dx}{1+\sin^x}]\\
    &+\pi[\int_0^{\frac{\pi}{2}}\frac{dx}{1+\cos^2x}+\int_0^{\frac{\pi}{2}}\frac{dx}{1+\sin^2x}]\\
    &=2[\int_{-\frac{\pi}{2}}^0\frac{xdx}{1+\sin^2x}+\frac{\pi}{2}\int_{-\frac{\pi}{2}}^0\frac{dx}{1+\sin^2x}\\
    &+\int_0^{\frac{\pi}{2}}\frac{xdx}{1+\sin^2x}+\frac{\pi}{2}\int_0^{\frac{\pi}{2}}\frac{dx}{1+\sin^2x}]\\
    &+2\pi\int_0^{\frac{\pi}{2}}\frac{dx}{1+\cos^2x}\\
    &=2[\int_{-\frac{\pi}{2}}^0\frac{xdx}{1+\sin^2x}+\frac{\pi}{2}\int_{\frac{\pi}{2}}^0\frac{d(-x)}{1+\sin^2x}\\
    &+\int_0^{\frac{\pi}{2}}\frac{xdx}{1+\sin^2x}+\frac{\pi}{2}\int_0^{\frac{\pi}{2}}\frac{dx}{1+\sin^2x}]\\
    &+2\pi\int_0^{\frac{\pi}{2}}\frac{dx}{2\cos^2x+\sin^2x}\\
    &=2\pi\int_0^{\frac{\pi}{2}}\frac{d\tan x}{\tan^2x+2}\\
    &=\sqrt{2}\pi(\arctan(\frac{\tan x}{\sqrt{2}}))|_0^{\frac{\pi}{2}}\\
    &=\frac{\pi^2}{\sqrt{2}}.\\
  \end{aligned}
  \end{equation}
  \end{example}
  \begin{example}
  \hfill\\
  若$f(x)$在$(0,+\infty)$上可微,$\displaystyle\lim_{x\rightarrow+\infty}\frac{f(x)}x=0$,求证:$(0,+\infty)$内存在一个单调数列${\xi_n}$,使得$\displaystyle\lim_{n\rightarrow\infty}\xi_n=+\infty$且$\displaystyle\lim_{n\rightarrow\infty}f'(\xi_n)=0$。 (浙江师范大学,2005)
  
  
  $\lim_{x\rightarrow\infty}\frac{f(x)}{x}\Longrightarrow\forall\varepsilon>0,\exists N,\forall m,n>N$,有$$|\frac{f(m)}{m}-\frac{f(n)}{n}|<\frac{\varepsilon}{2},|\frac{f(n)}n|<\frac{\varepsilon}{2}.$$
  
  于是取$m=2n$,有$|\frac{f(2n)-f(n)}{n}|=|\frac{f(2n)}{2n}-\frac{f(n)}{n}+\frac{f(2n)}{2n}|<\frac{\varepsilon}{2}+\frac{\varepsilon}{2}=\varepsilon.$
  
  取$\{\xi_n\}$,满足$f'(\xi_n)=\frac{f(2^{n+1})-f(2^n)}{2^n},\xi_n\in(2^n,2^{n+1}).$
  
  显见$\xi_n\rightarrow\infty,n\rightarrow\infty.$对上述$\varepsilon>0,\exists N_1>\max\{\log_2N,1\}$时有$|f'(\xi_n)|<\varepsilon$,从而$f'(\xi_n)\rightarrow0,(n\rightarrow\infty)$。
  \end{example}
  \begin{example}
  \hfill\\
  设函数$f(x)$在$[a,b]$上连续且严格递增,证明下式成立$$\int_a^bf(x)dx=bf(b)-af(a)-\int_{f(a)}^{f(b)}f^{-1}(x)dx.$$(河海大学,2006)

对$a,b$中插入$n$个分点$$a=x_0<x_1<x_2<\cdots<x_{n-1}<x_n=b,$$
从而对应$x=f^{-1}(y)$也有$n$个分点:
$$f(a)=f(x_0)<f(x_1)<f(x_2)<\cdots<f(x_{n-1})<f(x_n)=f(b).$$
记$\Delta_n=\max\{x_i-x_{i-1}:i=1,2,\cdots,n\},$因为$f(x)$连续,故
$$\Delta_n\rightarrow0,\max\{f(x_i)-f(x_{i-1}):i=1,2,\cdots,n\}\rightarrow0,n\rightarrow\infty.$$
从而
\begin{align*}
\int_a^bf(x)\mathrm{d}x+\int_{f(a)}^{f(b)}f^{-1}(x)\mathrm{d}x&=\lim_{\Delta_n\rightarrow0}\sum_{i=1}^nf(x_i)(x_i-x_{i-1})\\
&\qquad+\lim_{\Delta_n\rightarrow0}\sum_{i=1}^nx_{i-1}[f(x_i)-f(x_{i-1})]\\
&=\lim_{\Delta_n\rightarrow0}\sum_{i=1}^nf(x_i)x_i-f(x_{i-1})x_{i-1}\\
&=f(b)b-f(a)a.\\
\end{align*}
\end{example}
\begin{example}
\hfill\\
计算积分$\int_0^1\frac{\ln(1+x)}{1+x^2}dx$。(南京航空航天大学,2005)
  
  \begin{align*}
  \int_0^1\frac{\ln(1+x)}{1+x^2}dx&=\int_0^{\frac{\pi}{4}}\frac{\ln(1+\tan t)}{1+\tan^2t}d\tan t\\
  &=\int_0^{\frac{\pi}{4}}\ln(1+\tan t)dt\\
  &=\int_0^{\frac{\pi}{4}}\ln(\sin t+\cos t)dt-\int_0^{\frac{\pi}{4}}\ln\cos tdt\\
  &=\frac{\pi}{8}\ln2+\int_0^{\frac{\pi}{4}}\ln\cos(t-\frac{\pi}{4})-\int_0^{\frac{\pi}{4}}\ln\cos tdt\\
  &=\frac{\pi}{8}\ln2.\\
  \end{align*}
\end{example}
\begin{example}
  \hfill\\
   利用可积条件证明:函数$f(x)=\frac1x-[\frac1x],x\neq0;f(x)=0,x=0$在$[0,1]$上可积。(南京师范大学,2006)注:此题函数为分段函数,标准书写格式为大括号括起来书写。
   
     $0\leq f(x)\leq1$且$f(x)$在$[0,1]$上的不连续点为$$x=\frac{1}{2},\frac{1}{3},\cdots,\frac{1}{n},\cdots$$与$x=0$。$\forall\varepsilon>0$,取定$m>\frac{2}{\varepsilon}$,$f(x)$在区间$[\frac{1}{m},1]$上只有有限个不连续点,因此,$f(x)$在$[\frac{1}{m},1]$上可积,即存在$[\frac{1}{m},1]$的一个划分$P$,使得
  $$\sum_{i=1}^nw_i\Delta x_i<\frac{\varepsilon}{2},$$
  将$P$的分点和$0$合在一起,作为$[0,1]$的划分$P'$,则
  $$\sum_{i=1}^{n+1}w_i'\Delta x_i'=\sum_{i=1}^nw_i\Delta x_i+w_i'\Delta x_i'<\frac{\varepsilon}{2}+\frac{\varepsilon}{2}=\varepsilon.$$
  因此,$f(x)$在$[0,1]$上可积。
\end{example}
\begin{example}
\hfill\\
 设$f(x)\in C^1[0,1]$,试证明:$$\displaystyle\lim_{n\rightarrow\infty}n[\int_0^1f(t)dt-\frac1n\sum_{k=1}^{n-1}f(\frac kn)]=\frac{f(1)-f(0)}2.$$(南京大学,2005)
 
 由泰勒公式可得
$$f(t)=f(\frac{k}{n})+f'(\xi_k)(t-\frac{k}{n}),$$
其中$\xi_k$介于$t$和$\frac{k}{n}$之间,$k=0,1,2,\cdots,n-1$,则
\begin{align*}
\int_0^1f(t)\mathrm{d}t-\frac{1}{n}\sum_{k=0}^{n-1}f(\frac{k}{n})
&=\sum_{k=0}^{n-1}\int_{\frac{k}{n}}^{\frac{k+1}{n}}[f(t)-f(\frac{k}{n})\mathrm{d}t\\
&=\sum_{k=0}^{n-1}\int_{\frac{k}{n}}^{\frac{k+1}{n}}[f'(\xi_k)(t-\frac{k}{n})\mathrm{d}t\\
&=\sum_{k=0}^{n-1}\frac{1}{2n^2}f'(\xi_k)
\end{align*}
由于$f(t)\in C^1[0,1]$,进一步可得
$$f(1)-f(0)=\int_0^1f'(t)\mathrm{d}t=\lim_{n\rightarrow\infty}\sum_{k=0}^{n-1}\frac{1}{n}f'(\xi_k).$$

这样就完成了证明。
\end{example}
\begin{example}
\hfill\\
 求$\displaystyle\lim_{n\rightarrow\infty}\sum_{k=1}^n\sin\frac k{n^2}$.

  \begin{equation*}
  \begin{aligned}
  e^{\int_0^1\ln f(x)dx}&=\displaystyle e^{\lim_{n\rightarrow\infty}\frac1n\sum_{i=1}^n\ln f(\frac kn)}\\
  &=e^{\lim_{n\rightarrow\infty}[\ln(\prod_{i=1}^nf(\frac kn))]^{\frac1n}}\\
  &=\lim_{n\rightarrow\infty}[\prod_{i=1}^nf(\frac kn)]^{\frac1n}\\
  &\leq\lim_{n\rightarrow\infty}\frac1n\sum_{i=1}^nf(\frac kn)\\
  &=\int_0^1f(x)dx\\
  \end{aligned}
  \end{equation*}
%  $\displaystyle e^{\int_0^1\ln f(x)dx}=\displaystyle e^{\lim_{n\rightarrow\infty}\frac1n\sum_{i=1}^n\ln f(\frac kn)}=e^{\lim_{n\rightarrow\infty}[\ln(\prod_{i=1}^nf(\frac kn))]^{\frac1n}}=\lim_{n\rightarrow\infty}[\prod_{i=1}^nf(\frac kn)]^{\frac1n}\leq\lim_{n\rightarrow\infty}\frac1n\sum_{i=1}^nf(\frac kn)=\int_0^1f(x)dx$.
\end{example}
\begin{example}
\hfill\\
 设$f(x)$在$[a,b]$上可积,求证:$$\displaystyle\lim_{p\rightarrow+\infty}\int_a^bf(x)\sin pxdx=0,\lim_{p\rightarrow+\infty}\int_a^bf(x)\cos psdx=0.$$
 
 对任意有界区间$[\alpha,\beta]$有$$|\int_{\alpha}^{\beta}\sin px\mathrm{d}x|=|\frac{\cos p\alpha-\cos p\beta}{p}|\leq\frac{2}{p}.$$
设在$[a,b]$上,$|f(x)|\leq M$,任给$\varepsilon>0$,存在$[a,b]$的分割$T$:
$$a=x_0<x_1<x_2<\cdots<x_{n-1}<x_n=b,$$
使得
$$S(T,f)-s(T,f)<\frac{\varepsilon}{2},$$
其中$S(T,f)$与$s(T,f)$分别代表$f(x)$关于$T$的大和和小和,于是当$p\geq\frac{4nM}{\varepsilon}$,有
\begin{align*}
|\int_a^bf(x)\sin px\mathrm{d}x|&=
|\sum_{k=1}^n\int_{x_{k-1}}^{x_k}(f(x_k)+f(x)-f(x_k))\sin px\mathrm{d}x|\\
&\leq\sum_{k=1}^n(|f(x_k)||\int_{x_{k-1}}^{x_k}\sin px\mathrm{d}x|+\int_{x_{k-1}}^{x_k}|f(x)-f(x_k)||\sin px|\mathrm{d}x)\\
&<(S(T,f)-s(T,f))+\frac{2Mn}{p}\\
&<\varepsilon.\\
\end{align*}
同理可证后者极限等式成立。
\end{example}
\begin{example}
\hfill\\
 设$f(x)$在$[A,B]$上连续,$\phi(x)$在$[a,b]$上可积,而且$\phi([a,b])\subset[A,B]$,求证:$f(\phi(x))$在$[a,b]$上可积。
 
   设$|f(n)|\leq M$,$\forall u\in[A,B]$。由$f(u)$在$[A,B]$上连续可知,$\forall\varepsilon>0$,$\exists\delta>0$,当$u',u''\in[A,B]$且$|u'-u''|<\delta$时,有
  $$|f'(u')-f(u'')|<\frac{\varepsilon}{2(b-a)}.$$
  由$\phi(x)$在$[a,b]$上可积可知,$\exists\eta>0$对一切分割
  $$T:a=x_0<x_1<x_2<\cdots<x_n=b,$$
  只要$\|T\|<\eta$时,有
  $$\sum_{k=1}^nw_k(\phi)\Delta x_k<\frac{\delta\varepsilon}{4M},$$
  其中$$w_k(\phi)=\sup_{x_k\leq x\leq x_{k+1}}\phi(x)-\inf_{x_k\leq x\leq x_{k+1}}\phi(x).$$
  将 $\sum\limits_{k=1}^nw_k(\phi)\Delta x_k$分成$\sum\limits'$与$\sum\limits''$在$\sum\limits'$中有$w_k(\phi)\geq\delta$,在$\sum\limits''$中有$w_k(\phi)<\delta$,则
  $$\delta\sum'\Delta x_k=\sum'\delta\Delta x_k\leq\sum_{k=1}^nw_k(\phi)\Delta x_k\leq\frac{\delta\varepsilon}{4M}.$$
  即$$\sum'\Delta x_k<\frac{\varepsilon}{4M}.$$
  综上所述,可得
  \begin{align*}
  \sum_{k=1}^nw_k(f(\phi))\Delta x_k&=\sum'w_k(f(\phi))\Delta x_k+\sum''w_k(f(\phi))\Delta x_k\\
  &\leq2M\sum'\Delta x_k+\frac{\varepsilon\sum''\Delta x_k}{2(b-a)}\\
  &<\varepsilon.\\
  \end{align*}
  这就证得了$f(\phi(x))$在$[a,b]$上可积。
\end{example}  
 \hfill\\
 
    \section{反常积分}
    \begin{example}
    \hfill\\
    $\int_0^{\frac {\pi}2}\ln\sin xdx$
    
    
    \begin{equation}
  \begin{aligned}
  &\int_0^{\frac{\pi}{2}}\ln\sin xdx\\
  &=\int_0^{\frac{\pi}{2}}(\ln2+\ln\cos\frac{x}{2}+\ln\sin\frac{x}{2})dx\\
  &=2\int_0^{\frac{\pi}{4}}(\ln2+\ln\cos x+\ln\sin x)dx\\
  &=\frac{\pi\ln2}{2}+2\int_0^{\frac{\pi}{4}}\ln\cos xdx+2\int_0^{\frac{\pi}{4}}\sin xdx\\
  &=\frac{\pi\ln2}{2}-2\int_0^{-\frac{\pi}{4}}\ln\cos xdx+2\int_0^{\frac{\pi}{4}}\sin xdx\\
  &=\frac{\pi\ln2}{2}-2\int_{\frac{\pi}{2}}^{\frac{\pi}{4}}\ln\cos(x-\frac{\pi}{2})d(x-\frac{\pi}{2})+2\int_0^{\frac{\pi}{4}}\sin xdx\\
  &=\frac{\pi\ln2}{2}+2\int_0^{\frac{\pi}{2}}\ln\sin xdx\\
  &=-\frac{\pi\ln2}{2}\\
  \end{aligned}
  \end{equation}
    
    \end{example}
  
  
    \begin{example}
    \hfill\\
    设$a>0$,函数$f(x)$在$[0,a]$上连续可微,证明:$|f(0)|\leq\frac1a\int_0^a|f(x)|dx+\int_0^a|f'(x)|dx$
    
    
    $f(x)$在$[0,a]$上连续,由积分第一中值定理:
  
  存在$\xi\in(0,a)$使$\int_0^af(x)dx=f(\xi)(a-0)=f(\xi)a.$
  
  由于$f(0)=f(\xi)-\int_0^{\xi}f'(x)dx$,
  
  可得
  \begin{equation}
  \begin{aligned}
  |f(0)|&=|f(\xi)-\int_0^{\xi}f'(x)dx|\\
  &\leq|f(\xi)|+|\int_0^{\xi}f'(x)dx|\\
  &\leq\frac{1}{a}|\int_0^af(x)dx|+|\int_0^{\xi}f'(x)dx|\\
  &\leq\frac{0}{a}|f(x)|dx+\int_0^a|f'(x)|dx\\
  \end{aligned}
  \end{equation}
    
    \end{example}
  
  
     \begin{example}
    \hfill\\
    证明:$\int_0^1\frac{\ln\frac1x}{1-x}dx=\frac{\pi^2}6$。(复旦大学,2001)
    
    
    \begin{equation}
  \begin{aligned}
  \int_0^1\frac{\ln\frac1x}{1-x}dx&=-\int_0^1\frac{\ln(1-x)}{x}dx\\
  &=\int_0^1\sum_{n=1}^{\infty}\frac{x^{n-1}}{n}dx\\
  &=\int_0^1\sum_{n=0}^{\infty}\frac{x^n}{n+1}dx\\
  &=\sum_{n=0}^{\infty}\int_0^1\frac{x^n}{n+1}dx\\
  &=\sum_{n=0}^{\infty}\frac{1}{(n+1)^2}\\
  &=\sum_{n=1}^{\infty}\frac{1}{n^2}\\  
  &=\frac{\pi^2}6
  \end{aligned}
  幂级数在收敛域内可交换积分与求和次序。
  \end{equation}
    
    \end{example}
  
    \begin{example}
    \hfill\\
    计算$\int_0^{+\infty}\frac{\sin x}xdx$(浙江大学,2005)
  考虑二重积分$$\int_0^{+\infty}\int_0^{+\infty}e^{-xy}\sin xdxdy,$$
    
    
    $\int_0^{+\infty}e^{-xy}\sin xdx$关于$y$在$(0,+\infty)$上内闭一致收敛。从而
  \begin{equation}
  \begin{aligned}
  \int_0^{+\infty}\int_0^{+\infty}e^{-xy}\sin xdxdy&=\int_0^{+\infty}\sin xdx\int_0^{+\infty}e^{-xy}dy\\
  &=\int_0^{+\infty}\sin xdx(-\frac{e^{-xy}}{x}|_0^{+\infty}\\
  &=\int_0^{+\infty}\frac{\sin x}xdx,\\
  \end{aligned}
  \end{equation}
  又
  \begin{equation}
  \begin{aligned}
  &\int_0^{+\infty}\int_0^{+\infty}e^{-xy}\sin xdxdy\\
  &=\int_0^{+\infty}dy\int_0^{+\infty}e^{-xy}\sin xdx\\
  &=\int_0^{+\infty}dy[(-\frac{e^{-xy}}{y}\sin x)|_0^{+\infty}+\int_0^{+\infty}\frac{e^{-xy}\cos x}{y}dx]\\
  &=\int_0^{+\infty}dy[(-\frac{e^{-xy}\sin x}{y}|_0^{+\infty}-\int_0^{+\infty}\frac{e^{-xy}\sin x}{y}]\\
  &=\int_0^{+\infty}dy[\frac{1}{y^2}-\frac{1}{y^2}\int_0^{+\infty}e^{-xy}\sin xdx]\\
  &=\int_0^{+\infty}\frac{dy}{1+y^2}\\
  &=\frac{\pi}{2}
  \end{aligned}
  \end{equation}
    
    \end{example}  
    \begin{example}
    \hfill\\
    讨论不同$p$对$f(x)$在$[1,+\infty)$上积分的敛散性,$$\displaystyle f(x)=\ln(1+\frac{\sin x}{x^p}).$$(北京大学,2007)
    
    
    \begin{align*}
  \overline{\lim_{x\rightarrow\infty}}x^p|\ln(1+\frac{\sin x}{x^p})|
  &=\lim_{x\rightarrow\infty}x^p\ln(1-\frac{1}{x^p})^{-1}\\
  &=\lim_{x\rightarrow\infty}x^p\ln(1+\frac{1}{x^p-1})\\
  &=\left\{\begin{array}{ll}1,&p>0\\\ln2,&p=0\\0,&p<0.\end{array}\right.  
  \end{align*}

  从而当$p>1$时,$f(x)$绝对收敛;当$p\leq0$时,$f(x)$发散;当$p=1$时,$f(x)=\ln(1+\frac{\sin x}{x})$,考虑到

  \begin{align*}
  &\int_{2n\pi}^{2(n+1)\pi}\ln(1+\frac{\sin x}{x})dx\\&=\int_{2n\pi}^{(2n+1)\pi}\ln(1+\frac{\sin x}{x})dx+\int_{(2n+1)\pi}^{2(n+1)\pi}\ln(1+\frac{\sin x}{x}dx\\
  &=\int_{2n\pi}^{(2n+1)\pi}\ln(1+\frac{\sin x}{x})dx+\int_{2n\pi}^{(2n+1)\pi}\ln(1-\frac{\sin x}{x+\pi})dx\\
  &=\int_{2n\pi}^{(2n+1)\pi}\ln(1+\frac{\pi\sin x}{x(x+\pi)}-\frac{\sin^2x}{(x+\pi)x})dx\\
  &=\int_{2n\pi}^{(2n+1)\pi}\ln(1+\frac{(\pi-1)\sin x}{x(x+\pi)})dx\\
  &\leq\int_{2n\pi}^{(2n+1)\pi}\ln(1+\frac{\pi}{x(x+\pi)})dx\\
  &<\int_{2n\pi}^{(2n+1)\pi}(\frac{1}{x}-\frac{1}{x+\pi}dx\\
  &=\ln\frac{(2n+1)^2}{(2n+1)(2n)}\\
  &=\ln\frac{4n^2+4n+1}{4n^2+4n}\\
  &<\frac{1}{4n^2}\\
  \end{align*}

  
  从而$0<\int_{2\pi}^{2n\pi}\ln(1+\frac{\sin x}{x})dx<\frac{\pi^2}{24}$。
  
  而$\forall A>0,\exists n\in N_+$,有$A\in[2n\pi,2(n+1)\pi]$,从而!!!!!!
  
  
  
  !!
  
  @@
  
  未完成!!
    
    \end{example}  
\hfill\\
    
 \section{数项级数}
     \begin{example}
    \hfill\\
    利用数项级数$\displaystyle\sum_{n=1}^{\infty}\frac 1{n^2}=\frac{\pi^2}6$计算积分$I=\int_0^1\frac{\ln(x+1)}xdx$。(三峡大学,2006;厦门大学,2000)
    
    
     \begin{equation}
  \begin{aligned}
  I&=\int_0^1\frac{\ln(1+x)}{x}dx\\
  &=\int_0^1\sum_{n=1}^{\infty}\frac{(-1)^{n+1}x^{n-1}}{n}dx\\
  &=\sum_{n=1}^{\infty}\int_0^1\frac{(-1)^{n+1}x^{n-1}}{n}dx\\
  &=\sum_{n=1}^{\infty}\frac{(-1)^{n+1}}{n^2}dx\\
  &=\sum_{n=1}^{\infty}\frac{1}{(2n-1)^2}-\sum_{n=1}^{\infty}\frac{1}{(2n)^2}\\
  &=\sum_{n=1}^{\infty}\frac{1}{n^2}-\frac{1}{2}\sum_{n=1}^{\infty}\frac{1}{n^2}\\
  &=\frac{1}{2}\sum_{n=1}^{\infty}\frac{1}{n^2}=\frac{\pi^2}{12}.
  \end{aligned}
  \end{equation} 
    
    \end{example} 
    \begin{example}
    \hfill\\
    设$\displaystyle a_n>0,n=1,2,\cdot\cdot\cdot\lim_{n\rightarrow\infty}n(\frac{a_n}{a_{n-1}}-1)=c>0$。证明:$\displaystyle\sum_{n=1}^{\infty}(-1)^{n+1}a_n$收敛。(大连理工大学,2004)  
    
    
   由$\displaystyle a_n>0,n=1,2,\cdot\cdot\cdot\lim_{n\rightarrow\infty}n(\frac{a_n}{a_{n-1}}-1)=c>0$可知,当$n$充分大时,有$a_n>a_{n+1}$。取$a>0,b>0$使得$c>b>a>0$,当$n$充分大时,可得$$\frac{a_n}{a_{n+1}}>1+\frac{b}{n}>(1+\frac{1}{n})^{\alpha}=\frac{(n+1)^{\alpha}}{n^{\alpha}},$$
  从而得到$(n+1)^{\alpha}a_{n+1}<n^{\alpha}a_n$,即数列$\{n^{\alpha}a_n\}$对较大的$n$单调递减。从而存在$M>0$,使得$n^{\alpha}a_n\leq M$,即$$0<a_n<\frac{M}{n^{\alpha}}.$$
  由夹逼定理可知数列$\{a_n\}$趋于0,$\sum_{n=1}^{\infty}(-1)^{n+1}a_n$时Lebniz级数,因此$\sum_{n=1}^{\infty}(-1)^{n+1}a_n$收敛。   
    
    \end{example}  
    \begin{example}
    \hfill\\
  利用幂级数展开以及公式$\displaystyle\sum_{n=1}^{\infty}\frac1{n^2}=\frac{\pi^2}6$计算$\int_0^1\frac{\ln x}{1-x^2}dx$。    
    
    
  \begin{equation}
  \begin{aligned}
  \int_0^1\frac{\ln x}{1-x^2}dx&=\int_0^1\sum_{n=0}^{\infty}x^{2n}\ln xdx\\
  &=\sum_{n=0}^{\infty}\int_0^1x^{2n}\ln xdx\\
  &=\sum_{n=0}^{\infty}((\frac{x^{2n+1}}{2n+1}\ln x)|_0^1-\int_0^1\frac{x^{2n}}{2n+1}dx)\\
  &=\sum_{n=0}^{\infty}-\frac{1}{(2n+1)^2}\\
  &=\sum_{n=1}^{\infty}-\frac{1}{(2n-1)^2}-\sum_{n=1}^{\infty}-\frac{1}{(2n)^2}+\sum_{n=1}^{\infty}\frac{1}{n^2}+\frac{1}{(2n)^2}+\sum_{n=1}^{\infty}\frac{1}{n^2}\\
  &=-\sum_{n=1}^{\infty}\frac{1}{n^2}+\frac{1}{4}\sum_{n=1}^{\infty}\frac{1}{n^2}\\
  &=-\frac{\pi^2}6+\frac{\pi^2}{24}\\
  &=-\frac{\pi^2}{12}\\
  \end{aligned}
  \end{equation}   
    
    \end{example}  
    \begin{example}
    \hfill\\
  $\displaystyle\sum_{n=0}^{\infty}q^n\cos n\theta(|q|<1)$.    
    
    
  \begin{align*}
  \sum_{n=0}^{\infty}(qe^{i\theta})^n&=\frac{1}{1-qe^{i\theta}}=\frac{1-qe^{-i\theta}}{1-2q\cos\theta+q^2}\\
  &=\frac{1-q\cos\theta}{1-2q\cos\theta+q^2}+i\frac{q\sin\theta}{1-2q\cos\theta+q^2}\\
  &=\sum_{n=0}^{\infty}q^n\cos n\theta+i\sum_{n=0}^{\infty}q^n\sin n\theta\\
  \end{align*}    
    
    \end{example}  
  
\begin{example}
\[
\sum_{n=0}^{\infty}(1+\frac12+\cdot\cdot\cdot\frac1n)x^n
=\left(\sum_{n=0}^{\infty}x^n\right)\left(\sum_{n=1}^{\infty}\frac{x^n}n\right)
=\frac1{1-x}\ln\frac1{1-x}
\]
\end{example} 

  
\section{函数项级数}
\begin{exercise}
\hfill\\
  设函数列${a_n(x)}$在$[a,b]$上可导,且存在$M>0$,使对任意正整数$n$和$x\in[a,b]$,有$\displaystyle|\sum_{k=1}^na_k(x)|\leq M$ 成立,证明:如果级数$\displaystyle\sum_{n=1}^{\infty}a_n(x)$在$[a,b]$收敛,则必一致收敛。(大连理工大学,2006)
  
  对区间$[a,b]$作如下分割:
  $$a=x_0<x_1<\cdots x_{m-1}<x_m=b,$$
  使得$[a,b]$被分割成$m$个小区间$\Delta_i=[x_{i-1},x_i],i=1,2,\cdots m,$且$\Delta_i=x_i-x_{i-1}<\frac{\varepsilon}{2M}$。
  因为级数$\sum_{n=1}^{\infty}a_n(x)$在$[a,b]$上收敛,所以对$\Delta_i$上任意一点$\overline{x}_i$,存在$N_i>0$,使得$\forall m,n>N_i$,有$$|\sum_n^ma_n(\overline{x}_i)|<\frac{\varepsilon}{2}(\overline{x}_i\in\Delta_i).$$
  对函数$\sum_n^ma_n(x)$应用微分中值定理可知:
  
  对任意$x\in\Delta_i$,存在$\phi$介于$x$与$\overline{x}_i$之间,有$$|\sum_n^ma_n(x)-\sum_n^ma_n(\overline{x}_i|=|\sum_n^ma'_n(\phi)||x-\overline{x}_i|<M\cdot\frac{\varepsilon}{2M}=\frac{\varepsilon}{2}.$$
  
  于是$$|\sum_n^ma_n(x)|\leq|\sum_n^ma_n(x)-\sum_n^ma_n(\overline{x}_i|+|\sum_n^ma_n(\overline{x}_i|<\varepsilon.$$
  即$\forall\varepsilon>0,\exists N=\max\{N_1,N_2,\cdots N_m\},\forall m,n>N,\forall x\in[a,b],$$$|\sum_n^ma_n(x)|<\varepsilon.$$
  即$\sum_n^ma_n(x)$在$[a,b]$上一致收敛。
  \end{exercise}  
  \begin{exercise}
  \hfill\\
    设对每个自然数n,$f_n(x)$为$[a,b]$上的单调函数,${f_n(x)}$收敛于连续函数$f(x)$,求证${f_n(x)}$必在$[a,b]$上一致收敛。
  
  对$x_i\in(a,b)$,$\forall\varepsilon>0$,$\exists\delta>0$,$\forall x\in B(x_i,\delta)$,
  $$|f(x)-f(x_i)|<\frac{\varepsilon}3.$$
  $\exists N>0$,$\forall n>N$,
  $$|f_n(x_i)-f(x_i)|<\frac{\varepsilon}{3},$$
  $$|f_n(x_i-\frac{\delta}{2})-f(x_i-\frac{\delta}{2})|<\frac{\varepsilon}{3},$$
  $$|f_n(x_i+\frac{\delta}{2})-f(x_i+\frac{\delta}{2})|<\frac{\varepsilon}{3},$$
  于是$\forall x\in B(x_i,\frac{\delta}{2}),$
  \begin{align*}
  |f_n(x)-f(x)|&=|f_n(x)-f(x_i)+f(x_i)-f(x)|\\
  &\leq|f_n(x)-f(x_i)|+|f(x_i)-f(x)|\\
  &\leq\max\{|f_n(x_i-\frac{\delta}{2})-f(x_i)|,|f_n(x_i+\frac{\delta}{2})-f(x_i)|\}+\frac{\varepsilon}{3}\\
  &\leq\varepsilon.\\
  \end{align*}
  即对$x_i\in(a,b)$,$\forall\varepsilon>0,$ $\exists\delta_i>0,N_i>0$,$\forall n>N_i$,$x\in B(x_i,\frac{\delta}{2})$,
  $$|f_n(x)-f(x)|<\varepsilon.$$
  显然$\cup_{x_i\in(a,b)}B(x_i,\frac{\delta_i}{2})$是$[a,b]$的一个开覆盖。于是必有有限子覆盖不妨记为$\{B(x_i,\frac{\delta_i}{2}):i=1,2,\cdots,m\}$覆盖$[a,b]$。于是取$N=\max\{N_1,N_2,\cdots,N_m\}$。有$\forall\varepsilon>0,\exists N>0,\forall n>N,\forall x\in[a,b],|f_n(x)-f(x)|<\varepsilon.$
  \end{exercise}
  \hfill\\
    \section{Euclid空间上的极限与连续}
    \begin{exercise}
    \hfill\\
  设$f(x)$是$R^n$上的连续函数,满足$lim_{|x|\rightarrow+\infty}f(x)=+\infty$,其中$x=(x_1,x_2,\cdot\cdot\cdot,x_n),|x|=(x_1^2+x_2^2+\cdot\cdot\cdot x_n^2)^{\frac12}$。证明一定存在$x_0\in R^n$使得$f(x_0)=\inf_{x\in R^n}f(x)$。(兰州大学,2006)
	
	取$x'=(x_1',x_2',\cdots x_n'\in R^n$,因为$\lim_{|x|\rightarrow+\infty}f(x)=+\infty$,从而$\exists R>0,\forall|x|>R,f(x)>f(x')$。于是在紧集$\{x||x|\leq R\}$上,必有最小值$f(x_0)=\inf_{|x|\leq R}f(x)$。显然$\{x||x|\leq R|\}$上的最小值就是$R^n上$的最小值。
	\end{exercise}
	\begin{exercise}
	\hfill\\
  设$f$是$[a,b]\times[a,b]$上的二元连续函数,定义$$g(x)=\max\{f(x,y)|y\in[a,b]\}$$。证明:$g$在$[a,b]$上连续。(南京理工大学,2005)
	
	对固定的$x_0\in[a,b],g(x_0,y)$对$y\in[a,b]$上的最大值是存在且唯一的。因为$g(x_0,y)$对$y$是闭区间上的连续函数。从而$g(x)$是良定义的。
	
	因为$f$是区间$[a,b]\times[a,b]$上的二元连续函数,从而$f$在$[a,b]\times[a,b]$上一致连续。从而存在$\delta>0,\forall(x_1,y_1),(x_2,y_2)\in[a,b]\times[a,b]$,只要$|x_1-x_2|<\delta,|y_1-y_2|<delta,|f(x_1,y_1)-f(x_2,y_2)|<\frac{\varepsilon}{2}$成立。
	
	从而对上述$\varepsilon>0$,有$\forall|x-x_0|<\delta,x\in[a,b]$,有$|f(x,y)-f(x_0,y)|<\varepsilon$。
	
	对$g(x_0)=\max\{f(x_0,y):y\in[a,b]\}$,不妨令$g(x_0)=f(x_0,y_0),y_0\in[a,b]$。于是对$g(x)=\max\{f(x,y):y\in[a,b]\}=f(x_1,y_1),y_1\in[a,b]$。从而
	$$
	\begin{aligned}
	|g(x)-g(x_0)|&=|f(x,y_1)-f(x_0,y_0)|\\
	&\leq|f(x,y_1)-f(x_0,y_1)|+|f(x_0,y_1)-f(x_0,y_0)|\\
	&\leq3\varepsilon.\\
	\end{aligned}
	$$
	于是$g(x)$在$[a,b]$上连续。
	\end{exercise}
	
	\begin{exercise}
	\hfill\\
  设n元函数$f$在$R^n$上具有连续偏导数,证明对于任意的$x=(x_1,x_2,\cdot\cdot\cdot x_n),y=(y_1,y_2,\cdot\cdot\cdot y_n)\in R^n$成立下述Hadamard公式:
  $$\displaystyle f(\bar{y})-f(\bar{x})=\sum_{i=1}^n\int_0^1(y_i-x_i)\frac{\partial f}{\partial y_i}(\bar{x}+t(\bar{y}-\bar{x}))dt$$
	
设$$F(t)=f(\overline{x}+t(\overline{y}-\overline{x})),$$
则$$f(\overline{y})-f(\overline{x})=F(1)-F(0)=\int_0^1F'(t)\mathrm{d}t.$$
由于
\begin{align*}
F'(t)&=\sum_{i=1}^n\frac{\partial f}{\partial x_i}(\overline{x}+t(\overline{y}-\overline{x}))\frac{\partial(x_i-t(y_i-x_i))}{\partial t}\\
&=\sum_{i=1}^n(y_i-x_i)\frac{\partial f(\overline{x}+t(\overline{y}-\overline{x}))}{\partial x_i}.\\
\end{align*}
所以Hadamard公式得证。
	\end{exercise}
	\begin{exercise}
	\hfill\\
	  设函数$z=f(x,y)$在全平面上有定义,具有连续的偏导数,且满足方程$xf_x(x,y)+yf_y(x,y)=0$,证明:$f(x,y)$为常数。
	
	作极坐标变换:
	$$
	\begin{cases}
	x=r\cos\theta,&r\geq0\\
	y=r\sin\theta,&\theta\in[0,2\pi]\\
	\end{cases}	
	$$
	于是由$$f_r=f_x\cos\theta+f_y\sin\theta$$可知,$$rf_r=0,$$
	即$$f(r\cos\theta,r\sin\theta)=F(\theta).$$
	又因为$z=f(x,y)$在$(0,0)$处连续。
	从而$$f(0,0)=\lim_{r\rightarrow0}f(r\cos\theta,r\sin\theta)=\lim_{r\rightarrow0}F(\theta)=F(\theta).$$
	这就是说$$f(x,y)=f(0,0).$$
	\end{exercise}
	\hfill\\
  \section{多元函数的微分学}
  \begin{exercise}
  \hfill\\
    设函数$f(x)$在$[a,b]$上可导,$f'(x)$在$[a,b]$上单调下降,且$f'(x)\geq m>0.$求证:$|\int_a^b\cos f(x)dx|\leq\frac 2m$。

因为$f'(x)>0$,所以$f(x)$严格单调递增,设$f(a)=A$,$f(b)=B$,则$f(x)$的反函数$x=\phi(t)$在$[A,B]$上业单调递增,且可导。同时有$$0<\phi(t)=\frac{1}{f'(x)}\leq\frac{1}{m}.$$
但因为$f'(x)$在$[a,b]$上单调下降,所以$\phi'(x)$在$[A,B]$上严格单调递增,从而$\phi'(t)$若有间断点,只能是第一类间断点,但是因为$\phi(t)$在$[A,B]$上可导,所以$\phi'(t)$在$[A,B]$上不可能有第一类间断点,从而$\phi'(t)$在$[A,B]$上连续。于是
$$\int_a^b\cos f(x)\mathrm{d}x=\int_A^B\cos t\cdot\phi'(t)\mathrm{d}t,$$
根据积分第二中值定理,$\exists\xi\in[A,B]$,使得
$$\int_A^B\cos t\cdot\phi'(t)\mathrm{d}t=\phi'(B)\int_{\xi}^B\cos t\mathrm{d}t=\phi'(B)(\sin B-\sin\xi),$$
联立上式可得:
$$|\int_a^b\cos f(x)\mathrm{d}x|\leq2|\phi'(B)|\leq\frac{2}{m}.$$
\end{exercise}
\hfill\\
  \section{重积分}
  \begin{exercise}
  \hfill\\
    计算$\displaystyle\lim_{n\rightarrow\infty}\sum_{j=1}^{2n}\sum_{i=1}^n\frac2{n^2}[\frac{2i+j}n]$,这里$[x]$为不超过$x$的最大整数。(清华大学,2007)
  
  \begin{align*}
  \lim_{n\rightarrow\infty}\sum_{j=1}^{2n}\sum_{i=1}^n\frac{2}{n^2}[\frac{2i+j}{n}]&=4\lim_{n\rightarrow\infty}\frac{1}{n}\frac{1}{2n}\sum_{i=1}^{2n}\sum_{i=1}^n[\frac{2}{n}i+2\frac{j}{2n}]\\
  &=4\iint_D[2x+2y]dxdy,\\
  \end{align*}
  其中$D=\{(x,y):0\leq x\leq1,0\leq y\leq1\}$,因此
  $$\lim_{n\rightarrow\infty}\sum_{j=1}^{2n}\sum_{i=1}^n\frac{2}{n^2}[\frac{2i+j}{n}]=4\iint_D[2x+2y]dxdy=6.$$
  \end{exercise}
  
  \begin{exercise}
  \hfill\\
   设$f(t)$连续,试证$\displaystyle\iiint_{x^2+y^2+z^2\leq1}f(ax+by+cz)dxdydz=\pi\int_{-1}^1(1-u^2)f(ku)du$,其中$k=\sqrt{a^2+b^2+c^2}>0$。(南京大学,2007) 
  
   作旋转变换使平面$ax+by+cz=0$变成$Ox'y'z'$空间中的坐标平面$z'=0$。则
  \begin{align*}
  \iiint_{x^2+y^2+z^2\leq 1}f(ax+by+cz)dxdydz&=\iiint_{x'^2+y'^2+z'^2\leq1}f(kz')dx'dy'dz'\\
  &=\int_{-1}^1f(kz')\iint_{x'^2+y'^2\leq1-z'^2}dx'dy'\\
  &=\pi\int_{-1}^1(1-z'^2)f(kz')dz'\\
  &=\pi\int_{-1}^1(1-u^2)f(ku)du.\\
  \end{align*} 
  \end{exercise}
  
  \begin{exercise}
  \hfill\\
  求$\int\int_S\frac{ds}z$,其中$S$是球面$x^2+y^2+z^2=a^2$被平面$z=h(0<h<a)$截得的球冠部分。(华东师范大学,2007)  
  
   曲面$S$的方程为$z=\sqrt{a^2-x^2-y^2}$,定义域$D$为圆域$x^2+y^2\leq a^2-h^2$。由于$$z_x=\frac{-x}{\sqrt{a^2-x^2-y^2}},z_y=\frac{-y}{\sqrt{a^2-x^2-y^2}},$$
  从而得到$\sqrt{1+z_x^2+z_y^2}=\frac{a}{\sqrt{a^2-x^2-y^2}}$。因此
  \begin{align*}
  \iint_S\frac{dS}{z}&=\iint_D\frac{adxdy}{a^2-x^2-y^2}\\
  &=\int_0^{2\pi}d\theta\int_0^{\sqrt{a^2-h^2}}\frac{ar}{a^2-r^2}dr\\
  &=2\pi a\int_0^{\sqrt{a^2-h^2}}\frac{ardr}{a^2-r^2}\\
  &=-\pi a\ln(a^2-r^2)|_0^{\sqrt{a^2-h^2}}\\
  &=2a\pi\ln\frac{a}{h}.\\
  \end{align*} 
  \end{exercise}
  \begin{exercise}
  \hfill\\
   $I=\iint_{x^2+y^2+z^2=R^2}\frac{ds}{\sqrt{x^2+y^2+(z-h)^2}}$,其中$h\neq R$(浙江大学,2002)
   
  
  令
  \[
	\begin{cases}
	x=R\cos\phi\sin\theta,&\\
	y=R\sin\phi\sin\theta,&0\leq\theta\leq\pi,0\leq\phi\leq2\pi,\\
	z=R\cos\theta,\\
	\end{cases}  
  \]
则$$E=x_{\theta}^2+y_{\theta}^2+z_{\theta}^2=R^2,$$
$$F=x_{\theta}x_{\phi}+y_{\theta}y_{\phi}+z_{\theta}z_{\phi}=0,$$
$$G=x_{\phi}^2+y_{\phi}^2+z_{\phi}^2=R^2\sin^2\theta.$$

因此
\begin{align*}
I&=\iint\limits_{x^2+y^2+z^2=R^2}\frac{dS}{\sqrt{x^2+y^2+(z-h)^2}}\\
&=\iint\limits_{\substack{0\leq\theta\leq\pi\\0\leq\phi\leq\leq2\pi}}f(x(\theta,\phi),y(\theta,\phi),z(\theta,\phi))\sqrt{EG-F^2}\mathrm{d}\theta\mathrm{d}\phi\\
&=R^2\int_0^{2\pi}\mathrm{d}\phi\int_0^{\pi}\frac{\sin\theta\mathrm{d}\theta}{\sqrt{R^2-2Rh\cos\theta+h^2}}\\
&=4\pi R.\\
\end{align*}
 
  \end{exercise}
\hfill\\
\section{曲线积分、曲面积分与场论}  
  
  \begin{exercise}
  \hfill\\
  计算$\int_Lx^2ds$,其中$L$是球面$x^2+y^2+z^2=1$与平面$x+y+z=0$的交线。(北京大学,2005)
   
  
   由坐标对称性知:
  $$\int_Lx^2\mathrm{d}S=\int_Ly^2\mathrm{d}S=\int_Lz^2\mathrm{d}S,$$
  从而$$\int_Lx^2\mathrm{d}S=\frac{1}{3}\int_L(x^2+y^2+z^2)\mathrm{d}S=\frac{1}{3}\int_L\mathrm{d}S=\frac{2\pi}{3}.$$
 
  \end{exercise}
  \begin{exercise}
  \hfill\\
   设$n$是平面区域$D$的正向边界线$C$的外法线,则$$\int_C\frac{\partial u}{\partial n}ds=\int\int_D(\frac{\partial^2u}{\partial x^2}+\frac{\partial^2u}{\partial y^2})dxdy.$$(中国科学院,2006) 
  
记$n=(\cos\alpha,\cos\beta)$,设切线方向为$l=(\cos x,\cos y)$。则
$$\cos x=-\cos\beta,\cos y=\cos\alpha.$$
从而
\begin{align*}
\int_L\frac{\partial u}{\partial n}dS&=\int_C(\frac{\partial u}{\partial x}\cos\alpha+\frac{\partial u}{\partial y}\cos\beta)ds\\
&=\int_L\frac{\partial u}{\partial x}\cos y-\frac{\partial u}{\partial y}\cos x\mathrm{d}S\\
&=\int_L\frac{\partial u}{\partial x}dy-\frac{\partial u}{\partial y}dx\\
&=\iint_D(\frac{\partial^2u}{\partial x^2}+\frac{\partial^2u}{\partial y^2})dxdy.\\
\end{align*}  
  \end{exercise}

  \begin{exercise}
  \hfill\\
设区域$\Omega$由分片光滑封闭曲面$\sum$所围成,$u(x,y,z)$在$\overline{\Omega}$上具有二阶连续偏导数,且在$\overline{\Omega}$上调和,即满足
$$\frac{\partial^2u}{\partial x^2}+\frac{\partial^2u}{\partial y^2}+\frac{\partial^2u}{\partial z^2}=0.$$  
  
  证明:$\iint_{\Sigma}\frac{\partial u}{\partial n}ds=0$,其中$n$为$\Sigma$的单位外法向量。
  
  设$(x_0,y_0,z_0)\in\Omega$为一定点,证明
  $$u(x_0,y_0,z_0)=\frac1{4\pi}\int\int_{\Sigma}(u\frac{\cos(r,n)}{r^2}+\frac1r\frac{\partial u}{\partial n})dS,$$
  其中$r=(x-x_0,y-y_0,z-z_0),r=|r|.$  
  
   设$n=(\cos\alpha,\cos\beta,\cos\gamma)$,则
  $$\frac{\partial f}{\partial n}=(\frac{\partial f}{\partial x}\cos\alpha,\frac{\partial f}{\partial y}\cos\beta,\frac{\partial f}{\partial z}\cos\gamma).$$
  \begin{align*}
    \iint_{\Sigma}\frac{\partial u}{\partial n}ds&=\iint_{\Sigma}\frac{\partial f}{\partial x}dydz+\frac{\partial f}{\partial y}dzdx+\frac{\partial f}{\partial z}dxdy\\
    &=\iiint_V(\frac{\partial^2u}{\partial x^2}+\frac{\partial^2u}{\partial y^2}+\frac{\partial^2z}{\partial z^2})dxdydz.
  \end{align*} 
  
  记
$$S=\frac1{4\pi}\int\int_{\Sigma}(u\frac{\cos(r,n)}{r^2}+\frac1r\frac{\partial u}{\partial n})dS,$$
$$\sum':=\{(x,y,z):|r|=\delta,\delta\in(0,1)\}.$$
则
\begin{align*}
S&=\frac{1}{4\pi}\iint_{\sum}(u\frac{(x-x_0)\cos\alpha+(y-y_0)\cos\beta+(z-z_0)\cos\gamma}{r^3}+\frac{1}{r}\frac{\partial u}{\partial \textbf{n}})\mathrm{d}S\\
&=\frac{1}{4\pi}\iint_{\sum}(\frac{u(x-x_0)}{r^3}+\frac{1}{r}\frac{\partial u}{\partial x})\mathrm{d}y\mathrm{d}z
+(\frac{u(y-y_0)}{r^3}+\frac{1}{r}\frac{\partial u}{\partial y})\mathrm{d}z\mathrm{d}x\\
&+(\frac{u(z-z_0)}{r^3}+\frac{1}{r}\frac{\partial u}{\partial z})\mathrm{d}y\mathrm{d}z\\
&=\iiint_{\overline{\Omega/\sum'}}\frac{1}{r}(\frac{\partial^2u}{\partial x^2}+\frac{\partial^2u}{\partial y^2}+\frac{\partial^2u}{\partial z^2})\mathrm{d}x\mathrm{d}y\mathrm{d}z\\
&+\frac{1}{4\pi}\iint_{\sum'}(\frac{u\cos(r,n)}{r^2}+\frac{1}{r}\frac{\partial u}{\partial\textbf{n}})\mathrm{d}S\\
&=\frac{1}{4\pi}\iint_{\sum'}(\frac{u\cos(r,n)}{r^2}+\frac{1}{r}\frac{\partial u}{\partial\textbf{n}})\mathrm{d}S\\
&=\frac{1}{4\pi}\iint_{\sum'}\frac{u}{\delta^2}\mathrm{d}S.\\
\end{align*}
由$u$的连续性,令$\delta\rightarrow0$,可得$S=u(x_0,y_0,z_0)$。
  \end{exercise}
 \hfill\\
 
   \section{含参变量积分}
  \begin{exercise}
  \hfill\\
 $\displaystyle\int_0^{\pi}\ln(1-2a\cos x+a^2)dx$(华东师范大学,2003)  
  
   记$I(a)=\int_0^{\pi}\ln(1-2a\cos x+a^2)\mathrm{d}x$,则$I(0)=0$,$$I'(a)=\int_0^{\pi}\frac{-2\cos x+2a}{1-2a\cos x+a^2}\mathrm{d}x$$。
  作变换$t=\tan\frac{x}{2}$,则$$\mathrm{d}x=\frac{2\mathrm{d}t}{1+t^2},\cos x=\frac{1-t^2}{1+t^2},\sin x=\frac{2t}{1+t^2}.$$
  于是
  \begin{align*}
  I'(a)&=4\int_0^{\infty}\frac{t^2-1+a+at^2}{1+t^2-2a(1-t^2)+a^2(1+t^2)}\frac{\mathrm{d}t}{1+t^2}\\
  &=\frac{2}{a}\int_0^{\infty}\frac{\mathrm{d}t}{1+t^2}+2(a-\frac{1}{a})\int_0^{\infty}\frac{\mathrm{d}t}{(1-a)^2+(1+a)^2t^2}\\
  &=\frac{2}{a}\int_0^{\infty}\frac{\mathrm{d}t}{1+t^2}-\frac{2}{a}\int_0^{\infty}\frac{\mathrm{d}(\frac{1+a}{1-a}t)}{1+(\frac{1+a}{1-a})^2t^2}\\
  &=0\quad(|a|<1).
  \end{align*}
  从而$I(a)=0.$
 
  \end{exercise}

  \begin{exercise}
  \hfill\\
   设$f(x)$是$[-1,1]$上的连续函数,则$\displaystyle\lim_{y\rightarrow0^+}\int_{-1}^{1}\frac{yf(x)}{x^2+y^2}dx=\pi f(0)$。(中国科学院,2003) 
  
 因为$\forall\delta>0$,
$$\lim_{y\rightarrow0+}\int_{-1}^{-\delta}\frac{yf(x)}{x^2+y^2}\mathrm{d}x=\int_{-1}^{-\delta}\lim_{y\rightarrow0+}\frac{yf(x)}{x^2+y^2}\mathrm{d}x=0,$$
$$\lim_{y\rightarrow0+}\int^{1}_{\delta}\frac{yf(x)}{x^2+y^2}\mathrm{d}x=\int^{1}_{\delta}\lim_{y\rightarrow0+}\frac{yf(x)}{x^2+y^2}\mathrm{d}x=0,$$
从而
$$\lim_{y\rightarrow0+}\int_{-1}^{1}\frac{yf(x)}{x^2+y^2}\mathrm{d}x=\lim_{y\rightarrow0+}\int_{-\delta}^{\delta}\frac{yf(x)}{x^2+y^2}\mathrm{d}x.$$
又$\forall\varepsilon>0,\exists\delta>0,\forall x\in(-\delta,\delta),f(0)-\varepsilon<f(x)<f(0)+\varepsilon,$从而
\begin{align*}
\lim_{y\rightarrow0+}\int_{-\delta}^{\delta}\frac{yf(x)}{x^2+y^2}\mathrm{d}x&<(f(0)+\varepsilon)\lim_{y\rightarrow0+}\int_{-\delta}^{\delta}\frac{y\mathrm{d}x}{x^2+y^2}\\
&=(f(0)+\varepsilon)\lim_{y\rightarrow0+}\arctan\frac{\delta}{y}-\arctan\frac{-\delta}{y}\\
&=(f(0)+\varepsilon)\pi.\\
\end{align*}
同理$$\lim_{y\rightarrow0+}\int_{-\delta}^{\delta}\frac{yf(x)}{x^2+y^2}\mathrm{d}x>(f(0)-\varepsilon)\pi.$$
于是由$\varepsilon$的任意性,可得所求结果。 
  \end{exercise}

  \begin{exercise}
  \hfill\\
   设$0<r<1,x\in R$,求证:$\displaystyle\frac{1-r^2}{1-2r\cos x+r^2}=1+2\sum_{n=1}^{\infty}r^n\cos nx$; 
  

\begin{align*}
\frac{1-r^2}{1-2r\cos x+r^2}&=-1+\frac{2-2r\cos x}{1-2r\cos x+r^2}\\
&=-1+\frac{2-2r\cos x}{(1-r\cos x)^2+r^2\sin^2x}\\
&=-1+\frac{2-2r\cos x}{(1-r\cos x-i\sin x)(1-r\cos x+i\sin x)}\\
&=-1+\frac{1}{1-r\cos x-i\sin x}+\frac{1}{1-r\cos x+i\sin x}\\
&=-1+\frac{1}{1-re^{ix}}+\frac{1}{1-re^{-ix}}\\
&=-1+\sum_{n=0}^{\infty}(re^{ix})^n+\sum_{n=0}^{\infty}(re^{-ix})^n\\
&=-1+\sum_{n=0}^{\infty}r^n(e^{inx}+e^{-inx})\\
&=-1+2\sum_{n=0}^{\infty}r^n\cos nx\\
&=1+2\sum_{n=1}^{\infty}r^n\cos nx.\\
\end{align*}

  \end{exercise}
\begin{exercise}
\hfill\\
  设二元函数$f(x,y)$为$[a,b]\times[c,+\infty)$上的连续非负函数,$I(x)=\int_c^{+\infty}f(x,y)dy$在$[a,b]$上连续。证明:$I(x)$在$[a,b]$上一致连续。(南京师范大学,2004,2008)

设$x_0\in[a,b]$,由于$\int_c^{+\infty}f(x_0,y)\mathrm{d}y$收敛,因此对任意$\varepsilon>0$,存在$M_{x_0}>c$,对任意$M>M_{x_0}$,有$\int_M^{+\infty}f(x_0,y)\mathrm{d}y<\frac{\varepsilon}{4}.$

$f(x,y)$在$[a,b]\times[c,M]$上一致连续,从而存在$\delta>0,$对任意$x\in(x_0-\delta_1,x_0+\delta_1)\cap[a,b]$,有$$|f(x_0,y)-f(x,y)|<\frac{\varepsilon}{4(M_{x_0}-c)}.$$
$I(x)=\int_c^{\infty}f(x,y)\mathrm{d}y$在$[a,b]$上一致连续,对上述$\varepsilon>0$,存在$\delta_2>0$,对任意$x\in(x_0-\delta_2,x_0+\delta_2)\cap[a,b]$,有
$$|I(x)-I(x_0)|=|\int_c^{\infty}f(x,y)\mathrm{d}y-\int_c^{\infty}f(x_0,y)\mathrm{d}y|<\frac{\varepsilon}{2}.$$
取$\delta=\min\{\delta_1,\delta_2\}$,对任意$x_0\in[a,b]$,$x\in(x_0-\delta,x_0+\delta)\cap[a,b]$,有
\begin{align*}
\int_M^{\infty}f(x,y)\mathrm{d}y&=\int_c^{\infty}f(x,y)\mathrm{d}y-\int_c^Mf(x,y)\mathrm{d}y\\
&=|\int_c^{\infty}f(x,y)\mathrm{d}y-\int_c^{\infty}f(x_0,y)\mathrm{d}y\\
&+\int_c^M[f(x_0,y)-f(x,y)]\mathrm{d}y+\int_M^{\infty}f(x_0,y)\mathrm{d}y|\\
&<|\int_c^{\infty}f(x,y)\mathrm{d}y-\int_c^{\infty}f(x_0,y)\mathrm{d}y|\\
&+|\int_c^M[f(x_0,y)-f(x,y)\mathrm{d}y|+|\int_M^{\infty}f(x_0,y)\mathrm{d}y|\\
&<\frac{\varepsilon}{2}+(M-c)\frac{\varepsilon}{4(M_{x_0}-c)}+\frac{\varepsilon}{4}<\varepsilon.\\
\end{align*}
从而得到$[a,b]$的一个开覆盖$\{(x_0-\delta,x_0+\delta)|x\in[a,b]\}$,有有限开覆盖定理,存在其中有限个开区间$(x_i-\delta,x_i+\delta),i=1,2,\cdots,n$覆盖$[a,b]$。不妨令$M'=\max\{M_{x_1},M_{x_2},\cdots,M_{x_n}\}$,则对任意的$M>M'$,有$\int_M^{\infty}f(x,y)\mathrm{d}y<\frac{\varepsilon}{4}$。因此$I(x)$在$[a,b]$上一致收敛。
\end{exercise}
  \hfill\\
  \section{傅里叶级数}
  \begin{exercise}
  \hfill\\
  设$\psi(x)$在$[0,+\infty)$上连续且单调,$$\displaystyle\lim_{x\rightarrow+\infty}\psi(x)=0$$,证明:$$\displaystyle\lim_{p\rightarrow+\infty}\int_0^{+\infty}\psi(x)\sin pxdx=0.$$  
  
 因为$\lim_{x\rightarrow+\infty}\phi(x)=0$,所以存在$N>0$,使得$x\geq N$,$|\phi(x)|<1$。利用积分第二中值定理可得
\begin{align*}
|\int_N^A\phi(x)\sin px\mathrm{d}x|&=|\phi(N)\int_N^{\xi}\sin px\mathrm{d}x+\phi(A)\int_{\xi}^A\sin px\mathrm{d}x|\\
&<|\int_N^{\xi}\sin px\mathrm{d}x|+|\int_{\xi}^A\sin px\mathrm{d}x|\\
&\leq\frac{4}{p}(\forall A>N),
\end{align*}
因此$|\int_N^{+\infty}\phi(x)\sin px\mathrm{d}x|\leq\frac{4}{p}.$从而
$$\lim_{p\rightarrow+\infty}\int_N^{+\infty}\phi(x)\sin px\mathrm{d}x=0.$$
而由Riemann引理
$$\lim_{p\rightarrow+\infty}\int_0^N\phi(x)\sin px\mathrm{d}x=0.$$
这就得到了我们所要求的结果。 
  \end{exercise}

  \begin{exercise}
  \hfill\\
 三角级数$\displaystyle\frac{a_0}2+\sum_{n=1}^{\infty}(a_n\cos x+b_n\sin nx)$是某个在$[-\pi,\pi]$上可积且绝对可积函数的Fourier级数的必要条件是$\displaystyle\sum_{n=1}^{\infty}\frac{b_n}n$收敛。  
  
  设$f(x)\sim\frac{a_0}{2}+\sum_{n=1}^{\infty}(a_n\cos x+b_n\sin x)$。
  令$F(x)=\int_c^x[f(t)-\frac{a_0}{2}\mathrm{d}t.$(仅考虑$f(x)$只有有限个间断点的情况)
  $F(x)$是周期为$2\pi$的连续函数,在$f(x)$的连续点,成立$F'(x)=f(x)-\frac{a_0}{2}$。在$f(x)$的第一类间断点,$F(x)$的两个单侧导数$F'_{\pm}(x)=f(x\pm)-\frac{a_0}{2}$都存在。由Dini-Lipschitz判别法,$F(x)$可展开为收敛的Fourier级数$$F(x)=\frac{A_0}{2}+\sum_{n=1}^{\infty}(A_n\cos nx+B_n\sin nx).$$
  利用分步积分法
  \begin{align*}
  A_n&=\frac{1}{\pi}\int_{-\pi}^{\pi}F(x)\cos nx\mathrm{d}x\\
  &=[\frac{1}{\pi}\frac{\sin nx}{n}F(x)]|_{-\pi}^{\pi}-\frac{1}{n\pi}\int_{-\pi}^{\pi}f(x)\sin nx\mathrm{d}x\\
  &=-\frac{1}{n\pi}\int_{-\pi}^{\pi}(f(x)-\frac{a_0}{2})\sin nx\mathrm{d}x\\
  &=-\frac{b_n}{n}.
  \end{align*}
  类似可得:$B_n=\frac{a_n}{n}$。于是$$F(x)=\frac{A_0}{2}+\sum_{n=1}^{\infty}(-\frac{b_n}{n}\cos nx+\frac{a_n}{n}\sin nx).$$
  令$x=c$,有
  $$0=\frac{A_0}{2}+\sum_{n=1}^{\infty}(-\frac{b_n}{n}\cos nc+\frac{a_n}{n}\sin nc).$$
  从而
  \begin{align*}
  F(x)&=\int_c^x[f(t)-\frac{a_0}{2}]\mathrm{d}t\\
  &=\sum_{n=1}^{\infty}(a_n\frac{\sin nx-\sin nc}{n}+b_n\frac{-\cos nx+\cos nc}{n})\\
  &=\sum_{n=1}^{\infty}\int_c^x(a_n\cos nt+b_n\sin nt)dt.\\
  \end{align*}
  在此,令$x=0$,得到$F(0)=\frac{A_0}{2}-\sum_{n=1}^{\infty}\frac{b_n}{n}$,说明了级数$\sum_{n=1}^{\infty}\frac{b_n}{n}$收敛。  
  \end{exercise}
  \begin{exercise}
  \hfill\\
  
  
  
  \end{exercise}
  
   \begin{exercise}
  \hfill\\
  
  
  
  \end{exercise}
  
  \begin{exercise}
  \hfill\\
  
  
  
  \end{exercise}
  
  \begin{exercise}
  \hfill\\
  
  
  
  \end{exercise}
  
  
  \begin{exercise}
  \hfill\\
  
  
  
  \end{exercise} 
\begin{example}[id:20151012-201647] \label{20151012-201647}\index{Example!20151012-201647} \hfill \\
\%\%\%\%\%\%\%\%\%\%\%\%\%\%\%



\%\%\%\%\%\%\%\%\%\%\%\%\%\%\%


example \ref{20151012-190708}-20151012-190708

\end{example}



























%-=-=-= DEFINITION
\begin{definition}[Natural Numbers]\index{Number System! Natural Numbers}

\[
\mathbb{N}=\set{0, 1, 2, 3 \ldots}
\]

\end{definition}

\begin{remark}
It is not uncommon for zero to be excluded from the natural numbers.  In fact, some exclude zero from the natural numbers and then describe the set of natural numbers that include zero the whole numbers. \\

\[
\mathbb{W}=\set{0, 1, 2, 3, \ldots}
\]

For the purposes of these notes, zero will be included within the set of natural numbers.
\end{remark}

%-=-=-= DEFINITION 

%\part{linear algebra}
\chapter{高等代数}
数学的思维方法:观察客观世界的现象,抓住其主要的特征,抽象出概念或建立模型;运用直觉判断,归纳,类比,联想,推理等进行摸索,猜测可能有的规律;然后通过深入分析,逻辑推理和计算进行论证,揭示事物的内在规律,这就是数学思维方式的全过程。

按照“观察-抽象-探索-猜测-论证”的思维方式学习数学是学好数学的正确途径,而且可以培养正确处理工作和生活中遇到的各种问题的能力,从而终身受益。——丘维生。


\begin{exercise}
\hfill
设$n$个方程的$n$元齐次线性方程组的系数矩阵$A$的行列式等于0,并且$A$的$(k,l)$元的代数余子式$A_{kl}=0$。证明:$$\eta=(A_{k1},A_{k2},\cdots,A_{kn})'$$是这个其次线性方程组的一个基础解系。


因为$A$有一个$n-1$阶子式不为零,又$A$的行列式为0,所以$A$的秩为$n-1$.从而此齐次线性方程组的解空间维数为1.由按一行展开公式知,$\eta$是齐次线性方程组的一个解,且是非零解。从而线性无关。因此$\eta$构成了解空间的一组基。得证。
\end{exercise}
\begin{exercise}
\hfill\\
设$A=(a_{ij})$是实数域上的$n$阶矩阵,证明:
如果$a_{ii}>\sum_{j\not=i}|a_{ij}|,i=1,2,\cdots,n$,那么$|A|>0$。

令
$$
B(t)=
\left(
\begin{array}{llll}
a_{11}&a_{12}t&\cdots&a_{1n}t\\
a_{21}t&a_{22}&\cdots&a_{2n}t\\
\vdots&\vdots&\ddots&\vdots\\
a_{n1}t&a_{n2}t&\cdots&a_{nn}\\
\end{array}
\right)
$$
则$|B(t)|$是$t$的多项式,当$t\in(0,1]$时,由已知条件,$B(t)$是主对角占优矩阵,从而$|B(t)|\neq0$。又$|B(0)|=a_{11}a_{12}\cdots a_{nn}>0$。根据连续函数的中间值定理知:$|B(1)|>0$,即$|A|>0$。
\end{exercise}

\begin{exercise}
    if $x_1, x_2, x_3,\cdots, x_n\in\mathbb R$ and $x_1^2+x_2^2+\cdots x_n^2\neq0$, then 
    \begin{equation*}
        \frac{x_1\cdot x_2 + x_2\cdot x_3 + \cdots + x_{n-1}\cdot x_n}{x_1^2 + x_2^2 + x_3^2 + \cdots + x_n^2} \leq \cos\frac{\pi}{n+1}.
    \end{equation*}
\end{exercise}

\begin{proof}
    left side is equal to a half of $(x_1,x_2,\cdots, x_n)A(x_1,x_2,\cdots, x_n)^T$ with 
    \begin{equation*}
        A = \begin{pmatrix}
            0 & 1 & 0 & \cdots & 0 \\
            1 & 0 & 1 & \cdots & 0 \\
            0 & 1 & 0 & \cdots & 0 \\
            \vdots & \vdots & \vdots & \ddots & 1 \\
            0 & 0 & 0 & 1 & 0
            \end{pmatrix}.
    \end{equation*}
so it suffices to show the most large enginvalue is less than $2\cos\pi/(n+1)$. 
For the equation $Av = \lambda v$,
the corresponding enginfunction $v = (v_1,v_2,\cdots, v_n)$ has relations 
\[
v_{k-1} + v_{k+1} = \lambda v_k \quad (1 \leq k \leq n)
\]
with boundary conditions $v_0=0$ and $v_{n+1} = 0$. As the left hand is less than 1, we have $|\lambda| < 2$. We may assume $v_k = \sin k\theta$, then 
\[
\sin((k-1)\theta) + \sin((k+1)\theta) = \lambda \sin(k\theta)
\]
which reduce to 
\[
\lambda = 2\cos\theta
\]
by triangle inequality.
The boundary condition entails 
\[
\theta = \frac{k\pi}{n+1} \quad (k = 1, 2, \dots, n).
\]
Therefore,
\[
\boxed{\lambda_k = 2\cos\left(\frac{k\pi}{n+1}\right) \quad (k = 1, 2, \dots, n)}
\]
We complete the proof as it is obvious that $\lambda_k \leq \lambda_1$.
\end{proof}

%\part{complex analysis}
\chapter{复变函数}

\section{复变函数}
\begin{exercise}
\hfill\\
$\displaystyle|\frac{z-1}{z+1}|<1$表示点$z$的轨迹和图形是什么?它是不是区域?

解答,令$z=a+bi$,带入原不等式化简可得$0<4a[(a+1)^2+b^2]$,这等价于$a>0$,或者说$Rez>0$,此即是说原不等式表示的区域是z平面的右半平面。

证明:z平面上的圆周可以写成$Az\overline{z}+\beta\overline{z}+\overline{\beta}z+C=0$,其中$A>0,C$为实数,$A\neq0$,$\beta$为复数,且$\beta>AC$。
\end{exercise}
\begin{exercise}
\hfill\\
z平面上的圆周可以写成$a\overline{z}+\overline{a}z=C$(a是非零复常数,C是实常数)。

\[
\begin{aligned}
Az\overline{z}+\beta\overline{z}+\overline{\beta}z+c=0&\iff z\overline{z}+\frac{\beta}A\overline{z}+\frac{\overline{\beta}}Az+\frac CA=0\\
&\iff(z+\frac{\beta}A)(\overline{z}+\frac{\overline{\beta}}A)=\frac{\beta\overline{\beta}}{A^2}-\frac CA\\
&\iff(z+\frac{\beta}A)(\overline{z+\frac{\beta}A})=\frac{|\beta|^2-AC}{A^2}\\
&\iff|z+\frac{\beta}A|=\frac{\sqrt{|\beta|^2-AC}}A
\end{aligned}
\]
故原方程表示以$z=-\frac{\beta}A$为圆心以$\frac{\sqrt{|\beta|^2-AC}}A$为半径的圆。

一来,在扩充z平面上直线可表示为半径无限大的直线,此即证明了另一个问题。或者设$z=x+yi;a=\alpha+\beta i,x,y,\alpha,\beta\in R$,由$a\overline{z}+\overline{a}z=C$知,$\alpha x+\beta y=\frac C2$,而由a非零知,$\alpha\beta\neq0$,即$\alpha x+\beta y=\frac C2$表示为xoy平面的一条直线,即$a\overline{z}+\overline{a}z=C$表示为z平面的一条直线;另一方面,设z平面上的直线在xoy平面为$mx+ny=t$(m,n,t均为实数),若取$a=m+ni$则有$a\overline{z}+\overline{a}z=2(mx+ny)=2t$,再取$C=2t$即有$a\overline{z}+\overline{a}z=C$。这就完成了证明。


\end{exercise}







\begin{exercise}
\hfill\\
试证方程$z+e^{-z}=a(a>1)$在$Rez>0$内只有一个根,且为实根。

证明,考虑以$R$为半径原点为圆心的右半圆与虚轴所形成的闭曲线$\Gamma_R$。令$z=i\lambda,\lambda\in R$。代入原方程得:$i\lambda+\cos\lambda-i\sin\lambda=a$,即$\cos\lambda=a>1$,这就是说原方程在虚轴上无解。从而记$f(z)=z+e^{-z}-a$,有$f(z)$在$\Gamma_R$内解析,在$\Gamma_R$内连续且不为零。由幅角原理,当$R$足够大使得$\Gamma_R$包含$f(z)$的零点时,$f(z)$在$\Gamma_R$上的幅角变化应为$2\pi n$,其中$n$为$\Gamma_R$包含零点个数。当$R\rightarrow\infty$时,$n$表示原方程在虚轴右侧的所有零点个数,易知$n<+\infty$。考虑到对称性沿半圆弧的幅角变化等于沿虚轴的幅角变化,且$f(z)=z(1+g(z))$,其中$(1+\frac{e^{-z}-a}{z})$;我们有
\[
\begin{aligned}
\Delta_{\Gamma_R}argf(z)&=\Delta_{\Gamma_R}argz(1+g(z))\\
&=\Delta_{\Gamma_R}argz+\Delta_{\Gamma_R}arg(1+g(z)),
\end{aligned}
\]
其中$g(z)$在$R\rightarrow\infty$时$g(z)$沿$\Gamma_R$
一致趋于零。由此知\[\lim_{R\rightarrow\infty}\Delta_{\Gamma_R}arg(1+g(z))=0.\]
这样一来$\lim_{R\rightarrow\infty}\Delta_{\Gamma_R}f(z)=\pi$,这即是说$f(z)$
在右边平面仅有一解。如若该解$z_0$不在实轴上则方程两边取逆易见$\overline{z_0}$也是原方程的解。
与仅有一解矛盾。从而原方程在右半平面仅有一解且在实轴上。

\end{exercise}





%\part{real analysis}
\chapter{实变函数}







\section{集合}\index{Set}






\begin{example}
\hfill\\
非空完备集一定是不可数集。


设$E$是完备集,若$E=\{x_n\}$可数,$\forall B_0=B(y_0,r_0)$,$y_0\in E$,$r_0=1$,$\exists y_1\in E$,$0<r_1\leq\frac{1}{2}$,使$B_1=B(y_1,r)\subset B_0$,且$x_1\not\in\overline{B_1}$。对$B(y_1,r_1)$,$\exists y_2\in E$,$0<r_2\leq\frac{1}{2^2}$,使$B_2=B(y_2,r_2)\subset B_1$,且$x_2\not\in\overline{B_2}$。以此类推,$\exists y_n\in E$,$0<r_n\leq\frac{1}{2^n}$,使$B_n=B(y_n,r_n)\subset B_{n-1}$,且$x_n\not\in\overline{B_n}$。因为$B_0\supset B_1\supset\cdots\supset B_n\cdots$,且$r_n\leq\frac{1}{2^n}$。所以$\{y_n\}$是基本列,$y_n\to y_0$。因为$\{y_n\}\subset E$,且$y_0\in\overline{B_n}$。因为$x_n\not\in\overline{B_n}$,所以$y_0\neq x_n,\forall n\geq1.$与$y_0\in E$矛盾。
\end{example}




\begin{exercise}
\hfill\\
证明:$R^n$中存在可列个开球$\{B_n\}_{n\geq1}$使得对任意开球$E$,存在子列$\{B_{n_k}\}$使得$E=\cup_{k\geq1}B_{n_k}$。满足这种条件的开球列$\{B_n\}_{n\geq1}$称为$R^n$中的一个可数邻域基。



\end{exercise}



\begin{exercise}
\hfill\\
设$E\subset R^n$。$\{G_{\alpha}\}_{\alpha\in\Lambda}$是$E$的一个开集族覆盖。证明:存在至多可数个开集$\{G_{\alpha_k}\}_{k\geq1}\subset\{G_{\alpha}\}_{\alpha\in\Lambda}$使得它仍覆盖$E$,即$E$的任意开覆盖有至多可数的子覆盖。
\end{exercise}

\section{Measure}

A \textbf{measure space} consists of a set $X$ equipped with two fundamental objects:
\begin{enumerate}
  \item A $\sigma$-algebra $\mathcal M$ of ``measurable'' sets, which is a non-empty collection of 
  subsets of $X$ closed under complements and countable unions and intersections.
  \item A \textbf{measure} $\mu:\mathcal M\to[0,\infty]$ with the following defining property:
  if $E_1$, $E_2$, $\cdots$ is a countable family of disjoint sets in $\mathcal M$, then 
  \[
  \mu\left(\cup_{n=1}^\infty E_n\right) = \sum_{n=1}^\infty\mu(E_n).
  \]
\end{enumerate}
A measure space is therefore often denoted by the triple $(X,\mathcal M,\mu)$ to emphasize its three main components.
Sometimes, however, when there is no ambiguity we will abbreviate this notation by referring to the measure space as $(X,\mu)$, or simply $X$.

A feature that a measure space often enjoys is the property of being $\sigma$-finite.
This means that $X$ can be written as the union of countably many measurable sets of finite measure.

\begin{definition}
  Let $X$ be a set. 
  An \textbf{exterior measure} (or \textbf{outer measure}) $\mu_\ast$ on $X$ is a function $\mu_*$ 
  from the collection of all subsets of $X$ to $[0,\infty]$ that satisfies the following properties:
  \begin{enumerate}
    \item $\mu_\ast(\emptyset) = 0$.
    \item If $E_1\subset E_2$, then $\mu_\ast(E_1) \leq \mu_*(E_2)$.
    \item If $E_1$, $E_2$, $\cdots$ is a countable family of sets, then 
    \[
      \mu_*\left(\cup_{n=1}^\infty E_n\right) = \sum_{n=1}^\infty\mu_\ast(E_n).
    \]
  \end{enumerate}
\end{definition}

\begin{definition}
  A set $E$ in $X$ is \textbf{Carath\'eodory measurable} or simply \textbf{measurable} if one has 
  \begin{equation}
    \label{eq: def of being Caratheodory measurable}
  \mu_*(A) = \mu_*(E\cap A) + \mu_*(E^c\cap A)\quad \text{for every }A\subset X.
  \end{equation}
\end{definition}

A first observation we make is that to prove a set $E$ is measurable, it suffices to verify 
\[
  \mu_*(A) \geq \mu_*(E\cap A) + \mu_*(E^c\cap A)\quad \text{for all }A\subset X,
\]
since the reverse inequality is automatically verified by the sub-additivity property of the exterior measure.
We see immediately from the definition that sets of exterior measure zero are necessarily measurable.

\begin{theorem}
  Given an exterior measure $\mu_*$ on a set $X$, the collection $\mathcal M$ of Carath\'eodory measurable sets forms a $\sigma$-algebra. 
  Moreover, $\mu_*$ restricted to $\mathcal M$ is a measure.
\end{theorem}

Our previous observation that sets of exterior measure $0$ are Carath\'eodory measurable shows that
the measure space $(X,\mathcal M,\mu)$ in the theorem is \textbf{complete}: 
whenever $F\in\mathcal M$ satisfies $\mu(F) = 0$ and $E\subset F$, then $E\in\mathcal M$.

\subsection{Metric exterior measures}

If the underlying set $X$ is endowed with a ``distance function'' or ``metric'',
there is a particular class of exterior measures that is of interest in practice. 
The importance of these exterior measures is that they induce measures on the natural $\sigma$-algebra 
generated by the open sets in $X$.

On a metric space $X$ we can define the \textbf{Borel $\sigma$-algebra}, $\mathcal B_X$, 
that is the smallest $\sigma$-algebra of sets in $X$ that contains the open sets of $X$. 
In other words $\mathcal B_X$ is the intersection of all $\sigma$-algebras that contain the open sets.
Elements in $\mathcal B_X$ are called \textbf{Borel sets}.

Given two sets $A$ and $B$ in the metric space $(X, d)$, 
the \textbf{distance} between $A$ and $B$ is defined by 
\[
d(A,B) := \inf\{d(x,y) : x\in A \text{ and } y\in B\}.
\]
Then an exterior measure $\mu_*$ on $X$ is a \textbf{metric exterior measure} if it satisfies 
\[
\mu_*(A\cup B) = \mu_*(A) + \mu_*(B)\quad \text{whenever } d(A,B)>0.
\]
This property played a key role in the case of exterior Lebesgue measure.

Metric exterior measures is these exterior measures with the special property of being additive on sets that are ``well separated'', 
which guarantees that this exterior measure defines a measure on the Borel $\sigma$-algebra.
This is achieved by proving that all Borel sets are Carath\'eodory measurable.

\begin{theorem}
  If $\mu_*$ is a metric exterior measure on a metric space $X$, 
  then the Borel sets in $X$ are measurable.
  Hence $\mu_*$ restricted to $\mathcal B_X$ is a measure.
\end{theorem}

Given a metric space $X$, a measure $\mu$ defined on the Borel sets of $X$ will be referred to as a \textbf{Borel measure}.
Borel measures that assign a finite measure to all balls (of finite radius) also satisfy a useful regularity property. 
The requirement that $\mu(B) < \infty$ for all balls $B$ is satisfied in many (but not in all) circumstances that arise in practice.
\footnote{This restriction is not always valid for the Hausdorff measures that are considered in the next chapter.}
When it does hold, we get the following proposition.

\begin{proposition}
  Suppose the Borel measure $\mu$ is finite on all balls in $X$ of finite radius. 
  Then for any Borel set $E$ and any $varepsilon>0$, 
  there are an open set $\mathcal O$ and a closed set $F$ such that $E\subset\mathcal O$ and 
  $\mu(\mathcal O-E) < \varepsilon$, while $F\subset E$ and $\mu(E-F) < \varepsilon$.
\end{proposition}

\subsection{The extension theorem}

As we have seen, a class of measurable sets on $X$ can be constructed once we start with a given exterior measure.
However, the definition of an exterior measure usually depends on a more primitive idea of measure defined on a simper class of sets.
This is the role of a premeasure defined below.
As we will show, any premeasure can be extended to a measure on $X$. 
We begin with several definitions.

Let $X$ be a set. An \textbf{algebra} in $X$ is a non-empty collection of subsets of $X$ that is closed under complements, finite unions, and finite intersections.
Let $\mathcal A$ be an algebra in $X$. 
A \textbf{premeasure} on an algebra $\mathcal A$ is a function $\mu_0:\mathcal A\to[0,\infty]$ that satisfies:
\begin{enumerate}
  \item $\mu_0(\emptyset) = 0$.
  \item If $E_1$, $E_2$, $\cdots$ is a countable collection of disjoint sets in $\mathcal A$ with 
  $\cup_{k=1}^\infty E_k \subset\mathcal A$, then 
  \[
  \mu_0\left(\cup_{k=1}^\infty E_k\right) = \sum_{k=1}^\infty \mu_0(E_k).
  \]
\end{enumerate}

Premeasures give rise to exterior measures in a natural way.

\begin{lemma}
  If $\mu_0$ is a premeasure on an algebra $\mathcal A$, define $\mu_*$ on any subset $E$ of $X$ by 
  \[
  \mu_*(E) = \inf\left\{\sum_{j=1}^\infty \mu_0(E_j): E\subset\cup_{j=1}^\infty E_j, E_j\in\mathcal A \right\}.
  \]
  Then, $\mu_*$ is an exterior measure on $X$ that satisfies:
  \begin{enumerate}
    \item $\mu_*(E) = \mu_0(E)$ for all $E\in\mathcal A$.
    \item All sets in $\mathcal A$ are measurable in the sense of \eqref{eq: def of being Caratheodory measurable}.
  \end{enumerate}
\end{lemma}

The $\sigma$-algebra generated by an algebra $\mathcal A$ is by definition the smallest $\sigma$-algebra that contains $\mathcal A$.
The above lemma then provides the necessary step for extending $\mu_0$ on $\mathcal A$ to a measure on the $\sigma$-algebra generated by $\mathcal A$.

\begin{theorem}
  Suppose that $\mathcal A$ is an algebra of sets in $X$, 
  $\mu_0$ a premeasure on $\mathcal A$, 
  and $\mathcal M$ the $\sigma$-algebra generated by $\mathcal A$.
  Then there exists a measure $\mu$ on $\mathcal M$ that extends $\mu_0$.
  \footnote{One notes below that $\mu$ is the only such extension of $\mu_0$ under the assumption that $\mu$ is $\sigma$-finite.}
\end{theorem}

For later use we record the following observation about the premeasure $\mu_0$ on the algebra $\mathcal A$ 
and the resulting measure $\mu_*$ that is implicit in the argument given above. 
The details of the proof may be left to the reader.

We define $\mathcal A_\sigma$ as the collection of sets that are countable unions of sets in $\mathcal A$,
and $\mathcal A_{\sigma\delta}$ as the sets that arise as countable intersections of sets in $\mathcal A_\sigma$.

\begin{proposition}
  For any set $E$ and any $\varepsilon>0$, there are sets $E_1\in\mathcal A_\sigma$ and $E_2\in\mathcal A_{\sigma\delta}$,
  such that $E\subset E_1$, $E\subset E_2$, and $\mu_*(E_1)\leq \mu_*(E) + \varepsilon$,
  while $\mu_*(E_2) = \mu_*(E)$.
\end{proposition}

\subsection{Hausdorff measure}

\begin{definition}
  For any subset $E$ of $\mathbb R^n$, 
  define the \textbf{exterior $\alpha$-dimensional Hausdorff measure} of $E$ by 
  \[
  m^*_\alpha(E) = \lim_{\delta\to0}\inf\left\{\sum_k(\diam F_k)^\alpha : E\subset\cup_{k=1}^\infty F_k,
  \quad \diam F_k \leq \delta \text{ all } k\right\},
  \]
  where $\diam S$ denotes the diameter of the set $S$, that is, 
  \[
  \diam S := \sup\{|x-y|: x,y\in S\}.
  \]
\end{definition}

\begin{remark}
  The quantity 
  \[
  \mathcal{H}_\alpha^\delta(E) :=  \inf\left\{\sum_k(\diam F_k)^\alpha : E\subset\cup_{k=1}^\infty F_k,
  \quad \diam F_k \leq \delta \text{ all } k\right\}
  \]
  is increasing as $\delta$ decreases, so that the limit 
  \[
  m_\alpha^*(E) = \lim_{\delta\to0}\mathcal H_\alpha^\delta(E)
  \]
  exists,
  although $m_\alpha^*(E)$ could be infinite.
\end{remark}

The Cantor set $\mathcal C$ was constructed by successively removing the middle-third intervals in $[0,1]$.

\section{Integration on measure space}

To avoid unnecessary complications will assume throughout that the measure space $(X,\mathcal M,\mu)$ under consideration is $\sigma$-finite.

\subsection{Measurable functions}

\begin{definition}
  A function $f$ on $X$ with values in the extended real numbers is \textbf{measurable} if 
  \[
  f^{-1}([-\infty, a)) = \{x\in X: f(x) < a\}\in\mathcal{M}\quad \text{for all } a\in\mathbb R.
  \]
\end{definition}

\begin{remark}
  With this definition, the basic properties of measurable functions obtained in the case of $\mathbb R^d$
  with the Lebesgue measure continue to hold.
  For instance, the collection of measurable functions is closed under the basic algebraic manipulations.
  Also, the pointwise limits of measurable functions are measurable.
\end{remark}

The notation of ``almost everywhere'' that we use now is with respect to the measure $\mu$.
For instance, if $f$ and $g$ are measurable functions on $X$,
we write $f=g$ a.e. to say that 
\[
\mu(\{x\in X: f(x)\neq g(x)\}) = 0.
\]

A \textbf{simple function} on $X$ takes the form 
\[
\sum_{k=1}^N a_k\chi_{E_k},
\]
where $E_k$ are measurable sets of finite measure and $a_k$ are real numbers.
Approximations by simple functions played an important role in the definition of the Lebesgue integral.
Fortunately, this result continues to hold in our abstract setting.

\begin{theorem}
  Suppose $f$ is a non-negative measurable function on a measure space $(X,\mathcal M,\mu)$.
  Then there exists a sequence of simple functions $\{\phi_k\}^\infty_{k=1}$ that satisfies 
  \[
  \phi_k(x)\leq \phi_{k+1}(x) 
  \quad\text{and}\quad
  \lim_{k\to\infty}\phi_k(x) = f(x) \text{ for all }x.
  \]
  In general, if $f$ is only measurable, there exists a sequence of simple functions $\{\phi_k\}_{k=1}^\infty$ that satisfies 
  \[
  |\phi_k(x)| \leq |\phi_{k+1}(x)|
  \quad\text{and}\quad
  \lim_{k\to\infty}\phi_k(x) = f(x)\text{ for all }x.
  \]
\end{theorem}

Another important result that generalizes immediately is Egorov's theorem.

\begin{theorem}
  Suppose $\{f_k\}_{k=1}^\infty$ is a sequence of measurable functions defined on a measurable set $E\subset X$ with $\mu(E)<\infty$, and $f_k\to f$ a.e. 
  Then for each $\varepsilon > 0$ there is a set $A_\varepsilon$ with $A_\varepsilon\subset E$,
  $\mu(E-A_\varepsilon) \leq \varepsilon$, and such that $f_k\to f$ uniformly on $A_\varepsilon$.
\end{theorem}

The four-step approach to the construction of the Lebesgue integral that begins with its definition on simple functions carries over to the situation of a $\sigma$-finite measure space $(X,\mathcal M,\mu)$.
This leads to the notion of the integral, with respect to the measure $\mu$, 
of a non-negative measurable function $f$ on $X$. This integral is denoted by 
\[
\int_X f(x)\dd\mu(x),
\]
which we sometimes simplify as $\int_Xf\dd\mu$, $\int f\dd\mu$ or $\int f$, 
when no confusion is possible.
Finally, we say that a measurable function $f$ is \textbf{integrable} if 
\[
\int_X|f(x)|\dd\mu(x) < \infty.
\]
The elementary properties of the integral, such as linearity and monotonicity, 
continue to hold in this general setting, as well as the following basic limit theorems.
\begin{description}
  \item[Fatou's lemma.] If $\{f_n\}$ is a sequence of non-negative measurable functions on $X$, then 
  \[
  \int \liminf_{n\to\infty}f_n\dd\mu\leq\liminf_{n\to\infty}\int f_n\dd\mu.
  \]
  \item[Monotone convergence.] If $\{f_n\}$ is a sequence of non-negative measurable functions with $f_n\nearrow f$, then 
  \[
  \lim_{n\to\infty}\int f_n = \int f.
  \]
  \item[Dominated convergence.] If $\{f_n\}$ is a sequence of measurable functions with $f_n\to f$ a.e., 
  and such that $|f_n|\leq g$ for some integrable $g$, then 
  \[
  \int |f_n-f|\dd\mu\to0,\quad\text{as }n\to\infty,
  \]
  and consequently 
  \[
  \int f_n\dd\mu\to\int f\dd\mu,\quad\text{as }n\to\infty.
  \] 
\end{description}

\subsection{Product measures and Fubini theorem}

The construction of product measures leads to a general form of the theorem that express a multiple integral as a repeated integral.

Suppose $(X,\mathcal M, \mu)$ and $(Y,\mathcal N, \nu)$ are a pair of measure spaces.
We want to describe the \textbf{product measure} $\mu\times\nu$ on the space 
$Z = X\times Y = \{(x,y): x\in X, y\in Y\}$.
We will suppose here that the two measure spaces are each complete and $\sigma$-finite.

Let $\mathcal A$ denote the collection of all sets in $Z$ that are finite unions of disjoint measurable rectangles $A\times B$, with $A\in\mathcal M$ and $B\in\mathcal N$.
It is easy to check that $\mathcal A$ is an algebra of subsets of $X$.
From now on we abbreviate our terminology by referring to measurable rectangles simply as ``rectangles''.

On the rectangles we define the function $\mu_0$ by $\mu_0(A\times B) = \mu(A)\nu(B)$.
Since $\mu_0$ becomes a premeasure, $\mu_0$ has a unique extension to the algebra $\mathcal A$,
which we denote by $\gamma=\mu\times\nu$ on the $\sigma$-algebra $\mathcal O$ of sets generated by the algebra $\mathcal A$ of measurable rectangles.
In this way, we have defined the product measure space $(Z,\mathcal O,\mu) = (X\times Y,\mathcal O,\mu\times\nu)$.

\begin{theorem}
  In the setting above, suppose $f(x,y)$ is an integrable function on $(X\times Y,\mu\times\nu)$.
  \begin{enumerate}
    \item For almost every $x\in X$, the slice $f^y(x) = f(x,y)$ is integrable on $(X,\mu)$.
    \item $\int_Xf(x,y)\dd\mu$ is an integrable function on $Y$.
    \item $\int_Y\left(\int_Xf(x,y)\dd\mu\right)\dd\nu = \int_{X\times Y}f\dd\mu\times\nu$.
  \end{enumerate}
\end{theorem}

\subsection{Borel measures on $\mathbb R$ and the Lebesgue-Stieltjes integral}

\begin{theorem}
  Let $F$ be an increasing function on $\mathbb R$ that is normalized.
  \footnote{We say a function is normalized, if it is right-continuous.}
  Then there is a unique measure $\mu$ (also denoted by $\dd F$) on the Borel sets $\mathcal B$ on $\mathbb R$
  such that $\mu((a,b]) = F(b) - F(a)$ if $a<b$.
  Conversely, if $\mu$ is a measure on $\mathcal B$ that is finite on bounded intervals,
  then $F$ defined by $F(x) = \mu((0,x])$, $x>0$, $F(0) = 0$ and $F(x) = -\mu((-x,0])$, $x<0$, 
  is increasing and normalized.
\end{theorem}

\subsection{Absolute continuity of measures}

The generalization of the notion of absolute continuity requires that we extend the ideas of a measure to encompass set functions that may be positive or negative. 
We describe this notion first.

\subsubsection{Signed measures}

Loosely speaking, a signed measure possesses all the properties of a measure, 
except that it may take positive or negative values. 
More precisely, 
a \textbf{signed measure} $\nu$ on a $\sigma$-algebra $\mathcal M$ is a mapping that satisfies:
\begin{enumerate}
  \item The set function $\nu$ is extended-valued in the sense that $-\infty<\nu(E)\leq\infty$ for all $E\in\mathcal M$.
  \item If $\{E_j\}_{j=1}^\infty$ are disjoint subsets of $\mathcal M$, then 
  \[
  \nu\left(\cup_{j=1}^\infty E_j\right) = \sum_{j=1}^\infty\nu(E_j).
  \]
\end{enumerate}

Note that for this to hold the sum $\sum\nu(E_j)$ must be independent of the rearrangements of terms,
so that if $\nu(\cup_{j=1}^\infty E_j)$ is finite, 
it implies that the sum converges absolutely.

Given a signed measure $\nu$ on $(X,\mathcal M)$ it is always possible to find a 
(positive) measure $\mu$ that dominates $\nu$, in the sense that 
\[
\nu(E)\leq\mu(E)\quad \text{for all } E,
\]
and that in addition is the ``smallest'' $\mu$ that has this property.

The construction is in effect an abstract version of the decomposition of a function of bounded variation 
as the difference of two increasing functions. 
We proceed as follows.
We define a function $|\nu|$ on $\mathcal M$, called the \textbf{total variation} of $\nu$, by 
\[
|\nu|(E) := \sup\sum_{j=1}^\infty|\nu(E_j)|,
\]
where the supremum is taken over all partitions of $E$, 
that is, over all countable unions $E=\cup_{j=1}^\infty E_j$, 
where the sets $E_j$ are disjoint and belong to $\mathcal M$.

The fact that $|\nu|$ is actually additive is not obvious, and is given in the proof below.

\begin{proposition}
  The total variation $|\nu|$ of a signed measure $\nu$ is itself a (positive) measure that satisfies $\nu\leq|\nu|$.
\end{proposition}

It is now possible to write $\nu$ as the difference of two (positive) measures.
To see this, we define the \textbf{positive variation} and \textbf{positive variation} of $\nu$ by 
\[
\nu^+ = \frac12(|\nu|+\nu)
\quad\text{and}\quad
\nu^- = \frac12(|\nu|-\nu).
\]
By the proposition we see that $\nu^+$ and $\nu^-$ are measures, and they clearly satisfy 
\[
\nu = \nu^+ - \nu^-
\quad\text{and}\quad
|\nu| = \nu^+ + \nu^-.
\]
In the above if $\nu(E) = \infty$ for a set $E$, then $|\nu|(E) = \infty$, and $\nu^-(E)$ is defined to be zero.

We also make the following definition: we say that the signed measure $\nu$ is \textbf{$\sigma$-finite} if the measure $|\nu|$ is $\sigma$-finite.

\subsubsection{Absolute continuity}

Here we adopt the terminology that the measure $\nu$ is \textbf{supported} on a set $A$,
if $\nu(E) = \nu(E\cap A)$ for all $E\in\mathcal M$.

Two signed measures $\nu$ and $\mu$ on $(X,\mathcal M)$ are \textbf{mutually singular} if there are disjoint subsets $A$ and $B$ in $\mathcal M$ so that 
\[
\nu(E) = \nu(A\cap E)
\quad\text{and}\quad
\nu(E)=\mu(B\cap E)
\quad\text{for all } 
E\in\mathcal M.
\]
Thus, $\nu$ and $\mu$ are supported on disjoint subsets.
We use the symbol $\nu\bot \mu$ to denote the fact that the measures are mutually singular.

In contrast, if $\nu$ is \textbf{absolutely continuous} with respect to $\mu$ if 
\[
\nu(E) = 0\quad \text{whenever }E\in\mathcal M \text{ and }\mu(E) = 0.
\]
Thus, if $\nu$ is supported in a set $A$, then $A$ must be an essential part of the support of $\mu$ in the sense that $\mu(A)>0$.
We use the symbol $\nu\ll \mu$ to indicate that $\nu$ is absolutely continuous with respect to $\mu$.
Note that if $\nu$ and $\mu$ are mutually singular, and $\nu$ is also absolutely continuous with respect to $\mu$, then $\nu$ vanishes identically.

\begin{proposition}
  The assertion 
  ``For each $\varepsilon>0$, there is a $\delta>0$ such that $\mu(E)<\delta$ implies $|\nu(E)|<\varepsilon$.''
  implies $\nu\ll\mu$.
  Conversely, if $|\nu|$ is a finite measure, then $\nu\ll\mu$ implies this assertion.
\end{proposition}


\subsection{Lebesgue decomposition theorem}

It was proven in the case of $\mathbb R$ by Lebesgue, and in the general case by Radon and Nikodym.

\begin{theorem}
  Suppose $\mu$ is a $\sigma$-finite positive measure on the measure space $(X,\mathcal M)$ and 
  $\nu$ a $\sigma$-finite signed measure on $\mathcal M$.
  Then there exist unique signed measures $\nu_a$ and $\nu_s$ on $\mathcal M$ such that 
  $\nu_a\ll\mu$, $\nu_s\bot\mu$ and $\nu=\nu_a+\nu_s$.
  In addition, the measure $\nu_a$ takes the form $\dd\nu_a=f\dd\mu$; that is,
  \[
  \nu_a(E) = \int_Ef(x)\dd\mu(x)
  \]
  for some extended $\mu$-integrable function $f$.
\end{theorem}

Throughout this subsection we shall assume that $X$ is the real line, 
$S$ is the class of all Borel sets,
and $\mu$ is Lebesgue measure on $S$.

If $(X, S)$ is a measurable space and $\mu$ and $v$ are signed measures on $S$, 
we say that $v$ is absolutely continuous with respect to $\mu$,
in symbols $v \ll \mu$, if $v(E) = 0$ for every measurable set $E$ for which $|\mu|(E) = 0$.
We say that $\mu$ and $v$ are singular, in symbols $\mu \bot v$,
if there exist two disjoint sets $A$ and $B$ whose union is $X$ such that,
for every measurable set $E$, $A\cap E$ and $B\cap E$ are measurable and 
$|\mu|(A\cap E) = |v|(B\cap E) = 0$.


\begin{theorem}
  \label{thm: Lebesgue decomposition theorem}
  If $v$ is a finite measure on $S$, 
  then there exist three uniquely determined $v_1$, $v_2$ and $v_3$ on $S$
  whose sum is $v$ and which are such that $v_1$ is absolutely continuous with respect to $\mu$, 
  $v_2$ is purely atomic, 
  and $v_3$ is singular with respect to $\mu$ but $v_3(\{x\}) = 0$ for every point $x$.
  Here, we shall say that a finite measure $v$ on $S$ is \textbf{purely atomic} 
  if there exists a countable set $C$ such that $v(X-C) = 0$.
\end{theorem}

\begin{proof}
  According to the Lebesgue decomposition theorem there exist two measures $v_0$ and $v_1$ on $S$ 
  whose sum is $v$ and which are such that $v_0$ is singular 
  and $v_1$ is absolutely continuous with respect to $\mu$.
  Let $C$ be the set of those points $x$ for which $v_0(\{x\}) \neq 0$; the finiteness of $v$ implies that 
  $C$ is countable. If we write 
  \[
  v_2(E) = v_0(E\cap C) \quad \text{and}\quad v_3(E) = v_0(E-C),
  \]
  then it is clear that the decomposition $v = v_1 + v_2 + v_3$ has all the desired properties.
  Uniqueness follows from the uniqueness of the Lebesgue decomposition 
  and the easily verifiable uniqueness of $C$.
\end{proof}

\begin{example}
  Let $f\in L^1(\mathbb R)$ be a non negative function. 
  Define 
  \[
  \nu(E) := \int_Ef\dd x.
  \]
  Then $\nu_1(E) = \nu(E)$, $\nu_2(E) = 0$ and $\nu_3(E) = 0$.

  Let $r_1, r_2, \cdots$ be an enumeration of the set $R$ of all rational numbers. 
  Assign $v(\{r_i\}) = 2^{-i}$ and $v(\{p\}) = 0$ for any irrational number $p\in I$.
  Then $v(X) = 1$, $v_1(E) = 0$, $v_2(E) = v(I\cap E)$ and $v_3(E) = 0$.

  Let $C$ be the Cantor set and $H^{\alpha}$ be the Hausdorff measure with $\alpha=\ln 2/\ln3$.
  Define 
  \[
  v(E) := \int_E H(E\cap C)\dd H^\alpha.
  \]
  Then $v(X) = H^\alpha(C)$, $v_1(E) = 0$, $v_2(E) = 0$ and $v_3(E) = v(E\cap C)$.
\end{example}

\begin{example}
  \label{eg: Lebesgue decomposition of increasing function}
  Suppose $F$ is an increasing function on $[a,b]$.
  \begin{enumerate}
    \item Prove that we can write 
    \[
    F = F_A + F_C + F_J,
    \]
    where each of the function $F_A$, $F_C$ and $F_J$ is increasing and:
    \begin{enumerate}
      \item $F_A$ is absolutely continuous.
      \item $F_C$ is continuous, but $F'_C(x) = 0$ for a.e. $x$.
      \item $F_J$ is a jump function.
    \end{enumerate}
    \item Moreover, each component $F_A$, $F_C$, $F_J$ is uniquely determined up to an additive constant.
  \end{enumerate}
  The above is the \textbf{Lebesgue decomposition} of $F$. There is a corresponding decomposition for any $F$ of bounded variation.
\end{example}

\begin{proof}
  Since $F$ is an increasing function, $F'$ exists for a.e. $x \in [a, b]$.
  Let 
  \[
  F_A(x) = \int_a^xF'(s)\dd s.
  \]
  Then $F_A(x) \leq F(b) - F(a)$ is absolutely continuous.
  Since $F$ has at most countably many discontinuities, 
  let $\{x_1, x_2,\cdots\}\subset[a,b]$ denote the points where $F$ is discontinuous,
  and let $\alpha_n$ denote the jump of $F$ at $x_n$,
  that is $\alpha_n = F(x_n+) - F(x_n-)$,
  then  
  \[
  F(x_n) = F(x_n-) + \theta_n\alpha_n,
  \]
  for some $\theta_n\in[0,1]$,
  with the understanding $F(b+) = F(b)$ and $F(a-) = F(a)$.
  If we let 
  \begin{align*}
    j_n(x) = 
    \begin{cases}
      0, & x<x_n,\\
      \theta_n, & x=x_n,\\
      1, & x>x_n,
    \end{cases}
  \end{align*}
  then we define the \textbf{jump function} associated to $F$ by 
  \[
  F_J(x) := \sum_{n=1}^\infty \alpha_n j_n(x).
  \]
  Finally, we define 
  \[
  F_C = F - F_A - F_J.
  \]
  It is clear that $F_C$ is continuous, as desired.
\end{proof}

\begin{example}
  Suppose that $F$ is an increasing normalized function on $\mathbb R$,
  that is, $F$ is right-continuous at every point,
  and let $F = F_A + F_B + F_J$ be the decomposition of $F$ in Example~\ref*{eg: Lebesgue decomposition of increasing function};
  here $F_A$ is absolutely continuous, $F_C$ is continuous with $F'_C = 0$ a.e, and $F_J$ is a pure jump function.
  Let $\mu = \mu_A + \mu_C + \mu_J$ with $\mu$, $\mu_A$, $\mu_C$ and $\mu_J$ the Borel measures associated to $F$, $F_A$, $F_C$ and $F_J$, respectively.
  Verify that:
  \begin{enumerate}
    \item $\mu_A$ is absolutely continuous with respect to Lebesgue measure and $\mu_A(E) = \int_EF'(x)\dd x$ for every Lebesgue measurable set $E$.
    \item As a result, if $F$ is absolutely continuous, then $\int f\dd\mu = \int f\dd F = \int f(x)F'(x)\dd x$ whenever $f$ and $fF'$ are integrable.
    \item $\mu_C + \mu_J$ and Lebesgue measure are mutually singular.
  \end{enumerate}
\end{example}


\subsection{Harmonic measure}

Green function $G(x,y)$ (Poisson kernel, Newtonian potential) solves 
\[
-\Delta_y G(x,y) = \delta_x(y), \quad x,y\in\mathbb R^n,
\]
in the distribution sense. 
So the singular function $K(x) = G(x,0)$, consistent with Liouville theorem,
has a unit harmonic capacity.

\begin{proposition}
  Let $n\geq3$. 
  Write
  \begin{align*}
    K(x) &= \int_0^\infty\frac{1}{(4\pi t)^{n/2}}e^{-\frac{|x|^2}{4t}}\dd t 
    = - \frac{1}{4\pi^{n/2}}\int_0^\infty\frac{1}{(4t)^{n/2-2}}e^{-\frac{|x|^2}{4t}}\dd\frac1{4t}\\
    &= \frac1{4\pi^{n/2}}\int_0^\infty s^{n/2-2}e^{-|x|^2s}\dd s
    = \frac{|x|^{2-n}}{4\pi^{n/2}}\int_0^\infty \xi^{n/2-2}e^{-\xi}\dd\xi \\
    &= \frac{\Gamma(n/2-1)|x|^{2-n}}{4\pi^{n/2}} = \frac{\Gamma(n/2)|x|^{2-n}}{2(n-2)\pi^{n/2}} =: c_n|x|^{2-n}.
  \end{align*}
  It can be seen that 
  \[
  K(x) \in W_{\loc}^{1,p}(\mathbb R^n)\cap C_{\loc}^\infty(\mathbb R^n\setminus \{0\}),
  \quad 1\leq p < n/(n-1),
  \]
  and 
  \[
  \nabla K\cdot x/|x| = c_n(2-n)|x|^{1-n},\quad \Delta K = 0, \quad x\in  \mathbb R^n\setminus\{0\}.
  \]
  It can be observed that for any $\phi\in C_0^\infty(\mathbb R^n)$,
  \[
  \Delta K*\phi (x) = -\delta_x(\phi).
  \]
\end{proposition}

\begin{proof}
  We calculate 
  \begin{align*}
    &\quad\Delta K*\phi(x)\\
    &= \int_{\mathbb R^n}K(x-y)\Delta\phi(y)\dd y & \text{distribution derivative}\\ 
    &= -\int_{\mathbb R^n}\nabla K(y)\nabla\phi(x-y)\dd y
    & \text{weak derivative}\\
    &= - \lim_{\varepsilon\searrow0}\int_{|y|>\varepsilon}\nabla K(y)\nabla \phi(x-y)\dd y
    & \text{Lebesgue dominated convergence theorem}\\
    &= \lim_{\varepsilon\searrow0}\int_{|y|=\varepsilon}\nabla K(y)\cdot y/|y|\phi(x-y)\dd y
    & \text{integration by parts}\\
    &= c_n(2-n)\omega_n\lim_{\varepsilon\searrow0}\fint_{|x-y|=\varepsilon}\phi(y)\dd y
    = -\phi(x).
    & \text{continuity}
  \end{align*}
\end{proof}

\begin{remark}
  Let 
  \begin{equation*}
    f(x) = 
    \begin{cases}
      -x/2, & x\leq 0,\\
      x/2, & x>0.
    \end{cases}
  \end{equation*}
  Then $f\in W_{\loc}^{1,\infty}(\mathbb R)$ and $f''(x) = \delta_0(x)$.
  Indeed,
  we observe that 
  \begin{align*}
    f''(\phi) = \int_{\mathbb R}  f\phi''\dd x 
    = \int_{\mathbb R^-}  \phi'/2\dd x - \int_{\mathbb R^+}\phi'/2\dd x 
    = \phi(0) = \int_{\mathbb R}\phi\dd\delta_0(x),
  \end{align*}
  holds for any $\phi(x)\in C^\infty_0(\mathbb R)$.
\end{remark}

\begin{remark}
Let 
  \begin{align*}
    N(x) = -\frac{1}{2\pi}\ln|x|,\quad x\in\mathbb R^2.
  \end{align*}
Then $N\in W^{1,q}_{\loc}(\mathbb R^2)$, $q\in[1,2)$ and $\Delta N = -\delta_0(x)$,
as desired.
\end{remark}

\begin{remark}
  Let 
  \[
  K^\varepsilon := 
  \begin{cases}
    c_n\varepsilon^{2-n}, & |x|\leq\varepsilon,\\
    K(x), & |x|>\varepsilon.
  \end{cases}
  \]
  Then following the proof above, we can see that 
  \[
  -\Delta K^\varepsilon(\phi) = \fint_{|x|=\varepsilon}\phi(x)\dd S,\quad \phi\in C_0^\infty(\mathbb R^n).
  \] 
  It is clear that for any $\varepsilon>0$,
  $-\Delta K^\varepsilon$ is mutually singular with respect to Lebesgue measure $\mu$.
  The singular set is exactly $S^\varepsilon := \{|x|=\varepsilon\}$.
  When $\varepsilon>0$, for each $x\in S^\varepsilon$, $-\Delta^\varepsilon(\{x\}) = 0$.
  While $\varepsilon = 0$, $-\Delta(\{0\}) = 1$. 
  See Theorem~\ref{thm: Lebesgue decomposition theorem} for more details.
\end{remark}

\begin{proposition}
  Let $n\geq2$. 
  Suppose $P\in W_{\loc}^{1,2}(\mathbb R^n)$.
  Put $S:= \spt(\Delta P)$.
  If $S\subset L$, 
  then $P$ is a distribution Laplacian with zero harmonic capacity,
  where
  \[
  L\cong \mathbb{R}^k,\quad k\in\{0,1,2,3,\cdots, n-2\},
  \]
  with the understanding $\mathbb R^0=\{0\}$.
\end{proposition}

\begin{proof}
  Without loss of generality, we may assume that 
  \[
  L\subset\{(0,0,x_3,x_4,\cdots,x_n)\} := \tilde L.
  \]
  Denote $\pi(x) = x_{1,2} = (x_1, x_2, 0,0,\cdots,0)$ as projection of $x$ to $\tilde L^\bot $.
  %Since $\Delta P = 0$ over $\mathbb R^n\setminus S$, 
  Fubuni theorem implies 
  \[
  b(s) := \int_{|x_{1,2}|=s}\int_{\pi^{-1}(x_{1,2})}|\nabla P|^2\dd H^{n-2}\dd S
  \]
  is well-defined.
  $P\in W^{1,2}_{\loc}(\mathbb R^n)$ implies that 
  \[
  B := \liminf_{s\searrow0}sb(s) = 0.
  \]
  Otherwise,
  \begin{align*}
    \int_{\mathbb R^n}|\nabla P|^2\dd x 
    &= \int_{\mathbb R^2}\int_{\pi^{-1}(x_{1,2})}|\nabla P|^2\dd H^{n-2}\dd H^2\\
    &= \int_{\mathbb R^+}b(s)\dd s = \infty,
  \end{align*}
  which is absurd.
  Using H\"older inequality, we obtain for any $\phi\in C_0^{\infty}(\mathbb R^n)$,
  \begin{align*}
    \int_{|x_{1,2}|=\varepsilon}
    \left(\int_{\mathbb R^{n-2}}|\nabla P||\phi|\dd H^{n-2}\right)\dd S
    \leq C(\phi)\sqrt{\varepsilon b(\varepsilon)}
  \end{align*}
  We estimate for any $\phi\in C_0^{\infty}(\mathbb R^n)$,
  \begin{align*}
    \Delta P(\phi) 
    &= \int_{\mathbb R^n}P\Delta\phi\dd x &\text{distribution derivative}\\
    &= -\int_{\mathbb R^n}\nabla P\nabla \phi\dd x &\text{weak derivative}\\
    &= -\int_{\mathbb R^{2}}\left(\int_{\mathbb R^{n-2}}\nabla P\nabla \phi\dd H^{n-2} \right)\dd H^2
    &\text{Fubini theorem}\\
    &= -\lim_{\varepsilon\searrow0}
    \int_{|x_{1,2}| > \varepsilon} \left(\int_{\mathbb R^{n-2}}\nabla P\nabla \phi\dd H^{n-2}\right)\dd H^2
    &\text{Lebesgue dominated convergence theorem}\\
    &\leq \liminf_{\varepsilon\searrow0}\int_{|x_{1,2}|=\varepsilon}
    \left(\int_{\mathbb R^{n-2}}|\nabla P||\phi|\dd H^{n-2}\right)\dd S = 0.
    &\text{integration by parts}
  \end{align*}
\end{proof}

\begin{lemma}
  Let $w\in W^{1,q}(\Omega)$, $q\in[1,n/(n-1))$ be the weak solution of 
  \begin{equation*}
    \begin{cases}
      -\Delta w = f, & \Omega,\\
      w = 0, &\partial\Omega,
    \end{cases}
  \end{equation*}
  in the sense that 
  \[
  \int_\Omega \nabla w\nabla \phi = \int_\Omega f\phi, \quad\phi\in C_0^\infty(\Omega),
  \]
  where $f\in L^1(\Omega)$ and $\Omega = \{x\in \mathbb R^n: |x| < 1\}$.
  Then 
  \[
  w(x) = -\int_{\Omega}K(x-y)\Delta w(y)\dd y, \quad x\in\Omega.
  \]
\end{lemma}

\begin{proof}
  Denote 
  \[
  h(x) = \int_\Omega \nabla K(x-y) \cdot\nabla w(y)\dd y,\quad x\in\Omega.
  \]
  Then $h\in L^{p}(\Omega)$, $p\in[1,n/(n-2))$.
  It follows that $h$ is almost everywhere finite.
  Let $x$ be a Lebesgue point of $w$.
\end{proof}

\section{可测函数}

\begin{proposition}\label{prop: uniform continuity of integrable function}
  If $f\in L^1(\mathbb R^n)$, then 
  \[
  \lim_{\varepsilon\searrow0}\sup_{|y|<\varepsilon}\|f(x+y)-f(x)\|_{L^1(\mathbb R^n)} = 0
  \]
  holds.
\end{proposition}

\begin{proof}
  Let $g\in C^\infty_0(\mathbb R^n)$. 
  By triangle inequality,
  \begin{align*}
		\|f(x+y)-f(x)\|_{L^1(\mathbb R^n)} 
		\leq \|f(x+y)-g(x+y)\|_{L^1(\mathbb R^n)} 
		+ \|g(x+y)-g(x)\|_{L^1(\mathbb R^n)}\\
		+ \|g(x)-f(x)\|_{L^1(\mathbb R^n)}.
	\end{align*}
  The first and third terms of the right side of this inequality may be made arbitrarily small 
  by choosing a sufficiently close smooth approximation, e.g. by modifier action, 
  and for fixed $g$, the middle term may be made arbitrarily small by choosing $y$ suffciently small,
  which can be made uniformly small, due to uniform continuity of $g$.
\end{proof}

\begin{example}
  Let 
  \begin{equation}
    f_{n,k}(x) = 
    \begin{cases}
      0, & x\in(-\infty, (k-2)/n],\\
      nx-k+2,& x\in((k-2)/n, (k-1)/n],\\
      1,& x\in((k-1/n), k/n],\\
      -nx+k+1, & x\in(k/n, (k+1)/n],\\
      0, & x\in((k+2)/n, \infty),
    \end{cases}
    \quad k = 1,2,\cdots,n, \quad n\geq1.
  \end{equation}
  $\{f_{n,k}\}$ is a sequence of continuous functions that converges in measure, 
  but does not converge at any point.
\end{example}

\begin{exercise}
\hfill\\
设$f(x)$是$[a,b]$上的可测函数,试证明$f'(x)$是$[a,b]$上的可测函数。



\end{exercise}


\begin{exercise}
\hfill\\
$\forall\delta>0$,$\exists E_{\delta}\subset E$使得$m(E_{\delta})<\delta$,且在$E\backslash E_{\delta}$上,$\{f_n(x)\}$一致收敛于$f(x)$。证$\{f_n(x)\}$几乎处处收敛于$f$。



\end{exercise}


\begin{exercise}
\hfill\\
证鲁津定理的逆定理:若$\forall\delta>0$,存在闭子集$F_{\delta}\subset E$,使$m(E\backslash F_{\delta})\leq\delta$,且$f(x)$在$F_{\delta}$上连续,则$f(x)$在$E$上是可测函数。



\end{exercise}


\begin{exercise}
\hfill\\
设$\{f_n(x)\}$是$E$上的可测函数列,$m(E)<\infty$。试证明$$\lim_{n\to\infty}f_n(x)=0,a.e.x\in E$$的充分必要条件是:对任意的$\varepsilon>0$有$$\lim_{n\to\infty}m(\{x\in E:\sup_{k>n}|f_k(x)|\geq\varepsilon\})=0.$$



\end{exercise}


\begin{exercise}
\hfill\\
设$\{f_n(x)\}$在$[a,b]$上依测度收敛于$f(x)$,$g(x)$是$R$上的连续函数。证明$\{g(f_n(x))\}$在$[a,b]$上依测度收敛于$g(f(x))$。



\end{exercise}



\begin{exercise}
\hfill\\
设$f(x)=f(\xi_1,\xi_2)$是$R^2$上的连续函数。$g_1(x)$,$g_2(x)$是$[a,b]$上的实值可测函数,试证明$F(x)=f(g_1(x),g_2(x))$是$[a,b]$上的可测函数。



\end{exercise}

\hfill\\
\section{Lebesgue积分}


\begin{exercise}
\hfill\\
设$f(x)$在$[a,b]$上的$\mathbb{R}$反常积分存在。证明:$f(x)$在$[a,b]$上可积的充要条件为$|f(x)|$在$[a,b]$上的$\mathbb{R}$反常积分存在。并证明此时成立$$(L)\int_{[a,b]}f(x)\mathrm{d}x=(R)\int_a^bf(x)\mathrm{d}x.$$


不妨设$x=b$为$f(x)$的瑕点。

若$|f|$在$[a,b]$上的$\mathbb{R}$反常积分存在,则$\forall n\geq1$,$|f|$在$E_n=[a,b-\frac{1}{n}]$上$\mathbb{R}$可积。因为$\{|f(x)|X_{E_n}(x)\}$是非负递增函数,且有$$\lim_{n\to\infty}|f(x)|X_{E_n}(x)=|f(x)|,x\in E.$$所以由Levi定理知:
\begin{align*}
(L)\int_E|f(x)|\mathrm{d}x&=\lim_{n\to\infty}\int_E|f(x)|X_{E_n}(x)\mathrm{d}x\\
&=\lim_{n\to\infty}\int_{E_n}|f(x)|\mathrm{d}x\\
&=(R)\int_a^b|f(x)|\mathrm{d}x\\
&\leq\infty.
\end{align*}
这就说明了$|f(x)|\in L(E).$

若$f(x)$在$E$上$\mathbb{L}$可积,则$|f(x)|$在$[a,b]$上$L$可积。
定义$E_n=[a,b_n]$,其中$b_n\leq b$且$\lim\limits_{n\to\infty}b_n=b$。
因为$\{|f(x)|X_{E_n}(x)\}$是非负递增函数,且
$$\lim_{n\to\infty}|f(x)|X_{E_n}(x)=|f(x)|,$$
所以应用Levi定理可得:
$$+\infty>(L)\int_E|f(x)|\mathrm{d}x=\lim_{n\to\infty}\int_E|f(x)|X_{E_n}(x)\mathrm{d}x.$$
另一方面$$\int_{E_n}|f(x)|\mathrm{d}x=\int_E|f(x)|X_{E_n}(x)\mathrm{d}x.$$
这就说明了$$\lim_{n\to\infty}\int_a^{b_n}|f(x)|\mathrm{d}x=\int_E|f(x)|\mathrm{d}x<\infty$$
对任意的$b_n\to b^-$恒成立,即$|f(x)|$在$[a,b]$上$\mathbb{R}$可积。
\end{exercise}



\begin{exercise}
\hfill\\
设$f$是$E$上定义的函数。如果存在可积函数列$g_n$,$h_n$使得$g_n(x)\leq f(x)\leq h_n(x)\quad a.e.$,而且
$$\lim_{n\to\infty}(h_n(x)-g_n(x))\mathrm{d}x=0,$$
则$f$在$E$上可积。


定义$$K_n(x)=f(x)-g_n(x),$$
$$F_n(x)=h_n(x)-g_n(x),$$
则
$$0\leq K_n(x)\leq F_n(x)\quad a.e.,$$
且
\begin{equation}\label{lebesgue_integral_1}
\lim_{n\to\infty}F_n(x)\mathrm{d}x=0.
\end{equation}
由(\ref{lebesgue_integral_1})可知$F_n(x)\Rightarrow0$。
于是$K_n(x)\Rightarrow0$。即$g_n(x)\Rightarrow f(x)$。
那么由依测度收敛的Lebesgue控制收敛定理知$f$在$E$上可积,且积分满足:
$$\int_Ef(x)\mathrm{d}x=\lim_{n\to\infty}\int_Eg_n(x)\mathrm{d}x.$$
\end{exercise}



\begin{exercise}
\hfill\\
设$f\in L(\mathbb{R})$,若对$\mathbb{R}$上任意连续函数$g(x)$,有$\int_{\mathbb{R}}f(x)g(x)\mathrm{d}x=0$,证明$f(x)=0,\quad a.e.x\in\mathbb{R}.$


定义
\begin{equation}
h(x)=
\begin{cases}
1,&x\in R[f(x)\geq0],\\
-1,&x\in R[f(x)<0],\\
\end{cases}
\end{equation}
则$h(x)$为$\mathbb{R}$上简单函数。因为$f\in L(\mathbb{R})$,于是$\forall\varepsilon>0$,$\exists X>0$,使得$$\int_{\mathbb{R}\backslash E_X}|f|\mathrm{d}x<\frac{\varepsilon}{4},$$
其中$E_X=\{x\in\mathbb{R}:|x|<X\}$。
对在$E_X\subset\mathbb{R}$上的有界可测函数$h(x)$,有$\forall\delta>0$,存在闭集$F\subset E_X$满足$m(E_X\backslash F)<\delta$和$\mathbb{R}$上的连续函数$g(x)$满足$g(x)=h(x),\forall x\in F$. 进一步,
$$\inf_{\mathbb{R}}g(x)=\inf_Fh(x)\geq-1,\sup_{\mathbb{R}}g(x)=\sup_Fh(x)\leq1.$$
于是
\begin{align*}
|\int_{\mathbb{R}}f(x)h(x)\mathrm{d}x|&=|(\int_{\mathbb{R}\backslash E_X}+\int_{E_X\backslash F}+\int_F)f(x)(h(x)-g(x))\mathrm{d}x|\\
&=|(\int_{E_X\backslash F}+\int_{\mathbb{R}\backslash E_x})f(x)(h(x)-g(x))\mathrm{d}x|\\
&\leq2(\int_{E\backslash F}+\int_{\mathbb{R}\backslash E_X})|f(x)|\mathrm{d}x\\
&\leq\frac{\varepsilon}{2}+2\int_{E_X\backslash F}|f(x)|\mathrm{d}x\\
\end{align*}
又因为$|f(x)|\in L(\mathbb{R})$,从而由Lebesgue积分的绝对连续性知:对上述$\varepsilon>0$,$\exists\delta>0$,只要$m(E_X\backslash F)<\delta$就有
$$\int_{E_X\backslash F}|f(x)|\mathrm{d}x<\frac{\varepsilon}{4}.$$
于是$$|\int_{\mathbb{R}}f(x)h(x)\mathrm{d}x|<\varepsilon.$$
事实上,我们有$$\int_{\mathbb{R}}|f(x)|\mathrm{d}x=\int_{\mathbb{R}}f(x)h(x)\mathrm{d}x<\varepsilon.$$
于是由$\varepsilon$的任意性,就有$\int_{\mathbb{R}}|f(x)|\mathrm{d}x=0$,即$f(x)=0,a.e.$
\end{exercise}

\begin{exercise}
设 $f,f_k(k=1,2,\cdots)$ 在 $R^n$ 上可积, 且对于任一可测集$E\subset\mathbb{R}^n$,有
$$\int_Ef_k(x)\mathrm{d}x\leq\int_Ef_{k+1}\mathrm{d}x,\quad k=1,2,\cdots,$$
$$\lim_{k\to\infty}\int_Ef_k(x)\mathrm{d}x=\int_Ef(x)\mathrm{d}x,$$
试证明$\lim_{k\to\infty}f_k(x)=f(x),a.e.x\in\mathbb{R}^n.$

\begin{proof}
首先由题意,$\forall k\geq1$,有
$$\int_Ef_k(x)\mathrm{d}x\leq\lim_{k\to\infty}f_k(x)\mathrm{d}x=\int_Ef(x)\mathrm{d}x,$$对任意可测子集$E\subset\mathbb{R}^n$恒成立。
于是必然有
\begin{equation}\label{lebesgue_intergal_2}
f_k(x)\leq f_{k+1}(x) \leq f(x)\quad a.e.x\in\mathbb{R}^n.
\end{equation}
由 \eqref{lebesgue_intergal_2} 知
\[
  \lim_{k\to\infty}\int_E|f(x)-f_k(x)|\mathrm{d}x
  = \lim_{k\to\infty} \int_E(f(x)-f_k(x))\mathrm{d}x = 0,
\]
应用Fatou引理,
\[
0\geq\int_E\liminf_{k\to\infty}f(x)-f_k(x)\mathrm{d}x=\int_Ef(x)\mathrm{d}x-\int_E\limsup_{k\to\infty}f_k(x)\mathrm{d}x,
\]
即 
\[
\int_Ef(x)\mathrm{d}x
\leq\int_E\limsup_{k\to\infty}f_k(x)\mathrm{d}x
= \int_E\lim_{k\to\infty}f_k(x)\mathrm{d}x
\leq\int_Ef(x)\mathrm{d}x.
\] 
完成了证明.
\end{proof}
\end{exercise}

\begin{exercise}
\hfill\\
设$f(x)$,$g(x)$是$E$上非负可测函数且$f(x)g(x)$在$E$上可积。令$Ey=E[g\geq y]$。证明:$$F(y)=\int_{E_y}f(x)\mathrm{d}x$$对一切$y>0$都存在,且成立
$$\int_0^{\infty}F(y)\mathrm{d}y=\int_Ef(x)g(x)\mathrm{d}x.$$

$E_y$可测是显然的。$$F(y)=\int_Ef(x)X_{E_y}\mathrm{d}x,$$
而$|f(x)X_{E_y}|\leq|f(x)|$,故由$f(x)$的可积性知$f(x)X_{E_y}\in L(E),\forall y>0.$
考虑到$g(x)=\int_0^{\infty}X_{E_y}\mathrm{d}y$,于是
\begin{align*}
\int_Ef(x)g(x)\mathrm{d}x&=\int_Ef(x)\int_0^{\infty}X_{E_y}\mathrm{d}y\mathrm{d}x\\
&=\int_E\int_0^{\infty}f(x)X_{E_y}\mathrm{d}y\mathrm{d}x\\
&=\int_0^{\infty}\int_Ef(x)X_{E_y}\mathrm{d}x\mathrm{d}y\\
&=\int_0^{\infty}F(y)\mathrm{d}y.\\
\end{align*}

\end{exercise}


\begin{exercise}
若$f$在$R$上可积,证明$\int_R|f(x+h)-f(x)|\mathrm{d}x\to0,h\to0$.
\end{exercise}

\begin{proof}
首先注意到$0\leq|f(x+h)-f(x)|\leq|f(x+h)|+|f(x)|$,所以由$f\in L(R)$可知$f(x+h)\in L(R),\forall h\in R$,从而$|f(x+h)-f(x)|\in L(R).$于是对任给的$\varepsilon>0$,存在$X>0$使得$$\int_{|x|>X}|f(x+h)-f(x)|\mathrm{d}x<\frac{\varepsilon}{2}.$$

其次,$|f(x+h)-f(x)|$在$[-X-1,X+1]$上可积,由鲁津定理,对任意的$\delta>0$,存在闭集$E\subset[-X,X]$且$m([-X,X]\backslash E)<\delta$,有$f(x)$在$E$上连续。从而$f$在$E$上一致连续,即对上述$\varepsilon>0$,存在$\tau>0$,只要$|h|<\tau$,就有
$$|f(x+h)-f(x)|<\frac{\varepsilon}{8X}.$$
另一方面,由Lebesgue积分的绝对连续性,对上述的$\varepsilon>0$,存在$\delta>0$,只要可测集$F$的测度$m(F)<\delta$,就有
$$\int_{F}|f(x)|\mathrm{d}x<\frac{\varepsilon}{8}.$$
于是$$\int_{F}|f(x+h)-f(x)|\mathrm{d}x<\int_{F}(|f(x)|+|f(x+h)|)\mathrm{d}x\leq\frac{\varepsilon}{4}.$$

综合即有,对任给的$\varepsilon>0$,存在$\tau>0$,使得只要$|h|<\tau$,就有
\begin{align*}
\int_R|f(x+h)-f(x)|\mathrm{d}x&\leq\frac{\varepsilon}{2}+\int_{[-X,X]}|f(x+h)-f(x)|\mathrm{d}x\\
&=\frac{\varepsilon}{2}+\int_{[-X,X]\backslash E}|f(x+h)-f(x)|\mathrm{d}x+\int_E|f(x+h)-f(x)|\mathrm{d}x\\
&<\frac{\varepsilon}{2}+\frac{\varepsilon}{4}+2X*\frac{\varepsilon}{2X}\\
&=\varepsilon.
\end{align*}
\end{proof}




\begin{exercise}
\hfill\\



\end{exercise}



\begin{exercise}
\hfill\\



\end{exercise}



\begin{exercise}
\hfill\\



\end{exercise}




%\part{functional analysis}
\chapter{泛函分析}
\section{度量空间}
\subsection{压缩映像原理}
\begin{exercise}
\hfill\\
证明完备空间的闭子集是一个完备的子空间,而任一度量空间中的完备子空间必是闭子集。

设距离空间$(\mathscr{X},\rho)$是一个完备的距离空间,闭集$U\subset\mathscr{X}$。任取闭集$U$中的基本列$\{x_n\}\subset U$,则$\{x_n\}$也是定义在$(\mathscr{X},\rho)$中的基本列,从而存在$x\in\mathscr{X}$,使得$\lim_{n\rightarrow\infty}x_n=x$。又因为$U$是闭集,所以$x\in U$。这就是说距离空间$(U,\rho)$是完备的。

设$(U,\rho)$是度量空间$(\mathscr{X},\rho)$的一个完备子空间。则$\forall\{x_n\}\subset U$,若$x_n\rightarrow x$,则易知$\{x_n\}$是距离空间$(U,\rho)$的基本列,从而存在$x'\in U$使得$$\lim_{n\rightarrow\infty}x_n=x'.$$
于是$\rho(x',x)=\lim_{n\rightarrow\infty}\rho(x_n,x)=0$,从而$x'=x\in U$。于是$U$是闭子集。
\end{exercise}

\begin{exercise}
\hfill\\
(Newton)设$f$是定义在$[a,b]$上的二次连续可微的实值函数,$x_0\in(a,b)$使得$f(x_0)=0$,$f'(x_0)\neq0$。求证存在$x_0$的临域$U(x_0)$使得$\forall x_0\in U(x_0)$,迭代序列
$$x_{n+1}=x_n-\frac{f(x_n)}{f'(x_n)}\quad(n=0,1,2,\cdots)$$是收敛的,并且
$$\lim_{n\rightarrow\infty}x_n=x_0.$$

\begin{align*}
\shortintertext{记}
 Tx&=x-\frac{f(x)}{f'(x)},\\
\shortintertext{则}
T'x&=\frac{f(x)f''(x)}{f'^2(x)}\rightarrow0,x\rightarrow x_0.
\end{align*}
于是存在$\delta>0$,使得$T$在$(x_0-\delta,x_0+\delta)$上是压缩的。

又因为$f'(x_0)\neq0$,于是存在$h>0$,使得$\forall x\in(x_0-h,x_0+h)$,$f'(x)>0$。于是
$$|Tx-x_0|=|x-x_0-\frac{f(x)-f(x_0)}{f'(x)}|=|x-x_0||1-\frac{f'(\xi)}{f'(x)}|<|x-x_0|,$$
$\forall x\in(x_0-h,x_0+h)$,其中$\xi$介于$x$与$x_0$之间。这就是说$T$是在$(x_0-h,x_0+h)$到上的映射。

于是只要取$\tau=\frac{1}{2}\min\{\delta,h\}$,就有$T$是在$[x_0-\tau,x_0+\tau]$上的压缩映射。于是由压缩映象原理知,存在$x_0$的一个临域,使得迭代序列$\{x_n\}$收敛。收敛到$x_0$是显然的。
\end{exercise}
\begin{exercise}\label{13}
\hfill\\
设$M$是$(\mathbb{R}^n,\rho)$中的有界闭集,映射$T:M\rightarrow M$满足:
$$\rho(Tx,Ty)<\rho(x,y)(\forall x,y\in M,x\neq y).$$
求证$T$在$M$中存在唯一的不动点。

首先,如果$$\rho(x_n,x_0)\rightarrow0\quad(n\rightarrow\infty)$$,则$$\rho(Tx_n,Tx_0)<\rho(x_n,x_0)\rightarrow0.\quad(n\rightarrow\infty)$$
这就是说$T$是一个连续映射。又距离函数$\rho(x,y)$是连续的,从而映射
$$f(x)=\rho(Tx,x),x\in M$$是$M\mapsto\mathbb{R}^+\cup\{0\}$上的连续映射。又$M\subset\mathbb{R}^n$是有界闭集,从而$M$是紧集,于是存在$x_0\in M$,使得$\forall x\in M$,$f(x_0)\leq f(x)$。如果$Tx_0\neq x_0$,则
$$f(x_0)=\rho(Tx_0,x_0)>\rho(T(Tx_0),Tx_0)=f(Tx_0).$$
这就与$x_0$的定义矛盾。于是$x_0$是不动点。假设$x'\neq x_0$也是$T$的一个不动点,则
$$\rho(x',x_0)=\rho(Tx',Tx_0)<\rho(x',x_0),$$
矛盾。于是不动点是唯一的。
\end{exercise}

\begin{exercise}
\hfill\\
对于积分方程
$$x(t)-\lambda\int_0^1e^{t-s}x(s)\mathrm{d}s=y(t),$$
其中$y(t)\in C[0,1]$为一给定函数,$\lambda$为常数,$|\lambda|<1$,求证存在唯一解$x(t)\in C[0,1]$。

考虑原积分方程等价于
$$e^{-t}x(t)=\lambda\int_0^1e^{-s}x(s)\mathrm{d}s+e^{-t}y(t),$$
籍此记
$$Tx=\lambda\int_0^1x\mathrm{d}s+e^{-t}y(t),$$
于是
\begin{align*}
\rho(Tx-Ty)&=\max_{t\in[0,1]}|\lambda\int_0^1(x-y)\mathrm{d}s|\\
&\leq|\lambda|\max_{t\in[0,1]}\int_0^1|x-y|\mathrm{d}s\\
&\leq|\lambda|\max_{t\in[0,1]}|x-y|\int_0^1\mathrm{d}s\\
&=|\lambda|\rho(x-y),\\
\end{align*}
即$T$是完备内积空间$(C[0,1],\rho)$映上的压缩映射。于是$T$在$(C[0,1],\rho)$存在唯一不动点。

考虑
$$\int_0^1e^{-t}x(t)dt=\lambda\int_0^1\int_0^1e^{-s}x(s)\mathrm{d}s\mathrm{d}t+\int_0^1e^{-t}y(t)\mathrm{d}t,$$
即
$$\int_0^1e^{-t}x(t)\mathrm{d}t=\frac{1}{1-\lambda}\int_0^1e^{-t}y(t)\mathrm{d}t,$$
所以原积分方程的解为:
$$x(t)=\frac{1}{1-\lambda}\int_0^1e^{t-s}y(s)\mathrm{d}s+y(t).$$
\end{exercise}

\subsection{完备化}

\begin{exercise}
令空间$S$为一切实或复数列
$$x=(\xi_1,\xi_2,\cdots,\xi_n,\cdots)$$
组成的集合,在$S$中定义距离为
$$\rho(x,y)=\sum_{k=1}^{\infty}\frac{1}{2^k}\frac{|\xi_k-\eta_k|}{1+|\xi_k-\eta_k|},$$
其中$x=(\xi_1,\xi_2,\cdots,\xi_n,\cdots)$,$y=(\eta_1,\eta_2,\cdots,\eta_n,\cdots)$。求证$S$为一个完备的距离空间。

首先检验$\rho(x,y)$是一个距离。
\begin{enumerate}
\item[i] $\rho(x,y)\geq0$是显然的;进一步,$\rho(x,y)=0$ $\Longleftrightarrow|\xi_k-\eta_k|=0,\forall k\geq0$ $\Longleftrightarrow x=y.$
\item[ii] $\rho(x,y)=\rho(y,x)$是显然的。
\item[iii] 考虑到
\begin{align*}
\frac{|x-y|}{1+|x-y|}&=\frac{|x-z+z-y|}{1+|x-z+z-y|}\\
&\leq\frac{|x-z|+|z-y|}{1+|x-z|+|z-y|}\\
&=\frac{|x-z|}{1+|x-z|+|z-y|}+\frac{|z-y|}{1+|x-z|+|y-z|}\\
&\leq\frac{|x-z|}{1+|x-z|}+\frac{|z-y|}{1+|z-y|}\\
\end{align*}
于是容易得到
$$\rho(x,y)\leq\rho(x,z)+\rho(y,z).$$
\end{enumerate}
这就证实了$\rho(x,y)$是一个距离。接下来证明完备性。
设$\{x^{(n)}\}\subset S$为基本列,即
$$\rho(x^{(m)}-x^{(n)})\rightarrow0,\text{ if }n,m\rightarrow\infty.$$
对固定的$k$,由$$\frac{1}{2^k}\frac{|\xi^{(m)}_k-\xi^{(n)}_k|}{1+|\xi^{(m)}_k-\xi^{(n)}_k|}\leq\rho(x^{(m)}-x^{(n)})\rightarrow0,\text{ if }n,m\rightarrow\infty.$$
可知,$\{\xi_k^{(n)}\}$是Cauchy列。从而存在一个实(复)数$\xi_k$使得
$$\lim_{n\rightarrow\infty}\xi_k^{(n)}=\xi_k.$$
令$m\rightarrow\infty,$
于是我们有$$\rho(x_k^{(n)},x)\rightarrow0,\text{ if }n\rightarrow\infty,$$
其中,$x=(\xi_1,\xi_2,\cdots,\xi_n,\cdots)$。这就是说$x^{(n)}\rightarrow x$,显然$x\in S$。这就完成了完备性的证明。
\end{exercise}

\begin{exercise}
\hfill\\
设$F$是只有有限项不为0的实数列全体,在$F$上引进距离
$$\rho(x,y)=\sup_{k\geq1}|\xi_k-\eta_k|,$$
其中$x=\{\xi_k\}\in F,y=\{\eta_k\}\in F$,求证$(F,\rho)$不完备,并指出它的完备化空间。

要说明$(F,\rho)$不完备,只需给出一个反例。考虑$$x_n(1,\frac{1}{2},\cdots,\frac{1}{n},0,\cdots,0,\cdots),$$
显然$$\rho(x_m,x_n)\leq\min\{\frac{1}{n},\frac{1}{m}\}\rightarrow0,\textbf{ if }m,n\rightarrow0,$$
即$\{x_n\}\subset F$是基本列,然而
$$x_n\rightarrow(1,\frac{1}{2},\cdots\frac{1}{n},\cdots)\not\in F.$$

$F$的完备化空间为
$$\overline{F}=\{\{x_n\}\subset\mathbb{R}:\lim_{n\rightarrow\infty}x_n=0.\}.$$

首先,$$F\subset\overline{F}.$$
因为$\forall\{x_n\}\in F$,有$\lim_{n\rightarrow\infty}x_n=0$,所以$\{x_n\}\in\overline{F}$,即$F\subset\overline{F}$。

其次,$F$在$\overline{F}$中稠密。因为$\forall\{x_n\}\in\overline{F}$,我们总可以定义$X_n=(x_1,x_2,\cdots,x_n,0,0,\cdots)\in F$,$n=1,2,\cdots$。易见$X_n\rightarrow x=(x_1,x_2,\cdots),n\rightarrow\infty$。

最后,我们证明$\overline{F}$是完备的。设$\{x^{(n)}=(\xi_1^{(n)},\xi_2^{(n)},\cdots)\}_{n=1}^{\infty}$是$\overline{F}$中的基本列。则$$\rho(x^{(m)},x^{(n)})=\sup_{k\geq1}|\xi_k^{(m)}-\xi_k^{(n)}|\rightarrow0,n,m\rightarrow0.$$
于是对固定的$k$,有
$$|\xi_k^{(m)}-\xi_k^{(n)}|\leq\rho(x^{(m)},x^{(n)})\rightarrow0,n,m\rightarrow\infty,$$
即$\{\xi_k^{(n)}\}$是Cauchy列。从而存在$\xi_k\in\mathbb{R}$使得$$\lim_{n\rightarrow\infty}\xi_k^{(n)}=\xi_k.$$
记$x=(\xi_1,\xi_2,\cdots)$,则令$m\rightarrow\infty$,有
$$\rho(x^{(n)},x)\rightarrow0,n\rightarrow\infty.$$
下面只需证明$\{\xi_n\}$是Cauchy列即可。
\begin{align*}
|\xi_l-\xi_k|&=|\xi_l-\xi_l^{(n)}+\xi_l^{(n)}-\xi_k^{(n)}+\xi_k^{(n)}-\xi_k|\\
&\leq|\xi_l-\xi_l^{(n)}|+|\xi_l^{(n)}-\xi_k^{(n)}|+|\xi_k^{(n)}-\xi_k|\\
\end{align*}
因为$$\lim_{n\rightarrow\infty}\xi_l^{(n)}=\xi_l,$$$$\lim_{n\rightarrow\infty}\xi_k^{(n)}=\xi_k$$
且$\{\xi_k^{(n)}\}_{k=1}^{\infty}\in\overline{F}$是Cauchy列。所以有
$$|\xi_l-\xi_k|\rightarrow0,k,l\rightarrow\infty.$$
即$x\in\overline{F}$,$\overline{F}$是完备距离空间。

综上,$\overline{F}$是$F$的完备化空间。
\end{exercise}

\section{列紧性}

\begin{exercise}

设$(\mathscr{X},\rho)$是度量空间,$M$是$\mathscr{X}$中的列紧集,映射$f$满足
$$\rho(f(x_1),f(x_2))<\rho(x_1,x_2)\quad(\forall x_1,x_2\in\mathscr{X},x_1\neq x_2).$$
求证:$f$在$\mathscr{X}$中存在唯一的不动点。

定义
$$\overline{M}:=\{x\in\mathscr{X}:\exists x_n\in M,n=1,2,\cdots\text{ s.t. }\lim_{n\rightarrow\infty}x_n=x.\}.$$

因为$M$是$\mathscr{X}$中的列紧集,所以上述定义是合理的。易见$M\subset\overline{M}$。下证$\overline{M}$在$\mathscr{X}$上也是列紧的。

如果$I:=\overline{M}/M$是有限集,则结论是显然的。现在假设$I$是至少可数集。任取$I$中点列$\{\xi_i\}_{i=1}^{\infty}$,则对每个$\xi_i$,存在$M$中Cauchy列$\{x_i^{(n)}\}_{n=1}^{\infty}$,有
$$\lim_{n\rightarrow\infty}x_i^{(n)}=\xi_i.$$


对于$\{x_i^{(i)}\}_{i=1}^{\infty}$也是$M$中的点列,从而有收敛子列,不妨设为$\{x_{i_k}^{(i_k)}\}_{k=1}^{\infty}$,于是存在$\xi_0\in\mathscr{X}$,使得
$$\lim_{k\rightarrow\infty}x_{i_k}^{(i_k)}=\xi_0.$$

\begin{align*}
|\xi_{i_m}-\xi_{i_n}|&=|\xi_{i_m}-x_{i_m}^{(i_m)}+x_{i_m}^{(i_m)}-x_{i_n}^{(i_n)}+x_{i_n}^{(i_n)}-\xi_{i_n}|\\
&\leq|\xi_{i_m}-x_{i_m}^{(i_m)}|+|x_{i_m}^{(i_m)}-x_{i_n}^{(i_n)}|+|x_{i_n}^{(i_n)}-\xi_{i_n}|\\
&\rightarrow0,m,n\rightarrow\infty.\\
\end{align*}
于是$\{\xi_{i_k}\}$是Cauchy列,且
$$\lim_{k\rightarrow\infty}\xi_{i_k}=\xi_0.$$
这就是说$I$也是列紧集,从而$\overline{M}$也是列紧集。从以上证明过程还可看出,$\overline{M}$还是闭集。

于是将$f$限制在有界闭集$\overline{M}$上便有:$f_{\overline{M}}=f|_{\overline{M}}:\overline{M}\mapsto M$满足
$$\rho(f_{\overline{M}}(x_1),f_{\overline{M}}(x_2))<\rho(x_1,x_2)\quad(\forall x_1,x_2\in\overline{M},x_1\neq x_2).$$

再利用练习\ref{13}的结果(证明方法),就得到了我们所要的结论。
\end{exercise}

\section{线性赋范空间}

\begin{lemma}
  Let $1\leq p,q<\infty$. Then 
  \[
  (L^p((0,T); L^q(\Omega)))' = L^{p'}((0,T); L^{q'}(\Omega)),
  \quad 1/p+1/p' = 1,\quad 1/q+1/q'=1,
  \]
  where $\Omega\subset\mathbb R^n$ is a measurable set.
\end{lemma}

\begin{proof}
  Let $X_{p,q} = L^p((0,T); L^q(\Omega))$. 
  For $f\in X_{p,q}$ and $g\in X_{p',q'}$, denote 
  \[
  H_g(f) = \int_0^T\int_\Omega fg\dd x\dd t.
  \]
  Then 
  \begin{align*}
    |H_g(f)| &\leq \int_0^T\|f\|_q\|g\|_{q'}\dd t \leq \|f\|_{X_{p,q}}\|g\|_{X_{p',q'}}.
  \end{align*}

  Let $H\in X_{p,q}'$. 
  Define 
  \[
    \mu(E) := H(\chi_E),\quad E\in\Lambda.
  \]  
  Clearly, $\mu$ satisfies the countable additivity.
  Moreover, $\mu$ is absolutely continuous with respect to Lebesgue measure.
  By Radon-Nikodym theorem, there exists an integrable function $g$ such that 
  \[
    \mu(E) = \iint_E g\dd x\dd t,
  \]
  and therefore,
  \[
    H(f) = \iint_{(0,T)\times\Omega}fg\dd x\dd t,\quad f\in X_{p,q}.
  \]
\end{proof}

\begin{exercise}
\hfill\\
设$C(0,1]$表示$(0,1]$上连续且有界的函数$x(t)$全体。$\forall x\in C(0,1]$,令$\|x\|=\sup\limits_{0<t\leq1}|x(t)|$。求证:
\begin{enumerate}
\item[(1)] $\|\cdot\|$是$C(0,1]$空间上的范数。
\item[(2)] $l^{\infty}$与$C(0,1]$空间的一个子空间是等距同构的。
\end{enumerate}

\begin{enumerate}
\item[(1)]
\begin{enumerate}
\item[1.]$\|x\|\geq0\forall x\in C(0,1]$显然成立。特别的
$$\|x\|=0\Leftrightarrow x=0.$$
\item[2.]$\|ax\|=\sup\limits_{0<t\leq1}|ax(t)|=|a|\sup\limits_{0<t\leq1}|x(t)|=|a|\|x\|.$
\item[3.]$\|x+y\|=\sup\limits_{0<t\leq1}|x+y|\leq\sup\limits_{0<t\leq1}(|x|+|y|)=\|x\|+\|y\|.$
\end{enumerate}
\item[(2)] 取$l=(l_1,l_2,\cdots)\in l^{\infty}$,则$\|l\|=\sup\limits_{i\geq1}|l_i|<\infty.$定义$$\phi(l)(x)=\begin{cases}
4l_n-2^{n+1}l_nx,&3\leq2^{n+1}x\leq4,\\
2^{n+1}l_nx-2l_n,&2\leq2^{n+1}x\leq3,\\
\end{cases}$$
易知$\phi(l)(x)\in C(0,1]$。且显然$\phi(l)$是从$l^{\infty}$到$\phi(l^{\infty}\subset C(0,1]$上的一一映射。且有$$\|\phi(l_1)-\phi(l_2)\|=\|\phi(l_1-l_2)\|=\|l_1-l_2\|.$$
现在只需证明$\phi(l)$是连续映射。只需证$\phi(l)$在$l=\theta$处连续。这是显然的因为,对任意的$\varepsilon>0$,存在$\delta=\varepsilon$,只要$\|l\|<\delta$,就有$\|\phi(l)\|=\|l\|<\varepsilon$成立。于是$l^{\infty}$与$\phi(l^{\infty})\subset C(0,1]$同构。
\end{enumerate}
\end{exercise}

\begin{exercise}
\hfill\\
设$\mathscr{X}$是$B^*$空间。求证:$\mathscr{X}$是$B$空间,必须而且仅须
$$\forall \{x_n\}\subset\mathscr{X},\sum_{n=1}^{\infty}\|x_n\|<\infty\Longrightarrow\sum_{n=1}^{\infty}x_n\text{收敛}.$$

$\Leftarrow:$记$$S_n=\sum_{i=1}^{n}x_i,$$
由$$\sum_{n=1}^{\infty}\|x_n\|<\infty,$$我们有
$\forall\varepsilon>0$,$\exists N>0$,使得只要$m>n>N$,就有
$$\sum_{i=n+1}^{m}\|x_i\|<\varepsilon.$$于是对上述$\varepsilon>0$,我们有
$$\|S_n-S_m\|=\|\sum_{i=n+1}^{m}x_i\|\leq\sum_{i=n+1}^{m}\|x_i\|<\varepsilon.$$从而$\{S_n\}$是柯西列。由$\mathscr{X}$是$B$空间知,$S_n$收敛。

$\Rightarrow:$任取柯西列$\{x_n\}\subset\mathscr{X}$,则我们只要证明存在子列$\{x_{n_k}\}$收敛即可。因为$\forall\varepsilon>0$,$\exists N$,使得$m>n>N$,就有$$\|x_m-x_n\|<\varepsilon.$$于是对$\frac{1}{2}$,能找到$N_1$,使得只要$m>n>N_1$,就有$$\|x_m-x_n\|<\frac12,$$不妨取$n_1=N_1+1$。同样的,对$\frac{1}{4}$,能找到$N_2>N_1$,使得只要$m>n>N_2$,就有$$\|x_m-x_n\|<\frac{1}{4},$$不妨取$n_2=N_2+1$。依次进行下去,我们就得到子列$\{x_{n_k}\}$满足
$$\|x_{n_{k+1}}-x_{n_k}\|<\frac{1}{2^k}.$$
显然$$\sum_{k=1}^{\infty}\|x_{n_{k+1}}-x_{n_k}\|<1<\infty.$$于是$$S_n=\sum_{i=1}^{k}x_{n_{k+1}}-x_{n_k}=x_{n_{k+1}}-x_{n_1}$$收敛,即$\{x_{n_k}\}$收敛。
\end{exercise}

\begin{exercise}
\hfill\\
设$\mathscr{X}$是线性赋范空间,函数$\phi:\mathscr{X}\mapsto R^1$称为凸的,如果不等式
\begin{equation}\label{tu}
\phi(\lambda x+(1-\lambda)y)\leq\lambda\phi(x)+(1-\lambda)\phi(y)\quad(\forall 0\leq\lambda\leq1)
\end{equation}
成立。求证凸函数的局部极小值必然是全空间最小值。

设$x_0$是凸函数$f(x)$的一个局部极小值。则存在$\delta>0$,使得$\forall x\in B^0(x_0,\delta)$,有$f(x)\geq f(x_0)$。对任意固定的$y$,当$n$足够大时,总有
$$(1-\frac{1}{n})x_0+\frac{1}{n}y\in B^0(x_0,\delta).$$
于是$$f(x_0)\leq f((1-\frac{1}{n})x_0+\frac{1}{n}y)\leq (1-\frac1n)f(x_0)+\frac{1}{n}f(y),$$
即$f(x_0)\leq f(y)$。得证。
\end{exercise}

\begin{exercise}
\hfill\\
设$\mathscr{X}$是$B^*$空间,$\mathscr{X}_0$是$\mathscr{X}$的线性子空间,假定$\exists c\in(0,1)$,使得
\begin{equation}
\inf_{x\in\mathscr{X}_0}\|y-x\|\leq c\|y\|\quad(\forall y\in\mathscr{X}).
\end{equation}
求证:$\mathscr{X}_0$在$\mathscr{X}$中稠密。

反证,$\mathscr{X}$在$\mathscr{X}$中不稠密,则存在$x\in\mathscr{X}$但$x\not\in\overline{\mathscr{X}_0}$。显然$\overline{\mathscr{X}_0}$也是$\mathscr{X}$的子空间,即是说$\overline{\mathscr{X}_0}$是$\mathscr{X}$的真闭子空间。于是应用F.Riesz引理,$\forall0<\varepsilon<1$,$\exists y\in\mathscr{X}$,使得$\|y\|=1$,并且$$\|y-x\|\geq1-\varepsilon\quad(\forall x\in\overline{\mathscr{X}_0}).$$
于是对$\varepsilon=\frac{1-c}{2}$,$\exists y_0\in\overline{\mathscr{X}}$,使得$\|y_0\|=1$,并且
$$\|y_0-x\|\geq\frac{1+c}{2}>c\quad(\forall x\in\overline{\mathscr{X}_0}).$$
这就与$$\inf_{x\in\mathscr{X}_0}\|y_0-x\|\leq c.$$
矛盾。
\end{exercise}

\begin{exercise}
\hfill\\
设$C_0$表示以0为极限的实数全体,并在$C_0$中赋以范数
$$\|x\|=\max_{n\geq1}|\xi_n\|\quad(\forall x=(\xi_1,\xi_2,\cdots,\xi_n,\cdots)\in C_0).$$
又设$$M\overset{\Delta}{=}\left\{x=\{\xi_n\}_{n=1}^{\infty}\in C_0|\sum_{n=1}^{\infty}\frac{\xi_n}{2^n}=0\right\}.$$
\begin{enumerate}
\item[(1)] 求证:$M$是$C_0$的闭线性子空间。
\item[(2)] 设$x_0=(2,0,0,\cdots)$,求证:
\begin{equation}
\inf_{Z\in M}\|x_0-Z\|=1,
\end{equation}
但$\forall y\in M$有$\|x_0-y\|>1$。

注: 

本题提供一个例子说明:对于无穷维闭线性子空间来说,给定其外一点$x_0$,未必能在其上找到一点$y$适合
$$\|x_0-y\|=\inf_{Z\in M}\|x_0-Z\|.$$
换句话说,给定$x_0\not\in M$,未必能在$M$上找到最佳逼近元。
\end{enumerate}

\begin{enumerate}
\item[(1)] $\forall k,l\in\mathbb{K}$以及$x,y\in M$,易见
$$\sum_{n=1}^{\infty}\frac{k\xi_n+l\eta_n}{2^n}=k\sum_{n=1}^{\infty}\frac{\xi_n}{2^n}+l\sum_{n=1}^{\infty}\frac{\eta_n}{2^n}=0,$$
其中$x=(\xi_1,\xi_2,\cdots,\xi_n,\cdots)$,$y=(\eta_1,\eta_2,\cdots,\eta_n,\cdots)$即$M$是线性的。
下面说明$M$是闭的。
设$\{x_n\}_{n=1}^{\infty}\in M$是柯西列,其中$x_n=(\xi^{(n)}_1,\xi^{(n)}_2,\cdots)$。由范数定义易知,数列$\{\xi_i^{(n)}\}_{n=1}^{\infty}$是柯西列,于是不妨设
$$\lim_{n\to\infty}\xi_i^{(n)}=\xi_i,\forall i\geq1.$$
记$x=(\xi_1,\xi_2,\cdots)$,于是
$$\lim_{n\to\infty}x_n=x.$$
要证$M$闭,则只需证$$\lim_{n\to\infty}\xi_n=0.$$

因为$\{x_n\}$是柯西列,所以$\forall\varepsilon>0,$ $\exists N>0,$只要$m,n>N$,就有
$$|\xi_m^{(i)}-\xi_n^{(i)}|\leq\|x_m-x_n\|<\frac{\varepsilon}{2}.$$
令$i\to\infty$,则有
$$|\xi_m-\xi_n|\leq\frac{\varepsilon}2<\varepsilon.$$
即$\{\xi_n\}$收敛,不妨设
$$\lim_{n\to\infty}\xi_n=\xi.$$
另一方面考虑数列$\{\xi_n^{(n)}\}$,我们有
$$|\xi_m^{(m)}-\xi|\leq|\xi_m^{(m)}-\xi_m|+|\xi_m-\xi|\leq\|x_m-x\|+|\xi_m-\xi|<\varepsilon,$$
即数列$\{\xi_n^{(n)}\}$收敛且收敛到$\xi$。另外取定$i=m$,则令$n\to\infty$,我们有$$|\xi_m^{(m)}-\xi_n^{(m)}|\to|\xi_m^{(m)}-0|=|\xi_m^{(m)}|\leq\frac{\varepsilon}2.$$再令$m\to\infty$,我们就有
$|\xi|\leq\frac{\varepsilon}{2}<\varepsilon.$
这就是说$\xi=0$。证毕。

\item[(2)] 记$Z=(z_1,z_2,\cdots)$,如果存在$i>1$使得$|z_i|>1$,则$\|x_0-Z\|\geq|0-z_i|>1$。否则如果$\forall i>1$,$|z_i|\leq1$,那么要使
$$\sum_{i=1}^{\infty}=0,$$
有$$|\frac{z_1}{2}|=|\sum_{i=2}^{\infty}\frac{z_i}{2^i}|\leq\sum_{i=2}^{\infty}|\frac{z_i}{2^i}|\leq\sum_{i=2}^{\infty}\frac{1}{2^i}=\frac{1}{2},$$即$|z_1|\leq1$。这样也有
$$\|x_0-Z\|\geq1.$$
上述等号成立当且仅当$z_i=1\forall i>1$或$z_i=-1\forall i>1$,而这是不可能的,因为$\{z_i\}$收敛到$0$。故上述不等式严格成立,即$$\|x_0-Z\|>1,\forall Z\in M.$$

取$$x_n=(1-\frac{1}{2^n},-1,-1,\cdots,-1,0,0,\cdots)$$,其中$x_n$中共有$n$项的值为$-1$。易验证,$x_n\in M$,且
$$\|x_0-x_n\|=1+\frac{1}{2^n}\to1.$$
综上,$\inf_{Z\in M}\|x_0-Z\|=1.$
\end{enumerate}
\end{exercise}

\begin{exercise}
\hfill\\
设$\mathscr{X}$是$B^*$空间,$M$是$\mathscr{X}$上的有限维真子空间,求证$\exists y\in\mathscr{X}$,$\|y\|=1$,使得
$$\|y-x\|\geq1\quad(\forall x\in M).$$
\end{exercise}
\begin{proof}
  取定一个$y_0\in\mathscr{X}\\M$,则在$M$中存在最佳逼近元$x_0\in M$使得$$\|y_0-x_0\|\leq\|y_0-x\|,\forall x\in M.$$因为有限维子空间$M$是闭的,故$\|y_0-x_0\|>0$。取
  $$y=\frac{y_0-x_0}{\|y_0-x_0\|},$$
  则$\|y\|=1$。且
  \begin{align*}
  \|y-x\|&=\frac{1}{\|y_0-x_0\|}\|y_0-x_0-\|y_0-x_0\|x\|\\
  &\geq\frac{1}{\|y_0-x_0\|}\|y_0-x_0\|\\
  &=1.
  \end{align*} 
  证毕。 
\end{proof}




\section{凸集与不动点}
\begin{exercise}
\hfill\\
求证在$B$空间中,列紧集的凸包是列紧集。

设$\mathscr{X}$是$B$空间,$M\in\mathscr{X}$是列紧集。则$\forall\varepsilon>0,$存在$$N_{\varepsilon}=\{x_{\varepsilon}^1,x_{\varepsilon}^2,\cdots,x_{\varepsilon}^N\}$$为$M$的有穷$\varepsilon$网。易证$co(N_{\varepsilon})$是$co(M)$的一个$\varepsilon$网。而$co(N_{\varepsilon})$是有限维子空间$span(N_{\varepsilon})$上的有界闭凸集,从而列紧,即对上述$\varepsilon>0$,存在$co(N_{\varepsilon})$的有穷$\varepsilon$网,记为$$coN_{\varepsilon}.$$下面我们要证明$coN_{\varepsilon}$恰好是$co(M)$的有穷$2\varepsilon$网。

因为$\forall x\in co(M)$,$\exists x_1\in co(N_{\varepsilon})$,使得$$\|x-x_1\|<\varepsilon;$$又对$x_1$,存在$x_{\varepsilon}\in coN_{\varepsilon}$,使得$$\|x_1-x_{\varepsilon}\|<\varepsilon.$$
于是,$\forall x\in co(M)$,$\exists x_{\varepsilon}\in coN_{\varepsilon}$,使得
$$\|x-x_{\varepsilon}\|=\|x-x_1+x_1-x_{\varepsilon}\|\leq\|x-x_1\|+\|x_1-x_{\varepsilon}\|<2\varepsilon.$$
这就是说,$coN_\varepsilon$是$co(M)$的一个有穷$\varepsilon$网。由$\varepsilon$的任意性,我们知道,$co(M)$是列紧集。
\end{exercise}

\begin{exercise}
\hfill\\
设$K(x,y)$是$[0,1]\times[0,1]$上的正值连续函数,定义映射
$$(Tu)(x)=\int_0^1K(x,y)u(y)\mathrm{d}y\quad(\forall u\in C[0,1]).$$
求证:存在$\lambda>0$及非负但不恒为零的连续函数$u$,满足
$$Tu=\lambda u.$$

定义
$$C:=\{u(x)\in C[0,1]|\int_0^1u(x)\mathrm{d}x=1,u(x)\geq0\},$$
则$C$是$C[0,1]$的闭凸子集。
定义
$$f(u)=\frac{\int_0^1K(x,y)u(y)\mathrm{d}y}{\int_0^1\int_0^1K(x,y)u(y)\mathrm{d}y\mathrm{d}x}.$$
则$$\int_0^1f(u)\mathrm{d}x=1.$$显然$f(u)\geq0,$
从而$f(u):C\mapsto C.$
下面我们说明$f(C)$列紧。
首先因为$K(x,y)$是正值连续函数,所以存在$m,M>0$使得
$$m\leq f(x,y)\leq M\quad(\forall(x,y)\in[0,1]\times[0,1]),$$
任取$u\in C$则有
$$|f(u)|\leq\frac{M\int_0^1u(y)\mathrm{d}y}{m\int_0^1\int_0^1u(y)\mathrm{d}y\mathrm{d}x}=\frac{M}{m},$$
其次$K(x,y)$在$[0,1]\times[0,1]$上是一致连续的,所以$\forall\varepsilon>0$,$\exists\delta>0$,使得只要$x_1,x_2\in[0,1]$且$|x_1-x_2|<\delta$,就有$$|K(x_1,y)-K(x_2,y)|<m\varepsilon.$$
\begin{align*}
|f(u)(x_1)-f(u)(x_2)|&=\frac{|\int_0^1(K(x_1,y)-K(x_2,y))u(y)\mathrm{d}y|}{\int_0^1\int_0^1K(x,y)u(y)\mathrm{d}y\mathrm{d}x}\\
&\leq\frac{\int_0^1|K(x_1,y)-K(x_2,y)|u(y)\mathrm{d}y}{\int_0^1\int_0^1K(x,y)u(y)\mathrm{d}y\mathrm{d}x}\\
&\leq\frac{m\varepsilon\int_0^1u(y)\mathrm{d}y}{\int_0^1\int_0^1K(x,y)u(y)\mathrm{d}y\mathrm{d}x}\\
&\leq\varepsilon.\\
\end{align*}
以上说明了$C$是一致有界且等度连续的,由Arzela-Ascoli定理知$C$列紧。最后由Schauder不动点定理知,$\exists u_0\in C$使得$$f(u_0)=u_0,$$
即
$$(Tu_0)=\int_0^1K(x,y)u_0(y)\mathrm{d}y=\lambda u_0,$$其中$\lambda=\int_0^1\int_0^1K(x,y)u_0(y)\mathrm{d}y\mathrm{d}x>0.$
\end{exercise}

\section{内积空间}


\begin{exercise}
\hfill\\
设$\{e_n\}_1^{\infty}$,$\{f_n\}_1^{\infty}$是Hilbert空间$\mathscr{X}$中的两个正交规范集,满足条件
\begin{equation}
\sum_{n=1}^{\infty}\|e_n-f_n\|^2<1.
\end{equation}
求证:$\{e_n\}$和$\{f_n\}$两者中一个完备蕴含另一个完备。

不妨设$\{e_n\}$是完备的,而$\{f_n\}$不完备。于是存在$f\in\mathscr{X}$满足$(f,f)=1$,而$(f,f_n)=0,\forall n\geq1.$
因为$\{e_n\}$完备,所以
$$f=\sum_{i=1}^{\infty}(f,e_i)e_i,$$
%$$f_n=\sum_{i=1}^{\infty}(f_n,e_i)e_i,$$
定义$$\overline{f}=\sum_{i=1}^{\infty}(f,e_i)f_n,$$
则有$$(\overline{f},\overline{f})=\sum_{i=1}^{\infty}(f,e_i)\overline{(f,e_i)}=(f,f)=1,$$
$$(f,\overline{f})=\sum_{i=1}^{\infty}(f,(f,e_i)f_n)=0.$$
于是我们有
$$(f-\overline{f},f-\overline{f})=(f,f)+(\overline{f},\overline{f})=2.$$
另一方面
\begin{align*}
(f-\overline{f},f-\overline{f})&=\|f-\overline{f}\|^2\\
&=\|\sum_{i=1}^{\infty}(f,e_i)(e_i-f_i)\|^2\\
&\leq(\sum_{i=1}^{\infty}\|(f,e_i)(e_i-f_i)\|)^2\\
&=(\sum_{i=1}^{\infty}|(f,e_i)|\|e_i-f_i\|)^2\\
&\leq(\sum_{i=1}^{\infty}(f,e_i)\overline{(f,e_i)})(\sum_{i=1}^{\infty}\|e_i-f_i\|^2)\\
&\leq1.
\end{align*}
这就导出了矛盾。于是如果$\{e_n\}$完备,则必然$\{f_n\}$完备。
\end{exercise}

\begin{exercise}
\hfill\\
设$f(x)\in C^2[a,b]$,满足边界条件:
$$f(a)=f(b)=0,f'(a)=1,f'(b)=0.$$
求证:
$$\int_a^b|f''(x)|^2\mathrm{d}x\geq\frac{4}{b-a}.$$

只需注意到由题目条件定义的三次样条插值函数为
$$g(x)=\frac{(x-a)(x-b)^2}{(b-a)^2},$$
且
$$\int_a^b|g''(x)|^2\mathrm{d}x=\frac{4}{b-a}.$$
\end{exercise}

\begin{exercise}
\hfill\\
设$D$是$\mathbb{C}$中的单位开圆域,$H^2(D)$表示在$D$内满足
$$\iint_D|u(z)|^2\mathrm{d}x\mathrm{d}y<\infty\quad(z=x+iy)$$
的解析函数全体组成的空间。规定内积为
$$(u,v)=\iint_Du(z)\overline{v(z)}\mathrm{d}x\mathrm{d}y.$$
\begin{enumerate}
\item[1] 如果$u(z)$的泰勒展开式是
$$u(z)=\sum_{k=0}^{\infty}b_kz^k,$$
求证:
$$\sum_{k=0}^{\infty}\frac{|b_k|^2}{1+k}<\infty;$$
\item[2] 设$u(z),v(z)\in H^2(D)$,并且
$$u(z)=\sum_{k=0}^{\infty}a_kz^k,\quad v(z)=\sum_{k=0}^{\infty}b_kz^k,$$
求证:
$$(u,v)=\pi\sum_{k=0}^{\infty}\frac{a_k\overline{b_k}}{k+1};$$
\item[3] 设$u(z)\in H^2(D)$,求证:
$$|u(z)|\leq\frac{\|u\|}{\sqrt[2]{\pi}(1-|z|)}\quad(\forall|z|<1);$$
\item[4] 验证$H^2(D)$是Hilbert空间。
\end{enumerate}

\begin{enumerate}
\item[(1)]记 $u_k(z)=z^k,$则
$$
(u_m,u_n)=\iint_Du_m\overline{u_n}\mathrm{d}x\mathrm{d}y=\begin{cases}
0,&m\neq n;\\
\frac{\pi}{k+1},&m=n,\\
\end{cases}
$$
于是
\begin{align*}
(u,u)&=\iint_Du(z)\overline{u(z)}\mathrm{d}x\mathrm{d}y\\
&=\iint_D|u(z)|^2\mathrm{d}x\mathrm{d}y\\
&=\iint_Du(z)\sum_{k=0}^{\infty}\overline{b_k}\overline{z}^k\mathrm{d}x\mathrm{d}y\\
&=\sum_{k=0}^{\infty}\overline{b_k}\iint_Du(z)\overline{z}^k\mathrm{d}x\mathrm{d}y\\
&=\sum_{k=0}^{\infty}\overline{b_k}\iint_D\sum_{l=0}^{\infty}b_lz^l\overline{z}^k\mathrm{d}x\mathrm{d}y\\
&=\sum_{k=0}^{\infty}\overline{b_k}\sum_{l=0}^{\infty}b_l\iint_Dz^l\overline{z}^k\mathrm{d}x\mathrm{d}y\\
&=\sum_{k=0}^{\infty}b_k\overline{b_k}\frac{\pi}{1+k}\\
&=\pi\sum_{k=0}^{\infty}\frac{|b_k|^2}{1+k}\\
&<+\infty.\\
\end{align*}
\item[(2)]类似$(1)$计算可得结论。
\item[(3)]我们先证当$u(z)=u_k(z),k\geq1$时结论成立,
即证$$|z|^{2k}(1-|z|)^2\leq\frac{1}{k+1}.$$
记$f(x)=x^{2k}(1-x)^2,x\in(0,1)$,则
$$0\leq f(x)\leq(\frac{k}{k+1})^{2k}\frac{1}{(k+1)^2}<\frac{1}{k+1}.$$

那么就有
\begin{align*}
|u(z)|&=|\sum_{k=0}^{\infty}b_ku_k(z)|\\
&\leq\sum_{k=0}^{\infty}|b_k||u_k(z)|\\
&\leq\sum_{k=0}^{\infty}|b_k|\frac{\|u_k\|}{\sqrt{\pi}(1-|z|)}\\
&=\frac{1}{\sqrt{\pi}(1-|z|)}\sum_{k=0}^{\infty}\|b_ku_k\|\\
&=\sum_{k=0}^{\infty}|b_k
\end{align*}


未完待续
\end{enumerate}
\end{exercise}





%\part{note}

\chapter{note}

\section{notation}

\subsection{Marcinkiewicz space}

Here the Marcinkiewicz space $M^p\left(\mathbb{R}^d\right), 1<p<\infty$, is defined as the set of $f \in$ $L_{\text {loc }}^1(\mathbb{R}^d)$ such that
\[
\int_K|f(x)| d x \leq C|K|^{(p-1) / p}
\]
for all subsets $K$ of finite measure. The minimal $C$ in the above inequality gives a norm in this space, i.e.,
\[
\|f\|_{M^p\left(\mathbb{R}^d\right)}=\sup \left\{\operatorname{meas}(K)^{-(p-1) / p} \int_K|f| d x: K \subset \mathbb{R}^d, \operatorname{meas}(K)>0\right\}
\]

\section{calculus}
\begin{equation}
	\int_\Omega |\nabla u|^\alpha \nabla u \cdot \nabla\Delta u\mathrm{d}x = \frac{1}{2}\int_{\partial \Omega} |\nabla u|^\alpha \frac{\partial |\nabla u|^2}{\partial n}\mathrm{d}S - \int_\Omega |\nabla u|^\alpha |\nabla^2u|^2\mathrm{d}x - \alpha \int_\Omega |\nabla u|^\alpha (\nabla |\nabla u|)^2\mathrm{d}x
\end{equation}
as \begin{equation}
	\nonumber
	\Delta|\nabla u|^2 = 2|D^2u|^2 + 2\nabla u \cdot \nabla \Delta u.
\end{equation}

\begin{proposition}
	let 
	\[
		K(x,t) := \frac{1}{(4\pi t)^{n/2}}e^{-\frac{|x|^2}{4t}},
	\]
	then for any $f\in L^1(\mathbb{R}^n)$,
	\begin{enumerate}
		\item mass preserving: 	
			\[
			\iint_{\mathbb R^n\times\mathbb R^n}K(x-y)f(y)\dd y\dd x 
			= \int_{\mathbb R^n}f,
			\]
			by Fubini theorem.
		\item pointwise convergence: 
		\[
		K*f\to f,\quad{t\searrow0},\quad\text{for a.e. } x\in\mathbb{R}^n,
		\]
		by Lebesgue differential theorem.
		\item strong convergence: $K*f\in C([0,\infty);L^1(\mathbb{R}^n))$.
		\item hypercontraction:
		\[
		\|K*f\|_p\leq \|K\|_p\|f\|_1\lesssim t^{-\frac{n}{2}\left(1-\frac1p\right)}\|f\|_1,
		\]
		by Young inequality for convolution.
		Particularly, by \cite{Giga1988},
		\begin{equation}\label{eq: giga1988}
			\limsup_{t\searrow0}t^{\frac{n}{2}\left(1-\frac1p\right)}\|K*f\|_p = 0,\quad p>1.
		\end{equation}
	\end{enumerate}
\end{proposition}
\begin{proof}
	Without loss of generality, we may assume that $f\geq0$ in the distribution sense.
	
	2. The idea of this proof is borrowed from \cite[Theorem~2.1 in Chapter~3]{Stein2005}. Let $x\in\mathbb R^n$ be a Lebesgue point. Then  
	\[
	F(r) := \fint_{|y|\leq r} |f(x-y)-f(x)|\dd y,\quad r > 0,
	\]
	is a continuous function with respect to $r>0$,
	and 
	\[
	F(r)\to 0\quad \text{as }r\searrow0.
	\]
	Moreover, $F(r)$ is bounded, that is, $F(r)\leq M$ for some $M>0$ and all $r>0$. 
	The first two properties are consequences of absolute continuity and Lebesgue differential theorem, respectively, and the third follows from the first two properties and integration of $f$.
	We estimate,
	\begin{align*}
		K*f-f &= \int_{\mathbb{R}^n}K(x-y)f(y)\dd y - f(x) = \int_{\mathbb R^n}K(y)(f(x-y)-f(x))\dd y\\
		&\leq \int_{|y|\leq\sqrt {4\pi t}}K(y)|f(x-y)-f(x)|\dd y 
		+ \sum_{k=0}^{\infty}\int_{2^k\sqrt {4\pi t} < |y|\leq 2^{k+1}\sqrt {4\pi t}}K(y)|f(x-y)-f(x)|\dd y\\
		&\leq C\fint_{B_{\sqrt {4\pi t}}(x)}|f(x-y)-f(x)|\dd y 
			+ C\sum_{k=0}^\infty\fint_{|y|\leq 2^{k+1}\sqrt{4\pi t}}2^{(k+1)n}e^{-4^k}|f(x-y)-f(x)|\dd y \\
		&= CF(\sqrt{4\pi t}) + C\sum_{k=0}^\infty 2^{(k+1)n}e^{-4^k} F(2^{k+1}\sqrt{4\pi t})\\
		&\leq CF(\sqrt{4\pi t}) + C\sum_{k=0}^{L}F(2^{k+1}\sqrt{4\pi t}) + C M \sum_{k=L+1}^\infty 2^{(k+1)n}e^{-4^k}.
	\end{align*}
	The third term of the right side of this inequality may be made arbitrarily small 
  by choosing a sufficiently large integer $L$, 
  and for fixed $L$, the first two terms may be made arbitrarily small by choosing $t$ suffciently small,
	as desired.

	3. 	 
	For $\varepsilon>0$, there exists $R>0$ such that 
	\[
	(4\pi)^{-n/2}\int_{\mathbb R^n\setminus B_R}e^{-z^2/4}<\varepsilon,
	\]
	we have 
	\begin{align*}
		\|K*f-f\|_{L^1(\mathbb R^n)} 
		&= \int_{\mathbb{R}^n}\left|\int_{\mathbb{R}^n}K(y)f(x-y)\dd y - f(x)\right|\dd x \\
		&= \int_{\mathbb{R}^n}\left|\int_{\mathbb{R}^n}K(y)(f(x-y) - f(x))\dd y\right|\dd x\\
		&\leq \iint_{\mathbb{R}^n\times\mathbb{R}^n}K(y)|(f(x-y) - f(x))|\dd x\dd y\\
		&= (4\pi)^{-n/2}\iint_{\mathbb{R}^n\times\mathbb R^n}e^{-z^2/4}|(f(x-\sqrt tz)-f(x))|\dd x\dd z\\
		&\leq (4\pi)^{-n/2}\int_{B_R}e^{-z^2/4}\int_{\mathbb{R}^n}|f(x-\sqrt tz)-f(x)|\dd x\dd z\\
		&\quad + (4\pi)^{-n/2}\int_{\mathbb R^n\setminus B_R}e^{-z^2/4}\int_{\mathbb{R}^n}|f(x-\sqrt tz)-f(x)|\dd x\dd z,\\
		&\leq \sup_{|z|<R}\|f(x-\sqrt t z) - f(x)\|_{L^1(\mathbb R^n)} 
		+ 2\|f\|_1\int_{|z|\geq R}e^{-z^2/4}.
	\end{align*}
	The second term of the right side of this inequality 
	may be made arbitrarily small by choossing $R$ sufficiently large,
	and, for fixed $R$, 
	the first term may be made arbitrarily small by choosing $t$ sufficiently small, 
	because of the uniform continuity of integrable function proved in  proposition~\ref{prop: uniform continuity of integrable function}. 
	This proves convergence in norm.


	4. by Young inequality, we estimate
	\begin{align*}
		\|K*f\|_p &\leq \|K\|_p\|f\|_1\\
		&= \frac{1}{(4\pi t)^{n/2}}\left(\int_{\mathbb R^n}e^{-\frac{p|x|^2}{4t}}\dd x\right)^{1/p}\|f\|_1\\
		&= \frac{(4t)^{n/(2p)}}{(4\pi t)^{n/2}p^{n/(2p)}}\left(\int_{\mathbb R^n}e^{-|z|^2}\dd z\right)^{1/p}\|f\|_1\\
		&= (4\pi t)^{-\frac{n}2\left(1-\frac1p\right)}p^{-\frac{n}{2p}}\|f\|_1.
	\end{align*}
%	For any $\varepsilon>0$, there exists $R>1>\delta>0$ such that $\|f\|_{\mathbb R^n\setminus B_{R-1}}<\varepsilon$,
	We calculate
	\begin{align*}
		t^{\frac{n}{2}\left(1-\frac1p\right)}\|K*f\|_p 
		&\lesssim t^{-n/(2p)}\left(\int_{\mathbb R^n}\left|\int_{\mathbb R^n}e^{-\frac{|x-y|^2}{4t}}f(y)\dd y\right|^p\dd x\right)^{1/p}\\
		&\leq \left(t^{-n/2}\int_{\mathbb R^n}\left(\int_{\mathbb R^n\setminus B_\delta(x)}e^{-\frac{|x-y|^2}{4t}}f(y)\dd y\right)^p\dd x\right)^{1/p}\\
		&\quad + \left(t^{-n/2}\int_{\mathbb R^n}\left(\int_{B_\delta(x)}e^{-\frac{|x-y|^2}{4t}}f(y)\dd y\right)^p\dd x\right)^{1/p}\\
		&=: I_1^{1/p} + I_2^{1/p}.
	\end{align*}
	By Fubini theorem and H\"older inequality, we estimate
	\begin{align*}
		%t^{-n/2}\int_{\mathbb R^n}\left|\int_{\mathbb R^n\setminus B_\delta(x)}e^{-\frac{|x-y|^2}{4t}}f(y)\dd y\right|^p\dd x
		I_1
		&\leq t^{-n/2} \int_{\mathbb R^n}\int_{\mathbb R^n\setminus B_\delta(x)}e^{-\frac{p|x-y|^2}{4t}}f(y)\dd y
		\cdot \left(\int_{\mathbb R^n\setminus B_\delta(x)}f(y)\dd y\right)^{p-1}\dd x\\
		&\leq t^{-n/2}\|f\|^{p-1}_{L^1(\mathbb R^n)}\iint_{\mathbb R^n\times\mathbb R^n}
		\chi_{\mathbb R^n\setminus B_\delta(x)}(y)	e^{-\frac{p|x-y|^2}{4t}}f(y)\dd y\dd x\\
		&\leq t^{-n/2}\|f\|_{L^1(\mathbb R^n)}^{p}\int_{\mathbb R^n\setminus B_\delta}e^{-\frac{p|x|^2}{4t}}\dd x\\
		&\lesssim \|f\|_{L^1(\mathbb R^n)}^{p}\int_{\mathbb R^n\setminus B_\delta/\sqrt t}e^{-|z|^2}\dd z
		\to 0,\quad\text{as }t\searrow0,
	\end{align*}
	% \begin{align*}
	% 	t^{-n/2}\int_{\mathbb R^n\setminus B_R}\left|\int_{B_\delta(x)}e^{-\frac{|x-y|^2}{4t}}f(y)\dd y\right|^p\dd x
	% 	&\leq  t^{-n/2} \int_{\mathbb R^n\setminus B_R}\int_{B_\delta(x)}e^{-\frac{p|x-y|^2}{4t}}f(y)\dd y
	% 	\cdot \left(\int_{B_\delta(x)}f(y)\dd y\right)^{p-1}\dd x\\
	% 	&\lesssim t^{-n/2}\|f\|^{p-1}_{L^1(\mathbb R^n)} \iint_{\mathbb R^n\times\mathbb R^n}
	% 	\chi_{B_\delta(x)}(y)\chi_{\mathbb R^n\setminus B_R}(x)	e^{-\frac{p|x-y|^2}{4t}}f(y)\dd y\dd x\\
	% 	&\lesssim  \|f\|^{p-1}_{L^1(\mathbb R^n)} \int_{\mathbb R^n\setminus B_{R-\delta}}f(y)\dd y,
	% \end{align*}
	and 
	\begin{align*}
		%t^{-n/2}\int_{\mathbb{R}^n}\left|\int_{B_\delta(x)}e^{-\frac{|x-y|^2}{4t}}f(y)\dd y\right|^p\dd x
		I_2
		&\leq t^{-n/2} \int_{\mathbb R^n}\int_{B_\delta(x)}e^{-\frac{p|x-y|^2}{4t}}f(y)\dd y
		\cdot \left(\int_{B_\delta(x)}f(y)\dd y\right)^{p-1}\dd x\\
		&\lesssim \|f\|_{L^1(\mathbb R^n)}\sup_{x\in\mathbb R^n} \cdot \left(\int_{B_\delta(x)}f(y)\dd y\right)^{p-1}.
	\end{align*}
	Therefore,
	\[
	\limsup_{t\searrow0}	t^{\frac{n}{2}\left(1-\frac1p\right)}\|K*f\|_p 
	\leq \|f\|^{1/p}_{L^1(\mathbb R^n)}\sup_{x\in\mathbb R^n} 
	\cdot \left(\int_{B_\delta(x)}f(y)\dd y\right)^{1-1/p},
	\]
	and \eqref{eq: giga1988} follows by absolute continuity of Lebesgue integral.
\end{proof}

\begin{remark}
	Banach-Alaoglu theorem tells that 
	\[
	K*f \overset{\ast}{\rightharpoonup} f,\quad \text{in } (C_0(\mathbb R^n))^*,
	\]
	as $t\searrow0$.

	If $f_n\to f$ a.e. and $\|f_n\|_1\to\|f\|_1$ as $n\to\infty$, 
	then $\|f_n-f\|_1 \to 0$ as $n\to\infty$.

	Integration convergence and almost-everywhere convergence imply convergence in norm (strong convergence).
	Actually, by Fatou lemma,
	\begin{align*}
	2\|f\|_1 &= \left\|\liminf_{n\to\infty}(|f|+|f_n|-|f_n-f|)\right\|_1\\
	&\leq \liminf_{n\to\infty}\||f_n|+|f|-|f_n-f|\|_1\\
	&\leq \limsup_{n\to\infty}\||f_n|+|f|\|_1 - \limsup_{n\to\infty}\|f_n-f\|_1\\
	&= 2\|f\|_1 - \limsup_{n\to\infty}\|f_n-f\|_1,
	\end{align*}
	which implies the desired result.	
\end{remark}

\begin{proposition}
	Let $I\in\mathbb{R}$ be an open interval. Suppose that $f: I\mapsto\mathbb{R}$ is a convex function,
	i.e.,
	\[
		f(\lambda a + (1-\lambda)b) \leq \lambda f(a) + (1-\lambda)f(b),\quad a,b\in I, \quad \lambda\in(0,1).
	\]
	Then $f\in C^{0,1}_{\mathrm{loc}}(I,\mathbb R)$.
\end{proposition}

\begin{proof}
	Fix $a\in I$.
	If $a<b$, then we have by putting $\lambda\nearrow1$
	\[
		\limsup_{\lambda\nearrow1}f(\lambda a + (1-\lambda)b) = \limsup_{c\searrow a}f(c)\leq f(a),
	\]
	if $a>b$, then we have similarly
	\[
		\limsup_{\lambda\nearrow1}f(\lambda a + (1-\lambda)b) = \limsup_{c\nearrow a}f(c)\leq f(a).
	\]
	Particularly, using
	\begin{align*}
		f(a) &= f(\lambda(a-\varepsilon(1-\lambda)) + (1-\lambda)(a+\varepsilon\lambda))\\
			&\leq \lambda f(a-\varepsilon(1-\lambda)) + (1-\lambda)f(a+\varepsilon\lambda),
			\quad 0<\varepsilon\ll 1,
	\end{align*}
	we have by putting $\lambda\nearrow1$,
	\[
		f(a)\leq \liminf_{\lambda\nearrow1}\lambda f(a-\varepsilon(1-\lambda)) 
			+ \limsup_{\lambda\nearrow1}(1-\lambda)f(a+\varepsilon\lambda)
			= \liminf_{c\nearrow a}f(c),
	\]
	and by putting $\lambda\searrow0$,
	\[
		f(a)\leq \limsup_{\lambda\searrow0}\lambda f(a-\varepsilon(1-\lambda)) 
			+ \liminf_{\lambda\searrow0}(1-\lambda)f(a+\varepsilon\lambda)
			= \liminf_{c\searrow a}f(c).
	\]	
	Four estimates above imply the continuity of $f$ at $a$.

	Using 
	\[
	\frac{f(\lambda a + (1-\lambda)b) - f(a)}{\lambda a + (1-\lambda)b - a} \leq \frac{f(b) - f(a)}{b-a},\quad a < b,
	\]
	we have
	\[
	\limsup_{c\searrow a}\frac{f(c)-f(a)}{c-a} 
	= \limsup_{\lambda\nearrow 1}\frac{f(\lambda a + (1-\lambda)b) - f(a)}{\lambda a + (1-\lambda)b - a} 
	\leq \frac{f(b) - f(a)}{b-a},
	\]
	and hence 
	\[
	\limsup_{c\searrow a}\frac{f(c)-f(a)}{c-a} 
	\leq \liminf_{b\searrow a}\frac{f(b)-f(a)}{b-a},
	\]
	which implies $f'(a+)$ exists and satisfies 
	\[
	f'(a+)\leq \frac{f(b) - f(a)}{b-a},\quad b>a.
	\]
	If $a>c=b$, we similarly obtain $f'(a-)$ exists and satisfies 
	\[
	f'(a-)\geq \frac{f(c) - f(a)}{c-a},\quad c<a.
	\]
	Noting 
	\begin{align*}
	\frac{f(b) - f(a)}{b-a} - \frac{f(a) - f(c)}{a-c}
	&= \frac{f(b)(a-c) + f(c)(b-a) - f(a)(b-c)}{(b-a)(a-c)}\\
	&= \frac{(b-c)}{(b-a)(a-c)}\left(f(b)\frac{a-c}{b-c} + f(c)\frac{b-a}{b-c} - f(a) \right)\geq 0,
	\end{align*}
	we have $f'(a-)\leq f'(a+)$.
	Using 
	\[
	\frac{f(b) - f(\lambda a + (1-\lambda)b)}{b - \lambda a - (1-\lambda)b} \geq \frac{f(b) - f(a)}{b-a},\quad b>a,
	\]
	we obtain
	\[
	f'(b-) = \lim_{\lambda\searrow0}\frac{f(b) - f(\lambda a + (1-\lambda)b)}{b - \lambda a - (1-\lambda)b}\geq \frac{f(b) - f(a)}{b-a}\geq f'(a+).
	\]
	So $g(a) := f'(a-)$, $a\in I$ is a nondecreasing functions 
	which has at most countably many  discontinuous points.
	At its continuous point $p\in I$,
	\begin{align*}
		f'(p-) = g(p) = \lim_{a\searrow p}g(a) 
		= \lim_{a\searrow p} f'(a-) \geq \lim_{a\searrow p}f'(p+) = f'(p+),
	\end{align*} 
	which implies that $f'$ exists almost everywhere.
	Moreover, for any compact set $K\subset I$, 
	there exists an open interval $(c,b)$ such that $K\subset(c,b)$ and $\overline{(c,b)}\subset I$. 
	For any $a\in K$, if $f'(a)$ exists, 
	then  
	\[
	\frac{f(a) - f(c)}{a - c} 
	\leq f'(a) \leq \frac{f(b) - f(a)}{b - a}.
	\] 
	Therefore $f\in W^{1,\infty}_{\mathrm{loc}}(I) = C^{0,1}_{\mathrm{loc}}(I)$.
\end{proof}

\begin{lemma}
	The surface area $\omega_n$ and volume $V_n$ of the unit ball $\Omega = B_1 := \{x\in\mathbb{R}^n: |x|<1\}$ are given by
	\[
		V_n=\frac{\pi^{\frac{n}{2}}}{\Gamma\left(\frac{n}{2}+1\right)} = \frac{\omega_n}{n},
	\]
	where $\Gamma(s)$ is the usual Gamma function.
\end{lemma}
\begin{proof}
	Using the fact
	\begin{equation*}
		\int_{\mathbb{R}^2}e^{-|x|^2}\dd x 
			= 2\pi\int_0^\infty re^{-r^2}\dd r
			= \pi,
	\end{equation*}
	we have 
	\begin{align*}
		\int_{\mathbb{R}^n}e^{-|x|^2}\dd x
			&= \left(\int_{\mathbb{R}}e^{-s^2}\dd s\right)^n\\
			&= \left(\int_{\mathbb{R}^2}e^{-s^2}\dd s\right)^{n/2}\\
			&= \pi^{n/2},
	\end{align*}
	While 
	\begin{align*}
		\int_{\mathbb{R}^n}e^{-|x|^2}\dd x
			&= \omega_n\int_0^\infty r^{n-1}e^{-r^2}\dd r\\
			&= \frac{\omega_n}2\int_0^\infty \rho^{n/2-1}e^{-\rho}\dd\rho\\
			&= \frac{\omega_n\Gamma(n/2)}{2},
	\end{align*}
	we end up with
	\[
		\omega_n = \frac{2\pi^{n/2}}{\Gamma(n/2)} = \frac{n\pi^{n/2}}{\Gamma(n/2+1)},
	\]
	and thus
	\[
		V_n = \int_{\Omega}\dd x = \omega_n\int_0^1r^{n-1}\dd r = \frac{\omega_n}{n}.
	\]
\end{proof}

\section{the interpolation-trace lemma}
\begin{lemma}
	\cite{Diaz1985}
	\label{the interpolation-trace lemma}
	Suppose $G$ is a bounded open subset of $R^{N}, N \geq 1,$ with a $C^{1}$ houndary $\partial G$ and $0 \leq \sigma \leq q<\infty .$ Then there exists a constant $C$ depending on $\sigma$, $q$ and $G$ such that for any $v \in W^{1, q+1}(G)$ we have
(2.12)
\begin{equation*}
	\|v\|_{L^{q+1}(\partial G)} \leq C\left(\|D v\|_{L^{q+1}(G)}+\|v\|_{L^{\sigma+1}(G)}\right)^{\theta}\|v\|_{L^{\sigma+1}(G)}^{1-\theta}
\end{equation*}
where $\theta=(N(q-\sigma)+\sigma+1) / \kappa$, \(\kappa=N(q-\sigma)+(\sigma+1)(q+1)\).

\end{lemma}

\begin{proof}
	For the sake of simplicity we restrict ourselves to $v \in C^{1}(\bar{G})$ since $C^{1}(\bar{G})$ is dense in $W^{1, q+1}(G) .$ The proof of \Lref{the interpolation-trace lemma} is divided onto four steps (see [9, Appendix] for a similar result).
	
	First step. From a result of Ehrling's lemma, for any $\varepsilon>0$ there exists $C_{\varepsilon}>0$ such that for any $v \in C^{1}(\bar{G})$ the following holds:
	\begin{equation*}
		\|v\|_{L^{q+1}(G)} \leq \varepsilon\|D v\|_{L^{q+1}(G)}+C_{\varepsilon}\|v\|_{L^{\sigma+1}(G)}
	\end{equation*}
	If we set $C_{2}=\max \left(1+\varepsilon, C_{\varepsilon}|G|^{1-1 /(\sigma+1)}\right)$ we get
	\begin{equation}\label{ehrling inequality}
		\|v\|_{W^{1, q+1}(G)} \leq C_{2}\left(\|D v\|_{L^{q+1}(G)}+\|v\|_{L^{\sigma+1}(G)}\right)
	\end{equation}
Second step. We start from the elementary trace result \cite{Adams2003}: there exists $C_{3}>0$ such that for any $u \in C^{1}(\bar{G})$ we have
\begin{equation}\label{L1 trace}
	\|u\|_{L^{1}(\partial G)} \leq C_{3}\|u\|_{W^{1,1}(G)}
\end{equation}
and for $q>0$ we apply \eqref{L1 trace} to $u=v|v|^{q}, v \in C^{1}(\bar{G}),$ so
\begin{equation*}
	\int_{\partial G}|v|^{q+1} d \sigma \leq C_{3}\left\{(q+1) \int_{G}|v|^{q}|D v| d x+\int_{G}|v|^{q+1} d x\right\}
\end{equation*}
since
\begin{equation*}
	\int_{G}|v|^{q}|D v| d x \leq\|D v\|_{L^{q+1}(G)}\|v\|_{L^{q+1}(G)}^{q}
\end{equation*}
we get
\begin{equation*}
	\int_{\partial G}|v|^{q+1} d \sigma \leq C_{3}\left\{(q+1)\|D v\|_{L^{q+1}(G)}\|v\|_{L^{q+1}(G)}^{q}+\|v\|_{L^{q+1}(G)}^{q+1}\right\}
\end{equation*}
which implies
\begin{equation}\label{trace inequality of generalized l1}
	\|v\|_{L^{q+1}(\partial G)} \leq\left((q+1) C_{3}\right)^{1 /(q+1)}\|v\|_{W^{1, q+1}(G)}^{1 /(q+1)}\|v\|_{L^{q+1}(G)}^{q /(q+1)}
\end{equation}

Third step. Set $0 \leq \sigma \leq q<\infty$. We claim that there exists a constant $C_{4}>0$ such that for any $v \in C^{1}(\bar{G})$ we have
\begin{equation}\label{claim equation}
	\|v\|_{L^{q+1}(G)} \leq C_{4}\|v\|_{W^{1, q+1}(G)}^{((q+1) \theta-1) / q}\|v\|_{L^{\sigma+1}(G)}^{(q+1)(1-\theta) / q}
\end{equation}
\textrm{Case 1. } Assume $q+1<N$. From Sobolev's inequality we have $\|v\|_{L^{\tau}(G)} \leq$ $C\|v\|_{W^{1, q+1}(G)}$ with $1 / \tau=1 /(q+1)-1 / N .$ Moreover
\begin{equation}\label{lp interpolation inequality 1}
	\|v\|_{L^{q+1}(G)} \leq\|v\|_{L^{\tau}(G)}^{1-\lambda}\|v\|_{L^{\sigma+1}(G)}^{\lambda},
\end{equation}
where $1 /(q+1)=\lambda /(\sigma+1)+(1-\lambda) / \tau,$ that is
\begin{equation*}
	\lambda=(q+1)(\sigma+1) / N(q-\sigma)+(q+1)(\sigma+1)
\end{equation*}
Hence with Sobolev's inequality
\begin{equation}\label{sobolev inequality}
	\|v\|_{L^{q+1}(G)} \leq C^{1-\lambda}\|v\|_{W^{1, q+1}(G)}^{1-\lambda}\|v\|_{L^{\sigma+1}(G)}^{\lambda}
\end{equation}
and
\begin{equation*}
	1-\lambda=\frac{N(q-\sigma)}{N(q-\sigma)+(q+1)(\sigma+1)}=\frac{(q+1) \theta-1}{q}, \quad \lambda=\frac{(q+1)(1-\theta)}{q}
\end{equation*}
\textrm{Case } ~ 2 . ~ \text { Assume } ~ $q+1 \geq N \geq 1$. We set $\alpha=(N+1) / 2, \rho=2(q+1) /(N+1)$, $\beta=(\sigma+1)(N+1) / 2(q+1)$ and $\alpha^{*}=\alpha N /(N-\alpha)\left(\alpha^{*}=\infty\right.$ if $\left.N=1\right) .$ From
H\"{o}lder's interpolating inequality we have
\begin{equation}\label{holder interpolating inequality}
	\|u\|_{L^{\alpha}(G)} \leq\|u\|_{L^{\alpha^ *}(G)}^{1-\lambda}\|u\|_{L^{\beta}(G)}^{\lambda},
\end{equation}
where $1 / \alpha=(1-\lambda) / \alpha^{*}+\lambda / \beta$(\eqref{holder interpolating inequality} is valid even if $0<\beta<1$ with a simple change of function). From Sobolev's inequality we get
\begin{equation}
\|u\|_{L^{\alpha}(G)} \leq C_{5}\|u\|_{W^{1, \alpha}(G)}^{1-\lambda}\|u\|_{L^{\beta}(G)}^{\lambda}
\end{equation}
Now we set $u=v|v|^{\rho-1}$ and we have
\begin{equation*}
	\begin{array}{c}
		\|u\|_{L^{\alpha}(G)}=\|v\|_{L^{\alpha \rho}(G)}^{\rho}=\|v\|_{L^{q+1}(G)}^{\rho} \\
		\|u\|_{L^{\beta}(G)}=\|v\|_{L^{\beta \rho}(G)}^{\rho}=\|v\|_{L^{\sigma+1}(G)}^{\rho} \\
		\|u\|_{W^{1, \alpha}(G)}=\|v\|_{L^{q+1}(G)}^{\rho}+\left(\int_{G}\left(\rho|v|^{\rho-1}|D v|\right)^{\alpha} d x\right)^{1 / \alpha}
	\end{array}
\end{equation*}
and
\begin{equation*}
	\int_{G}\left(|v|^{\rho-1}|D v|\right)^{\alpha} d x \leq\left(\int_{G}|v|^{\alpha \rho} d x\right)^{1-1 / \rho}\left(\int_{G}|D v|^{\alpha \rho} d x\right)^{1 / \rho}
\end{equation*}
which yields $\|u\|_{W^{1, \alpha}(G)} \leq \rho\|v\|_{L^{q+1}(G)}^{\rho-1}\|v\|_{W^{1, q+1}(G)}$ and \eqref{sobolev inequality} becomes
\begin{equation}
	\|v\|_{L^{q+1}(G)}^{\rho} \leq C_{6} \rho^{1-\lambda}\|v\|_{L^{q+1}(G)}^{(\rho-1)(1-\lambda)}\|v\|_{W^{1, q+1}(G)}^{1-\lambda}\|v\|_{L^{\sigma +1}(G)}^{\lambda \rho}
\end{equation}
If we compute the exponents we get
\begin{equation*}
	\frac{1-\lambda}{\lambda \rho+1-\lambda}=\frac{N(q-\sigma)}{(q+1)(\sigma+1)+N(q-\sigma)}=\frac{(q+1) \theta-1}{q}
\end{equation*}
and
\begin{equation*}
	\frac{\lambda \rho}{\lambda \rho+1-\lambda}=\frac{(q+1)(\sigma+1)}{N(q-\sigma)+(q+1)(\sigma+1)}=\frac{(q+1)(1-\theta)}{q}
\end{equation*}
which is \eqref{claim equation}.

Fourth step. End of the proof. We use \eqref{trace inequality of generalized l1} and \eqref{claim equation} and get
\begin{equation}
\|v\|_{L^{q+1}(\partial G)} \leq C_{7}\|v\|_{W^{1, q+1}(G)}^{1 /(q+1)}\|v\|_{W^{1, q+1}(G)}^{(q \theta+\theta-1) /(q+1)}\|v\|_{L^{\sigma+1}(G)}^{1-\theta}
\end{equation}
where $\theta=N(q-\sigma)+\sigma+1 / N(q-\sigma)+(q+1)(\sigma+1) ;$ using \eqref{ehrling inequality} yields finally 
\begin{equation}
\|v\|_{L^{q+1}(\partial G)} \leq C\left(\|D v\|_{L^{q+1}(G)}+\|v\|_{L^{\sigma+1}(G)}\right)^{\theta}\|v\|_{L^{\sigma+1}(G)}^{1-\theta}
\end{equation}
\end{proof}

\section{Ehrling's lemma}
\begin{lemma}[Ehrling's lemma]
	Let $(X,\|\cdot\|_X),$ $(Y,\|\cdot\|_Y)$ and $(Z,\|\cdot\|_Z)$ be three banach spaces. Assume that:
	$X$ is compactly embedded in $Y$: i.e. $X \subset Y$ and every $\|\cdot\|_X$ -bounded sequence in$ X$ has a subsequence that is $\|\cdot\|_Y$ convergent; and
	$Y$ is continuously embedded in $Z$: i.e. $Y \subset Z$ and there is a constant $k$ so that 
	$\|y\|_Z \leqslant k\|y\|_Y$ for every $y \in Y$.
	Then, for every $\varepsilon > 0$, there exists a constant $C(\varepsilon)$ such that, for all $x \in X$,
	
	${\displaystyle \|x\|_{Y}\leqslant \varepsilon \|x\|_{X}+C(\varepsilon )\|x\|_{Z}}.$
\end{lemma}
\begin{proof}
	proof by contradiction.
	for some given $\varepsilon_0 > 0,$ for all $n \in N^ \star$, there exists $x_n \in X$, such that
	${\displaystyle \|x_n\|_{Y} > \varepsilon_0 \|x_n\|_{X}+n\|x_n\|_{Z}}.$ Let $\tilde{x}_n=\dfrac{x_n}{\|x_n\|_Y},$we have $\|\tilde{x}_n\|_X < 1$ and $\|\tilde{x}_n\|_Z < \dfrac{1}{n}.$ on the one hand, 	$X$ is compactly embedded in $Y$, therefore $\{\tilde{x}_n\}$ has a subsequence that is $\|\cdot\|_Y$ convergent, noted $\{\tilde{x}_n\}$ as well; on the other hand, 	$Y$ is continuously embedded in $Z$, so $\{\tilde{x}_n\}$ is $\|\cdot\|_Z$ convengent. noticing $\|\tilde{x}_n\|_Z\mapsto0,$ we have $ \tilde{x}_n \mapsto 0 $ in $Z$, thus $ \tilde{x}_n \mapsto 0 $ in $Y$. However, $\|\tilde{x}_n\|_Y=1$ which has a controdiction with $ \tilde{x}_n \mapsto 0 $ in $Y$ .
\end{proof}

\section{inequality}
\begin{lemma}
	\begin{equation}
		\|fgh\|_1\leqslant \|f\|_a \|g\|_b \|h\|_c, \quad \text{if } a^{-1} +b^{-1} + c^{-1}=1.
	\end{equation}
\end{lemma}
\begin{lemma}
	\label{le: Young inequality for convolutions}
	if $r,p,q\geqslant1$ and such that $1+r^{-1} = p^{-1}+q^{-1}$, then
	\begin{equation}
		\|F*G\|_r\leqslant\|F\|_p\|G\|_q.
	\end{equation}
	\end{lemma}
\begin{proof}
	using the generalized holder inequation above, with $f=F^{1-\dfrac{p}{r}},g=G^{1-\dfrac{q}{r}}$, $h=F^{\dfrac{p}{r}}G^{\dfrac{q}{r}}$ and $a=\dfrac{q}{q-1}$, $b=\dfrac{p}{p-1}$, $c= \dfrac{1}{r}$.
\end{proof}



\begin{lemma}
	for any $z\in L^p(\Omega)$, there exists $c_1>0$ independent of $p$ and $q$ such that
	\begin{equation}\label{greenapproximation}
		\left\|e^{t \Delta} z\right\|_{L^{p}(\Omega)} \leqslant c_{1} t^{-\frac{n}{2}\left(\frac{1}{q}-\frac{1}{p}\right)}\|z\|_{L^{q}(\Omega)},
	\end{equation}
with $1\leqslant q\leqslant p < \infty.$
\end{lemma}
\begin{proof}
	with the help of Green function $G$, we can express $e^{t\Delta}$ explictly 
	\begin{equation}
		e^{t\Delta}z = \int_{\Omega}G(x,t;0,y)z(y)\mathit{d}y,
	\end{equation}
by pointwise estimate of Green function of Neumman heat semigroup, we have $c_2,c_3>0$ only dependent of $\Omega$ such that
\begin{equation}
	|G(x,t;0,y)| \leqslant \dfrac{c_2}{t^\frac{n}{2}}e^{\frac{c_3|x-y|^2}{t}},
\end{equation}
so one can verfity \eqref{greenapproximation} with calculation.
\end{proof}


\begin{lemma}
	 Let $\left(e^{t \Delta}\right)_{t \geqslant 0}$ be the Neumann heat semigroup in $\Omega,$ and let $\lambda_{1}>0$ denote the first nonzero eigenvalue of $-\Delta$ in $\Omega$ under Neumann boundary conditions. Then there exist constants $C_{1}, \ldots, C_{4}$ depending on $\Omega$ only which have the following properties.
	 \begin{itemize}
	 	\item [i] If $1 \leqslant q \leqslant p \leqslant \infty$ then
	$$
	\left\|e^{t \Delta} w\right\|_{L^{p}(\Omega)} \leqslant C_{1}\left(1+t^{-\frac{n}{2}\left(\frac{1}{q}-\frac{1}{p}\right)}\right) e^{-\lambda_{1} t}\|w\|_{L^{q}(\Omega)} \quad \text { for all } t>0
	$$
	holds for all $w \in L^{q}(\Omega)$ satisfying $\int_{\Omega} w=0$
	\item [ii] If $1 \leqslant q \leqslant p \leqslant \infty$ then
	$$
	\left\|\nabla e^{t \Delta} w\right\|_{L^{p}(\Omega)} \leqslant C_{2}\left(1+t^{-\frac{1}{2}-\frac{n}{2}\left(\frac{1}{q}-\frac{1}{p}\right)}\right) e^{-\lambda_{1} t}\|w\|_{L^{q}(\Omega)} \quad \text { for all } t>0
	$$
	is true for each $w \in L^{q}(\Omega)$.
	\item [iii] If $2 \leqslant p<\infty$ then
	$$
	\left\|\nabla e^{t \Delta} w\right\|_{L^{p}(\Omega)} \leqslant C_{3} e^{-\lambda_{1} t}\|\nabla w\|_{L^{p}(\Omega)} \text { for all } t>0
	$$
	is valid for all $w \in W^{1, p}(\Omega)$.
	\item [iv] Let $1<q \leqslant p<\infty .$ Then
	$$
	\left\|e^{t \Delta} \nabla \cdot w\right\|_{L^{p}(\Omega)} \leqslant C_{4}\left(1+t^{-\frac{1}{2}-\frac{n}{2}\left(\frac{1}{q}-\frac{1}{p}\right)}\right) e^{-\lambda_{1} t}\|w\|_{L^{q}(\Omega)} \quad \text { for all } t>0
	$$
	 \end{itemize}
	holds for all $w \in\left(C_{0}^{\infty}(\Omega)\right)^{n} .$ Consequently, for all $t>0$ the operator $e^{t \Delta} \nabla \cdot$ possesses a uniquely determined extension to an operator from $L^{q}(\Omega)$ into $L^{p}(\Omega),$ with norm controlled according to (1.7)
	
\end{lemma}
\begin{proof}
	Proof. (i) For $t<2,(1.4)$ is a consequence of the well-known smoothing estimate
	$$
	\left\|e^{t \Delta} z\right\|_{L^{p}(\Omega)} \leqslant c_{1} t^{-\frac{n}{2}\left(\frac{1}{q}-\frac{1}{p}\right)}\|z\|_{L^{q}(\Omega)} \quad \text { for all } t<2
	$$
	which can be checked for some $c_{1}$ independent of $p$ and $q$ and all $z \in L^{q}(\Omega)$ using pointwise estimates for Green's function of the Neumann heat semigroup [22, Theorem 2.2]. As to $t \geqslant 2,$ we first note that upon integrating the heat equation and using the variational definition of $\lambda_{1}$ we obtain
	$$
	\frac{1}{2} \frac{d}{d t} \int_{\Omega}\left|e^{t \Delta} w\right|^{2}=-\int_{\Omega}\left|\nabla e^{t \Delta} w\right|^{2} \leqslant-\lambda_{1} \int_{\Omega}\left|e^{t \Delta} w\right|^{2}
	$$
	for all $t>0$ and each smooth $w$ satisfying $\int_{\Omega} w=0 .$ Therefore,
	$$
	\left\|e^{t \Delta} w\right\|_{L^{2}(\Omega)} \leqslant e^{-\lambda_{1} t}\|w\|_{L^{2}(\Omega)} \quad \text { for all } t>0
	$$
	holds for all $w \in L^{2}(\Omega)$ with $\int_{\Omega} w=0 .$ Now for $p<2,$ using Hölder's inequality and then and (1.8) we find
	$$
	\begin{aligned}
		\left\|e^{t \Delta} w\right\|_{L^{p}(\Omega)} & \leqslant|\Omega|^{\frac{2-p}{2 p}}\left\|e^{t \Delta} w\right\|_{L^{2}(\Omega)} \leqslant|\Omega|^{\frac{2-p}{2 p}} e^{-\lambda_{1}(t-1)}\left\|e^{\Delta} w\right\|_{L^{2}(\Omega)} \\
		& \leqslant|\Omega|^{\frac{2-p}{2 p}} c_{1} e^{-\lambda_{1}(t-1)}\|w\|_{L^{q}(\Omega)}
	\end{aligned}
	$$
	for all $t \geqslant 2 .$ By a similar reasoning, for $p \geqslant 2$ we derive
	$$
	\left\|e^{t \Delta} w\right\|_{L^{p}(\Omega)} \leqslant c_{1}\left\|e^{(t-1) \Delta} w\right\|_{L^{2}(\Omega)} \leqslant c_{1} e^{-\lambda_{1}(t-2)}\left\|e^{\Delta} w\right\|_{L^{2}(\Omega)} \leqslant c_{1}^{2} e^{-\lambda_{1}(t-2)}\|w\|_{L^{q}(\Omega)}
	$$
	for all $t \geqslant 2,$ from which the claim follows.
	(ii) We first note that for some $c_{2}>0$ independent of $p$
	$$
	\left\|\nabla e^{t \Delta} w\right\|_{L^{p}(\Omega)} \leqslant c_{2} t^{-\frac{1}{2}}\|w\|_{L^{p}(\Omega)} \quad \text { for all } t \leqslant 1
	$$
	holds for all $w \in L^{p}(\Omega) .$ In fact, for $p=1$ and for $p=\infty$ this can be seen using pointwise estimates for the spatial gradient of Green's function of $e^{t \Delta}[22,$ Theorem 2.2$],$ whereby (1.10) for $1<p<\infty$ follows from a Marcinkiewicz-type interpolation argument (cf. $[10,$ Theorem 9.8$]$ ). In order to combine this with $(1.4),$ we write $\bar{w}:=\frac{1}{|\Omega|} \int_{\Omega} w$ and thus have $\int_{\Omega}(w-\bar{w})=0 .$ since constants are invariant under $e^{t \Delta},$ we thus obtain from (1.10) and (1.4)
	
	$$
	\begin{aligned}
		\left\|\nabla e^{t \Delta} w\right\|_{L^{p}(\Omega)} &=\left\|\nabla e^{\frac{t}{2} \Delta} e^{\frac{t}{2} \Delta}(w-\bar{w})\right\|_{L^{p}(\Omega)} \\
		& \leqslant c_{2}\left(\frac{t}{2}\right)^{-\frac{1}{2}}\left\|e^{\frac{t}{2} \Delta(w-\bar{w})}\right\|_{L^{p}(\Omega)} \\
		& \leqslant c_{2} C_{1}\left(\frac{t}{2}\right)^{-\frac{1}{2}}\left(1+\left(\frac{t}{2}\right)^{-\frac{n}{2}\left(\frac{1}{q}-\frac{1}{p}\right)}\right) e^{-\lambda_{1} \frac{t}{2}}\|w-\bar{w}\|_{L^{q}(\Omega)}
	\end{aligned}
	$$
	which implies (1.5) for $t<2 .$ For $t \geqslant 2$ we split $e^{t \Delta}$ in a different way to see that
	$$
	\begin{aligned}
		\left\|\nabla e^{t \Delta} w\right\|_{L^{p}(\Omega)} &=\left\|\nabla e^{\Delta} e^{(t-1) \Delta}(w-\bar{w})\right\|_{L^{p}(\Omega)} \leqslant c_{2}\left\|e^{(t-1) \Delta}(w-\bar{w})\right\|_{L^{p}(\Omega)} \\
		& \leqslant c_{2} C_{1}\left(1+(t-1)^{-\frac{n}{2}\left(\frac{1}{q}-\frac{1}{p}\right)}\right) e^{-\lambda_{1}(t-1)}\|w-\bar{w}\|_{L^{q}(\Omega)} \\
		& \leqslant 4 c_{2} C_{1} e^{-\lambda_{1}(t-1)}\|w\|_{L^{q}(\Omega)}
	\end{aligned}
	$$
	for all $t \geqslant 2 .$ This together with (1.11) proves (1.5).
	(iii) Passing to $\hat{w}:=w-\frac{1}{12} \int_{\Omega} w$ if necessary, we may assume that $\int_{\Omega} w=0 .$ We first consider the case $t \geqslant 1$, in which we apply (ii), (i) and the Poincaré inequality to find
	$$
	\begin{aligned}
		\left\|\nabla e^{t \Delta} w\right\|_{L^{p}(\Omega)} & \leqslant 2 C_{2}\left\|e^{(t-1) \Delta} w\right\|_{L^{p}(\Omega)} \leqslant 4 C_{2} C_{1} e^{-\left(\lambda_{1}-1\right) t}\|w\|_{L^{p}(\Omega)} \\
		& \leqslant 4 C_{2} C_{1} c_{P} e^{-\left(\lambda_{1}-1\right) t}\|\nabla w\|_{L^{p}(\Omega)}
	\end{aligned}
	$$
	which yields (1.6) for all $t \geqslant 1$ and any $p \in(1, \infty),$ because, as can easily be verified, the Poincaré constant $c_{P}$ can be chosen independent of $p$ Next, for $p=2$, multiplying $\left(e^{t \Delta} w\right) t=\Delta e^{t \Delta} w$ by $-\Delta e^{t \Delta} w$ and integrating shows that
	$$
	\left\|\nabla e^{t \Delta} w\right\|_{L^{2}(\Omega)} \leqslant\|\nabla w\|_{L^{2}(\Omega)} \quad \text { for all } t \geqslant 0
	$$
	On the other hand, it is known [22, formula (2.39)] that for some $c_{3} \geqslant 1$,
	$$
	\left\|\nabla e^{t \Delta} w\right\|_{L^{\infty}(\Omega)} \leqslant c_{3}\|\nabla w\|_{L^{\infty}(\Omega)} \quad \text { for all } t \in(0,1)
	$$
	for each $w \in \hat{C}^{1}(\bar{\Omega}):=\left\{z \in C^{1}(\bar{\Omega}) \mid \frac{\partial z}{\partial v}=0\right.$ on $\left.\partial \Omega\right\} .$ A Marcinkiewicz interpolation as in
	(ii) now asserts that (1.6) is valid for each $p \in[2, \infty)$ and $t \in(0,1)$ and all $w \in \hat{C}^{1}(\bar{\Omega}),$ so that all that remains to be shown is that $\hat{C}^{1}(\bar{\Omega})$ is dense in $W^{1, p}(\Omega) .$ To sketch a possible way to see this, we let $w \in$ $W^{1, p}(\Omega)$ be given and fix $\varepsilon>0 .$ Then there exists $w_{1} \in C^{1}(\bar{\Omega})$ such that $\left\|w-w_{1}\right\|_{W^{1, p}(\Omega)}<\frac{\varepsilon}{2} .$ Given $x^{0} \in \partial \Omega,$ applying a shifting and local flattening procedure if necessary, we may assume that $x^{0}=0$ that $\Omega \subset\left\{x_{n}>0\right\}:=\left\{x=\left(x_{1}, \ldots, x_{n}\right) \in \mathbb{R}^{n} \mid x_{n}>0\right\}$ and that $\partial \Omega$ is a part of $\left\{x_{n}=0\right\}$ near $x^{0} .$ For
	near $x^{0},$ we define $w_{x^{0}}(x):=w_{1}\left(x_{1}, \ldots, x_{n-1}, 0\right)$ for $x \in \Omega,$ so that $w_{x^{0}}=w_{1}$ on $\partial \Omega$ and $\frac{\partial w_{x} 0}{\partial v}=0$
	on $\partial \Omega$ hold near $x^{0}$. The same is thus valid for $w_{x^{0}, k}(x):=w_{x^{0}}(x) \cdot\left(1-\chi\left(k x_{n}\right)\right)+w_{1}(x) \cdot \chi\left(k x_{n}\right)$, where $\chi \in C^{\infty}(\mathbb{R})$ satisfies $\chi_{[2, \infty)} \leqslant \chi \leqslant \chi_{[1, \infty)} .$ since $w_{1} \in C^{1}(\bar{\Omega}),$ it is easily checked that $w_{x^{0}, k} \rightarrow w_{1}$
	in $W^{1, p}$ in a neighborhood of $x^{0},$ so that returning to the original coordinates via a suitable partition of unity will provide some $w_{2} \in \hat{C}^{1}(\bar{\Omega})$ such that $\left\|w_{1}-w_{2}\right\|_{W^{1, p}(\Omega)}<\frac{\varepsilon}{2},$ which proves the claim.
	
	(iv) Given $\varphi \in C_{0}^{\infty}(\Omega),$ we use that $e^{t \Delta}$ is self-adjoint in $L^{2}(\Omega)$ and integrate by parts to see that
	$$
	\begin{aligned}
		\left|\int_{\Omega} e^{t \Delta} \nabla \cdot w \varphi\right| &=\left|-\int_{\Omega} w \cdot \nabla e^{t \Delta} \varphi\right| \leqslant\|w\|_{L^{q}(\Omega)} \cdot\left\|\nabla e^{t \Delta} \varphi\right\|_{L^{q^{\prime}}(\Omega)} \\
		& \leqslant C_{2}\left(1+t^{-\frac{1}{2}-\frac{\eta}{2}\left(\frac{1}{p^{\prime}}-\frac{1}{q^{\prime}}\right)}\right) e^{-\lambda_{1} t}\|w\|_{L^{q}(\Omega)}\|\varphi\|_{L^{p^{\prime}}(\Omega)}
	\end{aligned}
	$$
	for all $t>0$ holds in view of (ii), where $\frac{1}{p}+\frac{1}{p^{\prime}}=1$ and $\frac{1}{q}+\frac{1}{q^{\prime}}=1 .$ since $\frac{1}{p^{\prime}}-\frac{1}{q^{\prime}}=\frac{1}{q}-\frac{1}{p},$ taking the
	supremum over all such $\varphi$ satisfying $\|\varphi\|_{L^{\prime}(\Omega)} \leqslant 1$ we arrive at $(1.7) .$
\end{proof}
\section{uniform integrability}
\begin{theorem}[Dunford-Pettis theorem] 
A class of random variables $X_{n} \subset L^{1}(\mu)$ is uniformly integrable if and only if it is relatively compact for the weak topology $\sigma\left(L^{1}, L^{\infty}\right)$.
\end{theorem}
\begin{theorem}[de la Vallée-Poussin theorem] 
The family $\left\{X_{\alpha}\right\}_{\alpha \in \mathrm{A}} \subset L^{1}(\mu)$ is uniformly integrable if and only if there exists a non-negative increasing convex function $G(t)$ such that $\lim _{t \rightarrow \infty} \frac{G(t)}{t}=\infty$ and $\sup _{\alpha} \mathrm{E}\left(G\left(\left|X_{\alpha}\right|\right)\right)<\infty$
\end{theorem}
\section{Lp}

\begin{lemma}
	Let $\Omega \subset \mathbb{R}^{d}$ be a bounded domain and let $\left(u_{\varepsilon}\right)$ be bounded in $L^{p}(\Omega),$ where $1<p \leq \infty,$ such that $u_{\varepsilon} \rightarrow u$ a.e. in $\Omega$ as $\varepsilon \rightarrow 0 .$ Then, for $1 \leq q<p$
	$$
	u_{\varepsilon} \rightarrow u \text { strongly in } L^{q}(\Omega).
	$$
\end{lemma}
\begin{proof}
	set $M > 0$, such that $\|u_\varepsilon\|_p \leqslant M$. by Fatou's lemma,$u\in L^p(\Omega),$ as 
	\(\|u\|_p \leqslant \liminf\limits_{\varepsilon\rightarrow 0}\|u_\varepsilon\|_p \leqslant M\).
	by H\"{o}lder inequality, for $1 \leqslant q\leqslant p,$ we have \[\|u\|_q \leqslant |\Omega|^{\frac{p-q}{pq}} \|u\|_p,\]
	so $u,u_\varepsilon$ are in $L^q(\Omega)$.
	
	by Egoroff theorem, for any given $\tau > 0$, there exists a measurable set $\Omega_\tau$ with $|\Omega_\tau| < \tau$, such that $u_{\varepsilon_n}$ converges to $u$ uniformly on $\Omega\backslash \Omega_\tau$, while $\varepsilon_n \rightarrow 0$, as $n \rightarrow 0.$
	that is to say, for $\tau$ given above, there exists a positive integer $N$, such that for any $n > N$, we have $\sup\limits_{x\in\Omega\backslash\Omega_\tau}|u_{\varepsilon_n} - u| \leqslant \tau$. therefore
	\begin{equation*}
		\begin{split}
			\|u_{\varepsilon_n} - u\|_{L^q(\Omega)} &= \|u_{\varepsilon_n} - u\|_{L^q(\Omega\backslash\Omega_\tau)} + \|u_{\varepsilon_n} - u\|_{L^q(\Omega_\tau)}\\ 
			& \leqslant \tau|\Omega|^{\frac{1}{q}} + |\Omega_\tau|^{\frac{p-q}{pq}}\|u_{\varepsilon_n } - u\|_p\\
			& \leqslant \tau|\Omega|^{\frac{1}{q}} + 2M\tau^{\frac{p-q}{pq}},
		\end{split}
	\end{equation*}
we can conclude $u_{\varepsilon_n} \rightarrow u \text { strongly in } L^{q}(\Omega)$. the subscript $n$ can be droped by the common contradiction discussion.
\end{proof}

a simple example, consider $u_n(x)=n^\delta x^n\in L^2(0,1)$. clearly, $u_n\rightarrow0$ a.e. in $(0,1)$.
$u_n$ are bounded if $\delta \leqslant \dfrac{1}{2}.$

if $\delta = \dfrac{1}{2}$, $u_n\not\rightarrow0$ in $L^2(0,1)$; otherwise in $L^q(0,1),1\leqslant q < 2$.

if $\delta < \dfrac{1}{2}$, $u_n\rightarrow0$ in $L^q(0,1), 1\leqslant q \leqslant 2$.

\begin{lemma}
Let $\Omega$ be an open subset of $\mathcal{R}^n$, let $1<p<\infty$, and let functions $f_k\in L^p(\Omega)$, $k\geqslant1$, and $f\in L^p(\Omega)$ be such that the sequence $\{f_k\}_{k=1}^{\infty}$ is bounded in $L^p(\Omega)$ and $f_k$ converges almost everywhere in $\Omega$ to $f$ as $k\rightarrow \infty$. Show that 
\begin{equation}
	f_k\rightharpoonup f \text{ in } L^p(\Omega) \text{ as } k\rightarrow \infty.
\end{equation}	
\end{lemma}

\begin{proof}
	we only need to verfy that
	\begin{equation}
		\lim_{k\rightarrow\infty}\int_\Omega f_kg\rightarrow\int_\Omega fg \text{ for any } g\in L^q(\Omega) \text{ with } \frac1p+\frac1q=1.
	\end{equation}

	for one fixed $g\in L^q(\Omega)$, given $\varepsilon>0$, there exists a $R>0$ large enough such that 
	\begin{equation}
		\int_{\Omega\backslash B_R}|g|^q < (\frac\varepsilon2)^q.
	\end{equation}
therefore 
\begin{equation}
	\begin{split}
	|\int_\Omega(f_k-f)g|&\leqslant|\int_{\Omega\backslash B_R}(f_k-f)g| + |\int_{B_R}(f_k-f)g|\\
	&\leqslant 2M\frac\varepsilon2 + |\int_{B_R}(f_k-f)g|
\end{split}
\end{equation}
as $\{f_k\}_{k=1}^{\infty}$ is bounded in $L^p(\Omega)$. 
By Egoroff theorem, since $f_k$ converges almost everywhere in $\Omega$ to $f$ as $k\rightarrow \infty$, 
for any given $\tau>0$, there exists a measurable set $\Omega_\tau$ with $|\Omega_\tau|<\tau$, 
such that $f_n$ converges to $f$ uniformly on $B_R\backslash \Omega_\tau$. 
Furthermore, for $\varepsilon$ given above, there exists a $\tau>0$, 
such that for any $|\Omega_\tau|<\tau$, $\int_{\Omega_\tau}|g|^q<(\frac{\varepsilon}{2})^q.$ 
Consequently, there exists a large $K>0$, 
such that for any $k>K$, $|f_k-f|^p<\varepsilon/|B_R|$ for any $x\in B_R\backslash \Omega_\tau$, thus
\begin{equation}
	\begin{split}
		|\int_{B_R}(f_k-f)g|&\leqslant |\int_{B_R\backslash \Omega_\tau}(f_k-f)g| 
		+ |\int_{\Omega_\tau}(f_k-f)g|\\
		&\leqslant \|g\|_{L^q}\varepsilon +2M\frac{\varepsilon}{2}.
	\end{split}
\end{equation}
Now, we have for any $\varepsilon>0$, there exists a $K>0$ large enough such that 
\begin{equation}
	|\int_\Omega(f_k-f)g|\leqslant (2M+ \|g\|_{L^q})\varepsilon \text{ for all } k> K.
\end{equation}
\end{proof}


\section{a weak convergence lemma in \texorpdfstring{$L_1$}{} related to multiplication}

\begin{lemma}
denote $\mu$ Lebesgue measure, $\Omega$ a bounded domain in $\mathbb{R}^n$, abbreviate $L^1:= L^1(\Omega, \mu)$ and $L^\infty:= L^\infty(\Omega, \mu)$. if 
\begin{itemize}
\item $u_n\rightharpoonup u$ in $L^1$;
\item $u_nv_n\rightharpoonup z$ in $L^1$;
\item either $v_n\rightarrow v$ $\mu$-a.e.  or $v_n\stackrel{\mu}{\Longrightarrow}v$,
\end{itemize}
then $z = uv$ a.e..
\end{lemma}
\begin{proof}
for each $\Delta\subset\Omega$ is a measurable set, denote $\chi_\Delta$ set function, that is for each $x\in\Delta, \chi_\Delta(x) = 1$, otherwise $\chi_\Delta(x)=0$.

\textbf{Step 1.} supposed that $v\in L^\infty$,
if $v_n\rightarrow v$ a.e., by Egrov theorem, for each $\delta > 0$ there exist a 
measurable set $\Delta\subset\Omega$, such that $\mu(\Delta) < \delta$ and $v_n$ converges uniformly to $v$ in $\Delta^c$. Since weak convergence in $L^1$ implies uniformly integrable, that is for any $\varepsilon > 0$, there exists a $\delta>0$ and arbitrary measurable set $\Delta$ satisfied with $\mu(\Delta) < \delta$, we have $|(u_nv_n, \chi_\Delta)| < \varepsilon$ and $|(u_n, \chi_{\Delta})| < \varepsilon$.

for $\phi\in L^\infty$ and $|\phi|\leqslant1$, 
\begin{equation*}
\begin{split}
|(u_nv_n - uv, \phi)| & \leqslant |(u_nv_n - u_nv, \phi)| + |(u_nv - uv, \phi)|\\
&\leqslant |(u_nv_n - u_nv, \chi_\Delta\phi)| + |(u_nv_n - u_nv, \chi_{\Delta^c}\phi)| \\
&\quad + |(u_n - u, \phi v)|\\
&\leqslant |(u_nv_n, \chi_\Delta\phi)| + |(u_n, \chi_\Delta\phi v)| + |(u_nv_n - u_nv, \chi_{\Delta^c}\phi)| \\
&\quad+ |(u_n - u, \phi v)|\\
&\leqslant C\varepsilon \text{ as } n\rightarrow\infty.
\end{split}
\end{equation*}
Letting $\varepsilon\rightarrow0$, we have $u_nv_n\rightharpoonup uv$ in $L^1$ and hence 
$uv=z$ $\mu$-a.e..

or rather, if $v_n\stackrel{\mu}{\Longrightarrow}v$, denote $\Delta_\varepsilon:=\{|v_n - v| >\varepsilon\}$, then for each $\delta>0$, $ \varepsilon>0$, there exist $N$, such that  $$\mu(\Delta_\varepsilon) < \delta \text{ for each } n > N.$$
\begin{equation*}
\begin{split}
|(u_nv_n - uv, \phi)| & \leqslant |(u_nv_n - u_nv, \phi)| + |(u_nv - uv, \phi)|\\
&\leqslant |(u_nv_n - u_nv, \chi_{\Delta_\varepsilon}\phi)| + |(u_nv_n - u_nv, \chi_{\Delta_\varepsilon^c}\phi)| \\
&\quad + |(u_n - u, \phi v)|\\
&\leqslant |(u_nv_n, \chi_{\Delta_\varepsilon}\phi)| + |(u_n, \chi_{\Delta_\varepsilon}\phi v)| + |(u_nv_n - u_nv, \chi_{\Delta^c}\phi)| \\
&\quad+ |(u_n - u, \phi v)|\\
&\leqslant C\varepsilon \text{ for } n > N.
\end{split}
\end{equation*}
this implies $u_nv_n\rightharpoonup uv$ in $L^1$ and hence 
$uv=z$ $\mu$-a.e..

\textbf{Step 2.} let $\Omega_k:= \{|v|\leqslant k\}$ for $k > 0$. then $\mu(\Omega_k)\nearrow\mu(\Omega)$, $z=uv$ $\mu|_{\Omega_k}$-a.e. and hence $z=uv$ $\mu$-a.e.. 
\end{proof}

\begin{example}[stability of weak convergence or something]
if $u_\varepsilon\rightharpoonup u$ in $L^1$ and $u_\varepsilon\geqslant0$ in the sense of distribution, then $$\frac{u_\varepsilon}{1 + \varepsilon u_\varepsilon}\rightharpoonup u \text{ in } L^1.$$
\begin{proof}
since $\varepsilon u_\varepsilon\rightarrow 0$ in $L^1$, $$\frac{1}{1 + \varepsilon u_\varepsilon}\stackrel{\mu}{\Longrightarrow}1.$$
the desired result is a direct consequence of above lemma.
\end{proof}
\end{example}

\section{a counterpart about weak convergence}

if $f_n\to f$ in $L^p$, it is obvious that $|f_n|\to |f|$ by the triangle inequality.
But this is not true in the case of weak convergence.
Actually, writing $X:=L^2((0,1))$, putting 
\[
f_n = \sqrt2\sin(n\pi t),
\]
it is easy to check that 
\[
\|f_n\|_X = 1.
\]
Using Riemann-Lebesgue theorem,
we have 
\[
f_n \rightharpoonup 0\quad\text{in }X,
\]
but 
\[
|f_n| \not\rightharpoonup 0\quad \text{in }X,
\]
since 
\[
(|f_n|,1)_X = 2\sqrt2/\pi.
\]
Actually, putting $\phi\in C^0(\overline{\Omega})$, by uniformly continuity, 
for any $\varepsilon>0$, there exists $\delta>0$ such that for any $t', t''\in[0,1]$,
\[
|\phi(t')-\phi(t'')|<\varepsilon,\quad\text{provided } |t'-t''|<\delta.
\]
So we estimate for $n>N$ with $N\delta>1$,
\begin{align*}
(|f_n|,\phi)_X 
&= \int_0^1|f_n|\phi\dd t = \sum_{i=0}^{n-1}\int_{i/n}^{(i+1)/n}|f_n|\phi\dd t\\
&= \sum_{i=0}^{n-1}\int_0^{1/n} f_n(t)\phi(i/n+t)\dd t\\
&= \sum_{i=0}^{n-1}\int_0^{1/n} f_n(t)\phi(i/n)\dd t 
	+ \sum_{i=0}^{n-1}\int_0^{1/n}f_n(t)o (\varepsilon)\\
&= \|f_n\|_1\frac1n\sum_{i=1}^{n-1}\phi(i/n) + \|f_n\|_1o(\varepsilon)\\
&\to (|f_1|,1)_X (\phi,1)_X + o(\varepsilon),\quad n\to\infty.
\end{align*}
That is $|f_n|\rightharpoonup (|f_1|,1)_X$ in $X$, 
by the density of $C^0(\overline\Omega)\hookrightarrow X$.

\section{a compact property of unbounded domain}
\begin{lemma}
	$V:=\{f\in H^1(\mathbb{R}^n): \int_{\mathbb R^n}|x|f^2\dd x<\infty\}$ 
	is relatively compact in $L^2(\mathbb{R}^n)$.
\end{lemma}

\begin{proof}
	Let $S$ be a bounded subset of $V$.
	Then for each $\varepsilon>0$, there exists $R>0$ such that 
	for any $f\in S$,
	\[
	\int_{\mathbb{R}^n\setminus B_R}f^2 \leq \frac1R\int_{\mathbb{R}^n\setminus B_R}|x|f^2<\varepsilon.
	\]
	Since $S$ is relatively compact in $L^2(B_R)$, 
	for each $\varepsilon$ given as above,
	there exists a finite set $F_\varepsilon = \{f_1, f_2, \cdot, f_k\}\subset S$ such that 
	for any $f\in S$, 
	\[
	\min_{f_k\in F_\varepsilon}\|f-f_k\|_{L^2(B_R)}< \varepsilon.
	\] 
	Therefore,
	\[
	\min_{f_k\in F_\varepsilon}\|f-f_k\|_{L^{2}(\mathbb R^n)}
	\leq \min_{f_k\in F_\varepsilon}\|f-f_k\|_{L^{2}(B_R)}
		+ \sup_{f_k\in F_\varepsilon}\|f-f_k\|_{L^{2}(\mathbb R^n\setminus B_R)} 
		< 3\varepsilon,
	\]
	which implies $S$ is relatively compact in $L^2(\mathbb R^n)$.
\end{proof}

\section{The Kolmogorov–Riesz theorem}
\begin{theorem}[Kolmogorov-Riesz]
	 Let $1 \leq p<\infty .$ A subset $\mathcal{F}$ of $L^{p}\left(\mathbb{R}^{n}\right)$ is totally bounded if, and only if,
	 \begin{itemize}
	 	\item $\mathcal{F}$ is bounded,
	 	\item for every $\varepsilon>0$ there is some $R$ so that, for every $f \in \mathcal{F}$
	 	$$
	 	\int_{|x|>R}|f(x)|^{p} d x<\varepsilon^{p}
	 	$$
	 	\item for every $\varepsilon>0$ there is some $\rho>0$ so that, for every $f \in \mathcal{F}$ and $y \in \mathbb{R}^{n}$ with $|y|<\rho$
	 	$$
	 	\int_{\mathbb{R}^{n}}|f(x+y)-f(x)|^{p} d x<\varepsilon^{p}
	 	$$	
	 \end{itemize}
\end{theorem}

\begin{proof}
	sufficient. 
	see \cite{Hanche-Olsen2010}.
	necessary.
	for every $\varepsilon>0$, 
	there exists $\{f_\varepsilon^1, f_\varepsilon^2, \cdots, f_\varepsilon^k\}\subset\mathcal{F}$ such that 
	for any $f\in\mathcal{F}$, 
	there exists some $f_\varepsilon^i$, $1\leq i \leq k$ such that 
	\[
	\|f-f_\varepsilon^i\|_p < \varepsilon.
	\]
	Let $\varepsilon=1$, then $\|f\|_p\leq1+\sup_i\|f_1^i\|_p$.
	For any fixed $f_\varepsilon^i$, there $R_\varepsilon^i$ and $\rho_\varepsilon^i$ such that the last two assertations hold, respectively. An application of the triangle inequality extends the conclusion of $f_\varepsilon^i$ to any $f\in\mathcal{F}$.
\end{proof}
\section{vitali convergence theorem}

\begin{theorem}
Let $\left(f_{n}\right)_{n \in \mathbb{N}} \subseteq L^{p}(X, \tau, \mu), f \in L^{p}(X, \tau, \mu),$ with $1 \leq p<\infty .$ Then, $f_{n} \rightarrow f$ in $L^{p}$ if and only if we have
\begin{itemize}
	\item $f_{n}$ converge in measure to $f$.
	\item For every $\varepsilon>0$ there exists a measurable set $E_{\varepsilon}$ with $\mu\left(E_{\varepsilon}\right)<\infty$ such that for every $G \in \tau$ disjoint from $E_{\varepsilon}$ we have, for every $n \in \mathbb{N}$, $\int_{G}\left|f_{n}\right|^{p} d \mu<\varepsilon^{p}$
	\item  For every $\varepsilon>0$ there exists $\delta(\varepsilon)>0$ such that, if $E \in \tau$ and $\mu(E)<\delta(\varepsilon)$ then, for every $n \in \mathbb{N}$ we have $\int_{E}\left|f_{n}\right|^{p} d \mu<\varepsilon^{p}$	
\end{itemize}
\end{theorem}

\begin{proof}
	sufficient: as $ f\in L^p(X, \tau, \mu) $,
	
	as $ f_n \rightarrow f $ in $ L^p $, for any $ m\in N^\star $, we have $$\frac{1}{m^p} \mu(\{|f_n-f|>\frac{1}{m}\}) \leqslant \|f_n-f\|_{L^p}^p\rightarrow 0, n\rightarrow\infty,$$
	which entails $ f_n $ converge in measure to $ f $.
	
	denote $ B(r) = \{|x|\leqslant r\} $, then for any $ \varepsilon>0 $, there exists a $R_0>0$ such that \[ \|f\|_{L^p(X\backslash B(R_0))} < \frac{\varepsilon}{2},\]
	meanwhile, there exists a positive interger $ N $, such that for any $ n>N $, we have 
	\[ \|f_n-f\|_{L^p(X\backslash B(R_0))} < \frac{\varepsilon}{2};\]
	there exist $ R_1, R_2, \cdots, R_N > 0 $, such that 
	\[ \|f_i\|_{L^p(X\backslash B(R_i))} < \frac{\varepsilon}{2}, i = 1, 2, \cdots, N.\]
	now, we can conclude that for any $ \varepsilon>0 $, there exists a measure set $ E_\varepsilon = B(R)$ with $ R = \max\limits_{i=0}^N R_i,\mu(E_\varepsilon)<\infty $ such that for every $ G\in\tau $ disjoint from $ E_\varepsilon $, we have, for every $ n\in \mathbb{N} $, $ \|f_n\|_{L^p(G)}<\varepsilon $.
	
	the third one can be varified by the similar manner used in the above. by the absolute continuty of lebesgue integral, we assert that for any $ \varepsilon>0, $ there exists $ \delta_0>0 $ such that, if $ \mu(E)<\delta_0, $ we have $ \|f\|_{L^p(E)} < \dfrac\varepsilon2; $ there exists a positive integer $ N $ such that, for any $ n>N $, we have $ \|f_n-f\|_{L^p(E)} < \dfrac{\varepsilon}{2};$ there exist $ \delta_1 > 0  $ such that, if $ \mu(E)< \delta_1, $ 
	\[ \|f_i\|_{L^p(E)} < \frac{\varepsilon}{2}, i = 1, 2, \cdots, N.\]
	now, we can conclude that for  every $\varepsilon>0$ there exists $\delta=\min\{\delta_0, \delta_1\}>0$ such that, if $E \in \tau$ and $\mu(E)<\delta$ then, for every $n \in \mathbb{N}$ we have $\int_{E}\left|f_{n}\right|^{p} d \mu<\varepsilon^{p}$	
	
	necessary: for any $ \varepsilon>0 $, there exist $ \delta(\varepsilon)>0,$ a measurable set $ E_\varepsilon $ with $ \mu(E_\varepsilon)<\infty, $ $N\in\mathbb{N}$ such that, 
	\[ \mu(X_n) < \delta(\varepsilon) \text{for any } n > N \text{ with } X_n:=\{|f_n-f|>\dfrac{\varepsilon}{\mu(E_\varepsilon)^{1+\frac{1}{p}}}\},\]
	thus we have
	\begin{equation*}
		\begin{split}
			\|f_n-f\|_{L^p(X)} &= \|f_n-f\|_{L^p(E_\varepsilon\cap X)} + \|f_n-f\|_{X\backslash L^p(E_\varepsilon)}\\
			&= \|f_n-f\|_{L^p(E_\varepsilon\cap X_n)} + \|f_n-f\|_{L^p(E_\varepsilon\backslash X_n)} +
			\|f_n-f\|_{X\backslash L^p(E_\varepsilon)}\\
			&< 3\varepsilon, \text{for any } n>N,
		\end{split}
	\end{equation*}
\end{proof}

\section{\texorpdfstring{$W^{1,1}$}{W1,1} embeds into \texorpdfstring{$L^\infty$}{bounded space}}
\begin{lemma}
	If $f\in W^{1,1}(\mathbb{R})$, then $|f|\leq \|f\|_{W^{1,1}}$.
	Particularly, $\lim_{\xi\to\infty}f(\xi) = 0$.
\end{lemma}
\begin{proof}
	By density, without loss of generality,
	we may assume that $f\in C^1(\mathbb{R})$, then
	\[
	f(b) = f(a) + \int_a^b f'(\xi)\dd \xi,\quad (a,b)\subset(a, a+1)\subset\mathbb{R},
	\]
	and hence
	\[
	f(b) \leq f(a) + \int_a^{a+1}|f'(\xi)|\dd\xi,
	\]
	and integrating over $(a, a+1)$,
	\begin{align*}
		|f(b)| &\leq \int_a^{a+1}f(\xi)\dd\xi + \int_a^{a+2}|f'(\xi)|\dd\xi,
	\end{align*}
	which implies the desired conclusion.
\end{proof}

\section{Admissible Relative Entropies and Their Generating
	Functions}

\begin{definition}
	 Let $J$ be either $\mathbb{R}$ or $\mathbb{R}^{+}:=(0, \infty) .$ Let $\psi \in C(\bar{J}) \cap C^{4}(J)$ satisfy the conditions
	$$
	\begin{array}{l}
	\psi(1)=0 \\
	\psi^{\prime \prime} \geq 0, \quad \psi^{\prime \prime} \neq 0 \quad \text { on } J \\
	\left(\psi^{\prime \prime \prime}\right)^{2} \leq \frac{1}{2} \psi^{\prime \prime} \psi^{I V} \quad \text { on } J \quad (\text{or } (\dfrac1{\varphi''})''\leqslant0 \quad \text { on } J)
	\end{array}
	$$
	Let $\rho_{1} \in L^{1}\left(\mathbb{R}^{n}\right), \rho_{2} \in L_{+}^{1}\left(\mathbb{R}^{n}\right)$ with $\int \rho_{1} d x=\int \rho_{2} d x=1$ and $\rho_{1} / \rho_{2} \in$
	$\bar{J} \rho_{2}(d x)-$ a.e. Then
	$$
	e_{\psi}\left(\rho_{1} \mid \rho_{2}\right):=\int_{\mathbb{R}^{n}} \psi\left(\frac{\rho_{1}}{\rho_{2}}\right) \rho_{2}(d x)
	$$
	is called an admissible relative entropy (of $\rho_{1}$ with respect to $\rho_{2}$ ) with generating function $\psi$.
\end{definition}

if \(\varphi\) generates a admissible relative entropy, then its normalization $\tilde{\psi}(\sigma)=\psi(\sigma)-\psi^{\prime}(1)(\sigma-1)$ generates the same admissible relative entropy.

\begin{lemma}\label{entropy_on_R}
	For $J=\mathbb{R}$ all admissible entropies are generated by $\varphi(\sigma)=\alpha(\sigma-1)^{2}, \quad \sigma \geq 0 ; \alpha>0 .$
\end{lemma}

\begin{lemma}
 Let $\psi$ generate an admissible relative entropy with $J=\mathbb{R}^{+} .$ Then there exists a logarithmic-type function 
 \begin{equation}\label{logarithmic-type-entropy-generator}
 \chi(\sigma)=\alpha(\sigma+\beta) \ln \frac{\sigma+\beta}{1+\beta}-\alpha(\sigma-1), \quad \sigma>0 ; \alpha>0, \beta \geq 0\end{equation}
  and a quadratic function
  \begin{equation}\varphi(\sigma)=\alpha(\sigma-1)^{2}, \quad \sigma \geq 0 ; \alpha>0\end{equation}
   such that
$$
\chi(\sigma) \leq \psi(\sigma) \leq \varphi(\sigma), \quad \sigma \in J
$$
and hence
$$
0 \leq e_{\chi}\left(\rho_{1} \mid \rho_{2}\right) \leq e_{\psi}\left(\rho_{1} \mid \rho_{2}\right) \leq e_{\varphi}\left(\rho_{1} \mid \rho_{2}\right)
$$
\(\chi\) and $\varphi$ both satisfy the conditions of admissible relative entropy and thus generate, respectively, an admissible sub- and superentropy for $e_{\psi}$	
\end{lemma}

\begin{proof}
	since $J=\mathbb{R}^{+},$ the function $g$ from the proof of Lemma \ref{entropy_on_R} satisfies
	\begin{equation}\label{gproperty}
	g>0, \quad g^{\prime} \geq 0, \quad g^{\prime \prime} \leq 0 \quad \text { on } J	
	\end{equation}

	
	Now denote the derivatives of the given function $\psi$ by
\begin{equation}\label{varphiproperty}
	\psi(1)=0, \psi^{\prime}(1)=0, \psi^{\prime \prime}(1)=: \mu_{2}>0, \psi^{\prime \prime \prime}(1)=: \mu_{3} \leq 0
\end{equation}
	From \eqref{gproperty} we readily get the estimate
	$$
	\left.\begin{array}{ll}
	\sigma \mu_{2}^{-1}, & 0<\sigma<1 \\
	\mu_{2}^{-1}, & \sigma>1
	\end{array}\right\} \leq g(\sigma) \leq \gamma \sigma+\delta, \sigma>0
	$$
	with $\gamma:=-\mu_{3} \mu_{2}^{-2} \geq 0, \delta:=\left(\mu_{2}+\mu_{3}\right) \mu_{2}^{-2} \geq 0 .$ Integrating the corresponding
	estimate for $\psi^{\prime \prime}=\frac{1}{g}$
	$$
	(\gamma \sigma+\delta)^{-1} \leq \psi^{\prime \prime}(\sigma) \leq\left\{\begin{array}{ll}
	\mu_{2} / \sigma, & 0<\sigma<1 \\
	\mu_{2}, & \sigma>1
	\end{array}\right.
	$$
	we obtain with \eqref{varphiproperty} the upper bound for $\psi:$
\begin{equation}\label{varphiprimeprimeproperty}
	\psi(\sigma) \leq\left\{\begin{array}{cl}
\mu_{2}(\sigma \ln \sigma-\sigma+1), & 0<\sigma<1 \\
\frac{\mu_{2}}{2}(\sigma-1)^{2}, & \sigma>1
\end{array}\right\}
\end{equation}
	$$
	\leq \mu_{2}(\sigma-1)^{2}=: \varphi(\sigma), \quad \sigma>0
	$$
	To derive the lower bound of $\psi$ one integrates \eqref{varphiprimeprimeproperty} twice to show $\chi(\sigma) \leq$ $\psi(\sigma) .$ For $\gamma>0$ the function $\chi(\sigma)$ is given by \eqref{logarithmic-type-entropy-generator} with $\alpha=\frac{1}{\gamma}, \beta=\frac{\delta}{\gamma}$
	If $\gamma=0$ we set
	$$
	\chi(\sigma)=\frac{\mu_{2}}{2}(\sigma-1)^{2}.
	$$
\end{proof}

\begin{theorem}[Csiszár-Kullback inequality]
	Let $\psi:(0, \infty) \rightarrow \mathbb{R}$ be bounded below, convex on $(0, \infty)$ strictly convex at 1 with $\psi(1)=0 .$ Then there exists a function $\mathrm{W}_{\psi}: \mathbb{R} \rightarrow[0, \infty)$
	such that
	1. W $_{\psi}$ is increasing.
	2. $\mathbf{W}_{\psi}(0)=0$
	3. W $_{\psi}$ is continuous at 0.
	4. For all non-negative $u \in L^{1}(\Omega, \mathscr{S}, \mu)$ (where $(\Omega, \mathscr{S}, \mu)$ is a probability space $)$ with $\int_{\Omega} u d \mu=1$ the Csiszár-Kullback inequality holds:
	$$
	\|u-1\|_{L^{1}(d \mu)} \leqslant \mathbf{W}_{\psi}\left(e_{\psi}(u)\right)
	$$
	where $e_{\psi}(u):=\int_{\Omega} \psi(u) d \mu$ is the entropy of u (relative to the state 1) generated by the function $\psi$.
\end{theorem}

For the generating function $\psi(\sigma)=\sigma \ln \sigma-\sigma+1,$ e.g.,
$\mathrm{W}_{\psi}$ can be chosen such that
$$
\mathrm{W}_{\psi}(c) \leqslant \sqrt{2 c}
$$
that is 
\begin{equation}
\frac{1}{2}\left\|\rho_{1}-\rho_{2}\right\|_{L^{1}\left(\mathbb{R}^{n}\right)}^{2} \leq e\left(\rho_{1} \mid \rho_{2}\right)=\int_{\mathbb{R}^n}\ln\dfrac{\rho_1}{\rho_2}\rho_{1}\mathrm{d}x.
\end{equation}

notice that $u\ln u \geqslant u - 1 + \dfrac{1}{2}(u-1)^2, \forall u\in(0,1),$ 
\begin{equation}
\begin{split}
\left\|\rho_{1}-\rho_{2}\right\|_{L^{1}\left(\mathbb{R}^{n}\right)}^{2} &=
4 (\int_{\rho_{1}<\rho_{2}}\left|\rho_{1}-\rho_{2}\right| d x)^2\\
&= 4 (\int_{\rho_{1}<\rho_{2}}\left|\frac{\rho_{1}}{\rho_{2}}-1\right|\rho_{2} d x)^2\\
&\leqslant
4 \int_{\rho_{1}<\rho_{2}}\left|\frac{\rho_{1}}{\rho_{2}}-1\right|^2\rho_{2} d x\int_{\rho_{1}<\rho_{2}}\rho_{2}\mathrm{d}x\\
&\leqslant 4 \int_{\rho_{1}<\rho_{2}}\left|\frac{\rho_{1}}{\rho_{2}}-1\right|^2\rho_{2} d x\\
&\leqslant 2 \int_{\rho_{1}<\rho_{2}}(\dfrac{\rho_{1}}{\rho_{2}} \ln \dfrac{\rho_{1}}{\rho_{2}} - \dfrac{\rho_{1}}{\rho_{2}} + 1) \rho_{2} \mathrm{d}x\\
&\leqslant 2 \int_{\mathbb{R}^n} (\dfrac{\rho_{1}}{\rho_{2}} \ln \dfrac{\rho_{1}}{\rho_{2}} - \dfrac{\rho_{1}}{\rho_{2}} + 1) \rho_{2} \mathrm{d}x\\
&= 2 \int_{\mathbb{R}^n} \rho_{1} \ln \dfrac{\rho_{1}}{\rho_{2}} \mathrm{d}x\\
\end{split}
\end{equation}

\section{convex sobolev inequation}
The probably easiest equation for which the Bakry-Émery method leads to non-trivial results is the heat equation
\begin{equation}\label{heat-equation}
	\partial_{t} u=u_{x x}, \quad u(0 ; x)=\hat{u}(x)
\end{equation}

on the interval [0,1] with homogeneous Neumann boundary conditions.

\begin{theorem}
	 Assume that $\phi: \mathbb{R}_{\geq 0} \rightarrow \mathbb{R}_{\geq 0}$ is convex and s.t. $\left(\phi^{\prime \prime}\right)^{-1 / 2}$ is concave, and let $\psi$ be such that
	$$
	\psi^{\prime}(s)^{2}=\phi^{\prime \prime}(s)
	$$
	Then the following convex Sobolev inequality \eqref{convexsobolevinequation}
\begin{equation}\label{convexsobolevinequation}
	\int_{0}^{1} \phi(\hat{u}) d x-\phi\left(\int_{0}^{1} \hat{u} d x\right) \leq \frac{1}{2 \pi^{2}} \int_{0}^{1} \psi(\hat{u})_{x}^{2} d x
\end{equation}
	holds for all smooth, positive functions $\hat{u}$ on $[0,1] .$
\end{theorem}

\begin{proof}
	Let us start from the special case of \eqref{convexsobolevinequation} with $\phi(s)=\frac{1}{2} s^{2}$ and $\psi(s)=s$
	$$
	\int_{0}^{1} \hat{u}^{2} d x-\left(\int_{0}^{1} \hat{u} d x\right)^{2} \leq \frac{1}{\pi^{2}} \int_{0}^{1} \hat{u}_{x}^{2} d x
	$$
	This is the Poincaré inequality which we shall not prove again. Instead, we shall now generalize it to other convex functions $\phi: \mathbb{R}_{\geq 0} \rightarrow \mathbb{R}_{\geq 0}$
	
	Define the l.h.s of \eqref{heat-equation} as $H_{\phi}[u],$ and let $u$ be the unique solution to \eqref{heat-equation} . One finds
	$$
	D_{\phi}[u]:=-\frac{d}{d t} H_{\phi}[u]=-\int \phi^{\prime}(u) u_{t} d x=-\int \phi^{\prime}(u) u_{x x} d x=\int \phi^{\prime \prime}(u) u_{x}^{2} d x=\int \psi(u)_{x}^{2} d x
	$$
	as first time derivative and
	$$
	\begin{aligned}
		R_{\phi}[u] &:=-\frac{1}{2} \frac{d}{d t} D_{\phi}[u]=-\int \psi(u)_{x} \psi(u)_{x t} d x=\int \psi(u)_{x x} \psi^{\prime}(u) u_{x x} d x \\
		&=\int \psi^{\prime}(u)^{2} u_{x x}^{2} d x+\int \psi^{\prime \prime}(u) \psi^{\prime}(u) u_{x}^{2} u_{x x} d x
	\end{aligned}
	$$
	as second. The crucial step is to relate $R_{\phi}[u]$ to $D_{\phi}[u] .$ To this end, the expression
	$$
	\begin{aligned}
		0 &=\frac{1}{3} \int\left(\psi^{\prime}(u) \psi^{\prime \prime}(u) u_{x}^{3}\right)_{x} d x \\
		&=\int \psi^{\prime \prime}(u) \psi^{\prime}(u) u_{x}^{2} u_{x x} d x+\frac{1}{3} \int\left(\psi^{\prime \prime}(u)^{2}+\psi^{\prime}(u) \psi^{\prime \prime \prime}(u)\right) u_{x}^{4} d x
	\end{aligned}
	$$
	is added to $R_{\phi}[u],$ obviously without changing the value of the latter. Hence
	$$
	R_{\phi}[u]=\int \psi^{\prime}(u)^{2} u_{x x}^{2} d x+2 \int \psi^{\prime \prime}(u) \psi^{\prime}(u) u_{x x} u_{x}^{2} d x+\frac{1}{3} \int\left(\psi^{\prime \prime}(u)^{2}+\psi^{\prime}(u) \psi^{\prime \prime \prime}(u)\right) u_{x}^{4} d x
	$$
	On the other hand, one has
	$$
	0 \leq(\psi(u))_{x x}^{2}=\psi^{\prime}(u)^{2} u_{x x}^{2}+2 \psi^{\prime \prime}(u) \psi^{\prime}(u) u_{x x} u_{x}^{2}+\psi^{\prime \prime}(u)^{2} u_{x}^{4}
	$$
	In combination with Poincaré's inquality, one concludes
	\begin{equation}\label{entropy-production-prime-entropy-production-inequation}
	D_{\phi}[u] \leq \frac{1}{\pi^{2}} R_{\phi}[u]
	\end{equation}
	provided that
\begin{equation}\label{psi-prime-reverse-concavity}
		\frac{1}{3}\left(\left(\psi^{\prime \prime}\right)^{2}+\psi^{\prime} \psi^{\prime \prime \prime}\right) \geq\left(\psi^{\prime \prime}\right)^{2}
\end{equation}
	since $\psi^{\prime}=\left(\phi^{\prime \prime}\right)^{1 / 2}>0,$ it is easy to see that \eqref{psi-prime-reverse-concavity} is equivalent to the concavity of $\left(\psi^{\prime}\right)^{-1}=$ $\left(\phi^{\prime \prime}\right)^{-1 / 2}$
	To finish the argument, rewrite \eqref{entropy-production-prime-entropy-production-inequation} as
	$$
	-\frac{d}{d t} H_{\phi}[u(t)] \leq-\frac{1}{2 \pi^{2}} \frac{d}{d t} D_{\phi}[u(t)]
	$$
	and integrate both sides from $t=+\infty$ to $t=0 .$ This yields
\begin{equation}\label{entropy-entropy-production-inequation}
		H_{\phi}\left[u_{0}\right]-\lim _{t \rightarrow+\infty} H_{\phi}[u(t)] \leq \frac{1}{2 \pi^{2}}\left(D_{\phi}\left[u_{0}\right]-\lim _{t \rightarrow \infty} D_{\phi}[u(t)]\right)
\end{equation}
	By standard theory, the solution $u(t)$ to \eqref{heat-equation} converges to the homogeneous steady state $u_{\infty} \equiv$ $\int_{0}^{1} \hat{u}(x) d x$ in $C^{\infty},$ implying that $D_{\phi}[u(t)] \rightarrow 0$ and $H_{\phi}[u(t)] \rightarrow H_{\phi}\left[u_{\infty}\right]$ as $t \rightarrow \infty .$ Substituting
	these limits, \eqref{entropy-entropy-production-inequation} becomes \eqref{convexsobolevinequation} . $\square$

\end{proof}
	For example, $\phi(s)=s \log s$ with $\psi(s)=2 s^{1 / 2}$ is a possible choice, leading to a logarithmic Sobolev inequality,
\begin{equation}
	0 \leq \int_{0}^{1} \hat{u} \log \hat{u} d x-\left(\int_{0}^{1} \hat{u} d x\right) \log \left(\int_{0}^{1} \hat{u} d x\right) \leq \frac{2}{\pi^{2}} \int_{0}^{1} \sqrt{\hat{u}}_{x}^{2} d x\label{logarithmic-sobolev-inequation}
\end{equation}


Moreover, the pairs $\phi(s)=s^{\alpha} /(\alpha-1)$ and $\psi(s)=2 s^{\alpha / 2} / \sqrt{\alpha}$ are also allowed, when $1<\alpha<2$. These yield Beckner's interpolation inequalities,
\begin{equation}\label{beckner's-interpolation-inequation}
	0 \leq \frac{1}{\alpha-1}\left[\int_{0}^{1} \hat{u}^{\alpha} d x-\left(\int_{0}^{1} \hat{u} d x\right)^{\alpha}\right] \leq \frac{2}{\alpha \pi^{2}} \int_{0}^{1}\left(\sqrt{\hat{u}^{\alpha}}\right)_{x}^{2} d x
\end{equation}
Notice that from \eqref{beckner's-interpolation-inequation}, one obtains both the Poincare inequality for $\alpha \nearrow 2,$ as well as the logarithmic Sobolev inequality \eqref{logarithmic-sobolev-inequation} for $\alpha \searrow 1$ as limit cases.

\section{operator semigroup}

\begin{definition}
	
 We call a linear operator A in a Banach space $X$ a sectorial operator if it is a closed densely defined operator such that, for some $\phi $ in $(0, \pi / 2)$ and some $M \geq 1$ and real $a$, the
	Sector
	$$
	S_{a, \phi}=\{\lambda|\phi \leq| \arg (\lambda-a) \mid \leq \pi, \quad \lambda \neq a\}
	$$
	is in the resolvent set of $A$ and
	$$
	(\lambda-A)^{-1} \| \leq M /|\lambda-a| \quad \text { for all } \quad \lambda \in S_{a, \phi}
	$$
\end{definition}
\begin{definition}
	
	A \textbf{strongly continuous semigroup} on a Banach space $X$ is a map $T: \mathbb{R}_{+} \rightarrow L(X)$ such that
	\begin{itemize}
		\item [1.] $T(0)=I,$ (identity operator on $X$ )
		\item [2.] $\forall t, s \geq 0: T(t+s)=T(t) T(s)$
		\item [3.] $\forall x_{0} \in X:\left\|T(t) x_{0}-x_{0}\right\| \rightarrow 0,$ as $t \downarrow 0$
	\end{itemize}
	The first two axioms are algebraic, and state that $T$ is a representation of the semigroup ( $\left.\mathbb{R}_{+},+\right)$; the last is topological, and states that the map $T$ is continuous in the strong operator topology.

\end{definition}

The infinitesimal generator $A$ of a strongly continuous semigroup $T$ is defined by
$$
A x=\lim _{t \downarrow 0} \frac{1}{t}(T(t)-I) x
$$
whenever the limit exists. The domain of $A, D(A)$, is the set of $x \in X$ for which this limit does exist; $D(A)$ is a linear subspace and $A$ is linear on this domain.  The operator $A$ is closed, although not necessarily bounded, and the domain is dense in $X.$

The strongly continuous semigroup $T$ with generator $A$ is often denoted by the symbol $e^{A t}$. This notation is compatible with the notation for matrix exponentials, and for functions of an operator defined via functional calculus (for example, via the spectral theorem).




\begin{proposition}
	if $T$ is a strongly continuous semigroup with generator $A$, we have 
	
	\begin{itemize}
		\item [I] $A$ is closed;
		\item [II] $D(A)$ is densely in $X$; 
		\item [III] if $x\in D(A)$, then $AT(t)x = T(t)Ax;$
		\item [IV] if $x\in D(A)$, then $A\int_0^tT(s)x\mathrm{d}s=\int_0^tAT(s)x\mathrm{d}s= T(t)x - x;$
	\end{itemize}
\end{proposition}


\begin{proof}
	\begin{itemize}
		\item [I] if $x_n\in D(A), y\in X$ such that $Ax_n\rightarrow y$ and $x_n\rightarrow x,$ we have
	$$|\frac{1}{t}(T(t)-I)x-y| \leqslant |\frac{1}{t}(T(t)-I)(x - x_n)| + |\frac{1}{t}(T(t)-I)x_n-\dfrac{1}{t}\int_0^tT(s)y\mathrm{d}s| + |\dfrac{1}{t}\int_0^tT(s)y\mathrm{d}s-y|$$
	and $$\frac{1}{t}(T(t)-I) x_n = \dfrac{1}{t}\int_0^tT(s)Ax_n\mathrm{d}s\rightarrow \dfrac{1}{t}\int_0^tT(s)y\mathrm{d}s$$
	
	\item [II] select $x\in X$, define \begin{equation*}
		x_\lambda = \dfrac{1}{\lambda}\int_0^\lambda T(t)x\mathrm{d}t,
	\end{equation*}
obviously, $x_\lambda\rightarrow x,$ as $\lambda\rightarrow0.$ noticing that,
\begin{equation}
	\begin{split}
		\lim\limits_{t\downarrow0}\dfrac{1}{t}(T(t)-I)x_\lambda &= \lim\limits_{t\downarrow0}\dfrac{1}{t\lambda}\int_0^\lambda [T(t+s)-T(s)]x\mathrm{d}s\\
		&= \lim\limits_{t\downarrow0}\dfrac{1}{t\lambda}(\int_\lambda^{\lambda + t}T(s)x\mathrm{d}s -
		\int_0^t T(s)x\mathrm{d}s)\\
		&= \lim\limits_{t\downarrow0}\dfrac{T(\lambda)- I}{t\lambda}\int_0^t T(s)x\mathrm{d}s\\
		&= \dfrac{T(\lambda)x- x}\lambda,\\
	\end{split}
\end{equation}
thus $Ax_\lambda = \dfrac{T(\lambda)x- x}\lambda,$ $x_\lambda \in D(A).$
\item [III] \begin{equation}
	AT(t)x = \lim\limits_{s\downarrow0}\dfrac{T(s)T(t)x - T(t)x}{s}= T(t)\lim\limits_{s\downarrow0}\dfrac{T(s)x - x}{s} = T(t)Ax.
\end{equation}
\item [IV] from II, we have $\lambda Ax_\lambda = T(\lambda)x - x,$
\end{itemize}
\end{proof}

\begin{theorem}
	If $A$ is a sectorial operator, then $-A$ is the in finitesimal generator of an analytic semigroup $\left\{e^{-t A}\right\}_{t \geq 0},$ where
	$$
	e^{-A t}=\frac{1}{2 \pi i} \int_{\Gamma}(\lambda+A)^{-1} e^{\lambda t} d \lambda
	$$
	where $\Gamma$ is a contour in $\rho(-\mathrm{A}) \quad$ with arg $\lambda \rightarrow \pm \theta$ as $|\lambda| \rightarrow \infty$ for
	some $\theta \quad$ in $\quad(\pi / 2, \pi)$
	Further e $^{-A t}$ can be continued analytically into a sector
	$\{t \neq 0:|\arg t|<\varepsilon\} \quad$ containing the positive real axis, and if
	$\operatorname{Re} \sigma(\mathrm{A})>\mathrm{a}, \quad$ i. e. $\quad$ if $\quad \operatorname{Re} \lambda>\mathrm{a} \quad$ when ever $\quad \lambda \in \sigma(\mathrm{A}),$ then for $t>0$
	$$
	\left\|e^{-A t}\right\| \leq C e^{-a t},\left\|A e^{-A t}\right\| \leq \frac{C}{t} e^{-a t}
	$$
	for some constant $C$.
	
	Finally $\frac{d}{d t} e^{-A t}=-A e^{-A t}$ for $t>0$
	\end{theorem}

\begin{proof}
\begin{equation}
	\|e^{-At}\| \leqslant \frac{1}{2\pi}\int_\Gamma\frac{M}{|\lambda - a|}e^{Re \lambda t}\mathrm{d}\lambda\\
\end{equation}
\end{proof}

Let us regard $\gamma-\Delta$ as a closed operator in a Banach space. So define the
$$
A_{p}=\gamma-\Delta, \quad D\left(A_{p}\right)=\left\{u \in W^{2, p}(\Omega) ; \frac{\partial u}{\partial n}=0 \quad \text { on } \partial \Omega\right\}
$$

The operator $A_{p}$ is sectorial in $L^{p}(\Omega)$ and $\sigma(A) \subset\left\{z \in C ; \mathscr{R}(z)>\gamma_{0}\right\}$ for a positive number $\gamma_{0},$ where $\sigma(A)$ is the spectrum of $A_{p} .$ Then for $\beta \geqq 0$ the fractional powers $A_{p}^{\beta}$ of $A_{p}$ are defined, and the domain $X_{p}^{\beta}=D\left(A_{p}^{\beta}\right)$ is a Banach space under the norm $\|u\|_{X_{p}^{\beta}}=\left\|A_{p}^{\beta} u\right\|_{p} . \quad$ since $A_{p}$ is sectorial in $L^{p}(\Omega)$ the operator $-A_{p}$ generates the analytic semigroup $\left\{T_{p}(t)\right\} .$ 

\begin{definition}
	suppose $A$ is a sectorial operator and $Re(\sigma(A))>0,$ then for any $\alpha>0$, define 
	\begin{equation}
		A^{-\alpha} = \frac{1}{\Gamma(\alpha)}\int_0^{\infty}t^{\alpha-1}e^{-At}\mathrm{d}t.
	\end{equation}
\end{definition}

Lemma ( [10] ). Suppose $\Omega \subset R^{n}$ a bounded domain with smooth boundary. Then for $0 \leqq \beta \leqq 1$, the following holds:
$$
\begin{array}{ll}
	X_{p}^{\beta} \subset W^{k, q}(\Omega) & \text { when } k-n / q<2 \beta-n / p, \quad q \geqq p \\
	X_{p}^{\beta} \subset C^{\nu}(\Omega) & \text { when } 0 \leqq v<2 \beta-n / p
\end{array}
$$
and the inclusion is continuous.

\begin{example}
estimate \(0=\Delta v - v + u\) subjected to homogeneous Neumann boundary conditions.
if \(u\in L_1\), then 
\[  
    \|v\|_q<c(q) \quad\text{for each }q\in[1,\frac{n}{n-2})
\]
and
\[
    \|\nabla v\|_p<c(p)\quad\text{for each }p\in[1,\frac{n}{n-1}).
\]
\end{example}
\begin{proof}
  Using Laplace transform of operator semigroup, 
  \[
    v = R(1)u=(I-\Delta)^{-1} u = \int_0^\infty e^{(\Delta-1)t}u\dd t
  \]
  we estimate (based on heat kernel estimates and Young's inequality for convolution) as done in \cite[Lemma 1.3]{Winkler2010}
  \begin{align*}
    \|v\|_q &\leq\int_0^\infty \|e^{(\Delta-1)t}\|_q\dd t\\
    &\leq c\|u\|_1\int_0^\infty (1+t^{-\frac{n}{2}(1-\frac1q)})e^{-t}\dd t. 
  \end{align*}
  and
  \begin{align*}
    \|\nabla v\|_p & \leq \int_0^\infty\|\nabla e^{(\Delta-1)t}u\|_p\dd t\\
    &\leq c\|u\|_1\int_0^\infty (1+t^{-\frac{1}{2} - \frac{n}{2}(1-\frac{1}{p})})e^{-t}\dd t    
  \end{align*}
\end{proof}

\section{\texorpdfstring{$W^{1,1}$}{} embeding coefficient}
let $\Omega(t) = \{x\in\mathrm{R}^n : |u(x)| > t\},$ 
\begin{equation}
	\chi_E(x)=
	\begin{cases}
		0,& x\not\in E,\\
		1,& x\in E,
	\end{cases}
\end{equation}
and $w_n$ is the volume of unit ball in $\mathrm{R}^n$, then
\begin{equation}
	\begin{split}
		\|u\|_{L^{\frac{n}{n-1}}(\mathrm{R}^n)} &= \left(\int_{\mathrm{R}^n}|u(x)|^\frac{n}{n-1}\right)^{\frac{n-1}{n}}\\
		&= \left(\int_{\mathrm{R}^n}\left(\int_{0}^{\infty}\chi_{\Omega(t)}\mathrm{d}t\right)^{\frac{n}{n-1}}\right)\\
		&\leqslant \int_0^\infty\left(\int_{\mathrm{R}^n}\chi_{\Omega(t)}^{\frac{n}{n-1}}\mathrm{d}x\right)^{\frac{n-1}{n}}\mathrm{d}t\\
		&= \int_0^{\infty}|\Omega(t)|^{\frac{n-1}{n}}\mathrm{d}t\\
		&\leqslant n^{-1}w_n^{-\frac{1}{n}}\int_0^{\infty}|\partial\Omega(t)|\mathrm{d}t\\
		&= n^{-1}w_n^{-\frac{1}{n}} \int_{\mathrm{R}^n}|\nabla u|.
	\end{split}
\end{equation}


\section{a isomorphism to Laplace}
\begin{theorem}
Suppose $\Omega$ is a bounded, connected open set with smooth boundary, then
\begin{equation}
D := \left\{\phi\in W^{2,2}(\Omega)\mid \frac{\partial \phi}{\partial \boldsymbol{n}}\mid_{\partial\Omega} = 0, \int_\Omega\phi=0\right\}
\cong V :=\left\{f\in L^2(\Omega)\mid \int_\Omega f = 0\right\}
\end{equation}
\end{theorem}
\begin{proof}
if $u\in H^1(\Omega)$ is a weak solution to the Neumann boundary problem 
\begin{equation*}
\left\{
\begin{array}{ll}
-\Delta u =f, & \text{ in } \Omega\\
\frac{\partial u}{\partial \boldsymbol{n}} = 0, & \text{ on } \partial\Omega
\end{array}\right.
\end{equation*}
is to say, 
\begin{equation*}
\int_\Omega\nabla u\cdot\nabla v\mathrm{d}x = \int_\Omega fv\mathrm{d}x,\text{ for all } v\in H^1(\Omega).
\end{equation*}
Define $-\Delta u\in D \mapsto f\in V$. It will be clear that $-\Delta$ is a isomorphic map between $D$ and $V$.
By Lax-Milgram Theorem, for arbitrary $f\in V$, there exists a $u\in H^1(\Omega)$ solving the Poisson equation. In fact, elliptic regularity implies $u\in H^2(\Omega)$, and hence 
$-\Delta$ is surjective; Obviously, the Laplace equation has only zero solution in $D$, and thus $-\Delta$ is injective. A direct norm inequality deduces $-\Delta$ is bounded, and consequently, $(-\Delta)^{-1}$ exists and is a continuous linear functional invoked by Banach inverse mapping theorem.
\end{proof}
\begin{example}
a direct consequence is $\|u\|_{H^2}\leq C\|\Delta u\|_{L^2}$ for all $u\in D$.
\end{example}

\section{an equivalent norm}

\begin{lemma}
	%if $\|f\|_{L^2(\partial\Omega)} + \|\nabla f\|_{L^2(\Omega)} < C$, then $f\in H^1(\Omega)$ is bounded. precisely, 
	there exists $C>0$ such that 
	\[
	\|f\|_{L^2(\Omega)} \leq C\|f\|_{L^2(\partial\Omega)} + C\|\nabla f\|_{L^2(\Omega)}.
	\]
\end{lemma}

\begin{proof}
	By absurdom, for each $n\in \mathbb{N}^+$, there exists $f_n$ such that
	\[
	\|f_n\|_{L^{2}(\Omega)} > n (\|f_n\|_{L^2(\partial\Omega)} + \|\nabla f_n\|_{L^2(\Omega)}).
	\]
	Writing
	\[
	\tilde{f}_n = \frac{f_n}{\|f_n\|_{L^2(\Omega)}},
	\]
	we have 
	\[
	\|\tilde f_n\|_{L^{2}(\Omega)} = 1,
	\quad \|\tilde f_n\|_{L^2(\partial\Omega)} + \|\nabla \tilde f_n\|_{L^2(\Omega)} \to 0.
	\]
	So $\tilde f_n\in H^1(\Omega)$ implies there exists $\tilde f\in L^2(\Omega)\cap L^2(\partial\Omega)$ such that substracting a subsequence if necessary,
	\[
	\tilde f_n\to \tilde f \quad\text{in } L^2(\Omega)\cap L^2(\partial\Omega),
	\]
	and thus 
	\[
	\|\tilde f\|_{L^2(\Omega)} = 1,
	\quad \|\tilde f\|_{L^2(\partial\Omega)} = 0,
	\quad \|\nabla\tilde f\|_{L^2(\Omega)} = 0.
	\]
	The last identity follows from the definition of weak derivative, that is,
	for each $g\in C^\infty_0(\Omega)$,
	\begin{align*}
		\int_\Omega \tilde f\partial_i g \leftarrow \int_\Omega \tilde f_n\partial_i g 
		= - \int_\Omega \partial_i \tilde f_n g \to 0. 
	\end{align*}
	We end up with $\tilde f\in H^1_0(\Omega)$ and $\|\tilde f\|_{L^2(\Omega)} = 0$ by Poincar\'e inequality,
	which is absurd.
\end{proof}


\section{global existence of a chemotaxis system with indirect signal production}
\begin{align}\label{sys: ks with isp}
  u_t &= \nabla\cdot(\nabla u - u\nabla w) \\
  \varepsilon v_t &= u - v \\
  w_t &= \Delta w - w + v
\end{align}

testing the first equation in \eqref{sys: ks with isp} by $pu^{p-1}$ ($p>1$), integrating by parts and substituting $\Delta w$ by the third equation, we get
\begin{align*}
    \frac{\dd}{\dd t}\int u^p &= p\int u^{p-1}\nabla\cdot(\nabla u - u\nabla w)\\
    &= -p(p-1)\int u^{p-2}|\nabla u|^2 + p(p-1)\int u^{p-1}\nabla w\cdot \nabla u\\
    &= -\frac{4(p-1)}{p}\int |\nabla u^{\frac{p}2}|^2 - (p-1)\int u^p\Delta w\\
    &= -\frac{4(p-1)}{p}\int |\nabla u^{\frac{p}2}|^2
    - (p-1)\int u^pw - (p-1)\int u^pw_t + (p-1)\int u^pv\\
    &\leq \frac{4(p-1)}{p}\int |\nabla u^{\frac{p}2}|^2
     + \delta\int u^{p+1} + c(\delta,p)\int w_t^{p+1} + c(\delta,p)\int v^{p+1},
\end{align*}
with some $\delta>0$.
Using Gagliardo-Nirenberg interpolation inequality
\[
    \|u\|_{p+1}^{p+1} = \|u^{\frac{p}2}\|_{\frac{2(p+1)}{p}}^{\frac{2(p+1)}{p}}
    \leq c_{gn}(\|\nabla u^{\frac{p}{2}}\|^2_2\cdot\|u^{\frac{p}{2}}|_{\frac2p}^{\frac2p}
     + \|u^{\frac{p}{2}}\|_{\frac2p}^{\frac{2(p+1)}{p}})
     \leq c_m(\|\nabla u^{\frac{p}2}\|_2^2 + 1),
\]
and putting $\delta c_m < \frac{p-1}{p}$, testing the second equation by $(p+1)v^p$ and integrating by parts,
\begin{equation*}
  \varepsilon\frac{\dd}{\dd t}\int v^{p+1} = (p+1)\int v^p(u-v) \leq \delta\int u^{p+1} + c(\delta, p)\int v^{p+1}
\end{equation*}
we obtain 
\begin{align*}
  \frac{\dd}{\dd t}\int u^p + \varepsilon v^{p+1} &\leq 
  -\frac{2(p-1)}{p}\int |\nabla u^{\frac{p}2}|^2
  + c(p)\int |w_t|^{p+1} + c(p)\int v^{p+1}.
\end{align*}
Denote $\mathcal{F}(t) = \int u^p + \varepsilon v^{p+1}$, then 
\[
    \mathcal{F}'(t) \leq c\mathcal{F} + \|w_t\|_{p+1}^{p+1}
\]
and thus using Gronwall's lemma and parabolic $L_p$ estimates (or Amann's Sobolev maximal regularity theory[Amann 1995, Theorem 4.10.7 and Remark 4.10.9), this yields 
\begin{align*}
  \mathcal{F} &\leq \mathcal{F}(0)e^{ct} + c\int_0^te^{c(t-s)}\|w_t\|_{p+1}^{p+1}\dd s\\
  &\leq \mathcal{F}(0)e^{ct} + c\int_0^t\|w_t\|_{p+1}^{p+1}\dd s\\
  &\leq F(0)e^{ct} + ce^{ct}(\|w_0\|_\star^{p+1} + \int_0^t\|v\|_{p+1}^{p+1}\dd s)\\
  &\leq ce^{ct}\int_0^t\mathcal{F}\dd s + F(0)e^{ct} + ce^{ct}\|w_0\|_\star^{p+1},
\end{align*}
and 
\[
    \int_0^t\mathcal{F}\dd s\leq c(t),
\]
and consequently, \(\mathcal{F}(t)\leq c(t)\).
Here,
\begin{align*}
  \star &= (L^{p+1},D(\Delta-1))_{1-\frac{1}{p+1},p+1}\\
  &\cong (L^{p+1}, W^2_{p+1,\mathcal{B}})_{1-\frac{1}{p+1},p+1}\\
  &\cong B_{p,p,\mathcal{B}}^{\frac{2p}{p+1}}\\
  &\cong\begin{cases}
          W^{\frac{2p}{p+1}}_{p+1,\mathcal{B}}, & \mbox{if } p>1 \\
          H^1_{2,\mathcal{B}}\cong W^1_{2,\mathcal{B}}, & p=1.
        \end{cases}
\end{align*}

\textbf{a Lyapunov functional}

testing the first equation in \eqref{sys: ks with isp} by $\ln u -w$ and integrating by parts, we get
\[
    \frac{\dd}{\dd t}\int u(\ln u - w)
    = \int u_t(\ln u - w) + \int u_t - \int uw_t
    = -\int u|\nabla (\ln u - w)|^2 - \int uw_t.
\]
substituting $u$ and $v_t$ by the last two equations, respectively, we calculate by integration by parts
\begin{align*}
    \int uw_t &= \int (\varepsilon v_t + u)w_t\\
    &= \int \varepsilon(w_{tt} - \Delta w_t + w_t)w_t 
    + \int (w_t - \Delta w + w)w_t\\
    &= \frac{\varepsilon}{2}\frac{\dd}{\dd t}\int w_t^2 
    + \varepsilon\int |\nabla w_t|^2
    + (1+\varepsilon) \int w_t^2
    + \frac{1}{2}\frac{\dd}{\dd t}\int w^2
    + \varepsilon\int |\nabla w_t|^2.
\end{align*}
finally, we have 
\[
    \mathcal{E}' = -\mathcal{D},
\]
where
\[
    \mathcal{E} = \int u\ln u - \int uw + \frac{1}{2}\int |\nabla w|^2 
    + \frac12\int w^2 + \frac{\varepsilon}{2}\int w_t^2
\]
and 
\[
    \mathcal{D} = \int u|\nabla(\ln u - w)|^2 + (1 + \varepsilon)\int w_t^2 
    + \varepsilon\int |\nabla w_t|^2.
\]

Using Morse-Trudinger inequality, one can verify that if $M=\int u_0\in(0,4\pi)$, then there exists $c(M)>0$ such that \(\mathcal{F}\geq -c(M)\) and
\[
    \sup_{t>0}(\|u\ln u\|_1 + \|w\|_{W^1_2} + \|w_t\|_2) + \int_0^\infty\|w_t\|_{W^1_2}^2\dd t < \infty.
\]

\textbf{Uniformly boundedness under the subcritical condition.}

similarly, we test the first equation in \eqref{sys: ks with isp} by $pu^{p-1}$, integrate by parts and substitute $\Delta w$ by the third equation,
\begin{align*}
  \frac{\dd}{\dd t}\int u^p &= -p(p-1)\int u^{p-2}|\nabla u|^2 
  + p(p-1)\int u^{p-1}\nabla u\cdot\nabla w\\
  &= -\frac{4(p-1)}{p}\int |\nabla u^{\frac{p}2}|^2 - (p-1)\int u^p\Delta w\\
  &= -\frac{4(p-1)}{p}\int |\nabla u^{\frac{p}2}|^2 
  - (p-1)\int u^p w - (p-1)\int u^pw_t + (p-1)\int u^p v.
\end{align*} 
Using H\"older inequality, Sobolev imbedding inequality and Young's inequality, we estimate
\begin{align*}
    \int u^pw_t &\leq\left(\int u^{\frac{4p}3}\right)^{\frac34}\|w_t\|_4\\
    &\leq c_S\left(\int u^{\frac{4p}3}\right)^{\frac34}\|w_t\|_{W^1_2}^{\frac12}\|w_t\|_2^{\frac12}\\
    &\leq \int u^{\frac{4p}{3}} + c\|w_t\|_{W^1_2}^2.
\end{align*}
Testing the second equation by $(p+1)v^p$, we have
\[
    \varepsilon\frac{\dd}{\dd t}\int v^{p+1} = (p+1)\int uv^p - (p+1)\int v^{p+1}.
\]
we gather above estimates and obtain 
\begin{align*}
    \mathcal{F}' + \mathcal{F} &\leq
    - \frac{4(p-1)}{p}\int\|\nabla u^{\frac{p}{2}}\|^2
    + \int u^{\frac{4p}3} + c\int u^{p+1} + c\|w_t\|_{W^1_2}^2 + c.
\end{align*} 
Noting a logarithm-type variant of Gagliardo-Nirenberg interpolation inequality,
\[
    \int u^{p+1} = \|u^{\frac{p}{2}}\|_{\frac{2(p+1)}{p}}^{\frac{2(p+1)}{2}}
    \leq c_{GN}\|\nabla u^{\frac{p}{2}}\|_2^2\cdot
    \|(u\ln u)^{\frac{p}{2}}\|_{\frac{2}{p}}^{\frac2p}
    + c_{GN}\|u^{\frac{p}{2}}\|_{\frac2p}^{\frac{2(p+1)}{p}},
\] 
putting $p=3$ and use the Gronwall's inequality, we have
\begin{align*}
    \int u^3 + \varepsilon\int v^4 &\leq
    e^{-t}(\int u^3 + \varepsilon v^4)(0)
    + c(\int_0^te^{s-t}\|w_t\|_{W^1_2}^2\dd s + 1 - e^{-t})\\
    &\leq (\int u^3 + \varepsilon v^4)(0)
    + c(\int_0^t\|w_t\|_{W^1_2}^2\dd s + 1).
\end{align*}

\section{\texorpdfstring{$L^2$}{L2} theory}
\label{sec: L2 theory}

consider \cite{Winkler2023}
\begin{equation}
	v_t = \Delta v - v + f,
\end{equation}
with boundary conditions
\begin{equation*}
	\frac{\partial v}{\partial\nu} = 0,
\end{equation*}
where 
\begin{equation*}
	f\in L^\infty((0,T); L^1(\Omega)),
\end{equation*}
using \[
\nabla v \cdot \nabla \Delta v=\Delta\left(\frac{1}{2}|\nabla v|^2\right)-\left|D^2 v\right|^2,
\]
we get
\begin{align*}
	\frac{\dd}{\dd t}\int|\nabla v|^p 
	&= p\int |\nabla v|^{p-2}\nabla v\cdot \nabla v_t 
	= p\int |\nabla v|^{p-2} (\nabla \Delta v - \nabla v + \nabla f) \\
	&= \frac12\int |\nabla v|^{p-2}\Delta (|\nabla v|^2) - p\int |\nabla v|^{p-2}|D^2v|^2 
		-p\int |\nabla v|^p + p\int |\nabla v|^{p-2}\nabla v \nabla f,
\end{align*}
by additional assumption of convex domain 
or a pointwise estimate of boundary trace alongwith trace imbedding lemma and Sobolev imbedding inequality,
we can control the first term of right hand, and estimate
\begin{align*}
	p\int |\nabla v|^{p-2}\nabla v \cdot \nabla f 
	&= - p\int \nabla\cdot(|\nabla v|^{p-2}\nabla v) f\\
	&= -p(p-2)\int |\nabla v|^{p-4}\nabla v\cdot (D^2v\nabla v) f 
		- p \int |\nabla v|^{p-2}\Delta v f\\
	&\leq c(p,n)\int |\nabla v|^{p-2} |D^2v| |f|\\
	&\leq c(p,n)\left(\int |\nabla v|^{p-2}|D^2v|^2\right)^{\frac12}
	\left(\int |\nabla v|^{p-2}f^2\right)^{\frac12}\\
	&= c(p,n) \|\nabla w\|_{L^2} \left(\int |\nabla v|^{p-2}f^2\right)^{\frac12},
\end{align*}
here, 
\[
	w = |\nabla v|^{\frac{p}{2}},	
\]
let $q\in(2,p)$ to be specified below, we estimate
\begin{align*}
	\left(\int |\nabla v|^{p-2}f^2\right)^{\frac12}
	&\leq \left(\int |\nabla v|^{\frac{(p-2)q}{q-2}}\right)^{\frac{q-2}{2q}}
	\left(\int f^q\right)^{\frac{1}{q}}\\
	&= \|w\|_{L^{\frac{2(p-2)q}{p(q-2)}}}^{\frac{p-2}{p}}
		\|f\|_{L^q},
\end{align*}
using G-N inequality
\[
	\|u\|_{L^{\frac{2(p-2)q}{p(q-2)}}}
	\leq C_g \|u\|_{W^{1,2}}^\theta\|u\|_{L^2}^{1-\theta},\quad u\in W^{1,2}, 
\]
with
\[
	\frac{p(q-2)}{2(p-2)q} = \theta\left(\frac12-\frac1n\right) + \frac{1-\theta}{2}.
\]
i.e.,
\[
	\theta = \frac{n(p-q)}{(p-2)q},
\]
we get
\begin{equation*}
	p\int |\nabla v|^{p-2}\nabla v \cdot \nabla f 
	\leq c(p,n) \|w\|_{W^{1,2}}^{1+\frac{n(p-q)}{pq}}
		\|w\|_{L^2}^{\frac{p-2}{p} - \frac{n(p-q)}{pq}}
		\|f\|_{L^q}.
\end{equation*}
Fixing 
\[
	q = \frac{(n+2)p}{n+p},
\]
which is such that
\[
	\alpha := \frac{2}{1-\frac{n(p-q)}{pq}} = q,\quad \theta = \frac{n}{n+2},
\]
and 
\begin{equation}\label{eq: gn inequality Lpw12L2}
	\|u\|_{L^{\frac{2(n+2)}{n}}} \leq C_g \|u\|_{W^{1,2}}^{\frac{n}{n+2}}\|u\|_{L^2}^{\frac{2}{n+2}},
\end{equation}
using Young inequality, we get
\begin{align*}
	p\int |\nabla v|^{p-2}\nabla v \cdot \nabla f 
	&\leq \varepsilon \|w\|_{W^{1,2}}^2
		\|w\|_{L^2}^{\left(\frac{p-2}{p} - \frac{n(p-q)}{pq}\right)\alpha}
		\|f\|_{L^q}^\alpha\\
	&= \varepsilon \|w\|_{W^{1,2}}^2
	+ c(\varepsilon) \|w\|_{L^2}^{\frac{2(p-2)}{n+p}}
	\|f\|_{L^q}^q,
\end{align*}
and therefore
\begin{equation*}
	\mathcal{F}' + \frac1C F \leq C \mathcal{F}^{\frac{p-2}{n+p}} \|f\|_q^q, 
\end{equation*} 
where
\[
	\mathcal{F} := \int w^2.
\]
So we obtain by a direct application of ODE comparable method,
\[
	\mathcal{F}^{\frac{n+2}{n+p}} \leq C \int_0^T\|f\|_q^q,
\]
and consequently,
\begin{align*}
	\sup_{(0,T)}\int |\nabla v|^p + \int_0^T\int |\nabla v|^{p-2}|D^2 v|^2 
	\leq C \left(\int_0^T\int |f|^{\frac{(n+2)p}{n+p}}\right)^{\frac{n+p}{n+2}}
\end{align*}
and by \eqref{eq: gn inequality Lpw12L2},
\begin{equation*}
	\left(\int_0^T\int |\nabla v|^{\frac{(n+2)p}{n}}\right)^{\frac{n}{n+2}} 
	\leq C \left(\int_0^T\int |f|^{\frac{(n+2)p}{n+p}}\right)^{\frac{n+p}{n+2}}. 
\end{equation*}

\section{quadratic logistic dampening prevents chemotactic collapse}

consider 
\begin{equation}
	\begin{cases}
		u_t = \Delta u - \nabla\cdot(u\nabla v) + \mu u (1-u), \\
		v_t = \Delta v - v + u,\\
		\partial_\nu u = \partial_\nu v = 0,
	\end{cases}
\end{equation}
in a bounded domain $\Omega\subset\mathbb{R}^2$ with smooth boundary.
It is obvious that
\[
	\int_t^{t+\tau}\int u^2 < C,
\]
for any $t\in(0, T-\tau)$ with $\tau = \min\{1, T/2\}$.
By conclusions in Section~\ref{sec: L2 theory}, 
we have 
\begin{align*}
	\sup_{(0,T)}\|\nabla v\|_{L^2} 
	+ \sup_{(0,T-\tau)}\int_t^{t+\tau}\int |D^2v|^2 
	+ \sup_{(0, T-\tau)}\int_t^{t+\tau}\int |\nabla v|^4 < C.
\end{align*}


\subsection{method 1: direct \texorpdfstring{$L_p$}{Lp} estimates}
calculate
\begin{align*}
	\frac{\dd}{\dd t}\int u^p 
	&= p\int u^{p-1}\nabla\cdot(\nabla u - u\nabla v) + \mu p \int u^p(1-u)\\
	&= - p(p-1)\int u^{p-2}|\nabla u|^2 
		+ p(p-1) \int u^{p-1}\nabla u\cdot\nabla v 
		+ \mu p\int u^p(1-u)
\end{align*}
estimate
\begin{align*}
	p(p-1) \int u^{p-1}\nabla u\cdot\nabla v
	&= (p-1)\int \nabla u^p\cdot\nabla v
	 = -(p-1)\int u^p\Delta v\\
	&\leq (p-1)\left(\int u^{2p}\right)^{\frac12}\left(\int |\Delta v|^2\right)^\frac{1}{2}
	 = (p-1)\|\Delta v\|_2\cdot\|u^{\frac{p}{2}}\|_4^2\\
	&\leq c(p) \|\Delta v\|_2\cdot \|u^{\frac{p}{2}}\|_{W^{1,2}}\cdot\|u^{\frac{p}{2}}\|_2,
\end{align*}
and
\begin{align*}
	\|u^{\frac{p}{2}}\|_2 
	\leq C\|u^{\frac{p}{2}}\|_{W^{1,2}}^{\frac{p-1}{p}}\|u^{\frac{p}{2}}\|_{\frac2p}^{\frac1p},
\end{align*}
which implies 
\begin{align*}
	\|u^{\frac{p}{2}}\|_{W^{1,2}}^2 \geq c(m,p) \|u^{\frac{p}{2}}\|_2^{\frac{2p}{p-1}}.
\end{align*}
let 
\[
	\mathcal{F}(u) := \int u^p + c,
\]
we have
\begin{align*}
	\mathcal{F}' + \frac1C\mathcal{F}^\frac{p}{p-1} 
	\leq \mathcal{F} \int |\Delta v|^2 + C,
\end{align*}
let
\[
	\mathcal{G} := \ln\mathcal{F},
\]
we have 
\begin{equation*}
	\mathcal{G}' + \frac{\mathcal{G}}{C} \leq \int|\Delta v|^2 + C,
\end{equation*}
which implies the uniform-in-time boundedness of $\|u\|_p$.

\subsection{method 2: \texorpdfstring{$L\log L$}{LlogL}-type estimate}
Let 
\[
	\phi(\xi) := \int_0^\xi \ln^\alpha(1+\eta) \dd\eta,\quad\alpha \in (0,2],
\]
then
\begin{align*}
	\frac{\dd}{\dd t}\int\phi(u)
	&= \int \nabla\cdot(\nabla u - u\nabla v) \ln^\alpha(1+u) + \mu\int u(1-u)\ln^\alpha(1+u)\\
	&= - \int \frac{\alpha|\nabla u|^2\ln^{\alpha-1}(1+u)}{1+u}
		+ \int \frac{\alpha u}{1+u} \ln^{\alpha-1}(1+u)\nabla u\cdot\nabla v  + \mu\int u(1-u)\ln^\alpha(1+u)\\
	&\leq - \alpha \int \frac{|\nabla u|^2\ln^{\alpha-1}(1+u)}{1+u}
		+ \alpha\left(\int \frac{|\nabla u|^2\ln^{\alpha-1}}{1+u}\right)^{\frac12}
			\left(\int \frac{|\nabla v|^2u^2\ln^{\alpha-1}(1+u)}{1+u}\right)^{\frac12}\\
	&\quad + \mu\int u(1-u)\ln^\alpha(1+u)\\
	&\leq \frac{\mu}{2}\int u^2\ln^{2\alpha-2}(1+u) + c(\alpha, \mu)\int|\nabla v|^4 
		+ \mu\int u(1-u)\ln^\alpha(1+u)\\
	&\leq c(\alpha, \mu)\int|\nabla v|^4 - \int\phi(u) + c(\alpha, \mu),
\end{align*}
which implies
\begin{align*}
	\int u\ln^\alpha(1+u) \leq C\int \phi(u) + C \leq C.
\end{align*}
combining with a logarithmic variant of G-N inequality, one can get 
\[
	\|u\|_p + \|\nabla v\|_{2p} < C(p), \quad p>1,
\]
which implies uniform-in-time boundedness of $u$ 
by well-established parabolic regularity theory in the context of chemotaxis models.

or

denote 
\[
	\psi(\xi) := \int_0^\xi \frac{\alpha \eta}{1+\eta}\ln^{\alpha-1}(1+\eta),
\]
then 
\begin{align*}
	\int \frac{\alpha u}{1+u} \ln^{\alpha-1}(1+u)\nabla u\cdot\nabla v
	&= - \int \psi(u) \Delta v\\
	&\leq \varepsilon\int \psi^2(u) + c(\varepsilon)\int |\Delta v|^2\\
	&\leq \varepsilon \int u^2\ln^{2\alpha -2}(1+u) + c(\varepsilon, n) \int |D^2 v|^2,
\end{align*}


\subsection{is the title of this section right when \texorpdfstring{$n\geq3$}{n>=3}?}

\begin{align*}
	\frac2n\frac{\dd}{\dd t}\int u^{\frac n2} 
	&= \int u^{\frac n2-1}\nabla\cdot(\nabla u - u\nabla v) + \mu\int u^{\frac n2} - \mu\int u^{\frac n2+1}\\
	&= -\frac{n-2}{2}\int u^{\frac n2-2}|\nabla u|^2 
		+ \frac{n-2}{2}\int u^{\frac n2-1}\nabla u\cdot\nabla v + \mu\int u^{\frac n2} - \mu\int u^{\frac n2+1}
\end{align*}
\begin{align*}
	\frac{n-2}{2}\int u^{\frac n2-1}\nabla u\cdot\nabla v
	&\leq c(n) \left(\int u^{\frac n2-2}|\nabla u|^2\right)^{\frac12}
		\left(\int u^{\frac n2}|\nabla v|^2\right)^{\frac12}\\
	&\leq c(n) \left(\int u^{\frac n2-2}|\nabla u|^2\right)^{\frac12}
		\left(\int u^{\frac{n+2}{2}}\right)^{\frac{n}{2(n+2)}}
		\left(\int |\nabla v|^{n+2}\right)^{\frac1{n+2}}\\
	&\leq c\|\nabla u^{\frac n4}\|_2
		\cdot \|u^{\frac{n+2}{2}}\|_1^{\frac{n}{2(n+2)}}
		\cdot \||\nabla v|^{\frac n2}\|_2^{\frac{4}{(n+2)n}}
		\cdot \||\nabla v|^{\frac n2}\|_{W^{1,2}}^{\frac{2}{n+2}},
\end{align*}

\begin{align*}
	\frac{\dd}{\dd t}\int |\nabla v|^n 
	= \frac{n}{2} \int |\nabla v|^{n-2}\Delta |\nabla v|^2 - n\int |\nabla v|^{n-2}|D^2v|^2
		- n\int |\nabla v|^n 
		+ n\int |\nabla v|^{n-2}\nabla v\cdot\nabla u,
\end{align*}
\begin{align*}
	\frac{\dd}{\dd t}\int |\nabla v|^n 
		+ n\int |\nabla v|^n
	\leq - n\int |\nabla v|^{n-2}|D^2v|^2
		+ n\int |\nabla v|^{n-2}\nabla v\cdot\nabla u,
\end{align*}
let 
\[
	w=|\nabla v|^{\frac n2},
\]
\begin{align*}
	n\int |\nabla v|^{n-2}\nabla v\cdot\nabla u
	&= -n\int \nabla\cdot(|\nabla v|^{n-2}\nabla v)u \\
	&= -n\int((n-2)|\nabla v|^{n-4}\nabla v \cdot (D^2v\cdot\nabla v) + |\nabla v|^{n-2}\Delta v) u\\
	&\leq c(n)\int |\nabla v|^{n-2}|D^2v|u\\
	&\leq \|\nabla|\nabla v|^{\frac{n}{2}}\|_2\left(\int |\nabla v|^{n-2}u^2\right)^{\frac12}\\
	&\leq c(n)\|\nabla w\|_2\left(\int w^{\frac{2(n-2)}{n}}u^2\right)^{\frac12}\\
	&\leq c(n)\|\nabla w\|_2 \left(\int w^{\frac{2(n+2)}{n}}\right)^{\frac{n-2}{2(n+2)}}
		\left(\int u^{\frac{n+2}{2}}\right)^{\frac{2}{n+2}}\\
	&= c(n)\|\nabla w\|_2 
		\cdot \|w\|_{\frac{2(n+2)}{n}}^{\frac{n-2}{n}}
		\cdot \|u\|_{\frac{n+2}{2}}\\
	&\leq c(n)\|w\|_{W^{1,2}}^{\frac{2n}{n+2}}
		\cdot \|w\|_2^{\frac{2(n-2)}{n(n+2)}}
		\cdot \|u\|_{\frac{n+2}{2}}
\end{align*}
\subsection{一个问题}
\begin{align*}
	\|u\|_{n/2} < C \Rightarrow \|u\|_\infty < C ?
\end{align*}

let 
\[
	\phi(\xi):= \int_0^\xi \eta^{\frac{n}{2}-1}\ln^{\alpha}(e+\eta), \quad\alpha>1,
\]
then
\[
	\frac{\xi^{\frac{n}{2}}\ln^{\alpha}(e+\xi)}{C} - C 
	\leq \phi(\xi) 
	\leq \xi^{\frac{n}{2}}\ln^{\alpha}(e+\xi),
\]
calculate
\begin{align*}
	\frac{\dd}{\dd t}\int \phi(u) 
	&= \int u^{\frac{n}{2} - 1}\ln^{\alpha}(e+u) \nabla\cdot(\nabla u - u\nabla v)
		+ \mu\int u^{\frac{n}{2}}(1-u)\ln^{\alpha}(e+u)\\
	&= -\frac{n-2}{2}\int |\nabla u|^2u^{\frac{n}{2}-2}\ln^\alpha(e+u)
		- \alpha\int\frac{u^{\frac{n}{2}-1}\ln^{\alpha-1}(e+u)}{e+u}|\nabla u|^2\\
		&\quad + \frac{n-2}{2}\int u^{\frac{n}{2}-1}\ln^\alpha(e+u)\nabla u\cdot\nabla v 
			+ \alpha\int \frac{u^{\frac{n}{2}}\ln^{\alpha-1}(e+u)}{e+u} \nabla u \cdot \nabla v\\
		&\quad + \mu\int u^{\frac{n}{2}}(1-u)\ln^{\alpha}(e+u),
\end{align*}
Noting
\begin{align*}
	\|u\|_{n/2} < C 
	&\Rightarrow \sup_{(0,T-\tau)}\int_t^{t+\tau} \int u^{\frac{n+2}{2}} < C \\
	&\Rightarrow \|\nabla v\|_n 
		+ \sup_{(0,T-\tau)}\int_t^{t+\tau}\int |\nabla v|^{n+2}
		+ \sup_{(0,T-\tau)}\int_t^{t+\tau}\int  |\nabla v|^{n-2}|D^2v|^2 < C,
\end{align*}

\subsection{疑惑哪里错了}
\begin{equation}
	\begin{cases}
		u_t = \Delta u - \nabla\cdot(u\nabla v) + \mu u - \mu u^\alpha,\\
		0 = \Delta v - \overline u + u,
	\end{cases}
\end{equation}
let 
\[
	\phi(t) = \int_0^R ur^{2n-1}\dd r,
\]
and
\[
	m(t) = \int_0^R ur^{n-1}\dd r,
\]
we calculate
\begin{align*}
	\phi'(t) &=
	\int_0^R(u_rr^{n-1})_rr^n - (uv_rr^{n-1})_rr^n\dd r 
		+ \mu\int_0^R u r^{2n-1}\dd r - \mu \int_0^Ru^\alpha r^{2n-1}\dd r,
\end{align*}
estimate
\begin{align*}
	\int_0^R(u_rr^{n-1})_rr^n 
		&= - n\int_0^Ru_rr^{2n-2}\dd r\\
		&= -nu(R) + 2n(n-1)\int_0^Rur^{2n-3}\dd r\\
		&\leq 2n(n-1)\left(\int_0^Rur^{n-1}\dd r\right)^{\frac{2}{n}}
			\cdot \left(\int_0^Rur^{2n-1}\dd r\right)^{\frac{n-2}{n}}.
\end{align*}
Using 
\[
	(v_rr^{n-1})_r -\overline{u}r^{n-1} + ur^{n-1} = 0,
\]
we get
\begin{align*}
	- \int_0^R (uv_rr^{n-1})_rr^n\dd r 
		&= n\int_0^Rur^{n-1}\cdot v_rr^{n-1}\dd r \\
		&= n\int_0^R ur^{n-1} \int_0^r(\overline{u}\eta^{n-1} - u\eta^{n-1})\dd\eta\dd r\\
		&= - \frac{n}2\left(\int_0^Rur^{n-1}\dd r\right)^2 + \overline{u}\int_0^R ur^{2n-1}\dd r, 
\end{align*}
and therefore
\begin{align*}
	\phi'(t) \leq \Phi(t, \phi) := 2n(n-1)m^{\frac2n}\cdot \phi^{\frac{n-2}{n}} - \frac{nm^2}2 + (\overline{u} + \mu)\phi,
\end{align*}

while
\[
	m' = \mu m - \mu m^\alpha,
\]
we have 
\[
	m\in\left[\min\{m(0),1\}, \max\{m(0),1\}\right],
\]
and thus for any $m(0)>0$, there exists $\varepsilon>0$ 
such that for any $\xi\in(0,\varepsilon)$,
$\Phi(t,\xi) < 0$,
that is for any initial data satisfying $\phi(0) < \varepsilon$, 
there exists $t_0\in(0,\infty)$ such that 
\[
	\phi(t_0) = 0,
\]
which implies such solution is not globally well-posed.

\section{local existence}
\begin{align*}
	u_t &= \Delta u - \nabla\cdot(u\nabla v),\\
	v_t &= \Delta v - v + u,
\end{align*}

let 
\[
	X := C^0(\overline\Omega\times[0,T])\times C^0([0,T]; W^{1,p}(\Omega)), \quad p>n,
\]
define
\begin{align*}
	\Psi(u,v) :=
		\left(e^{\Delta t}u_0 + \int_0^te^{\Delta(s-t)}\nabla\cdot(u\nabla v)\dd s, 
		e^{(\Delta-1) t}v_0 + \int_0^te^{(\Delta-1)(s-t)}u\dd s\right),
		\quad (u,v)\in X,
\end{align*}
with 
\[
	(u_0, v_0) \in C^0(\overline{\Omega})\times W^{1,p}.
\]
Here, $\Omega$ is a bounded domain with smooth boundary $\partial\Omega$ in $\mathbb{R}^n$, and $e^{\Delta t}$ denotes the semigroup generated by $-\Delta$ in $L^p$ with domain 
\[
	W^{2,p}_N := \{f\in W^{2,p}: \partial_\nu f = 0\text{ on }\partial\Omega\}.
\]
we estimate 
\begin{align*}
	\|\Psi_1\|_{\infty} &= \left\| e^{\Delta t}u_0 + \int_0^te^{\Delta(s-t)}\nabla\cdot(u\nabla v)\dd s\right\|_\infty\\
	&\leq \|e^{\Delta t}u_0\|_\infty 
		+ \left\|  \int_0^te^{\Delta(s-t)}\nabla\cdot(u\nabla v)\dd s\right\|_\infty\\
	&\leq \|u_0\|_\infty 
		+ \int_0^t \| e^{\Delta(s-t)}\nabla\cdot(u\nabla v)\dd s\|_\infty\dd s\\
	&\leq \|u_0\|_\infty 
		+ C\int_0^t \left(1+(t-s)^{-\frac{1}{2}-\frac{n}{2p}}\|u\nabla v\|_p\right)\dd s\\
	&\leq \|u_0\|_\infty 
		+ C\|u\|_{X_1} \|v\|_{X_2} \int_0^t\left(1+s^{-\frac{1}{2} - \frac{n}{2p}}\right)\dd s,
\end{align*}
and 
\begin{align*}
	\|\Psi_2\|_{W^{1,p}} &= \left\| e^{(\Delta-1) t}v_0 + \int_0^te^{(\Delta-1)(s-t)}u\dd s\right\|_{W^{1,p}}\\
	&\leq \| e^{(\Delta-1) t}v_0 \|_{W^{1,p}}
		+ \left\| \int_0^te^{(\Delta-1)(s-t)}u\dd s\right\|_{W^{1,p}}\\
	&\leq C\|v_0\|_{W^{1,p}} 
		+ C\|u\|_{X_1}\int_0^t\left(1+s^{1/2}\right)e^{-s}\dd s.
\end{align*}
Similarly, we get
\begin{align*}
	\|\Psi_1(u_1,v_1) - \Psi_1(u_2,v_2)\|_\infty
	&= \left\|\int_0^te^{\Delta(s-t)}\nabla\cdot(u_1\nabla v_1 - u_2\nabla v_2)\dd s\right\|_\infty\\
	&\leq \left\|\int_0^te^{\Delta(s-t)}\nabla\cdot(u_1\nabla (v_1 -  v_2))\dd s\right\|_\infty
		+ \left\|\int_0^te^{\Delta(s-t)}\nabla\cdot((u_1 - u_2)\nabla v_2)\dd s\right\|_\infty\\
	&\leq C(\|u_1\|_{X_1} \|v_1-v_2\|_{X_2} + \|u_1-u_2\|_{X_1} \|v_2\|_{X_2}) 
		\int_0^t\left(1+s^{-\frac{1}{2} - \frac{n}{2p}}\right)\dd s
\end{align*}
and
\begin{align*}
	\|\Psi_2(u_1,v_1) - \Psi_2(u_2,v_2)\|_{W^{1,p}}
	&\leq C\|u_1-u_2\|_{X_1}\int_0^t\left(1+s^{1/2}\right)e^{-s}\dd s.
\end{align*}

define the topology
\[
	(u,v)_X := \|u\|_{X_1} + \|v\|_{X_2},
\]
suppose
\[
	(u_0,v_0)_X = \|u_0\|_\infty + \|v_0\|_{W^{1,p}} < M,
\]
then we can choose $T>0$ small enough such that 
$\psi$ has a fixed point on the set 
\[
	S := \{ (u,v)\in X: (u,v)_X \leq M + C + 1\}.
\]
To see this, we can find that 
$\Psi$ is a onto contracting map, i.e.,
\begin{align*}
	\|\Psi(u_1,v_1) - \Psi(u_2,v_2)\|_X 
	&\leq C(M+C+1)(u_1-u_2,v_1-v_2)_X\int_0^t\left(1+s^{-\frac{1}{2}-\frac{n}{2p}}\right)\dd s,
\end{align*}
and 
\begin{align*}
	\|\Psi\|_X \leq (C+1)(u_0,v_0)_X + C(M+C+1)^2\int_0^t\left(1+s^{-\frac{1}{2}-\frac{n}{2p}}\right)\dd s,
\end{align*}
provided that $T$ is less than a given number dependent on $M$.

By Banach fixed-point theorem, the integral equations
\[
	\Psi(u,v) = (u,v)
\]
has a solution $(u,v)\in X$, which can be extended to its maximal existence time $T_m\in(0,\infty]$.
Here, if $T_m < \infty$, then
\[
	\limsup_{t\nearrow T_m} (u,v)_X = \infty.
\]

\section{comparable criterion of elliptic equation under Neumann boundary conditions}

\begin{lemma}
	Let $u\in H^1(\Omega)$ be the weak solution of the following elliptic equation
	\begin{equation*}
		\begin{cases}
			-\Delta u = u = f, & \Omega,\\
			\frac{\partial u}{\partial \nu} = 0, & \partial\Omega,
		\end{cases}
	\end{equation*} 
	i.e.,
	\begin{align*}
		\int \nabla u\cdot\nabla \phi + \int u\phi = \int f\phi,\quad \phi\in H^1(\Omega),
	\end{align*}
	where $f\in L^2(\Omega)$.
	If $f\geq0$ in the sense of distribution,
	then $u\geq0$ a.e..
\end{lemma}
\begin{proof}
	let $\underline{u} = \min\{u,0\}$, 
	set $\phi = \underline{u}$, 
	then
	\begin{align*}
		\int \nabla u\cdot\nabla\phi + \int_\Omega u\phi 
		= \int|\nabla\phi|^2 + \int_\Omega\phi^2 
		= \int f\phi \leq 0, 
	\end{align*}
	which implies $\phi = 0$ a.e., i.e., $u\geq0$ a.e..
\end{proof}

\section{Helly's compactness theorem}
\begin{theorem}
	单调有界序列存在子列点点收敛.
\end{theorem}
证明思路:
\begin{itemize}
	\item 先构造在有理点处逐点收敛的子序列
	\begin{itemize}
		\item 利用有界性逐点迭代构造
		\item 取对角线序列
	\end{itemize}
	\item 利用单调性扩充定义
	\begin{itemize}
		\item 任取无理点, 若该点处的有理点左右极限相等, 则可将该点扩充到定义域中,
		\item 如若不然, 由单调函数的不连续点至多可数, 
		再利用第一步的技巧, 通过取对角线子列的办法, 
		把剩下的所有无理点扩充到定义中, 得到点点收敛的子序列.
	\end{itemize}
\end{itemize}

\section{BMO space}
\begin{theorem}
	\label{thm: bmo estimates}
	Let $u\in W^{1,1}(\Omega)$ where $\Omega$ is convex, and suppose there exists a constant $K$ such that
	\[
		\int_{\Omega\cap B_R}|Du|\dd x \leq KR^{n-1}\quad\text{for all balls }B_R.
	\]
	Then there exist positive constants $\sigma_0$ and $C$ depending only on $n$ such that
	\[
		\int_\Omega \exp\left(\frac{\sigma}{K}|u-u_\Omega|\right)\dd x \leq C(diam \Omega)^n,
	\]
	where $\sigma = \sigma_0|\Omega|(diam \Omega)^{-n}$.
\end{theorem}
\begin{proof}
	We first show for $\Omega$ be convex and $u\in W^{1,1}(\Omega)$, 
	\begin{equation}
		\label{eq: estimate local oscillation by gradient}
		|u(x) - u_S| \leq \frac{d^n}{n|S|}\int_\Omega|x-y|^{1-n}|Du(y)|\dd y\quad\text{a.e. }\Omega,
	\end{equation}
	where 
	\[
		u_S = \frac{1}{|S|}\int_S u\dd x,\quad d = diam \Omega,
	\]
	and $S$ is any measurable subset of $\Omega$.
	By dense argument, it is enough to establish \eqref{eq: estimate local oscillation by gradient} for $u\in C^1(\Omega)$.
	We then have for $x,y\in\Omega$,
	\[
		u(x) - u(y) = - \int_0^{|x-y|} D_ru(x + r\omega)\dd r, \quad \omega = \frac{y-x}{|y-x|}.
	\]
	Integrating with respect to $y$ over $S$, we obtain
	\[
		|S|(u(x) - u_S) = - \int_S\dd y \int_0^{|x-y|}D_r u(x + r\omega)\dd r.
	\]
	Writing 
	\begin{equation*}
		V(x) = 
		\begin{cases}
			|D_ru(x)|, & x\in\Omega,\\
			0, & x\not\in\Omega,
		\end{cases}
	\end{equation*}
	we thus have 
	\begin{align*}
		|u(x) - u_S| 
		&\leq \frac{1}{|S|}\int_{|x-y|<d}\dd y \int_0^\infty V(x+r\omega)\dd r\\
		&= \frac{1}{|S|}\int_0^\infty\int_{|\omega|=1}\int_0^dV(x+r\omega)\rho^{n-1}\dd\rho\dd\omega\dd r\\
		&= \frac{d^n}{n|S|} \int_0^\infty\int_{|\omega|=1}V(x+r\omega)\dd\omega\dd r\\
		&= \frac{d^n}{n|S|} \int_\Omega |x-y|^{1-n}|Du(y)|\dd y.
	\end{align*}
	Using 
	\[
		|x-y|^{1-n} = |x-y|^{(1/q-n)/q} \cdot |x-y|^{(1-1/q)(1+1/q-n)},
	\]
	we have by H\"older inequality
	\begin{align*}
		\int_\Omega |x-y|^{1-n}|Du(y)|\dd y
			&\leq \left(\int_\Omega|x-y|^{1/q-n}|Du(y)|\dd y\right)^{1/q}
				\cdot \left(\int_\Omega |x-y|^{1/q+1-n}|Du(y)|\dd y\right)^{1-1/q}.
	\end{align*}
	Extend $f$ to be zero outside $\Omega$ and write
	\[
		v(\rho) = \int_{B_\rho(x)}|Du(y)|\dd y.
	\]
	Then 
	\begin{align*}
		\int_\Omega|x-y|^{1+1/q-n}|Du(y)|\dd y 
			&= \int_\Omega \rho^{1+1/q-n}|Du(y)|\dd y,\quad \rho = |x-y|\\
			&= \int_0^d\rho^{1+1/q-n}v'(\rho)\dd\rho,\quad d=diam \Omega\\
			&= d^{1+1/q-n}v(d) + (n-1-1/q)\int_0^d\rho^{1/q-n}v(\rho)\dd\rho\\
			&\leq Kd^{1/q} + (n-1-1/q)\int_0^d\rho^{1/q-1}\dd\rho\\
			&\leq q(n-1)Kd^{1/q},
	\end{align*}
	For, choose $R_0>0$ so that $|\Omega| = |B_{R_0}(x)| = \omega_nR^n$. Then
	\begin{align*}
		\int_\Omega |x-y|^{1/q-n}\dd y 
			&\leq \int_{B_{R_0}(x)}|x-y|^{1/q-n}\dd y\\
			&= q\omega_nR_0^{1/q} = q\omega_n^{1-1/(qn)}|\Omega|^{1/(qn)}.
	\end{align*}
	Then it follows that %% by adapting the usual proof of the Young inequality for convolutions
	\begin{align*}
		\int_\Omega\int_\Omega|x-y|^{1/q-n}|Du(y)|\dd y \dd x 
		 &\leq \int_\Omega\int_\Omega |x-y|^{1/q-n}\dd x |Du(y)|\dd y\\
		 &\leq q\omega_nd^{1/q}\|Du\|_1.
	\end{align*}
	Hence 
	\begin{align*}
		\left\|\int_\Omega |x-y|^{1-n}|Du(y)|\dd y\right\|_q^q
			&\leq \int_\Omega\int_\Omega|x-y|^{1/q-n}|Du(y)|\dd y
			\cdot \left(\int_\Omega |x-y|^{1/q+1-n}|Du(y)|\dd y\right)^{q-1}\dd x\\
			&\leq q^q(n-1)^{q-1}K^{q-1}d\omega_n\|Du\|_1.
	\end{align*}
	Consequently,
	\begin{align*}
		\int_\Omega\sum_{m=0}^N \frac{\sigma_0^m|u-u_\Omega|^m}{m!K^m}
			&\leq d\omega_n\|Du\|_1 \sum_{m=0}^N \frac{\sigma_0^md^{nm}m^m(n-1)^{m-1}K^{m-1}}{n^m|\Omega|^mm!K^m}
			\leq C,
	\end{align*}
	if $\sigma_0>|\Omega|/(nd^n)$.
\end{proof}

\section{Moser's Iteration for elliptic equation}
\begin{theorem}[\cite[Theorem~8.15]{Gilbarg2001}]
	Let $u$ be a $W^{1,2}$ solution of 
	\[
		-\Delta u = \partial_if^i + g\quad\text{in }\Omega
	\]
	satisfying $u=0$ on $\partial\Omega$,
	in the sense of
	\begin{equation}\label{eq: formula of weak solutions}
		\int_\Omega \nabla u\cdot\nabla v = - \int_\Omega f^i\partial_iv + \int_\Omega gv 
		\quad\text{for all }v\in C^1_0(\Omega).
	\end{equation}
	Suppose that $f^i\in L^q(\Omega)$, $i=1,2,\cdots,n$, $g\in L^{q/2}(\Omega)$ for some $q>n$.
	Then 
	\[
		\sup_\Omega u \leq C\|u\|_2 + Ck,
	\]
	where $k=\lambda^{-1}(\|\mathbf{f}\|_q + \|g\|_{q/2})$ 
	and $C=C(n,q,|\Omega|)$.
\end{theorem}
\begin{proof}
	For $\beta\geq1$ and $N>k$, let us define a function $H\in C^1([k,\infty))$ by setting
	$H(z) = z^\beta - k^\beta$ for $z\in[k,N]$ and taking $H$ to be linear for $z\geq N$. 
	Let us next set $w=u+k$ and take
	\begin{equation*}
		v = G(w) = \int_k^w|H'(s)|^2\dd s 
	\end{equation*} 
	in the integral equality\eqref{eq: formula of weak solutions}. 
	By the chain rule, $v$ is a legitimate test function in \eqref{eq: formula of weak solutions}
	and on substitution we obtain,
	\begin{align*}
		\int_\Omega |\nabla w|^2 G'(w) &= - \int_\Omega f^i\partial_iwG'(w) 
			+ \int_\Omega G(w)g\\
			&\leq \frac{1}{2}\int_\Omega |\nabla w|^2G'(w) + \int_\Omega |\mathbf{f}|^2G'(w)
				+ \int_\Omega gwG'(w), 
	\end{align*}
	since $G(s)\leq sG'(s)$ and $Du = Dw$ when $v = G(w) >0$. Hence, we obtain,
	\[
		\int_\Omega |\nabla w|^2 G'(w) \leq 6\int_\Omega b G'(w)w^2,
	\]
	with 
	\[
		b = |f|^2/k^2 + |g|/k,
	\]
	that is,
	\[
		\int_\Omega |DH(w)|^2\dd x \leq 6\int_\Omega b|H'(w)w|^2\dd x.
	\]
	Since $H(w)\in W^{1,2}_0(\Omega)$, we may apply the Sobolev inequality and the H\"older inequality to obtain 
	\begin{align*}
		\|H(w)\|_{2n/(n-2)}&\leq C\left(\int_\Omega b(H'(w)w)^2\dd x\right)^{1/2}\\
		&\leq C\|b\|_{q/2}^{1/2} \|H'(w)w\|_{2q(q-2)}.
	\end{align*}
	To proceed further, we recall the definition of $H$ and let $N\to\infty$ in the estimate above.
	It follows then, for any $\beta\geq1$, 
	that the inclusion $w\in L^{2\beta q/(q-2)}(\Omega)$ implies the stronger inclusion, $w\in L^{2\beta n/(n-2)}(\Omega)$, 
	and moreover, setting $q^\ast = 2q/(q-2)$, $\chi=n(q-2)/q(n-2)>1$, we obtain
	\[
		\|w\|_{\beta\chi q^\ast} \leq (C\beta)^{1/\beta}\|w\|_{\beta q^\ast}.
	\] 
	The result is now obtained by iteration of the estimate above. 
	Namely, by induction, we may assume $w\in\cap_{1\leq p<\infty}L^p(\Omega)$. 
	Let us take $\beta = \chi^m$, $m=0,1,2,\cdots$, so that by the estimate above,
	\begin{align*}
		\|w\|_{\chi^N q^\ast} &\leq \Pi_0^{N-1}(C\chi^{m})^{\chi^{-m}}\|w\|_{q^\ast}\\
			&\leq C^{\sigma}\chi^\tau\|w\|_{q^\ast},\quad \sigma = \sum_0^{N-1}\chi^{-m},
				\quad \tau = \sum_0^{N-1}m\chi^{-m}\\
			&\leq C\|w\|_{q^\ast},
	\end{align*}
	where $C=C(n,q,|\Omega|)$. Letting $N\to\infty$, we therefore obtain
	\[
		\sup_\Omega w \leq C\|w\|_{q^\ast},
	\]
	whence by the interpolation inequality we have
	\[
		\sup_\Omega w \leq C\|w\|_2.
	\]
	The desired estimate follows from the definition $w = u + k$.
\end{proof}

The above technique of iteration of $L^p$ norms was introduced by Moser~\cite{Moser1960}.
The proof may also be effected by other choices of test functions.

\subsection{local properties of weak solutions}

Boardly speaking, the scheme of the joint proof follows the Moser iteration method (see \cite{Moser1961}) introduced in the previous section combined with the John-Nirenberg result, which is employed to bridge a vital gap in the iteration scheme.

\begin{lemma}
	if $u$ is a $W^{1,2}(\Omega)$ solution of equation \eqref{eq: formula of weak solutions} in $\Omega$,
	for any ball $B_{2R}(y)\subset\Omega$ and $p>1$,
	\[
		\sup_{B_R(y)} u \leq CR^{-n/p}\|u\|_{p, B_{2R}(y)} + Ck(R),
	\]
	where $C=C(n, \Lambda/\lambda, R, q, p)$.
\end{lemma}
\begin{proof}
	We assume initially that $R=1$ and $k>0$. The general case is later recovered through a simple coordinate transformation: $x\mapsto x/R$, and by letting $k$ tend to zero.
	Let us define, for $\beta\neq0$ and non-negative $\eta\in C^1_0(B_4)$, the test function
	\[
		v = \eta^2\overline{u}^\beta, \quad(\overline{u} = u + k).
	\]
	By the chain and product rules, $v$ is a valid test function in \eqref{eq: formula of weak solutions} 
	and also
	\[
		Dv = 2\eta D\eta \overline{u}^\beta + \beta\eta^2\overline{u}^{\beta-1}Du,
	\]
	so that by substitution into \eqref{eq: formula of weak solutions} we obtain
	\begin{align*}
		\beta\int_\Omega \eta^2 \overline{u}^{\beta-1}|Du|^2
			+ 2\int_\Omega \eta D\eta\cdot Du \overline{u}^\beta 
			= - \beta\int_\Omega \eta^2f^iu_i\overline{u}^{\beta-1}
			- 2\int_\Omega \eta \eta_i\cdot f^i \overline{u}^\beta
			+ \int_\Omega \eta^2\overline{u}^\beta g. 
	\end{align*}
	We can estimate, for any $0<\varepsilon\leq1$,
	\begin{align*}
		- \int_\Omega \eta^2f^iu_i\overline{u}^{\beta-1}
		&\leq \frac{\varepsilon}{2}\int_\Omega \eta^2 \overline{u}^{\beta-1}|Du|^2 
			+ \frac1\varepsilon\int_\Omega \eta^2 \overline{u}^{\beta-1}|\mathbf{f}|^2,
	\end{align*}
	By choosing $\varepsilon = \min\{1,|\beta|<4\}$, we then obtain from estimates above
	\begin{equation*}
		\int_\Omega \eta^2\overline{u}^{\beta-1}|Du|^2\dd X
		\leq C(\beta)\int_\Omega (b\eta^2 + |D\eta|^2)\overline{u}^{\beta +1}\dd x,
	\end{equation*}
	where $C(|\beta|)$ is bounded if $|\beta|$ is bounded away from zero.
	It is now convenient to introduce a function $w$ defined by
	\begin{equation*}
		w = 
		\begin{cases}
			\overline{u}^{(\beta+1)/2}, & \text{if } \beta\not=-1,\\
			\log\overline{u}, & \text{if }\beta = -1.
		\end{cases}
	\end{equation*}
	Letting $\gamma = \beta + 1$, we may rewrite the estimate above
	\begin{equation}
		\label{eq: local gradient estimates}
		\int_\Omega |\eta Dw|^2 \dd x \leq
		\begin{cases}
			C(|\beta|)\gamma^2\int_\Omega (b\eta^2 + |D\eta|^2)w^2\dd x, & \text{if }\beta\neq-1,\\
			C\int_\Omega (b\eta^2 + |D\eta|^2)\dd x, & \text{if }\beta = -1.
		\end{cases}
	\end{equation}
	The desired iteration process can now be developed from the first part of the estimate above.
	For from the Sobolev inequality we have
	\[
		\|\eta w\|^2_{2n/(n-2)} \leq C\int_\Omega (|\eta Dw|^2 + |wD\eta|^2)\dd x.
	\]
	Using the H\"older inequality followed by the interpolation inequality, we obtain, for any $\varepsilon>0$,
	\begin{align*}
		\int_\Omega b(\eta w)^2\dd x 
			&\leq \|b\|_{q/2}\|\eta w\|_{2q/(q-2)}^2\\
			&\leq \|b\|_{q/2}(\varepsilon\|\eta w\|_{2n/(n-2)} + \varepsilon^{-\sigma}\|\eta w\|_2)^2
	\end{align*}
	where $\sigma = n/(q-n)$. 
	Hence, by substitution into the last second estimate and appropriate choice of $\varepsilon$, 
	we obtain
	\begin{equation*}
		\|\eta w\|_{2n/(n-2)} \leq C(1+|\gamma|)^{\sigma+1}\|(\eta + |D\eta|)w\|_2,
	\end{equation*}
	where $C=C(n,q,|\beta|)$ is bounded when $|\beta|$ is bounded away from zero.
	It is now desirable to specify the cut-off function $\eta$ more precisely. 
	Let $r_1$, $r_2$ be such that $1\leq r_1 < r_2 \leq 3$ and set $\eta\equiv 1$ in $B_{r_1}$,
	$\eta\equiv0$ in $\Omega-B_{r_2}$ with $D\eta\leq 2/(r_2-r_1)$.
	Writing $\chi=n/(n-2)$ we then have from the estimate above
	\[
		\|w\|_{L^{2\chi}(B_{r_1})} \leq \frac{C(1+|\gamma|)^{\sigma + 1}}{r_2-r_1}\|w\|_{L^2(B_{r_2})}.
	\]
	For $r<4$ and $p\neq0$, let us now introduce the quantities
	\begin{equation}
		\Phi(p,r) = \left(\int_{B_r}|\overline{u}|^p\right)^{1/p}.
	\end{equation}
	Then we have
	\[
		\Phi(\infty, r) = \lim_{p\to\infty} \Phi(p,r) = \sup_{B_r}\overline{u},
	\]
	and 
	\[
		\Phi(-\infty, r) = \lim_{p\to -\infty}\Phi(p,r) = \inf_{B_r}\overline{u}.
	\]
	From inequality above, we now obtain
	\begin{align*}
		\Phi(\chi\gamma, r_1) &\leq \left(\frac{C(1+|\gamma|)^{\sigma+1}}{r_2-r_1}\right)^{2/|\gamma|}\Phi(\gamma,r_2)\quad \text{if } \gamma > 0,\\
		\Phi(\gamma, r_2) &\leq \left(\frac{C(1+|\gamma|)^{\sigma+1}}{r_2-r_1}\right)^{2/|\gamma|}\Phi(\chi\gamma,r_1)\quad \text{if } \gamma < 0.
	\end{align*}
	These inequalities can now be iterated to yield the desired estimates. 
	For example, taking $p>1$, we set $\gamma = \gamma_m = \chi^mp$ and $r_m = 1 + 2^{-m}$, $m = 0, 1, \cdots$,
	so that, 
	\begin{align*}
		\Phi(\chi^mp,1) &\leq (C\chi)^{2(1+\sigma)\sum m\chi^{-m}}\Phi(p, 2)\\
			& = C\Phi(p,2), \quad C = C(n,q,p).	
	\end{align*}
	Consequently, letting $m$ tend to infinity, we have
	\[
		\sup_{B_1}\overline{u} \leq C\|\overline{u}\|_{L^p(B_2)},
	\]
	and, by means of the transformation: $x\mapsto x/R$, the desired estimate is established.
\end{proof}

\begin{lemma}
	if $u$ is a $W^{1,2}(\Omega)$ solution of equation \eqref{eq: formula of weak solutions} in $\Omega$,
	non-negative in a ball $B_{4R}(y)\subset\Omega$ and $1\leq p<n/(n-2)$, 
	\[
		R^{-np}\|u\|_{L^p(B_{2R}(y))} \leq C\inf_{B_R(y)}u + Ck(R),
	\]
	where $C=C(n, \Lambda/\lambda, R, q, p)$.
\end{lemma}
\begin{proof}
	When $\beta<0$ and $\gamma<1$, we may prove in a similar manner, for any $p$, $p_0$ such that 
	$0<p_0 < p < \chi$,
	\begin{align*}
		&\Phi(p,2)\leq C\Phi(p_0,3)\\
		&\Phi(-p_0, 3) \leq C\Phi(-\infty, 1),\quad C = C(q,p,p_0).
	\end{align*}
	The conclusion of this lemma will thus follow if we can show that, for some $p_0 > 0$,
	\[
		\Phi(p_0, 3) \leq C\Phi(-p_0, 3).
	\]
	In order to establish the estimate above, we turn to the second of the estimates \eqref{eq: local gradient estimates}.
	Let $B_{2r}$ be any ball of radius $2r$, lying in $B_4(=B_4(y))$, and choose the cut-off function $\eta$
	so that $\eta\equiv1$ in $B_r$, $\eta\equiv0$ in $\Omega-B_4$ and $|D\eta\leq2/r$.
	From \eqref{eq: local gradient estimates}, with the aid of the H\"older inequality, we then obtain
	\begin{align*}
		\int_{B_r}|Dw|\dd x &\leq Cr^{n/2}\left(\int_{B_r}|Dw|^2\dd x\right)^{1/2}\\
			&\leq Cr^{n-1},\quad C=C(n).
	\end{align*}
	Hence, by Theorem~\ref{thm: bmo estimates}, there exists a constnat $p_0>0$ depending on $n$ such that,
	for 
	\[
		w_0 = \frac{1}{|B_3|}\int_{B_3}w\dd x,
	\]
	we have 
	\[
		\int_{B_3}e^{p_0|w-w_0|}\dd x \leq C(n, |\Omega|),
	\]
	and thus
	\[
		\int_{B_3}e^{p_0w}\dd x \int_{B_3}e^{-p_0w}\dd x 
			\leq C e^{p_0w_0}e^{-p_0w_0} = C.
	\]
	Recalling the definition of $w$, we obtain the desired estimate. 
	The full result then follows by means of the transformation: $x\mapsto x/R$ and by letting $k$ tend to zero.
\end{proof}

\subsection{Harnack inequality}

By combining two lemmas above, we obtain the full Harnack inequality.

\begin{theorem}
	Let $u\in W^{1,2}(\Omega)$ satisfy $u\geq0$ in $\Omega$ and 
	\[
		-\Delta u = \partial_if^i + g,\quad \text{in }\Omega.
	\]
	Then for any ball $B_{4R}(y)\subset\Omega$, 
	\[
		\sup_{B_R(y)}u \leq C\inf_{B_R(y)}u,
	\]
	where $C=C(n,|\Omega|,R)$.
\end{theorem}

\subsection{H\"older estimate}

\begin{theorem}
	Suppose that $f^i\in L^q(\Omega)$, $i=1, \cdot, n$, $g\in L^{q/2}(\Omega)$ for some $q>n$.
	Then if $u$ is a $W^{1,2}$ solution of the equation
	\[
		-\Delta u = D_if^i + g,\quad\text{in }\Omega,
	\]
	it follows that $u$ is locally H\"older continuous in $\Omega$, 
	and for any ball $B_0=B_{R_0}(y)\subset\Omega$ and $R\leq R_0$, we have
	\[
		osc_{B_R(y)}u\leq CR^{\alpha}(R_0^{-\alpha}\sup_{B_0}|u| + k),
	\]
	where $C=C(n,q,R_0)$ and $\alpha=\alpha(n,q,R_0)$ are positive constants, 
	and $k=\|f\|_q + \|g\|_{q/2}$. 
\end{theorem}

\begin{proof}
	We may assume without loss of generality that $R\leq R_0/4$.
	Let us write $M_0=\sup_{B_0}|u|$, $M_4 = \sup_{B_{4R}}u$, $m_4 = \inf_{B_{4R}}u$,
	$M_1=\sup_{B_R}u$, $m_1 = \inf_{B_R}u$.
	Then we have 
	\begin{gather*}
		-\Delta(M_4-u) = -D_if^i - g,\\
		-\Delta(u-m_4) = D_if^i + g.
	\end{gather*}
	Hence, if we set
	\begin{align*}
		K(R) &= R^\delta\|f\|_q + R^{2\delta}\|g\|_{q/2},\\
		\delta &= 1-n/q 
	\end{align*}
	and apply the weak Harnack inequality with $p=1$ to the functions 
	$M_4-u$, $u-m_4$ in $B_{4R}$, we obtain
	\begin{gather*}
		R^{-n}\int_{B_{2R}}(M_4-u)\dd x \leq C(M_4 - M_1 + K(R)),\\
		R^{-n}\int_{B_{2R}}(u-m_4)\dd x \leq C(m_1-m_4) + K(R).
	\end{gather*}
	Hence by addition,
	\[
		M_4 - m_4 \leq C(M_4 - m_4 + m_1 - M_1 + K(R)),
	\]
	so that, writing 
	\[
		\omega(R) = osc_{B_R} u = M_1 - m_1,
	\]
	we have 
	\[
		\omega(R) \leq \gamma\omega(4R) + K(R),
	\]
	where $\gamma = 1- C^{-1}$, $C = C(n,R_0,q)$.
	The following simple lemma then implies the desired result.
\end{proof}

\begin{lemma}
	Let $\omega$ be a non-decreasing function on an interval $(0,R_0]$ satisfying,
	for all $R\leq R_0$, the inequality 
	\[
		\omega(\tau R) \leq \gamma \omega(R) + \sigma(R),
	\]
	where $\sigma$ is also non-decreasing and $0<\gamma, \tau < 1$.
	Then, for any $\mu\in(0,1)$ and $R\leq R_0$,
	we have 
	\[
		\omega(R)\leq C\left(\left(\frac{R}{R_0}\right)^\alpha \omega(R_0) + \sigma(R^\mu R_0^{1-\mu})\right),
	\]
	where $C=C(\gamma, \tau)$ and $\alpha=\alpha(\gamma, \tau, \mu)$ are positive constants.
\end{lemma}

\begin{proof}
	Let us fix initially some number $R_1\leq R_0$.
	Then for any $R\leq R_1$ we have
	\[
		\omega(\tau R) \leq \gamma \omega(R) + \sigma(R_1),
	\]
	since $\sigma$ is non-decreasing. 
	We now iterate this inequality to get,
	for any positive integer $m$,
	\begin{align*}
		\omega(\tau^mR_1)
			&\leq \gamma^m\omega(R_1) + \sigma(R_1)\sum_{i=0}^{m-1}\gamma^i\\
			&\leq \gamma^m\omega(R_0) + \frac{\sigma(R_1)}{1-\gamma}.
	\end{align*}	
	For any $R\leq R_1$, we can choose $m$ such that
	\[
		\tau^mR_1<R\leq\tau^{m-1}R_1.
	\]
	Hence
	\begin{align*}
		\omega(R) &\leq \omega(\tau^{m-1}R_1)\\
			&\leq \gamma^{m-1}\omega(R_0) + \frac{\sigma(R_1)}{1-\gamma}\\
			&\leq \frac{1}{\gamma}\left(\frac{R}{R_1}\right)^{\log\gamma/\log\tau}\omega(R_0)
				+ \frac{\sigma(R_1)}{1-\gamma}.
	\end{align*}
	Now let $R_1=R_0^{1-\mu}R^{\mu}$ so that we have from the preceding
	\[
		\omega(R) \leq \frac{1}{\gamma}\left(\frac{R}{R_0}\right)^{(1-\mu)\log\gamma/\log\tau}\omega(R_0)
		+ \frac{\sigma(R_0^{1-\mu}R^\mu)}{1-\gamma}.
	\]
\end{proof}


\section{Moser's iteration for parabolic equation}

\begin{lemma}
	suppose $u\in C^0(\overline{\Omega}\times[0,T))\cap C^{2,1}(\Omega\times[0,T))$ satisfy 
\begin{equation*}
	\begin{cases}
		u_t \leq \Delta u + \nabla\cdot F, & \Omega,\\
		\nabla u\cdot n \leq 0, \quad F\cdot n \leq 0, & \partial\Omega,
	\end{cases}
\end{equation*}
if 
\[
	f\in L^\infty((0,T); L^q(\Omega)),\quad q>n,
\]
then
$u\in L^\infty(\Omega\times(0,T))$.
\end{lemma}
\begin{proof}
	\begin{align*}
		\frac{\dd}{\dd t}\int u^p &+ p(p-1)\int u^{p-1}|\nabla u|^2 
		\leq - p(p-2)\int u^{p-2}\nabla u\cdot F\\
		&\leq p(p-1)\left(\int u^{\frac{q(p-2)}{q-1}}|\nabla u|^{\frac{q}{q-1}}\right)^{\frac{q-1}{q}}
			\left(\int F^q\right)^{1/q}\\
		&= p(p-1)\left(\int \left(u^{\frac{p-2}{2}}|\nabla u|\right)^{\frac{q}{q-1}}
			\cdot u^{\frac{q(p-2)}{2(q-1)}}\right)^{\frac{q-1}{q}}\|F\|_q\\
		&\leq p(p-1)\left(\int u^{p-2}|\nabla u|^2\right)^{\frac12}
			\cdot\left(\int u^{\frac{q(p-2)}{q-2}}\right)^{\frac{q-2}{2q}}\|F\|_q,
	\end{align*}
	\textbf{integrability.} using
	\begin{align*}
		\left(\int u^{\frac{q(p-2)}{q-2}}\right)^{\frac{q-2}{2q}}
		&= \|u^{\frac{p}{2}}\|_{\frac{2q(p-2)}{p(q-2)}}^{\frac{p-2}{p}}\\
		&\leq C\|\nabla u^{\frac{p}{2}}\|_2^{\frac{(p-2)\theta}{p}}\cdot\|u^{\frac{p}{2}}\|_2^{\frac{(1-\theta)(p-2)}{p}} + C\|u^{\frac{p}{2}}\|_2^{\frac{p-2}{p}},
	\end{align*}
	with 
	\[
		\frac{p(q-2)}{2q(p-2)} = \theta\left(\frac12-\frac1n\right) + 1-\theta,
	\]
	noting $p>q>n$ ensures that
	\[
		0<\gamma := \frac{(p-2)\alpha}{p} = \frac{n}{q} - \frac{n}{p} < 1,
	\]
	we have 
	\begin{align*}
		\frac{\dd}{\dd t}\int u^p &+ \frac{4(p-1)}{p}\int |\nabla u^{\frac{p}{2}}|^2
		\leq Cp(p-1)\left(\int u^{p-2}|\nabla u|^2\right)^{\frac12+\frac\gamma2}\cdot\|u^{\frac{p}{2}}\|_2^{\frac{p-2}{p} - \gamma}\cdot\|F\|_q \\
		&\quad +Cp(p-1) \left(\int u^{p-2}|\nabla u|^2\right)^{\frac12}\cdot\|u^{\frac{p}{2}}\|_2^{\frac{p-2}{p}}\cdot\|F\|_q \\
		&\leq \frac{2(p-1)}{p}\int |\nabla u^{\frac{p}{2}}|^2 
			+ C(F)p^2\|u^{\frac{p}{2}}\|_2^{\frac{2(p-2)}{p(1-\gamma)}-\frac{2\gamma}{1-\gamma}}
			+ C(F)p^2\|u^{\frac{p}{2}}\|_2^{\frac{2(p-2)}{p}},
	\end{align*}
	and obtain $\|u\|_p\leq C(p)$.

	\textbf{boundedness.}
	\begin{align*}
		\left(\int u^{\frac{q(p-2)}{q-2}}\right)^{\frac{q-2}{2q}}
		&= \|u^{\frac{p}{2}}\|_{\frac{2q(p-2)}{p(q-2)}}^{\frac{p-2}{p}}\\
		&\leq C\|\nabla u^{\frac{p}{2}}\|_2^{\frac{(p-2)\alpha}{p}}\cdot\|u^{\frac{p}{2}}\|_1^{\frac{(1-\alpha)(p-2)}{p}} + C\|u^{\frac{p}{2}}\|_1^{\frac{p-2}{p}},
	\end{align*}
	with 
	\[
		\frac{p(q-2)}{2q(p-2)} = \alpha\left(\frac12-\frac1n\right) + 1-\alpha,
	\]
	noting $q>n$ ensures that
	\[
		\kappa := \frac{(p-2)\alpha}{p} = \frac{\frac{p-2}{p}-\frac{q-2}{2q}}{\frac12+\frac1n} <\gamma < 1,
		\quad p > 1,
	\]
	we get
	\begin{align*}
		\frac{\dd}{\dd t}\int u^p &+ \frac{4(p-1)}{p}\int |\nabla u^{\frac{p}{2}}|^2
		\leq Cp(p-1)\left(\int u^{p-2}|\nabla u|^2\right)^{\frac12+\frac\kappa2}\cdot\|u^{\frac{p}{2}}\|_1^{\frac{p-2}{p} - \kappa}\cdot\|F\|_q \\
		&\quad +Cp(p-1) \left(\int u^{p-2}|\nabla u|^2\right)^{\frac12}\cdot\|u^{\frac{p}{2}}\|_1^{\frac{p-2}{p}}\cdot\|F\|_q \\
		&\leq \frac{2(p-1)}{p}\int |\nabla u^{\frac{p}{2}}|^2 
			+ C(F)p^2\|u^{\frac{p}{2}}\|_1^{\frac{2(p-2)}{p(1-\kappa)}-\frac{2\kappa}{1-\kappa}}
			+ C(F)p^2\|u^{\frac{p}{2}}\|_1^{\frac{2(p-2)}{p}}.
	\end{align*}
	Using 
	\[
		\|f\|_2\leq C\|\nabla f\|_2^{\frac{n}{n+2}}\cdot\|f\|_1^{\frac{2}{n+2}} + C\|f\|_1,\quad f\in W^{1,2}(\Omega),
	\]
	we arrive 
	\begin{align*}
		\frac{\dd}{\dd t}\int u^p &+ \int  u^{p}
			\leq Cp^2 \|u^{\frac{p}{2}}\|_1^{\frac{2(p-2)}{p(1-\kappa)}-\frac{\kappa}{1-\kappa}}
				+ Cp^2\|u^{\frac{p}{2}}\|_1^{\frac{2(p-2)}{p}}
				+ C\|u^{\frac{p}{2}}\|_1^2,
	\end{align*}
	Let
	\[
		M_j(t) := \int u^{2^j} + 1, \quad j\geq1,
	\]
	noting 
	\[
		\frac{2(p-2)}{p(1-\kappa)}-\frac{\kappa}{1-\kappa} < 2,\quad p>2,
	\]
	we have 
	\[
		M_j' + M_j \leq C4^jM_{j-1}^2,
	\]
	which implies 
	\begin{align*}
		N_j := \sup_{t\in(0,T)}M_j \leq \max\left\{ M_j(0), C4^jM_{j-1}^2\right\},
	\end{align*}
	and hence
	\[
		N_j \leq \max\left\{ M_j(0), C4^jN_{j-1}^2\right\}.
	\]
	If there exist countably many $j$ such that $N_j \leq M_j(0)$, 
	then $\|u\|_\infty \leq \|u_0\|_\infty + 1$.
	On the other hand, there exists $j_0$ such that for any $j>j_0$, 
	\[
		N_j \leq C4^jN_{j-1}^2,
	\]
	it is no harm to assume that $j_0=1$ since we can appropriately enlarge the value of $C$.
	Therefore,
	\[
		N_j\leq C^{\sum_{i=1}^j2^{i-1}}4^{\sum_{i=1}^{j}(j+1-i)2^{i-1}}N_1^{2^{j-1}}
			\leq C^{2^j}4^{2^{j+1}}N_1^{2^{j-1}},
	\]
	and 
	\[
		\|u\|_\infty\leq\limsup_{j\to\infty}N_j^{2^{-j}}
			\leq\limsup_{j\to\infty}C4^2N_1^{2^{-1}}=16C\sqrt{N_1}.
	\]
\end{proof}

\section{De Giorgi iteration}

\begin{lemma}
	suppose $u\in C^0(\overline{\Omega}\times[0,T))\cap C^{2,1}(\Omega\times[0,T))$ satisfy 
\begin{equation*}
	\begin{cases}
		u_t \leq \Delta u + \nabla\cdot F, & \Omega,\\
		\nabla u\cdot n \leq 0, \quad F\cdot n \leq 0, & \partial\Omega,
	\end{cases}
\end{equation*}
if 
\[
	f\in L^p(\Omega)\times(0,T)),\quad p>n+2,
\]
then
$u\in L^\infty(\Omega\times(0,T))$.
\end{lemma}

\begin{proof}
	\begin{align*}
		-\int(u-k)_+^2(0) + \int (u-k)_+^2
		&= \iint (u-k)_+u_t = \iint (u-k)_+\Delta u + \iint(u-k)_+\nabla\cdot F\\
		&\leq - \iint |\nabla(u-k)_+|^2 - \iint \nabla(u-k)_+\cdot F\\
		&\leq - \iint |\nabla(u-k)_+|^2 
			- \left(\iint |\nabla(u-k)_+|^2\right)^{\frac12}
			\cdot\left(\iint |F|^p\right)^{\frac1p}
			\cdot |\{u\geq k\}|^{\frac{1}{2}-\frac1p},
	\end{align*}
using 
\[
	\|f\|_{\frac{2(n+2)}{n}}\leq C\|\nabla f\|_2^{\frac{n}{n+2}}\cdot\|f\|_2^{\frac{2}{n+2}} 
		+ C\|f\|_2,	\quad f\in W^{1,2},
\]
we have
\begin{align*}
	\left(\iint (u-k)_+^{\frac{2(n+2)}{n}}\right)^{\frac{n}{n+2}}
	\leq \int(u-k)_+^2(0) + \|F\|_p^2\cdot|\{u\geq k\}|^{1-\frac2p}.
\end{align*}
fix $l$ sufficiently large such that 
\[
	\int (u-l)_+^2(0) = 0,
\]
and noting that for any $h>k$,
\[
	\left(\iint (u-k)_+^{\frac{2(n+2)}{n}}\right)^{\frac{n}{n+2}}
	\geq (h-k)^2|\{u\geq h\}|^{\frac{n}{n+2}},
\]
denote 
\[
	\psi(\eta) = |\{u\geq \eta\}|,
\]
we get
\[
	\psi(h) \leq \frac{C(F)\psi(k)^{\frac{n+2}{n}\cdot\frac{p-2}{p}}}{(h-k)^{\frac{2(n+2)}{n}}},
	\quad h>k>l.
\]
Here, $p>n+2$ gurantees that 
\[
	\frac{n+2}{n}\cdot\frac{p-2}{p} > 1,
\]
which implies the desired assertation by the following lemma.
\end{proof}

\begin{lemma}
	if there exist $\varepsilon>0$, $p>0$, $c>0$ and $s_0>0$ such that
	for each $b>a>s_0$, the non-increasing function $f:\mathbb{R}^+\mapsto\mathbb{R}^+$ has the following property,
	\[
		f(b) \leq \frac{cf(a)^{1+\varepsilon}}{(b-a)^p},
	\]
	then there exists $d>0$ such that $f(s_0 + d) = 0$.
\end{lemma}
\begin{proof}
	let
	\[
		b_k = s_0+d-\frac{d}{2^k},
	\]
	then 
	\begin{align*}
		f(b_k) &\leq \frac{cf^{1+\varepsilon}(b_{k-1})}{\left(\frac{d}{2^k}\right)^p}
			= \frac{c}{d^p}2^{pk}f^{1+\varepsilon}(b_{k-1})\\
			&\leq  \frac{c}{d^p}2^{pk}\left(\frac{c}{d^p}2^{p(k-1)}f^{1+\varepsilon}(b_{k-2})\right)^{1+\varepsilon}\\
			&= \left(\frac{c}{d^p}\right)\cdot \left(\frac{c}{d^p}\right)^{1+\varepsilon}
				\cdot 2^{pk + p(k-1)(1+\varepsilon)}f^{(1+\varepsilon)^2}(b_{k-2})\\
			&= \left(\frac{c}{d^p}\right)^{\sum_{j=0}^{k-1}(1+\varepsilon)^j}
				\cdot 2^{p\sum_{j=0}^k(k-j)(1+\varepsilon)^j}
				\cdot f^{(1+\varepsilon)^k}(b_0)\\
			&\leq \left(\frac{c}{d^p}\right)^{\frac{(1+\varepsilon)^k-1}{\varepsilon}}
				\cdot 2^{\frac{p(1+\varepsilon)^k}{\varepsilon^2}-\frac{1+\varepsilon}{\varepsilon^2}-\frac{k}{\varepsilon}}
				\cdot f^{(1+\varepsilon)^k}(b_0)\\
			&= \left(\left(\frac{c}{d^p}\right)^{\frac{1}{\varepsilon}}
				\cdot 2^{\frac{p}{\varepsilon^2}}
				\cdot f(b_0)\right)^{(1+\varepsilon)^k},
	\end{align*}
	which implies $f(s_0 + d) = 0$, if $d$ is chosen so large that 
	\[
		d > c^{\frac{1}{p}} 
			\cdot 2^{\frac{1}{\varepsilon}}
			\cdot f^{\frac{\varepsilon}{p}}(b_0).
	\] 
\end{proof}

\section{porous medium diffusion}
the chemotaxis model with proous medium diffusion ($m>1$)
\begin{equation}
	\begin{cases}
		u_t = \Delta u^m - \nabla \cdot ( u\nabla v), & (x,t)\in\Omega\times(0,T),\\
		\varepsilon v_t = \Delta v - \alpha v + u, & (x,t)\in\Omega\times(0,T),
	\end{cases}
\end{equation}
has a free energy functional
\begin{equation}
	\mathcal{F}_\varepsilon(u,v)(t) :=  \int_\Omega \frac{u^m-1}{m-1} - \int_\Omega uv 
		+ \frac{1}{2}\int_\Omega |\nabla v|^2 + \frac{\alpha}{2}\int_\Omega v^2,
\end{equation}
which satisfies 
\begin{equation}
	\frac{\dd}{\dd t}\mathcal{F}_\varepsilon = - \mathcal{D}_\varepsilon.
\end{equation}
Here, 
\begin{equation*}
	\mathcal{D}_\varepsilon(u,v) 
		= - \int_\Omega u\left|\nabla\left(\frac{m}{m-1}u^{m-1} - v\right)\right|^2
			- \varepsilon\int_\Omega v_t^2.
\end{equation*}

When $\varepsilon = 0$, due to
\[
	\int_\Omega |\nabla v|^2 + \alpha\int_\Omega v^2 = \int_\Omega uv,
\]
we have
\begin{equation}
	\mathcal{F}_0(u) = \int_\Omega \frac{u^m-1}{m-1} - \frac{1}{2}\int_\Omega uv 
		= \int_\Omega\frac{u^m-1}{m-1} - \frac{1}{2}\int_\Omega u(-\Delta + 1)^{-1}u.
\end{equation}
Thanks to properties of the green function \cite{Grueter1982},
we estimate
\begin{align*}
	\int_\Omega u(-\Delta+1)^{-1}u 
		&\leq C\int_\Omega u\cdot |x|^{n-2}*u\\
		&\leq C\|u\|_m\cdot \||x|^{2-n}*u\|_{\frac{m}{m-1}}\\
		&\leq C\|u\|_m \cdot \||x|^{2-n}\|_r\cdot\|u\|_s,\quad 1+\frac{m-1}{m} = \frac{1}{r} + \frac{1}{s},
			\quad (2-n)r + n-1 > -1,\\
		&\leq C\|u\|_m\cdot \||x|^{2-n}\|_r\cdot\|u\|_m^{\theta}\cdot\|u\|_1^{1-\theta},
			\quad \frac{1}{s} = \frac{\theta}{m} + \frac{1-\theta}{1}, \theta = \frac{s-1}{s}\frac{m}{m-1}\\
		&\leq C\|u\|_m^{1+\theta}\cdot \||x|^{2-n}\|_r\cdot\|u\|_1^{1-\theta}.
\end{align*}
If 
\[
	m > 2-\frac2n,
\]
we can choose appropriately $r$ such that 
\begin{align*}
	1+\theta 
		&= 1+\frac{s-1}{s}\cdot\frac{m}{m-1}\\
		&= 1 + \left(\frac1r-\frac{m-1}{m}\right)\cdot\frac{m}{m-1}\\
		&< 1 + \left(\frac{n-2}{n} - \frac{m-1}{m}\right)\cdot\frac{m}{m-1}\\
		&= \frac{n-2}{n}\frac{m}{m-1}\leq m,
\end{align*} 
and hence $\mathcal{F}_0(u) \geq - C(\lambda)$ with $\lambda := \int_\Omega u$.

\textbf{method 2.}
\begin{align*}
	\int_\Omega uv &\leq \|u\|_{\frac{2n}{n+2}}\cdot\|v\|_{\frac{2n}{n-2}}\\
		&\leq C\|u\|_{\frac{2n}{n+2}}\cdot\|v\|_{W^{1,2}}\\
		&\leq C\|u\|_m^\theta\cdot\|u\|_1^{1-\theta}\cdot\|v\|_{W^{1,2}},
			\quad \frac{n+2}{2n} = \frac{\theta}{m} + \frac{1-\theta}{1},\\
		&\leq \frac{1}{2}\|v\|_{W^{1,2}}^2 + C(\lambda)\|u\|_m^{2\theta}.
\end{align*}
If 
	\[
		m\geq 2-\frac2n,
	\]
we have 
\[
	2\theta = \frac{n-2}{n}\cdot\frac{m}{m-1}\leq m,
\]
and hence $\mathcal{F}_0(u) \geq - C(\lambda)$ with $\lambda$ appropriately small 
when $m=2-2/n$. 

Particularly, when $m=2-2/n$,
\[
	\int_\Omega uv \leq C\|u\|_m^m \|u\|_1^{2-m},
\]
we end up with 
\begin{equation}
	C_\ast := \sup_{f\neq0}\left\{\frac{\int_\Omega f(-\Delta+1)^{-1}f}{\|f\|_m^m\|f\|_1^{2-m}}: f\in L^m(\Omega)\cap L^1(\Omega) = L^m(\Omega)\right\} < \infty.
\end{equation}

Suppose $\Omega = B_R := \{x\in\mathbb{R}^n: |x|<R\}$. 
Then if $(u,v)$ is a solution of the stationary system
\begin{equation}
	\begin{cases}
		0 = \Delta u^m - \nabla\cdot(u\nabla v), & x\in B_R\\
		0 = \Delta v - \alpha v + u, & x\in B_R.
	\end{cases}
\end{equation}
Then $(\tilde{u}, \tilde{v}) = (R^nu(Rx), R^{n-2}v(Rx))$ is a solution of the following problem
\begin{equation*}
	\begin{cases}
		0 = \Delta \tilde{u}^m - \nabla\cdot(\tilde u\nabla \tilde v), & x\in B_1,\\
		0 = \Delta \tilde v - R^2\alpha \tilde v + \tilde u, & x\in B_1.
	\end{cases}
\end{equation*}

for $f\in L^p(\Omega)$, Schwarz Symmetrization~\cite{Kesavan2006}, or, the spherically symmetric and decreasing rearrangement is the function $f^*: \Omega \mapsto \mathbb{R}$ defined by 
\[
	f^\ast(x) = f^\sharp (V_n |x|^n),\quad x\in\Omega,
\]
where
\begin{equation*}
	f^\sharp(s) := 
	\begin{cases}
		\esssup(f),& s=0,\\
		\inf\{t:\mu_f(t) <s\}, & s>0,
	\end{cases}
	\quad \mu_f(t) = |\{|f|>t\}| = |\{x\in\Omega: f(x)>t\}|.
\end{equation*}
Here, $V_n$ denotes the volume of the unit ball in $\mathbb{R}^n$,
\[
	V_n=\frac{\pi^{\frac{n}{2}}}{\Gamma\left(\frac{n}{2}+1\right)},
\]
where $\Gamma(s)$ is the usual Gamma function.

Then for $f\in L^m(\Omega)$, 
\[
	\|f\|_1 = \|f^*\|_1, \quad \|f\|_m = \|f^*\|_m.
\]
Writing
\[
	\tilde G_\alpha(x) = G_\alpha(x)\chi_\Omega,\quad \tilde f = f\chi_\Omega, 
\]
where $G_\alpha$ is the usual green function of $-\Delta + \alpha I$ associated with homogeneous Neumann boundary conditions.
Then by Resiz rearrangement properties \cite[Lemma~2.1]{Lieb1983}
\begin{align*}
	\int_\Omega f(-\Delta + \alpha I)^{-1}f 
		&= \int_\Omega f \cdot G_\alpha * f\\
		&= \iint_{\mathbb{R}^n\times\mathbb{R}^n} \tilde{f}(x)\cdot\tilde{G}_\alpha(x-y)\tilde{f}(y)\\
		&\leq \iint_{\mathbb{R}^n\times\mathbb{R}^n} \tilde{f}^\ast(x)\cdot\tilde{G}^\ast_\alpha(x-y)\tilde{f}^\ast(y)\\
		&= \int_\Omega f^\ast\cdot G_\alpha^\ast*f^\ast.
\end{align*}

Since $G_\alpha\in C^{\infty}_{\mathrm{loc}}(\overline{\Omega}\setminus\{0\})$ solves
\begin{align*}
	-\Delta G_\alpha + \alpha G_\alpha &= \delta_0, \quad x\in\Omega,\\
	\frac{\partial G_\alpha}{\partial n} &= 0,\quad x\in\partial\Omega,
\end{align*}
we have $G_\alpha$ is positive and radially symmetric function, 
and integrating over $\Omega\setminus B_s$, for $s\in(0,1)$, get 
\begin{align*}
	0 &= \int_{\Omega\setminus B_s} \delta_0\dd x 
		= - \int_{\Omega\setminus B_s} \Delta G_\alpha + \alpha\int_{\Omega\setminus B_s} G_\alpha\\
		&= -\left(\int_{\partial\Omega}\frac{\partial G_\alpha}{\partial n}\dd S 
			- \int_{\partial B_s}\frac{\partial G_\alpha}{\partial n}\dd S\right) 
			+ \alpha\int_{\Omega\setminus B_s} G_\alpha\\
		&= \omega_n G_\alpha'(s)s^{n-1} + \alpha\int_{\Omega\setminus B_s} G_\alpha,
\end{align*}  
which implies $G_\alpha$ is radially decreasing, and therefore $G_\alpha = G_\alpha^\ast$.
We end up with 
\[
\int_\Omega f(-\Delta + \alpha I)^{-1}f\leq \int_\Omega f^\ast(-\Delta +\alpha I)^{-1}f^\ast.
\]
Define 
\[
\Lambda(f) := \frac{\int_\Omega f_j(-\Delta + \alpha I)^{-1}f_j}{\|f_j\|_m^m\cdot\|f_j\|_1^{2-m}},
	\quad f\in L^m(\Omega),
\]
and consider a maximising sequence $\{f_j\}$ in $L^m$, that is
\[
\Lambda(f_j) \to C_\ast.
\]
So we may assume without loss of generality,
$\{f_j\}$ is a family of nonnegative, radially symmetric and non-increasing functions such that $\|f_j\|_m=1$. 
Then for any $R\in(0,1)$,
\begin{align*}
	f_j(R) \leq \left(\frac{\int_{B_R}f_j^m}{|B_R|}\right)^{1/m}
		\leq CR^{-n/m}.
\end{align*}
Relying on Helly's compactness method, there exists a subsequence (still denoted by $f_j$) 
and a measurable function $f$ such that 
\[
f_j \to f, \quad \text{p.p. } x\in\Omega.
\]
By Fatou lemma, $f\in L^m(\Omega)$ and $\|f\|_m\leq\liminf\|f_j\|_m = 1$. 
Moreover, $\|f_j-f\|_1 \to 0$.
Using $f\in L^{2n/(n+2)}$ and Hardy-Littlewood-Sobolev inequality, by Lebesgue dominated convergence theorem,
we have 
\[
\int_\Omega f_j(-\Delta + \alpha I)^{-1}f_j \to \int_\Omega f (-\Delta + \alpha I)^{-1}f,
\]
and thus 
\[
C_\ast = \lim_{j\to\infty}\Lambda(f_j) \leq 
\]
\emph{how to rule out $\|f\|_1 = 0$?}





%\part{template manual}
\include{template-manual}

\printbibliography[heading=bibintoc, title=\ebibname]
\appendix

\chapter{The Fourier transform on $L^2$}

The Fourier transform of a function $f$ on $\mathbb R^d$ is defined by 
\begin{equation}
  \hat f(\xi) = \int_{\mathbb R^d}f(x)e^{-2\pi ix\cdot\xi}\dd x,
\end{equation}
and its attached inversion is given by 
\begin{equation}
  f(x) = \int_{\mathbb R^d}\hat f(\xi)e^{2\pi ix\cdot\xi}\dd \xi.
\end{equation}

These formulas have already appeared in several contexts.
We consider first the properties of the Fourier transform in the elementary setting 
by restricting to functions in the Schwartz class $\mathcal S(\mathbb R^d)$.
The class $\mathcal S$ consists of functions $f$ that are smooth (indefinitely differentiable)
and such that for each multi-index $\alpha$ and $\beta$, the function $x^\alpha\left(\frac{\partial}{\partial x} \right)^\beta f$ is bounded on $\mathbb R^d$.
We saw that on this class the Fourier transform is a bijection,
that the inversion formula holds,
and moreover we have the Plancherel identity 
\begin{equation}
  \int_{\mathbb R^d}|\hat f(\xi)|^2\dd\xi = \int_{\mathbb R^d}|f(x)|^2\dd x.
\end{equation}

Turning now to more general (in particular, non-continuous) functions,
we note that the largest class for which the integral defining $\hat f(\xi)$ converges (absolutely) is the space $L^1(\mathbb R^d)$. 
For it, we saw that a (relatively) inversion formula is valid, 
provided $\hat f\in L^1(\mathbb R^d)$. 
In this case, 
since $\hat f$ is continuous, bounded, and moreover decays to zero at infinity,
$f$ could be modified on a set of measure zero to become continuous everywhere,
which is of course impossible for the general function $f\in L^1(\mathbb R^d)$.

Beyond these particular facts, what we would like here is to reestablish in the general context the symmetry between $f$ and $\hat f$ that holds for $\mathcal S$.
This is where the special role of the Hilbert space $L^2(\mathbb R^d)$ enters.

We shall define the Fourier transform on $L^2(\mathbb R^d)$ as an extension of its definition on $\mathcal S$.
For this purpose, we temporarily adopt the notational device of denoting by $\mathcal F_0$ and $\mathcal F$ the Fourier transform on $\mathcal S$ and its extension to $L^2$, respectively.

The main results we prove are the following.

\begin{theorem}
  The Fourier transform $\mathcal F_0$, initially defined on $\mathcal S(\mathbb R^d)$,
  has a (unique) extension $\mathcal F$ to a unitary mapping of $L^2(\mathbb R^d)$ to itself.
  In particular,
  \[
  \|\mathcal F(f)\|_{L^2(\mathbb R^d)} = \|f\|_{L^2(\mathbb R^d)}
  \]
  for all $f\in L^2(\mathbb R^d)$.
\end{theorem}

The extension $\mathcal F$ will be given by a limiting process: if $\{f_n\}$ is a sequence in the Schwartz space that converges to $f$ in $L^2(\mathbb R^d)$, 
then $\{\mathcal F_0(f_n)\}$ will converge to an element in $L^2(\mathbb R^d)$ which we will define as the Fourier transform of $f$.



\end{document}
