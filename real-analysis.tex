\chapter{实变函数}







\section{集合}\index{Set}






\begin{example}
\hfill\\
非空完备集一定是不可数集。


设$E$是完备集,若$E=\{x_n\}$可数,$\forall B_0=B(y_0,r_0)$,$y_0\in E$,$r_0=1$,$\exists y_1\in E$,$0<r_1\leq\frac{1}{2}$,使$B_1=B(y_1,r)\subset B_0$,且$x_1\not\in\overline{B_1}$。对$B(y_1,r_1)$,$\exists y_2\in E$,$0<r_2\leq\frac{1}{2^2}$,使$B_2=B(y_2,r_2)\subset B_1$,且$x_2\not\in\overline{B_2}$。以此类推,$\exists y_n\in E$,$0<r_n\leq\frac{1}{2^n}$,使$B_n=B(y_n,r_n)\subset B_{n-1}$,且$x_n\not\in\overline{B_n}$。因为$B_0\supset B_1\supset\cdots\supset B_n\cdots$,且$r_n\leq\frac{1}{2^n}$。所以$\{y_n\}$是基本列,$y_n\to y_0$。因为$\{y_n\}\subset E$,且$y_0\in\overline{B_n}$。因为$x_n\not\in\overline{B_n}$,所以$y_0\neq x_n,\forall n\geq1.$与$y_0\in E$矛盾。
\end{example}




\begin{exercise}
\hfill\\
证明:$R^n$中存在可列个开球$\{B_n\}_{n\geq1}$使得对任意开球$E$,存在子列$\{B_{n_k}\}$使得$E=\cup_{k\geq1}B_{n_k}$。满足这种条件的开球列$\{B_n\}_{n\geq1}$称为$R^n$中的一个可数邻域基。



\end{exercise}



\begin{exercise}
\hfill\\
设$E\subset R^n$。$\{G_{\alpha}\}_{\alpha\in\Lambda}$是$E$的一个开集族覆盖。证明:存在至多可数个开集$\{G_{\alpha_k}\}_{k\geq1}\subset\{G_{\alpha}\}_{\alpha\in\Lambda}$使得它仍覆盖$E$,即$E$的任意开覆盖有至多可数的子覆盖。
\end{exercise}

\section{measure}

\begin{remark}
  Let 
  \begin{equation*}
    f(x) = 
    \begin{cases}
      0, & x\leq 0,\\
      x, & x>0.
    \end{cases}
  \end{equation*}
  Then $ f''(x) = \delta_0(x)$.
  Indeed,
  we observe that 
  \begin{align*}
    f'(\phi) = - \int_{\mathbb R}  f\phi'\dd x 
    = - \int_{\mathbb R^+}  x \phi'\dd x 
    = \int_{\mathbb R^+}\phi\dd x,
  \end{align*}
  holds for any $\phi(x)\in C^\infty_0(\mathbb R)$, and 
  \begin{align*}
    f''(\phi) = \int_{\mathbb R}  f\phi''\dd x 
    = \int_{\mathbb R^+}  x\phi''\dd x 
    = - \int_{\mathbb R^+}\phi'\dd x = \phi(0) = \int_{\mathbb R}\phi\dd\delta_0(x),
  \end{align*}
  holds for any $\phi(x)\in C^\infty_0(\mathbb R)$.
\end{remark}

\begin{remark}
  Writing 
  \begin{align*}
    K(x) &= \int_0^\infty\frac{1}{(4\pi t)^{n/2}}e^{-\frac{|x|^2}{4t}}\dd t 
    = - \frac{1}{4\pi^{n/2}}\int_0^\infty\frac{1}{(4t)^{n/2-2}}e^{-\frac{|x|^2}{4t}}\dd\frac1{4t}\\
    &= \frac1{4\pi^{n/2}}\int_0^\infty s^{n/2-2}e^{-|x|^2s}\dd s
    = \frac{|x|^{2-n}}{4\pi^{n/2}}\int_0^\infty \xi^{n/2-2}e^{-\xi}\dd\xi \\
    &= \frac{\Gamma(n/2-1)|x|^{2-n}}{4\pi^{n/2}} =: c|x|^{2-n}.
  \end{align*}
  We observe that for any $\phi\in C_0^\infty(\mathbb R^n)$,
  \begin{align*}
    \Delta K[\phi] (x)
    &= c\int_{\mathbb R^n}|x-y|^{2-n}\Delta\phi(y)\dd y 
    = c\lim_{\varepsilon\searrow0}\int_{|x-y|>\varepsilon}|x-y|^{2-n}\Delta \phi(y)\dd y\\
    &= c\lim_{\varepsilon\searrow0}\left(-\int_{|x-y|=\varepsilon}|x-y|^{2-n}\partial_\nu\phi(y)\dd S(y) 
    - \int_{|x-y|>\varepsilon}\nabla |x-y|^{2-n}\nabla\phi(y)\dd y\right)\\
    &= c\lim_{\varepsilon\searrow0}\left(-\varepsilon\omega_n\fint_{|x-y|=\varepsilon}\partial_\nu\phi(y)\dd S(y) 
    - (n-2)\int_{|x-y|>\varepsilon}|x-y|^{-n}(x-y)\nabla\phi(y)\dd y\right)\\
    &= c(n-2)\lim_{\varepsilon\searrow0}\left(\int_{|x-y|=\varepsilon}|x-y|^{1-n}\phi(y)\dd y 
    + \int_{|x-y|>\varepsilon}\nabla\cdot(|x-y|^{-n}(x-y))\phi(y)\dd y\right)\\
    &= c(n-2)\omega_n\lim_{\varepsilon\searrow0}\fint_{|x-y|=\varepsilon}\phi(y)\dd y\\
    &= \frac{\Gamma(n/2-1)(n-2)}{4\pi^{n/2}}\frac{2\pi^{n/2}}{\Gamma(n/2)}\phi(x) = \phi(x).
  \end{align*}
\end{remark}

\section{可测函数}

\begin{example}
  Let 
  \begin{equation}
    f_{n,k}(x) = 
    \begin{cases}
      0, & x\in(-\infty, (k-2)/n],\\
      nx-k+2,& x\in((k-2)/n, (k-1)/n],\\
      1,& x\in((k-1/n), k/n],\\
      -nx+k+1, & x\in(k/n, (k+1)/n],\\
      0, & x\in((k+2)/n, \infty),
    \end{cases}
    \quad k = 1,2,\cdots,n, \quad n\geq1.
  \end{equation}
  $\{f_{n,k}\}$ is a sequence of continuous functions that converges in measure, 
  but does not converge at any point.
\end{example}

\begin{exercise}
\hfill\\
设$f(x)$是$[a,b]$上的可测函数,试证明$f'(x)$是$[a,b]$上的可测函数。



\end{exercise}


\begin{exercise}
\hfill\\
$\forall\delta>0$,$\exists E_{\delta}\subset E$使得$m(E_{\delta})<\delta$,且在$E\backslash E_{\delta}$上,$\{f_n(x)\}$一致收敛于$f(x)$。证$\{f_n(x)\}$几乎处处收敛于$f$。



\end{exercise}


\begin{exercise}
\hfill\\
证鲁津定理的逆定理:若$\forall\delta>0$,存在闭子集$F_{\delta}\subset E$,使$m(E\backslash F_{\delta})\leq\delta$,且$f(x)$在$F_{\delta}$上连续,则$f(x)$在$E$上是可测函数。



\end{exercise}


\begin{exercise}
\hfill\\
设$\{f_n(x)\}$是$E$上的可测函数列,$m(E)<\infty$。试证明$$\lim_{n\to\infty}f_n(x)=0,a.e.x\in E$$的充分必要条件是:对任意的$\varepsilon>0$有$$\lim_{n\to\infty}m(\{x\in E:\sup_{k>n}|f_k(x)|\geq\varepsilon\})=0.$$



\end{exercise}


\begin{exercise}
\hfill\\
设$\{f_n(x)\}$在$[a,b]$上依测度收敛于$f(x)$,$g(x)$是$R$上的连续函数。证明$\{g(f_n(x))\}$在$[a,b]$上依测度收敛于$g(f(x))$。



\end{exercise}



\begin{exercise}
\hfill\\
设$f(x)=f(\xi_1,\xi_2)$是$R^2$上的连续函数。$g_1(x)$,$g_2(x)$是$[a,b]$上的实值可测函数,试证明$F(x)=f(g_1(x),g_2(x))$是$[a,b]$上的可测函数。



\end{exercise}

\hfill\\
\section{Lebesgue积分}


\begin{exercise}
\hfill\\
设$f(x)$在$[a,b]$上的$\mathbb{R}$反常积分存在。证明:$f(x)$在$[a,b]$上可积的充要条件为$|f(x)|$在$[a,b]$上的$\mathbb{R}$反常积分存在。并证明此时成立$$(L)\int_{[a,b]}f(x)\mathrm{d}x=(R)\int_a^bf(x)\mathrm{d}x.$$


不妨设$x=b$为$f(x)$的瑕点。

若$|f|$在$[a,b]$上的$\mathbb{R}$反常积分存在,则$\forall n\geq1$,$|f|$在$E_n=[a,b-\frac{1}{n}]$上$\mathbb{R}$可积。因为$\{|f(x)|X_{E_n}(x)\}$是非负递增函数,且有$$\lim_{n\to\infty}|f(x)|X_{E_n}(x)=|f(x)|,x\in E.$$所以由Levi定理知:
\begin{align*}
(L)\int_E|f(x)|\mathrm{d}x&=\lim_{n\to\infty}\int_E|f(x)|X_{E_n}(x)\mathrm{d}x\\
&=\lim_{n\to\infty}\int_{E_n}|f(x)|\mathrm{d}x\\
&=(R)\int_a^b|f(x)|\mathrm{d}x\\
&\leq\infty.
\end{align*}
这就说明了$|f(x)|\in L(E).$

若$f(x)$在$E$上$\mathbb{L}$可积,则$|f(x)|$在$[a,b]$上$L$可积。
定义$E_n=[a,b_n]$,其中$b_n\leq b$且$\lim\limits_{n\to\infty}b_n=b$。
因为$\{|f(x)|X_{E_n}(x)\}$是非负递增函数,且
$$\lim_{n\to\infty}|f(x)|X_{E_n}(x)=|f(x)|,$$
所以应用Levi定理可得:
$$+\infty>(L)\int_E|f(x)|\mathrm{d}x=\lim_{n\to\infty}\int_E|f(x)|X_{E_n}(x)\mathrm{d}x.$$
另一方面$$\int_{E_n}|f(x)|\mathrm{d}x=\int_E|f(x)|X_{E_n}(x)\mathrm{d}x.$$
这就说明了$$\lim_{n\to\infty}\int_a^{b_n}|f(x)|\mathrm{d}x=\int_E|f(x)|\mathrm{d}x<\infty$$
对任意的$b_n\to b^-$恒成立,即$|f(x)|$在$[a,b]$上$\mathbb{R}$可积。
\end{exercise}



\begin{exercise}
\hfill\\
设$f$是$E$上定义的函数。如果存在可积函数列$g_n$,$h_n$使得$g_n(x)\leq f(x)\leq h_n(x)\quad a.e.$,而且
$$\lim_{n\to\infty}(h_n(x)-g_n(x))\mathrm{d}x=0,$$
则$f$在$E$上可积。


定义$$K_n(x)=f(x)-g_n(x),$$
$$F_n(x)=h_n(x)-g_n(x),$$
则
$$0\leq K_n(x)\leq F_n(x)\quad a.e.,$$
且
\begin{equation}\label{lebesgue_integral_1}
\lim_{n\to\infty}F_n(x)\mathrm{d}x=0.
\end{equation}
由(\ref{lebesgue_integral_1})可知$F_n(x)\Rightarrow0$。
于是$K_n(x)\Rightarrow0$。即$g_n(x)\Rightarrow f(x)$。
那么由依测度收敛的Lebesgue控制收敛定理知$f$在$E$上可积,且积分满足:
$$\int_Ef(x)\mathrm{d}x=\lim_{n\to\infty}\int_Eg_n(x)\mathrm{d}x.$$
\end{exercise}



\begin{exercise}
\hfill\\
设$f\in L(\mathbb{R})$,若对$\mathbb{R}$上任意连续函数$g(x)$,有$\int_{\mathbb{R}}f(x)g(x)\mathrm{d}x=0$,证明$f(x)=0,\quad a.e.x\in\mathbb{R}.$


定义
\begin{equation}
h(x)=
\begin{cases}
1,&x\in R[f(x)\geq0],\\
-1,&x\in R[f(x)<0],\\
\end{cases}
\end{equation}
则$h(x)$为$\mathbb{R}$上简单函数。因为$f\in L(\mathbb{R})$,于是$\forall\varepsilon>0$,$\exists X>0$,使得$$\int_{\mathbb{R}\backslash E_X}|f|\mathrm{d}x<\frac{\varepsilon}{4},$$
其中$E_X=\{x\in\mathbb{R}:|x|<X\}$。
对在$E_X\subset\mathbb{R}$上的有界可测函数$h(x)$,有$\forall\delta>0$,存在闭集$F\subset E_X$满足$m(E_X\backslash F)<\delta$和$\mathbb{R}$上的连续函数$g(x)$满足$g(x)=h(x),\forall x\in F$. 进一步,
$$\inf_{\mathbb{R}}g(x)=\inf_Fh(x)\geq-1,\sup_{\mathbb{R}}g(x)=\sup_Fh(x)\leq1.$$
于是
\begin{align*}
|\int_{\mathbb{R}}f(x)h(x)\mathrm{d}x|&=|(\int_{\mathbb{R}\backslash E_X}+\int_{E_X\backslash F}+\int_F)f(x)(h(x)-g(x))\mathrm{d}x|\\
&=|(\int_{E_X\backslash F}+\int_{\mathbb{R}\backslash E_x})f(x)(h(x)-g(x))\mathrm{d}x|\\
&\leq2(\int_{E\backslash F}+\int_{\mathbb{R}\backslash E_X})|f(x)|\mathrm{d}x\\
&\leq\frac{\varepsilon}{2}+2\int_{E_X\backslash F}|f(x)|\mathrm{d}x\\
\end{align*}
又因为$|f(x)|\in L(\mathbb{R})$,从而由Lebesgue积分的绝对连续性知:对上述$\varepsilon>0$,$\exists\delta>0$,只要$m(E_X\backslash F)<\delta$就有
$$\int_{E_X\backslash F}|f(x)|\mathrm{d}x<\frac{\varepsilon}{4}.$$
于是$$|\int_{\mathbb{R}}f(x)h(x)\mathrm{d}x|<\varepsilon.$$
事实上,我们有$$\int_{\mathbb{R}}|f(x)|\mathrm{d}x=\int_{\mathbb{R}}f(x)h(x)\mathrm{d}x<\varepsilon.$$
于是由$\varepsilon$的任意性,就有$\int_{\mathbb{R}}|f(x)|\mathrm{d}x=0$,即$f(x)=0,a.e.$
\end{exercise}

\begin{exercise}
设 $f,f_k(k=1,2,\cdots)$ 在 $R^n$ 上可积, 且对于任一可测集$E\subset\mathbb{R}^n$,有
$$\int_Ef_k(x)\mathrm{d}x\leq\int_Ef_{k+1}\mathrm{d}x,\quad k=1,2,\cdots,$$
$$\lim_{k\to\infty}\int_Ef_k(x)\mathrm{d}x=\int_Ef(x)\mathrm{d}x,$$
试证明$\lim_{k\to\infty}f_k(x)=f(x),a.e.x\in\mathbb{R}^n.$

\begin{proof}
首先由题意,$\forall k\geq1$,有
$$\int_Ef_k(x)\mathrm{d}x\leq\lim_{k\to\infty}f_k(x)\mathrm{d}x=\int_Ef(x)\mathrm{d}x,$$对任意可测子集$E\subset\mathbb{R}^n$恒成立。
于是必然有
\begin{equation}\label{lebesgue_intergal_2}
f_k(x)\leq f_{k+1}(x) \leq f(x)\quad a.e.x\in\mathbb{R}^n.
\end{equation}
由 \eqref{lebesgue_intergal_2} 知
\[
  \lim_{k\to\infty}\int_E|f(x)-f_k(x)|\mathrm{d}x
  = \lim_{k\to\infty} \int_E(f(x)-f_k(x))\mathrm{d}x = 0,
\]
应用Fatou引理,
\[
0\geq\int_E\liminf_{k\to\infty}f(x)-f_k(x)\mathrm{d}x=\int_Ef(x)\mathrm{d}x-\int_E\limsup_{k\to\infty}f_k(x)\mathrm{d}x,
\]
即 
\[
\int_Ef(x)\mathrm{d}x
\leq\int_E\limsup_{k\to\infty}f_k(x)\mathrm{d}x
= \int_E\lim_{k\to\infty}f_k(x)\mathrm{d}x
\leq\int_Ef(x)\mathrm{d}x.
\] 
完成了证明.
\end{proof}
\end{exercise}

\begin{exercise}
\hfill\\
设$f(x)$,$g(x)$是$E$上非负可测函数且$f(x)g(x)$在$E$上可积。令$Ey=E[g\geq y]$。证明:$$F(y)=\int_{E_y}f(x)\mathrm{d}x$$对一切$y>0$都存在,且成立
$$\int_0^{\infty}F(y)\mathrm{d}y=\int_Ef(x)g(x)\mathrm{d}x.$$

$E_y$可测是显然的。$$F(y)=\int_Ef(x)X_{E_y}\mathrm{d}x,$$
而$|f(x)X_{E_y}|\leq|f(x)|$,故由$f(x)$的可积性知$f(x)X_{E_y}\in L(E),\forall y>0.$
考虑到$g(x)=\int_0^{\infty}X_{E_y}\mathrm{d}y$,于是
\begin{align*}
\int_Ef(x)g(x)\mathrm{d}x&=\int_Ef(x)\int_0^{\infty}X_{E_y}\mathrm{d}y\mathrm{d}x\\
&=\int_E\int_0^{\infty}f(x)X_{E_y}\mathrm{d}y\mathrm{d}x\\
&=\int_0^{\infty}\int_Ef(x)X_{E_y}\mathrm{d}x\mathrm{d}y\\
&=\int_0^{\infty}F(y)\mathrm{d}y.\\
\end{align*}

\end{exercise}


\begin{exercise}
若$f$在$R$上可积,证明$\int_R|f(x+h)-f(x)|\mathrm{d}x\to0,h\to0$.
\end{exercise}

\begin{proof}
首先注意到$0\leq|f(x+h)-f(x)|\leq|f(x+h)|+|f(x)|$,所以由$f\in L(R)$可知$f(x+h)\in L(R),\forall h\in R$,从而$|f(x+h)-f(x)|\in L(R).$于是对任给的$\varepsilon>0$,存在$X>0$使得$$\int_{|x|>X}|f(x+h)-f(x)|\mathrm{d}x<\frac{\varepsilon}{2}.$$

其次,$|f(x+h)-f(x)|$在$[-X-1,X+1]$上可积,由鲁津定理,对任意的$\delta>0$,存在闭集$E\subset[-X,X]$且$m([-X,X]\backslash E)<\delta$,有$f(x)$在$E$上连续。从而$f$在$E$上一致连续,即对上述$\varepsilon>0$,存在$\tau>0$,只要$|h|<\tau$,就有
$$|f(x+h)-f(x)|<\frac{\varepsilon}{8X}.$$
另一方面,由Lebesgue积分的绝对连续性,对上述的$\varepsilon>0$,存在$\delta>0$,只要可测集$F$的测度$m(F)<\delta$,就有
$$\int_{F}|f(x)|\mathrm{d}x<\frac{\varepsilon}{8}.$$
于是$$\int_{F}|f(x+h)-f(x)|\mathrm{d}x<\int_{F}(|f(x)|+|f(x+h)|)\mathrm{d}x\leq\frac{\varepsilon}{4}.$$

综合即有,对任给的$\varepsilon>0$,存在$\tau>0$,使得只要$|h|<\tau$,就有
\begin{align*}
\int_R|f(x+h)-f(x)|\mathrm{d}x&\leq\frac{\varepsilon}{2}+\int_{[-X,X]}|f(x+h)-f(x)|\mathrm{d}x\\
&=\frac{\varepsilon}{2}+\int_{[-X,X]\backslash E}|f(x+h)-f(x)|\mathrm{d}x+\int_E|f(x+h)-f(x)|\mathrm{d}x\\
&<\frac{\varepsilon}{2}+\frac{\varepsilon}{4}+2X*\frac{\varepsilon}{2X}\\
&=\varepsilon.
\end{align*}
\end{proof}




\begin{exercise}
\hfill\\



\end{exercise}



\begin{exercise}
\hfill\\



\end{exercise}



\begin{exercise}
\hfill\\



\end{exercise}


