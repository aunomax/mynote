\chapter{The Fourier transform}

The Fourier transform of a function $f$ on $\mathbb R^d$ is defined by 
\begin{equation}
  \hat f(\xi) = \int_{\mathbb R^d}f(x)e^{-2\pi ix\cdot\xi}\dd x,
\end{equation}
and its attached inversion is given by 
\begin{equation}
  f(x) = \int_{\mathbb R^d}\hat f(\xi)e^{2\pi ix\cdot\xi}\dd \xi.
\end{equation}

These formulas have already appeared in several contexts.
We consider first the properties of the Fourier transform in the elementary setting 
by restricting to functions in the Schwartz class $\mathcal S(\mathbb R^d)$.
The class $\mathcal S$ consists of functions $f$ that are smooth (indefinitely differentiable)
and such that for each multi-index $\alpha$ and $\beta$, the function $x^\alpha\left(\frac{\partial}{\partial x} \right)^\beta f$ is bounded on $\mathbb R^d$.
We saw that on this class the Fourier transform is a bijection,
that the inversion formula holds,
and moreover we have the Plancherel identity 
\begin{equation}
  \int_{\mathbb R^d}|\hat f(\xi)|^2\dd\xi = \int_{\mathbb R^d}|f(x)|^2\dd x.
\end{equation}

Turning now to more general (in particular, non-continuous) functions,
we note that the largest class for which the integral defining $\hat f(\xi)$ converges (absolutely) is the space $L^1(\mathbb R^d)$. 
For it, we saw that a (relatively) inversion formula is valid, 
provided $\hat f\in L^1(\mathbb R^d)$. 
In this case, 
since $\hat f$ is continuous, bounded, and moreover decays to zero at infinity,
$f$ could be modified on a set of measure zero to become continuous everywhere,
which is of course impossible for the general function $f\in L^1(\mathbb R^d)$.

Beyond these particular facts, what we would like here is to reestablish in the general context the symmetry between $f$ and $\hat f$ that holds for $\mathcal S$.
This is where the special role of the Hilbert space $L^2(\mathbb R^d)$ enters.

We shall define the Fourier transform on $L^2(\mathbb R^d)$ as an extension of its definition on $\mathcal S$.
For this purpose, we temporarily adopt the notational device of denoting by $\mathcal F_0$ and $\mathcal F$ the Fourier transform on $\mathcal S$ and its extension to $L^2$, respectively.

The main results we prove are the following.

\begin{theorem}
  The Fourier transform $\mathcal F_0$, initially defined on $\mathcal S(\mathbb R^d)$,
  has a (unique) extension $\mathcal F$ to a unitary mapping of $L^2(\mathbb R^d)$ to itself.
  In particular,
  \[
  \|\mathcal F(f)\|_{L^2(\mathbb R^d)} = \|f\|_{L^2(\mathbb R^d)}
  \]
  for all $f\in L^2(\mathbb R^d)$.
\end{theorem}

The extension $\mathcal F$ will be given by a limiting process: if $\{f_n\}$ is a sequence in the Schwartz space that converges to $f$ in $L^2(\mathbb R^d)$, 
then $\{\mathcal F_0(f_n)\}$ will converge to an element in $L^2(\mathbb R^d)$ which we will define as the Fourier transform of $f$.