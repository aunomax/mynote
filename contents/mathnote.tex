
\chapter{note}

\section{notation}

\subsection{Cut-off function}

For convenience, we introduce the following cut-off functions.

\begin{lemma}
  \label{le: cut-off function}
  Let $n\geq1$ and $R>0$.
  There exists a nonnegative and radially symmetric function 
  $\phi(x)\in C_0^\infty(\mathbb R^n)$ such that 
  \begin{displaymath}
    \phi(x) = 
    \begin{cases}
      1, & x\in B_{1/2},\\
      0, & x\in\mathbb R^n\setminus B_1.
    \end{cases}
  \end{displaymath}
  Moreover, for
  \begin{displaymath}
    \phi^{(\rho)}(x) := \phi(Rx/\rho), \quad x\in \mathbb R^n,\quad  \rho\in(0,R),
  \end{displaymath}
  it holds 
  \begin{displaymath}
    \rho\|\nabla \phi^{(\rho)}\|_{L^\infty(\mathbb R^n)} 
    + \rho^2\|\Delta \phi^{(\rho)}\|_{L^\infty(\mathbb R^n)} 
    \leq C_\phi := 
    R\|\nabla \phi\|_{L^\infty(\mathbb R^n)} 
    + R^2\|\Delta \phi\|_{L^\infty(\mathbb R^n)}  
  \end{displaymath}
  for each $\rho\in(0,R)$.
\end{lemma}

\subsection{Marcinkiewicz space}

Here the Marcinkiewicz space $M^p\left(\mathbb{R}^d\right), 1<p<\infty$, is defined as the set of $f \in$ $L_{\text {loc }}^1(\mathbb{R}^d)$ such that
\[
\int_K|f(x)| d x \leq C|K|^{(p-1) / p}
\]
for all subsets $K$ of finite measure. The minimal $C$ in the above inequality gives a norm in this space, i.e.,
\[
\|f\|_{M^p\left(\mathbb{R}^d\right)}=\sup \left\{\operatorname{meas}(K)^{-(p-1) / p} \int_K|f| d x: K \subset \mathbb{R}^d, \operatorname{meas}(K)>0\right\}
\]

\section{calculus}
\begin{equation}
	\int_\Omega |\nabla u|^\alpha \nabla u \cdot \nabla\Delta u\mathrm{d}x = \frac{1}{2}\int_{\partial \Omega} |\nabla u|^\alpha \frac{\partial |\nabla u|^2}{\partial n}\mathrm{d}S - \int_\Omega |\nabla u|^\alpha |\nabla^2u|^2\mathrm{d}x - \alpha \int_\Omega |\nabla u|^\alpha (\nabla |\nabla u|)^2\mathrm{d}x
\end{equation}
as \begin{equation}
	\nonumber
	\Delta|\nabla u|^2 = 2|D^2u|^2 + 2\nabla u \cdot \nabla \Delta u.
\end{equation}

\begin{proposition}
	let 
	\[
		K(x,t) := \frac{1}{(4\pi t)^{n/2}}e^{-\frac{|x|^2}{4t}},
	\]
	then for any $f\in L^1(\mathbb{R}^n)$,
	\begin{enumerate}
		\item mass preserving: 	
			\[
			\iint_{\mathbb R^n\times\mathbb R^n}K(x-y)f(y)\dd y\dd x 
			= \int_{\mathbb R^n}f,
			\]
			by Fubini theorem.
		\item pointwise convergence: 
		\[
		K*f\to f,\quad{t\searrow0},\quad\text{for a.e. } x\in\mathbb{R}^n,
		\]
		by Lebesgue differential theorem.
		\item strong convergence: $K*f\in C([0,\infty);L^1(\mathbb{R}^n))$.
		\item hypercontraction:
		\[
		\|K*f\|_p\leq \|K\|_p\|f\|_1\lesssim t^{-\frac{n}{2}\left(1-\frac1p\right)}\|f\|_1,
		\]
		by Young inequality for convolution.
		Particularly, by \cite{Giga1988},
		\begin{equation}\label{eq: giga1988}
			\limsup_{t\searrow0}t^{\frac{n}{2}\left(1-\frac1p\right)}\|K*f\|_p = 0,\quad p>1.
		\end{equation}
	\end{enumerate}
\end{proposition}
\begin{proof}
	Without loss of generality, we may assume that $f\geq0$ in the distribution sense.
	
	2. The idea of this proof is borrowed from \cite[Theorem~2.1 in Chapter~3]{Stein2005}. Let $x\in\mathbb R^n$ be a Lebesgue point. Then  
	\[
	F(r) := \fint_{|y|\leq r} |f(x-y)-f(x)|\dd y,\quad r > 0,
	\]
	is a continuous function with respect to $r>0$,
	and 
	\[
	F(r)\to 0\quad \text{as }r\searrow0.
	\]
	Moreover, $F(r)$ is bounded, that is, $F(r)\leq M$ for some $M>0$ and all $r>0$. 
	The first two properties are consequences of absolute continuity and Lebesgue differential theorem, respectively, and the third follows from the first two properties and integration of $f$.
	We estimate,
	\begin{align*}
		K*f-f &= \int_{\mathbb{R}^n}K(x-y)f(y)\dd y - f(x) = \int_{\mathbb R^n}K(y)(f(x-y)-f(x))\dd y\\
		&\leq \int_{|y|\leq\sqrt {4\pi t}}K(y)|f(x-y)-f(x)|\dd y 
		+ \sum_{k=0}^{\infty}\int_{2^k\sqrt {4\pi t} < |y|\leq 2^{k+1}\sqrt {4\pi t}}K(y)|f(x-y)-f(x)|\dd y\\
		&\leq C\fint_{B_{\sqrt {4\pi t}}(x)}|f(x-y)-f(x)|\dd y 
			+ C\sum_{k=0}^\infty\fint_{|y|\leq 2^{k+1}\sqrt{4\pi t}}2^{(k+1)n}e^{-4^k}|f(x-y)-f(x)|\dd y \\
		&= CF(\sqrt{4\pi t}) + C\sum_{k=0}^\infty 2^{(k+1)n}e^{-4^k} F(2^{k+1}\sqrt{4\pi t})\\
		&\leq CF(\sqrt{4\pi t}) + C\sum_{k=0}^{L}F(2^{k+1}\sqrt{4\pi t}) + C M \sum_{k=L+1}^\infty 2^{(k+1)n}e^{-4^k}.
	\end{align*}
	The third term of the right side of this inequality may be made arbitrarily small 
  by choosing a sufficiently large integer $L$, 
  and for fixed $L$, the first two terms may be made arbitrarily small by choosing $t$ sufficiently small,
	as desired.

	3. 	 
	For $\varepsilon>0$, there exists $R>0$ such that 
	\[
	(4\pi)^{-n/2}\int_{\mathbb R^n\setminus B_R}e^{-z^2/4}<\varepsilon,
	\]
	we have 
	\begin{align*}
		\|K*f-f\|_{L^1(\mathbb R^n)} 
		&= \int_{\mathbb{R}^n}\left|\int_{\mathbb{R}^n}K(y)f(x-y)\dd y - f(x)\right|\dd x \\
		&= \int_{\mathbb{R}^n}\left|\int_{\mathbb{R}^n}K(y)(f(x-y) - f(x))\dd y\right|\dd x\\
		&\leq \iint_{\mathbb{R}^n\times\mathbb{R}^n}K(y)|(f(x-y) - f(x))|\dd x\dd y\\
		&= (4\pi)^{-n/2}\iint_{\mathbb{R}^n\times\mathbb R^n}e^{-z^2/4}|(f(x-\sqrt tz)-f(x))|\dd x\dd z\\
		&\leq (4\pi)^{-n/2}\int_{B_R}e^{-z^2/4}\int_{\mathbb{R}^n}|f(x-\sqrt tz)-f(x)|\dd x\dd z\\
		&\quad + (4\pi)^{-n/2}\int_{\mathbb R^n\setminus B_R}e^{-z^2/4}\int_{\mathbb{R}^n}|f(x-\sqrt tz)-f(x)|\dd x\dd z,\\
		&\leq \sup_{|z|<R}\|f(x-\sqrt t z) - f(x)\|_{L^1(\mathbb R^n)} 
		+ 2\|f\|_1\int_{|z|\geq R}e^{-z^2/4}.
	\end{align*}
	The second term of the right side of this inequality 
	may be made arbitrarily small by choosing $R$ sufficiently large,
	and, for fixed $R$, 
	the first term may be made arbitrarily small by choosing $t$ sufficiently small, 
	because of the uniform continuity of integrable function proved in  proposition~\ref{prop: uniform continuity of integrable function}. 
	This proves convergence in norm.


	4. by Young inequality, we estimate
	\begin{align*}
		\|K*f\|_p &\leq \|K\|_p\|f\|_1\\
		&= \frac{1}{(4\pi t)^{n/2}}\left(\int_{\mathbb R^n}e^{-\frac{p|x|^2}{4t}}\dd x\right)^{1/p}\|f\|_1\\
		&= \frac{(4t)^{n/(2p)}}{(4\pi t)^{n/2}p^{n/(2p)}}\left(\int_{\mathbb R^n}e^{-|z|^2}\dd z\right)^{1/p}\|f\|_1\\
		&= (4\pi t)^{-\frac{n}2\left(1-\frac1p\right)}p^{-\frac{n}{2p}}\|f\|_1.
	\end{align*}
%	For any $\varepsilon>0$, there exists $R>1>\delta>0$ such that $\|f\|_{\mathbb R^n\setminus B_{R-1}}<\varepsilon$,
	We calculate
	\begin{align*}
		t^{\frac{n}{2}\left(1-\frac1p\right)}\|K*f\|_p 
		&\lesssim t^{-n/(2p)}\left(\int_{\mathbb R^n}\left|\int_{\mathbb R^n}e^{-\frac{|x-y|^2}{4t}}f(y)\dd y\right|^p\dd x\right)^{1/p}\\
		&\leq \left(t^{-n/2}\int_{\mathbb R^n}\left(\int_{\mathbb R^n\setminus B_\delta(x)}e^{-\frac{|x-y|^2}{4t}}f(y)\dd y\right)^p\dd x\right)^{1/p}\\
		&\quad + \left(t^{-n/2}\int_{\mathbb R^n}\left(\int_{B_\delta(x)}e^{-\frac{|x-y|^2}{4t}}f(y)\dd y\right)^p\dd x\right)^{1/p}\\
		&=: I_1^{1/p} + I_2^{1/p}.
	\end{align*}
	By Fubini theorem and H\"older inequality, we estimate
	\begin{align*}
		%t^{-n/2}\int_{\mathbb R^n}\left|\int_{\mathbb R^n\setminus B_\delta(x)}e^{-\frac{|x-y|^2}{4t}}f(y)\dd y\right|^p\dd x
		I_1
		&\leq t^{-n/2} \int_{\mathbb R^n}\int_{\mathbb R^n\setminus B_\delta(x)}e^{-\frac{p|x-y|^2}{4t}}f(y)\dd y
		\cdot \left(\int_{\mathbb R^n\setminus B_\delta(x)}f(y)\dd y\right)^{p-1}\dd x\\
		&\leq t^{-n/2}\|f\|^{p-1}_{L^1(\mathbb R^n)}\iint_{\mathbb R^n\times\mathbb R^n}
		\chi_{\mathbb R^n\setminus B_\delta(x)}(y)	e^{-\frac{p|x-y|^2}{4t}}f(y)\dd y\dd x\\
		&\leq t^{-n/2}\|f\|_{L^1(\mathbb R^n)}^{p}\int_{\mathbb R^n\setminus B_\delta}e^{-\frac{p|x|^2}{4t}}\dd x\\
		&\lesssim \|f\|_{L^1(\mathbb R^n)}^{p}\int_{\mathbb R^n\setminus B_\delta/\sqrt t}e^{-|z|^2}\dd z
		\to 0,\quad\text{as }t\searrow0,
	\end{align*}
	% \begin{align*}
	% 	t^{-n/2}\int_{\mathbb R^n\setminus B_R}\left|\int_{B_\delta(x)}e^{-\frac{|x-y|^2}{4t}}f(y)\dd y\right|^p\dd x
	% 	&\leq  t^{-n/2} \int_{\mathbb R^n\setminus B_R}\int_{B_\delta(x)}e^{-\frac{p|x-y|^2}{4t}}f(y)\dd y
	% 	\cdot \left(\int_{B_\delta(x)}f(y)\dd y\right)^{p-1}\dd x\\
	% 	&\lesssim t^{-n/2}\|f\|^{p-1}_{L^1(\mathbb R^n)} \iint_{\mathbb R^n\times\mathbb R^n}
	% 	\chi_{B_\delta(x)}(y)\chi_{\mathbb R^n\setminus B_R}(x)	e^{-\frac{p|x-y|^2}{4t}}f(y)\dd y\dd x\\
	% 	&\lesssim  \|f\|^{p-1}_{L^1(\mathbb R^n)} \int_{\mathbb R^n\setminus B_{R-\delta}}f(y)\dd y,
	% \end{align*}
	and 
	\begin{align*}
		%t^{-n/2}\int_{\mathbb{R}^n}\left|\int_{B_\delta(x)}e^{-\frac{|x-y|^2}{4t}}f(y)\dd y\right|^p\dd x
		I_2
		&\leq t^{-n/2} \int_{\mathbb R^n}\int_{B_\delta(x)}e^{-\frac{p|x-y|^2}{4t}}f(y)\dd y
		\cdot \left(\int_{B_\delta(x)}f(y)\dd y\right)^{p-1}\dd x\\
		&\lesssim \|f\|_{L^1(\mathbb R^n)}\sup_{x\in\mathbb R^n} \cdot \left(\int_{B_\delta(x)}f(y)\dd y\right)^{p-1}.
	\end{align*}
	Therefore,
	\[
	\limsup_{t\searrow0}	t^{\frac{n}{2}\left(1-\frac1p\right)}\|K*f\|_p 
	\leq \|f\|^{1/p}_{L^1(\mathbb R^n)}\sup_{x\in\mathbb R^n} 
	\cdot \left(\int_{B_\delta(x)}f(y)\dd y\right)^{1-1/p},
	\]
	and \eqref{eq: giga1988} follows by absolute continuity of Lebesgue integral.
\end{proof}

\begin{remark}
	Banach-Alaoglu theorem tells that 
	\[
	K*f \overset{\ast}{\rightharpoonup} f,\quad \text{in } (C_0(\mathbb R^n))^*,
	\]
	as $t\searrow0$.

	If $f_n\to f$ a.e. and $\|f_n\|_1\to\|f\|_1$ as $n\to\infty$, 
	then $\|f_n-f\|_1 \to 0$ as $n\to\infty$.

	Integration convergence and almost-everywhere convergence imply convergence in norm (strong convergence).
	Actually, by Fatou lemma,
	\begin{align*}
	2\|f\|_1 &= \left\|\liminf_{n\to\infty}(|f|+|f_n|-|f_n-f|)\right\|_1\\
	&\leq \liminf_{n\to\infty}\||f_n|+|f|-|f_n-f|\|_1\\
	&\leq \limsup_{n\to\infty}\||f_n|+|f|\|_1 - \limsup_{n\to\infty}\|f_n-f\|_1\\
	&= 2\|f\|_1 - \limsup_{n\to\infty}\|f_n-f\|_1,
	\end{align*}
	which implies the desired result.	
\end{remark}


\begin{lemma}
	The surface area $\omega_n$ and volume $V_n$ of the unit ball $\Omega = B_1 := \{x\in\mathbb{R}^n: |x|<1\}$ are given by
	\[
		V_n=\frac{\pi^{\frac{n}{2}}}{\Gamma\left(\frac{n}{2}+1\right)} = \frac{\omega_n}{n},
	\]
	where $\Gamma(s)$ is the usual Gamma function.
\end{lemma}
\begin{proof}
	Using the fact
	\begin{equation*}
		\int_{\mathbb{R}^2}e^{-|x|^2}\dd x 
			= 2\pi\int_0^\infty re^{-r^2}\dd r
			= \pi,
	\end{equation*}
	we have 
	\begin{align*}
		\int_{\mathbb{R}^n}e^{-|x|^2}\dd x
			&= \left(\int_{\mathbb{R}}e^{-s^2}\dd s\right)^n\\
			&= \left(\int_{\mathbb{R}^2}e^{-s^2}\dd s\right)^{n/2}\\
			&= \pi^{n/2},
	\end{align*}
	While 
	\begin{align*}
		\int_{\mathbb{R}^n}e^{-|x|^2}\dd x
			&= \omega_n\int_0^\infty r^{n-1}e^{-r^2}\dd r\\
			&= \frac{\omega_n}2\int_0^\infty \rho^{n/2-1}e^{-\rho}\dd\rho\\
			&= \frac{\omega_n\Gamma(n/2)}{2},
	\end{align*}
	we end up with
	\[
		\omega_n = \frac{2\pi^{n/2}}{\Gamma(n/2)} = \frac{n\pi^{n/2}}{\Gamma(n/2+1)},
	\]
	and thus
	\[
		V_n = \int_{\Omega}\dd x = \omega_n\int_0^1r^{n-1}\dd r = \frac{\omega_n}{n}.
	\]
\end{proof}

\section{the interpolation-trace lemma}
\begin{lemma}
	\cite{Diaz1985}
	\label{the interpolation-trace lemma}
	Suppose $G$ is a bounded open subset of $R^{N}, N \geq 1,$ with a $C^{1}$ boundary $\partial G$ and $0 \leq \sigma \leq q<\infty .$ Then there exists a constant $C$ depending on $\sigma$, $q$ and $G$ such that for any $v \in W^{1, q+1}(G)$ we have
(2.12)
\begin{equation*}
	\|v\|_{L^{q+1}(\partial G)} \leq C\left(\|D v\|_{L^{q+1}(G)}+\|v\|_{L^{\sigma+1}(G)}\right)^{\theta}\|v\|_{L^{\sigma+1}(G)}^{1-\theta}
\end{equation*}
where $\theta=(N(q-\sigma)+\sigma+1) / \kappa$, \(\kappa=N(q-\sigma)+(\sigma+1)(q+1)\).

\end{lemma}

\begin{proof}
	For the sake of simplicity we restrict ourselves to $v \in C^{1}(\bar{G})$ since $C^{1}(\bar{G})$ is dense in $W^{1, q+1}(G) .$ The proof of \Lref{the interpolation-trace lemma} is divided onto four steps (see [9, Appendix] for a similar result).
	
	First step. From a result of Ehrling's lemma, for any $\varepsilon>0$ there exists $C_{\varepsilon}>0$ such that for any $v \in C^{1}(\bar{G})$ the following holds:
	\begin{equation*}
		\|v\|_{L^{q+1}(G)} \leq \varepsilon\|D v\|_{L^{q+1}(G)}+C_{\varepsilon}\|v\|_{L^{\sigma+1}(G)}
	\end{equation*}
	If we set $C_{2}=\max \left(1+\varepsilon, C_{\varepsilon}|G|^{1-1 /(\sigma+1)}\right)$ we get
	\begin{equation}\label{ehrling inequality}
		\|v\|_{W^{1, q+1}(G)} \leq C_{2}\left(\|D v\|_{L^{q+1}(G)}+\|v\|_{L^{\sigma+1}(G)}\right)
	\end{equation}
Second step. We start from the elementary trace result \cite{Adams2003}: there exists $C_{3}>0$ such that for any $u \in C^{1}(\bar{G})$ we have
\begin{equation}\label{L1 trace}
	\|u\|_{L^{1}(\partial G)} \leq C_{3}\|u\|_{W^{1,1}(G)}
\end{equation}
and for $q>0$ we apply \eqref{L1 trace} to $u=v|v|^{q}, v \in C^{1}(\bar{G}),$ so
\begin{equation*}
	\int_{\partial G}|v|^{q+1} d \sigma \leq C_{3}\left\{(q+1) \int_{G}|v|^{q}|D v| d x+\int_{G}|v|^{q+1} d x\right\}
\end{equation*}
since
\begin{equation*}
	\int_{G}|v|^{q}|D v| d x \leq\|D v\|_{L^{q+1}(G)}\|v\|_{L^{q+1}(G)}^{q}
\end{equation*}
we get
\begin{equation*}
	\int_{\partial G}|v|^{q+1} d \sigma \leq C_{3}\left\{(q+1)\|D v\|_{L^{q+1}(G)}\|v\|_{L^{q+1}(G)}^{q}+\|v\|_{L^{q+1}(G)}^{q+1}\right\}
\end{equation*}
which implies
\begin{equation}\label{trace inequality of generalized l1}
	\|v\|_{L^{q+1}(\partial G)} \leq\left((q+1) C_{3}\right)^{1 /(q+1)}\|v\|_{W^{1, q+1}(G)}^{1 /(q+1)}\|v\|_{L^{q+1}(G)}^{q /(q+1)}
\end{equation}

Third step. Set $0 \leq \sigma \leq q<\infty$. We claim that there exists a constant $C_{4}>0$ such that for any $v \in C^{1}(\bar{G})$ we have
\begin{equation}\label{claim equation}
	\|v\|_{L^{q+1}(G)} \leq C_{4}\|v\|_{W^{1, q+1}(G)}^{((q+1) \theta-1) / q}\|v\|_{L^{\sigma+1}(G)}^{(q+1)(1-\theta) / q}
\end{equation}
\textrm{Case 1. } Assume $q+1<N$. From Sobolev's inequality we have $\|v\|_{L^{\tau}(G)} \leq$ $C\|v\|_{W^{1, q+1}(G)}$ with $1 / \tau=1 /(q+1)-1 / N .$ Moreover
\begin{equation}\label{lp interpolation inequality 1}
	\|v\|_{L^{q+1}(G)} \leq\|v\|_{L^{\tau}(G)}^{1-\lambda}\|v\|_{L^{\sigma+1}(G)}^{\lambda},
\end{equation}
where $1 /(q+1)=\lambda /(\sigma+1)+(1-\lambda) / \tau,$ that is
\begin{equation*}
	\lambda=(q+1)(\sigma+1) / N(q-\sigma)+(q+1)(\sigma+1)
\end{equation*}
Hence with Sobolev's inequality
\begin{equation}\label{sobolev inequality}
	\|v\|_{L^{q+1}(G)} \leq C^{1-\lambda}\|v\|_{W^{1, q+1}(G)}^{1-\lambda}\|v\|_{L^{\sigma+1}(G)}^{\lambda}
\end{equation}
and
\begin{equation*}
	1-\lambda=\frac{N(q-\sigma)}{N(q-\sigma)+(q+1)(\sigma+1)}=\frac{(q+1) \theta-1}{q}, \quad \lambda=\frac{(q+1)(1-\theta)}{q}
\end{equation*}
\textrm{Case } ~ 2 . ~ \text { Assume } ~ $q+1 \geq N \geq 1$. We set $\alpha=(N+1) / 2, \rho=2(q+1) /(N+1)$, $\beta=(\sigma+1)(N+1) / 2(q+1)$ and $\alpha^{*}=\alpha N /(N-\alpha)\left(\alpha^{*}=\infty\right.$ if $\left.N=1\right) .$ From
H\"{o}lder's interpolating inequality we have
\begin{equation}\label{holder interpolating inequality}
	\|u\|_{L^{\alpha}(G)} \leq\|u\|_{L^{\alpha^ *}(G)}^{1-\lambda}\|u\|_{L^{\beta}(G)}^{\lambda},
\end{equation}
where $1 / \alpha=(1-\lambda) / \alpha^{*}+\lambda / \beta$(\eqref{holder interpolating inequality} is valid even if $0<\beta<1$ with a simple change of function). From Sobolev's inequality we get
\begin{equation}
\|u\|_{L^{\alpha}(G)} \leq C_{5}\|u\|_{W^{1, \alpha}(G)}^{1-\lambda}\|u\|_{L^{\beta}(G)}^{\lambda}
\end{equation}
Now we set $u=v|v|^{\rho-1}$ and we have
\begin{equation*}
	\begin{array}{c}
		\|u\|_{L^{\alpha}(G)}=\|v\|_{L^{\alpha \rho}(G)}^{\rho}=\|v\|_{L^{q+1}(G)}^{\rho} \\
		\|u\|_{L^{\beta}(G)}=\|v\|_{L^{\beta \rho}(G)}^{\rho}=\|v\|_{L^{\sigma+1}(G)}^{\rho} \\
		\|u\|_{W^{1, \alpha}(G)}=\|v\|_{L^{q+1}(G)}^{\rho}+\left(\int_{G}\left(\rho|v|^{\rho-1}|D v|\right)^{\alpha} d x\right)^{1 / \alpha}
	\end{array}
\end{equation*}
and
\begin{equation*}
	\int_{G}\left(|v|^{\rho-1}|D v|\right)^{\alpha} d x \leq\left(\int_{G}|v|^{\alpha \rho} d x\right)^{1-1 / \rho}\left(\int_{G}|D v|^{\alpha \rho} d x\right)^{1 / \rho}
\end{equation*}
which yields $\|u\|_{W^{1, \alpha}(G)} \leq \rho\|v\|_{L^{q+1}(G)}^{\rho-1}\|v\|_{W^{1, q+1}(G)}$ and \eqref{sobolev inequality} becomes
\begin{equation}
	\|v\|_{L^{q+1}(G)}^{\rho} \leq C_{6} \rho^{1-\lambda}\|v\|_{L^{q+1}(G)}^{(\rho-1)(1-\lambda)}\|v\|_{W^{1, q+1}(G)}^{1-\lambda}\|v\|_{L^{\sigma +1}(G)}^{\lambda \rho}
\end{equation}
If we compute the exponents we get
\begin{equation*}
	\frac{1-\lambda}{\lambda \rho+1-\lambda}=\frac{N(q-\sigma)}{(q+1)(\sigma+1)+N(q-\sigma)}=\frac{(q+1) \theta-1}{q}
\end{equation*}
and
\begin{equation*}
	\frac{\lambda \rho}{\lambda \rho+1-\lambda}=\frac{(q+1)(\sigma+1)}{N(q-\sigma)+(q+1)(\sigma+1)}=\frac{(q+1)(1-\theta)}{q}
\end{equation*}
which is \eqref{claim equation}.

Fourth step. End of the proof. We use \eqref{trace inequality of generalized l1} and \eqref{claim equation} and get
\begin{equation}
\|v\|_{L^{q+1}(\partial G)} \leq C_{7}\|v\|_{W^{1, q+1}(G)}^{1 /(q+1)}\|v\|_{W^{1, q+1}(G)}^{(q \theta+\theta-1) /(q+1)}\|v\|_{L^{\sigma+1}(G)}^{1-\theta}
\end{equation}
where $\theta=N(q-\sigma)+\sigma+1 / N(q-\sigma)+(q+1)(\sigma+1) ;$ using \eqref{ehrling inequality} yields finally 
\begin{equation}
\|v\|_{L^{q+1}(\partial G)} \leq C\left(\|D v\|_{L^{q+1}(G)}+\|v\|_{L^{\sigma+1}(G)}\right)^{\theta}\|v\|_{L^{\sigma+1}(G)}^{1-\theta}
\end{equation}
\end{proof}

\section{Ehrling's lemma}
\begin{lemma}[Ehrling's lemma]
	Let $(X,\|\cdot\|_X),$ $(Y,\|\cdot\|_Y)$ and $(Z,\|\cdot\|_Z)$ be three banach spaces. Assume that:
	$X$ is compactly embedded in $Y$: i.e. $X \subset Y$ and every $\|\cdot\|_X$ -bounded sequence in$ X$ has a subsequence that is $\|\cdot\|_Y$ convergent; and
	$Y$ is continuously embedded in $Z$: i.e. $Y \subset Z$ and there is a constant $k$ so that 
	$\|y\|_Z \leqslant k\|y\|_Y$ for every $y \in Y$.
	Then, for every $\varepsilon > 0$, there exists a constant $C(\varepsilon)$ such that, for all $x \in X$,
	
	${\displaystyle \|x\|_{Y}\leqslant \varepsilon \|x\|_{X}+C(\varepsilon )\|x\|_{Z}}.$
\end{lemma}
\begin{proof}
	proof by contradiction.
	for some given $\varepsilon_0 > 0,$ for all $n \in N^ \star$, there exists $x_n \in X$, such that
	${\displaystyle \|x_n\|_{Y} > \varepsilon_0 \|x_n\|_{X}+n\|x_n\|_{Z}}.$ Let $\tilde{x}_n=\dfrac{x_n}{\|x_n\|_Y},$we have $\|\tilde{x}_n\|_X < 1$ and $\|\tilde{x}_n\|_Z < \dfrac{1}{n}.$ on the one hand, 	$X$ is compactly embedded in $Y$, therefore $\{\tilde{x}_n\}$ has a subsequence that is $\|\cdot\|_Y$ convergent, noted $\{\tilde{x}_n\}$ as well; on the other hand, 	$Y$ is continuously embedded in $Z$, so $\{\tilde{x}_n\}$ is $\|\cdot\|_Z$ convergent. noticing $\|\tilde{x}_n\|_Z\mapsto0,$ we have $ \tilde{x}_n \mapsto 0 $ in $Z$, thus $ \tilde{x}_n \mapsto 0 $ in $Y$. However, $\|\tilde{x}_n\|_Y=1$ which has a contradiction with $ \tilde{x}_n \mapsto 0 $ in $Y$ .
\end{proof}

\section{inequality}
\begin{lemma}
	\begin{equation}
		\|fgh\|_1\leqslant \|f\|_a \|g\|_b \|h\|_c, \quad \text{if } a^{-1} +b^{-1} + c^{-1}=1.
	\end{equation}
\end{lemma}
\begin{lemma}
	\label{le: Young inequality for convolutions}
	if $r,p,q\geqslant1$ and such that $1+r^{-1} = p^{-1}+q^{-1}$, then
	\begin{equation}
		\|F*G\|_r\leqslant\|F\|_p\|G\|_q.
	\end{equation}
	\end{lemma}
\begin{proof}
	using the generalized H\"older inequality above, 
	with $f=F^{1-\dfrac{p}{r}},g=G^{1-\dfrac{q}{r}}$, $h=F^{\dfrac{p}{r}}G^{\dfrac{q}{r}}$ and $a=\dfrac{q}{q-1}$, $b=\dfrac{p}{p-1}$, $c= \dfrac{1}{r}$.
\end{proof}



\begin{lemma}
	for any $z\in L^p(\Omega)$, there exists $c_1>0$ independent of $p$ and $q$ such that
	\begin{equation}\label{greenapproximation}
		\left\|e^{t \Delta} z\right\|_{L^{p}(\Omega)} \leqslant c_{1} t^{-\frac{n}{2}\left(\frac{1}{q}-\frac{1}{p}\right)}\|z\|_{L^{q}(\Omega)},
	\end{equation}
with $1\leqslant q\leqslant p < \infty.$
\end{lemma}
\begin{proof}
	with the help of Green function $G$, we can express $e^{t\Delta}$ explicitly 
	\begin{equation}
		e^{t\Delta}z = \int_{\Omega}G(x,t;0,y)z(y)\mathit{d}y,
	\end{equation}
by pointwise estimate of Green function of Neumman heat semigroup, we have $c_2,c_3>0$ only dependent of $\Omega$ such that
\begin{equation}
	|G(x,t;0,y)| \leqslant \dfrac{c_2}{t^\frac{n}{2}}e^{\frac{c_3|x-y|^2}{t}},
\end{equation}
so one can verify \eqref{greenapproximation} with calculation.
\end{proof}


\begin{lemma}
	 Let $\left(e^{t \Delta}\right)_{t \geqslant 0}$ be the Neumann heat semigroup in $\Omega,$ and let $\lambda_{1}>0$ denote the first nonzero eigenvalue of $-\Delta$ in $\Omega$ under Neumann boundary conditions. Then there exist constants $C_{1}, \ldots, C_{4}$ depending on $\Omega$ only which have the following properties.
	 \begin{itemize}
	 	\item [i] If $1 \leqslant q \leqslant p \leqslant \infty$ then
	$$
	\left\|e^{t \Delta} w\right\|_{L^{p}(\Omega)} \leqslant C_{1}\left(1+t^{-\frac{n}{2}\left(\frac{1}{q}-\frac{1}{p}\right)}\right) e^{-\lambda_{1} t}\|w\|_{L^{q}(\Omega)} \quad \text { for all } t>0
	$$
	holds for all $w \in L^{q}(\Omega)$ satisfying $\int_{\Omega} w=0$
	\item [ii] If $1 \leqslant q \leqslant p \leqslant \infty$ then
	$$
	\left\|\nabla e^{t \Delta} w\right\|_{L^{p}(\Omega)} \leqslant C_{2}\left(1+t^{-\frac{1}{2}-\frac{n}{2}\left(\frac{1}{q}-\frac{1}{p}\right)}\right) e^{-\lambda_{1} t}\|w\|_{L^{q}(\Omega)} \quad \text { for all } t>0
	$$
	is true for each $w \in L^{q}(\Omega)$.
	\item [iii] If $2 \leqslant p<\infty$ then
	$$
	\left\|\nabla e^{t \Delta} w\right\|_{L^{p}(\Omega)} \leqslant C_{3} e^{-\lambda_{1} t}\|\nabla w\|_{L^{p}(\Omega)} \text { for all } t>0
	$$
	is valid for all $w \in W^{1, p}(\Omega)$.
	\item [iv] Let $1<q \leqslant p<\infty .$ Then
	$$
	\left\|e^{t \Delta} \nabla \cdot w\right\|_{L^{p}(\Omega)} \leqslant C_{4}\left(1+t^{-\frac{1}{2}-\frac{n}{2}\left(\frac{1}{q}-\frac{1}{p}\right)}\right) e^{-\lambda_{1} t}\|w\|_{L^{q}(\Omega)} \quad \text { for all } t>0
	$$
	 \end{itemize}
	holds for all $w \in\left(C_{0}^{\infty}(\Omega)\right)^{n} .$ Consequently, for all $t>0$ the operator $e^{t \Delta} \nabla \cdot$ possesses a uniquely determined extension to an operator from $L^{q}(\Omega)$ into $L^{p}(\Omega),$ with norm controlled according to (1.7)
	
\end{lemma}
\begin{proof}
	Proof. (i) For $t<2,(1.4)$ is a consequence of the well-known smoothing estimate
	$$
	\left\|e^{t \Delta} z\right\|_{L^{p}(\Omega)} \leqslant c_{1} t^{-\frac{n}{2}\left(\frac{1}{q}-\frac{1}{p}\right)}\|z\|_{L^{q}(\Omega)} \quad \text { for all } t<2
	$$
	which can be checked for some $c_{1}$ independent of $p$ and $q$ and all $z \in L^{q}(\Omega)$ using pointwise estimates for Green's function of the Neumann heat semigroup [22, Theorem 2.2]. As to $t \geqslant 2,$ we first note that upon integrating the heat equation and using the variational definition of $\lambda_{1}$ we obtain
	$$
	\frac{1}{2} \frac{d}{d t} \int_{\Omega}\left|e^{t \Delta} w\right|^{2}=-\int_{\Omega}\left|\nabla e^{t \Delta} w\right|^{2} \leqslant-\lambda_{1} \int_{\Omega}\left|e^{t \Delta} w\right|^{2}
	$$
	for all $t>0$ and each smooth $w$ satisfying $\int_{\Omega} w=0 .$ Therefore,
	$$
	\left\|e^{t \Delta} w\right\|_{L^{2}(\Omega)} \leqslant e^{-\lambda_{1} t}\|w\|_{L^{2}(\Omega)} \quad \text { for all } t>0
	$$
	holds for all $w \in L^{2}(\Omega)$ with $\int_{\Omega} w=0$. 
	Now for $p<2$, using H\"older's inequality and then and (1.8) we find
	$$
	\begin{aligned}
		\left\|e^{t \Delta} w\right\|_{L^{p}(\Omega)} & \leqslant|\Omega|^{\frac{2-p}{2 p}}\left\|e^{t \Delta} w\right\|_{L^{2}(\Omega)} \leqslant|\Omega|^{\frac{2-p}{2 p}} e^{-\lambda_{1}(t-1)}\left\|e^{\Delta} w\right\|_{L^{2}(\Omega)} \\
		& \leqslant|\Omega|^{\frac{2-p}{2 p}} c_{1} e^{-\lambda_{1}(t-1)}\|w\|_{L^{q}(\Omega)}
	\end{aligned}
	$$
	for all $t \geqslant 2 .$ By a similar reasoning, for $p \geqslant 2$ we derive
	$$
	\left\|e^{t \Delta} w\right\|_{L^{p}(\Omega)} \leqslant c_{1}\left\|e^{(t-1) \Delta} w\right\|_{L^{2}(\Omega)} \leqslant c_{1} e^{-\lambda_{1}(t-2)}\left\|e^{\Delta} w\right\|_{L^{2}(\Omega)} \leqslant c_{1}^{2} e^{-\lambda_{1}(t-2)}\|w\|_{L^{q}(\Omega)}
	$$
	for all $t \geqslant 2,$ from which the claim follows.
	(ii) We first note that for some $c_{2}>0$ independent of $p$
	$$
	\left\|\nabla e^{t \Delta} w\right\|_{L^{p}(\Omega)} \leqslant c_{2} t^{-\frac{1}{2}}\|w\|_{L^{p}(\Omega)} \quad \text { for all } t \leqslant 1
	$$
	holds for all $w \in L^{p}(\Omega) .$ In fact, for $p=1$ and for $p=\infty$ this can be seen using pointwise estimates for the spatial gradient of Green's function of $e^{t \Delta}[22,$ Theorem 2.2$],$ whereby (1.10) for $1<p<\infty$ follows from a Marcinkiewicz-type interpolation argument (cf. $[10,$ Theorem 9.8$]$ ). In order to combine this with $(1.4),$ we write $\bar{w}:=\frac{1}{|\Omega|} \int_{\Omega} w$ and thus have $\int_{\Omega}(w-\bar{w})=0 .$ since constants are invariant under $e^{t \Delta},$ we thus obtain from (1.10) and (1.4)
	
	$$
	\begin{aligned}
		\left\|\nabla e^{t \Delta} w\right\|_{L^{p}(\Omega)} &=\left\|\nabla e^{\frac{t}{2} \Delta} e^{\frac{t}{2} \Delta}(w-\bar{w})\right\|_{L^{p}(\Omega)} \\
		& \leqslant c_{2}\left(\frac{t}{2}\right)^{-\frac{1}{2}}\left\|e^{\frac{t}{2} \Delta(w-\bar{w})}\right\|_{L^{p}(\Omega)} \\
		& \leqslant c_{2} C_{1}\left(\frac{t}{2}\right)^{-\frac{1}{2}}\left(1+\left(\frac{t}{2}\right)^{-\frac{n}{2}\left(\frac{1}{q}-\frac{1}{p}\right)}\right) e^{-\lambda_{1} \frac{t}{2}}\|w-\bar{w}\|_{L^{q}(\Omega)}
	\end{aligned}
	$$
	which implies (1.5) for $t<2 .$ For $t \geqslant 2$ we split $e^{t \Delta}$ in a different way to see that
	$$
	\begin{aligned}
		\left\|\nabla e^{t \Delta} w\right\|_{L^{p}(\Omega)} &=\left\|\nabla e^{\Delta} e^{(t-1) \Delta}(w-\bar{w})\right\|_{L^{p}(\Omega)} \leqslant c_{2}\left\|e^{(t-1) \Delta}(w-\bar{w})\right\|_{L^{p}(\Omega)} \\
		& \leqslant c_{2} C_{1}\left(1+(t-1)^{-\frac{n}{2}\left(\frac{1}{q}-\frac{1}{p}\right)}\right) e^{-\lambda_{1}(t-1)}\|w-\bar{w}\|_{L^{q}(\Omega)} \\
		& \leqslant 4 c_{2} C_{1} e^{-\lambda_{1}(t-1)}\|w\|_{L^{q}(\Omega)}
	\end{aligned}
	$$
	for all $t \geqslant 2 .$ This together with (1.11) proves (1.5).
	(iii) Passing to $\hat{w}:=w-\frac{1}{12} \int_{\Omega} w$ if necessary, we may assume that $\int_{\Omega} w=0 .$ We first consider the case $t \geqslant 1$, in which we apply (ii), (i) and the Poincaré inequality to find
	$$
	\begin{aligned}
		\left\|\nabla e^{t \Delta} w\right\|_{L^{p}(\Omega)} & \leqslant 2 C_{2}\left\|e^{(t-1) \Delta} w\right\|_{L^{p}(\Omega)} \leqslant 4 C_{2} C_{1} e^{-\left(\lambda_{1}-1\right) t}\|w\|_{L^{p}(\Omega)} \\
		& \leqslant 4 C_{2} C_{1} c_{P} e^{-\left(\lambda_{1}-1\right) t}\|\nabla w\|_{L^{p}(\Omega)}
	\end{aligned}
	$$
	which yields (1.6) for all $t \geqslant 1$ and any $p \in(1, \infty),$ because, as can easily be verified, the Poincaré constant $c_{P}$ can be chosen independent of $p$ Next, for $p=2$, multiplying $\left(e^{t \Delta} w\right) t=\Delta e^{t \Delta} w$ by $-\Delta e^{t \Delta} w$ and integrating shows that
	$$
	\left\|\nabla e^{t \Delta} w\right\|_{L^{2}(\Omega)} \leqslant\|\nabla w\|_{L^{2}(\Omega)} \quad \text { for all } t \geqslant 0
	$$
	On the other hand, it is known [22, formula (2.39)] that for some $c_{3} \geqslant 1$,
	$$
	\left\|\nabla e^{t \Delta} w\right\|_{L^{\infty}(\Omega)} \leqslant c_{3}\|\nabla w\|_{L^{\infty}(\Omega)} \quad \text { for all } t \in(0,1)
	$$
	for each $w \in \hat{C}^{1}(\bar{\Omega}):=\left\{z \in C^{1}(\bar{\Omega}) \mid \frac{\partial z}{\partial v}=0\right.$ on $\left.\partial \Omega\right\} .$ A Marcinkiewicz interpolation as in
	(ii) now asserts that (1.6) is valid for each $p \in[2, \infty)$ and $t \in(0,1)$ and all $w \in \hat{C}^{1}(\bar{\Omega}),$ so that all that remains to be shown is that $\hat{C}^{1}(\bar{\Omega})$ is dense in $W^{1, p}(\Omega) .$ To sketch a possible way to see this, we let $w \in$ $W^{1, p}(\Omega)$ be given and fix $\varepsilon>0 .$ Then there exists $w_{1} \in C^{1}(\bar{\Omega})$ such that $\left\|w-w_{1}\right\|_{W^{1, p}(\Omega)}<\frac{\varepsilon}{2} .$ Given $x^{0} \in \partial \Omega,$ applying a shifting and local flattening procedure if necessary, we may assume that $x^{0}=0$ that $\Omega \subset\left\{x_{n}>0\right\}:=\left\{x=\left(x_{1}, \ldots, x_{n}\right) \in \mathbb{R}^{n} \mid x_{n}>0\right\}$ and that $\partial \Omega$ is a part of $\left\{x_{n}=0\right\}$ near $x^{0} .$ For
	near $x^{0},$ we define $w_{x^{0}}(x):=w_{1}\left(x_{1}, \ldots, x_{n-1}, 0\right)$ for $x \in \Omega,$ so that $w_{x^{0}}=w_{1}$ on $\partial \Omega$ and $\frac{\partial w_{x} 0}{\partial v}=0$
	on $\partial \Omega$ hold near $x^{0}$. The same is thus valid for $w_{x^{0}, k}(x):=w_{x^{0}}(x) \cdot\left(1-\chi\left(k x_{n}\right)\right)+w_{1}(x) \cdot \chi\left(k x_{n}\right)$, where $\chi \in C^{\infty}(\mathbb{R})$ satisfies $\chi_{[2, \infty)} \leqslant \chi \leqslant \chi_{[1, \infty)} .$ since $w_{1} \in C^{1}(\bar{\Omega}),$ it is easily checked that $w_{x^{0}, k} \rightarrow w_{1}$
	in $W^{1, p}$ in a neighborhood of $x^{0},$ so that returning to the original coordinates via a suitable partition of unity will provide some $w_{2} \in \hat{C}^{1}(\bar{\Omega})$ such that $\left\|w_{1}-w_{2}\right\|_{W^{1, p}(\Omega)}<\frac{\varepsilon}{2},$ which proves the claim.
	
	(iv) Given $\varphi \in C_{0}^{\infty}(\Omega),$ we use that $e^{t \Delta}$ is self-adjoint in $L^{2}(\Omega)$ and integrate by parts to see that
	$$
	\begin{aligned}
		\left|\int_{\Omega} e^{t \Delta} \nabla \cdot w \varphi\right| &=\left|-\int_{\Omega} w \cdot \nabla e^{t \Delta} \varphi\right| \leqslant\|w\|_{L^{q}(\Omega)} \cdot\left\|\nabla e^{t \Delta} \varphi\right\|_{L^{q^{\prime}}(\Omega)} \\
		& \leqslant C_{2}\left(1+t^{-\frac{1}{2}-\frac{\eta}{2}\left(\frac{1}{p^{\prime}}-\frac{1}{q^{\prime}}\right)}\right) e^{-\lambda_{1} t}\|w\|_{L^{q}(\Omega)}\|\varphi\|_{L^{p^{\prime}}(\Omega)}
	\end{aligned}
	$$
	for all $t>0$ holds in view of (ii), where $\frac{1}{p}+\frac{1}{p^{\prime}}=1$ and $\frac{1}{q}+\frac{1}{q^{\prime}}=1 .$ since $\frac{1}{p^{\prime}}-\frac{1}{q^{\prime}}=\frac{1}{q}-\frac{1}{p},$ taking the
	supremum over all such $\varphi$ satisfying $\|\varphi\|_{L^{\prime}(\Omega)} \leqslant 1$ we arrive at $(1.7) .$
\end{proof}
\section{uniform integrability}
\begin{theorem}[Dunford-Pettis theorem] 
A class of random variables $X_{n} \subset L^{1}(\mu)$ is uniformly integrable if and only if it is relatively compact for the weak topology $\sigma\left(L^{1}, L^{\infty}\right)$.
\end{theorem}
\begin{theorem}[de la Vallée-Poussin theorem] 
The family $\left\{X_{\alpha}\right\}_{\alpha \in \mathrm{A}} \subset L^{1}(\mu)$ is uniformly integrable if and only if there exists a non-negative increasing convex function $G(t)$ such that $\lim _{t \rightarrow \infty} \frac{G(t)}{t}=\infty$ and $\sup _{\alpha} \mathrm{E}\left(G\left(\left|X_{\alpha}\right|\right)\right)<\infty$
\end{theorem}
\section{Lp}

\begin{lemma}
	Let $\Omega \subset \mathbb{R}^{d}$ be a bounded domain and let $\left(u_{\varepsilon}\right)$ be bounded in $L^{p}(\Omega),$ where $1<p \leq \infty,$ such that $u_{\varepsilon} \rightarrow u$ a.e. in $\Omega$ as $\varepsilon \rightarrow 0 .$ Then, for $1 \leq q<p$
	$$
	u_{\varepsilon} \rightarrow u \text { strongly in } L^{q}(\Omega).
	$$
\end{lemma}
\begin{proof}
	set $M > 0$, such that $\|u_\varepsilon\|_p \leqslant M$. by Fatou's lemma,$u\in L^p(\Omega),$ as 
	\(\|u\|_p \leqslant \liminf\limits_{\varepsilon\rightarrow 0}\|u_\varepsilon\|_p \leqslant M\).
	by H\"{o}lder inequality, for $1 \leqslant q\leqslant p,$ we have \[\|u\|_q \leqslant |\Omega|^{\frac{p-q}{pq}} \|u\|_p,\]
	so $u,u_\varepsilon$ are in $L^q(\Omega)$.
	
	by Egoroff theorem, for any given $\tau > 0$, there exists a measurable set $\Omega_\tau$ with $|\Omega_\tau| < \tau$, such that $u_{\varepsilon_n}$ converges to $u$ uniformly on $\Omega\backslash \Omega_\tau$, while $\varepsilon_n \rightarrow 0$, as $n \rightarrow 0.$
	that is to say, for $\tau$ given above, there exists a positive integer $N$, such that for any $n > N$, we have $\sup\limits_{x\in\Omega\backslash\Omega_\tau}|u_{\varepsilon_n} - u| \leqslant \tau$. therefore
	\begin{equation*}
		\begin{split}
			\|u_{\varepsilon_n} - u\|_{L^q(\Omega)} &= \|u_{\varepsilon_n} - u\|_{L^q(\Omega\backslash\Omega_\tau)} + \|u_{\varepsilon_n} - u\|_{L^q(\Omega_\tau)}\\ 
			& \leqslant \tau|\Omega|^{\frac{1}{q}} + |\Omega_\tau|^{\frac{p-q}{pq}}\|u_{\varepsilon_n } - u\|_p\\
			& \leqslant \tau|\Omega|^{\frac{1}{q}} + 2M\tau^{\frac{p-q}{pq}},
		\end{split}
	\end{equation*}
we can conclude $u_{\varepsilon_n} \rightarrow u \text { strongly in } L^{q}(\Omega)$. the subscript $n$ can be dropped by the common contradiction discussion.
\end{proof}

a simple example, consider $u_n(x)=n^\delta x^n\in L^2(0,1)$. clearly, $u_n\rightarrow0$ a.e. in $(0,1)$.
$u_n$ are bounded if $\delta \leqslant \dfrac{1}{2}.$

if $\delta = \dfrac{1}{2}$, $u_n\not\rightarrow0$ in $L^2(0,1)$; otherwise in $L^q(0,1),1\leqslant q < 2$.

if $\delta < \dfrac{1}{2}$, $u_n\rightarrow0$ in $L^q(0,1), 1\leqslant q \leqslant 2$.

\begin{lemma}
Let $\Omega$ be an open subset of $\mathcal{R}^n$, let $1<p<\infty$, and let functions $f_k\in L^p(\Omega)$, $k\geqslant1$, and $f\in L^p(\Omega)$ be such that the sequence $\{f_k\}_{k=1}^{\infty}$ is bounded in $L^p(\Omega)$ and $f_k$ converges almost everywhere in $\Omega$ to $f$ as $k\rightarrow \infty$. Show that 
\begin{equation}
	f_k\rightharpoonup f \text{ in } L^p(\Omega) \text{ as } k\rightarrow \infty.
\end{equation}	
\end{lemma}

\begin{proof}
	we only need to verify that
	\begin{equation}
		\lim_{k\rightarrow\infty}\int_\Omega f_kg\rightarrow\int_\Omega fg \text{ for any } g\in L^q(\Omega) \text{ with } \frac1p+\frac1q=1.
	\end{equation}

	for one fixed $g\in L^q(\Omega)$, given $\varepsilon>0$, there exists a $R>0$ large enough such that 
	\begin{equation}
		\int_{\Omega\backslash B_R}|g|^q < (\frac\varepsilon2)^q.
	\end{equation}
therefore 
\begin{equation}
	\begin{split}
	|\int_\Omega(f_k-f)g|&\leqslant|\int_{\Omega\backslash B_R}(f_k-f)g| + |\int_{B_R}(f_k-f)g|\\
	&\leqslant 2M\frac\varepsilon2 + |\int_{B_R}(f_k-f)g|
\end{split}
\end{equation}
as $\{f_k\}_{k=1}^{\infty}$ is bounded in $L^p(\Omega)$. 
By Egoroff theorem, since $f_k$ converges almost everywhere in $\Omega$ to $f$ as $k\rightarrow \infty$, 
for any given $\tau>0$, there exists a measurable set $\Omega_\tau$ with $|\Omega_\tau|<\tau$, 
such that $f_n$ converges to $f$ uniformly on $B_R\backslash \Omega_\tau$. 
Furthermore, for $\varepsilon$ given above, there exists a $\tau>0$, 
such that for any $|\Omega_\tau|<\tau$, $\int_{\Omega_\tau}|g|^q<(\frac{\varepsilon}{2})^q.$ 
Consequently, there exists a large $K>0$, 
such that for any $k>K$, $|f_k-f|^p<\varepsilon/|B_R|$ for any $x\in B_R\backslash \Omega_\tau$, thus
\begin{equation}
	\begin{split}
		|\int_{B_R}(f_k-f)g|&\leqslant |\int_{B_R\backslash \Omega_\tau}(f_k-f)g| 
		+ |\int_{\Omega_\tau}(f_k-f)g|\\
		&\leqslant \|g\|_{L^q}\varepsilon +2M\frac{\varepsilon}{2}.
	\end{split}
\end{equation}
Now, we have for any $\varepsilon>0$, there exists a $K>0$ large enough such that 
\begin{equation}
	|\int_\Omega(f_k-f)g|\leqslant (2M+ \|g\|_{L^q})\varepsilon \text{ for all } k> K.
\end{equation}
\end{proof}

\section{Sobolev inequality}

\begin{lemma}
  \label{le: gninequality}
  Let $\Omega = B_R\subset\mathbb R^n$ for some $R>0$ and $n>2$.
  There exists $C_e > 0$ such that 
  \begin{equation}
    \label{eq: grad gn e}
    \|\nabla \varphi\|_{L^2(\Omega)} 
    \leq C_e \|\Delta \varphi - \varphi\|_{L^2(\Omega)}^{(n+2)/(n+4)} 
    \|\varphi\|_{L^1(\Omega)}^{2/(n+4)}
  \end{equation}
  holds for each $\varphi\in W^{2,2}(\Omega)$ 
  with $\partial_\nu \varphi = 0$ on the boundary $\partial\Omega$ in the sense of trace.  
  For any $s>0$ there exists $C(s) > 0$ such that 
  \begin{equation}
    \label{eq: L2 gn}
    \|\psi\|_{L^2(\Omega)}^2
    \leq C(s) \|\nabla \psi\|_{L^2(\Omega)}^{2n/(n+2)}\|\psi\|_{L^1(\Omega)}^{4/(n+2)} 
    + C(s)\|\psi\|_{L^s(\Omega)}^2
  \end{equation}
  and 
  \begin{equation}
    \label{eq: L2 gn rho}
    \|\psi\|_{L^2(B_\rho)}^2 
    \leq C(s) \|\nabla \psi\|_{L^2(B_\rho)}^{2n/(n+2)}\|\psi\|_{L^1(B_\rho)}^{4/(n+2)} 
    + C(s)\rho^{(s-2)n/s}R^{-(s-2)n/s}\|\psi\|_{L^{s}(B_\rho)}^2,
  \end{equation}
  are valid for all $\psi\in W^{1,2}(\Omega)$ and $\rho > 0$, 
  where $C_{s}$ is independent of $\rho > 0$.
\end{lemma}

\begin{proof}
  Gagliardo-Nirenberg inequality \cite{Nirenberg1959} and 
  global $L^p$-estimates for elliptic problem subject to homogeneous Neumann boundary conditions provide $C_{gn} > 0$ and $C_e > 0$ 
  such that 
  \begin{displaymath}
    \|\nabla \varphi\|_{L^2(\Omega)} 
    \leq C_{gn} \|\varphi\|_{W^{2,2}(\Omega)}^{(n+2)/(n+4)} 
    \|\varphi\|_{L^1(\Omega)}^{2/(n+4)}
    \leq C_e \|\Delta \varphi - \varphi\|_{L^2(\Omega)}^{(n+2)/(n+4)} 
    \|\varphi\|_{L^1(\Omega)}^{2/(n+4)},
  \end{displaymath}
  which implies \eqref{eq: grad gn e}.
  \eqref{eq: L2 gn} is exactly one of Gagliardo-Nirenberg inequalities.
  We intend to verify \eqref{eq: L2 gn rho} by a scaling argument.
  Let $\psi\in W^{1,2}(\Omega)$, 
  then $\psi^{(\rho)} := \psi(Rx/\rho) \in W^{1,2}(B_\rho)$.
  We compute 
  \begin{align*}
    % \|\psi^{(\rho)}\|_{L^1(B_\rho)}
    % &= \rho^nR^{-n}\|\psi\|_{L^1(\Omega)},
    % \|\psi^{(\rho)}\|_{L^2(B_\rho)}^2
    % &= \rho^nR^{-n}\|\psi\|_{L^2(\Omega)}^2,\\
    \|\psi^{(\rho)}\|_{L^s(B_\rho)}^s
    = \rho^nR^{-n}\|\psi\|_{L^s(\Omega)}^s
    \quad \text{and}\quad
    \|\nabla\psi^{(\rho)}\|_{L^2(B_\rho)}^2
    = \rho^{n-2}R^{2-n}\|\nabla\psi\|_{L^2(\Omega)}^2,
  \end{align*} 
  and thus it follows from \eqref{eq: L2 gn} that  
  \begin{displaymath}
    \begin{aligned} 
    \|\psi^{(\rho)}\|_{L^2(B_\rho)}^2 
    &\leq C(s) \rho^nR^{-n}\|\nabla \psi\|_{L^2(\Omega)}^{2n/(n+2)}\|\psi\|_{L^1(\Omega)}^{4/(n+2)} 
    + C(s)\rho^nR^{-n}\|\psi\|_{L^{s}(\Omega)}^2\\
    &= C(s) \|\nabla \psi^{(\rho)}\|_{L^2(B_\rho)}^{2n/(n+2)}\|\psi^{(\rho)}\|_{L^1(B_\rho)}^{4/(n+2)} 
    + C(s)\rho^{(s-2)n/s}R^{-(s-2)n/s}\|\psi\|_{L^{s}(B_\rho)}^2
    \end{aligned}
  \end{displaymath}
  for all $\rho > 0$.
  and 
  \begin{displaymath}
    \|\psi\|_{L^2(\Omega)} 
    \leq C_{gn} \|\nabla \psi\|_{L^2(\Omega)}^{n/(n+2)}\|\psi\|_{L^1(\Omega)}^{2/(n+2)} 
    + C_{gn}\|\psi\|_{L^1(\Omega)}
  \end{displaymath}
  are valid for all $\psi\in W^{1,2}(\Omega)$ and some $C_{gn} > 0$, 
  where it can be verified by a scaling argument that 
$C_{gn}$ is independent of $\rho\in(0,R)$,
\end{proof}


\section{a weak convergence lemma in \texorpdfstring{$L_1$}{} related to multiplication}

\begin{lemma}
denote $\mu$ Lebesgue measure, $\Omega$ a bounded domain in $\mathbb{R}^n$, abbreviate $L^1:= L^1(\Omega, \mu)$ and $L^\infty:= L^\infty(\Omega, \mu)$. if 
\begin{itemize}
\item $u_n\rightharpoonup u$ in $L^1$;
\item $u_nv_n\rightharpoonup z$ in $L^1$;
\item either $v_n\rightarrow v$ $\mu$-a.e.  or $v_n\stackrel{\mu}{\Longrightarrow}v$,
\end{itemize}
then $z = uv$ a.e..
\end{lemma}
\begin{proof}
for each $\Delta\subset\Omega$ is a measurable set, denote $\chi_\Delta$ set function, that is for each $x\in\Delta, \chi_\Delta(x) = 1$, otherwise $\chi_\Delta(x)=0$.

\textbf{Step 1.} supposed that $v\in L^\infty$,
if $v_n\rightarrow v$ a.e., by Egrov theorem, for each $\delta > 0$ there exist a 
measurable set $\Delta\subset\Omega$, such that $\mu(\Delta) < \delta$ and $v_n$ converges uniformly to $v$ in $\Delta^c$. Since weak convergence in $L^1$ implies uniformly integrable, that is for any $\varepsilon > 0$, there exists a $\delta>0$ and arbitrary measurable set $\Delta$ satisfied with $\mu(\Delta) < \delta$, we have $|(u_nv_n, \chi_\Delta)| < \varepsilon$ and $|(u_n, \chi_{\Delta})| < \varepsilon$.

for $\phi\in L^\infty$ and $|\phi|\leqslant1$, 
\begin{equation*}
\begin{split}
|(u_nv_n - uv, \phi)| & \leqslant |(u_nv_n - u_nv, \phi)| + |(u_nv - uv, \phi)|\\
&\leqslant |(u_nv_n - u_nv, \chi_\Delta\phi)| + |(u_nv_n - u_nv, \chi_{\Delta^c}\phi)| \\
&\quad + |(u_n - u, \phi v)|\\
&\leqslant |(u_nv_n, \chi_\Delta\phi)| + |(u_n, \chi_\Delta\phi v)| + |(u_nv_n - u_nv, \chi_{\Delta^c}\phi)| \\
&\quad+ |(u_n - u, \phi v)|\\
&\leqslant C\varepsilon \text{ as } n\rightarrow\infty.
\end{split}
\end{equation*}
Letting $\varepsilon\rightarrow0$, we have $u_nv_n\rightharpoonup uv$ in $L^1$ and hence 
$uv=z$ $\mu$-a.e..

or rather, if $v_n\stackrel{\mu}{\Longrightarrow}v$, denote $\Delta_\varepsilon:=\{|v_n - v| >\varepsilon\}$, then for each $\delta>0$, $ \varepsilon>0$, there exist $N$, such that  $$\mu(\Delta_\varepsilon) < \delta \text{ for each } n > N.$$
\begin{equation*}
\begin{split}
|(u_nv_n - uv, \phi)| & \leqslant |(u_nv_n - u_nv, \phi)| + |(u_nv - uv, \phi)|\\
&\leqslant |(u_nv_n - u_nv, \chi_{\Delta_\varepsilon}\phi)| + |(u_nv_n - u_nv, \chi_{\Delta_\varepsilon^c}\phi)| \\
&\quad + |(u_n - u, \phi v)|\\
&\leqslant |(u_nv_n, \chi_{\Delta_\varepsilon}\phi)| + |(u_n, \chi_{\Delta_\varepsilon}\phi v)| + |(u_nv_n - u_nv, \chi_{\Delta^c}\phi)| \\
&\quad+ |(u_n - u, \phi v)|\\
&\leqslant C\varepsilon \text{ for } n > N.
\end{split}
\end{equation*}
this implies $u_nv_n\rightharpoonup uv$ in $L^1$ and hence 
$uv=z$ $\mu$-a.e..

\textbf{Step 2.} let $\Omega_k:= \{|v|\leqslant k\}$ for $k > 0$. then $\mu(\Omega_k)\nearrow\mu(\Omega)$, $z=uv$ $\mu|_{\Omega_k}$-a.e. and hence $z=uv$ $\mu$-a.e.. 
\end{proof}

\begin{example}[stability of weak convergence or something]
if $u_\varepsilon\rightharpoonup u$ in $L^1$ and $u_\varepsilon\geqslant0$ in the sense of distribution, then $$\frac{u_\varepsilon}{1 + \varepsilon u_\varepsilon}\rightharpoonup u \text{ in } L^1.$$
\begin{proof}
since $\varepsilon u_\varepsilon\rightarrow 0$ in $L^1$, $$\frac{1}{1 + \varepsilon u_\varepsilon}\stackrel{\mu}{\Longrightarrow}1.$$
the desired result is a direct consequence of above lemma.
\end{proof}
\end{example}

\section{a counterpart about weak convergence}

if $f_n\to f$ in $L^p$, it is obvious that $|f_n|\to |f|$ by the triangle inequality.
But this is not true in the case of weak convergence.
Actually, writing $X:=L^2((0,1))$, putting 
\[
f_n = \sqrt2\sin(n\pi t),
\]
it is easy to check that 
\[
\|f_n\|_X = 1.
\]
Using Riemann-Lebesgue theorem,
we have 
\[
f_n \rightharpoonup 0\quad\text{in }X,
\]
but 
\[
|f_n| \not\rightharpoonup 0\quad \text{in }X,
\]
since 
\[
(|f_n|,1)_X = 2\sqrt2/\pi.
\]
Actually, putting $\phi\in C^0(\overline{\Omega})$, by uniformly continuity, 
for any $\varepsilon>0$, there exists $\delta>0$ such that for any $t', t''\in[0,1]$,
\[
|\phi(t')-\phi(t'')|<\varepsilon,\quad\text{provided } |t'-t''|<\delta.
\]
So we estimate for $n>N$ with $N\delta>1$,
\begin{align*}
(|f_n|,\phi)_X 
&= \int_0^1|f_n|\phi\dd t = \sum_{i=0}^{n-1}\int_{i/n}^{(i+1)/n}|f_n|\phi\dd t\\
&= \sum_{i=0}^{n-1}\int_0^{1/n} f_n(t)\phi(i/n+t)\dd t\\
&= \sum_{i=0}^{n-1}\int_0^{1/n} f_n(t)\phi(i/n)\dd t 
	+ \sum_{i=0}^{n-1}\int_0^{1/n}f_n(t)o (\varepsilon)\\
&= \|f_n\|_1\frac1n\sum_{i=1}^{n-1}\phi(i/n) + \|f_n\|_1o(\varepsilon)\\
&\to (|f_1|,1)_X (\phi,1)_X + o(\varepsilon),\quad n\to\infty.
\end{align*}
That is $|f_n|\rightharpoonup (|f_1|,1)_X$ in $X$, 
by the density of $C^0(\overline\Omega)\hookrightarrow X$.

\section{a compact property of unbounded domain}
\begin{lemma}
	$V:=\{f\in H^1(\mathbb{R}^n): \int_{\mathbb R^n}|x|f^2\dd x<\infty\}$ 
	is relatively compact in $L^2(\mathbb{R}^n)$.
\end{lemma}

\begin{proof}
	Let $S$ be a bounded subset of $V$.
	Then for each $\varepsilon>0$, there exists $R>0$ such that 
	for any $f\in S$,
	\[
	\int_{\mathbb{R}^n\setminus B_R}f^2 \leq \frac1R\int_{\mathbb{R}^n\setminus B_R}|x|f^2<\varepsilon.
	\]
	Since $S$ is relatively compact in $L^2(B_R)$, 
	for each $\varepsilon$ given as above,
	there exists a finite set $F_\varepsilon = \{f_1, f_2, \cdot, f_k\}\subset S$ such that 
	for any $f\in S$, 
	\[
	\min_{f_k\in F_\varepsilon}\|f-f_k\|_{L^2(B_R)}< \varepsilon.
	\] 
	Therefore,
	\[
	\min_{f_k\in F_\varepsilon}\|f-f_k\|_{L^{2}(\mathbb R^n)}
	\leq \min_{f_k\in F_\varepsilon}\|f-f_k\|_{L^{2}(B_R)}
		+ \sup_{f_k\in F_\varepsilon}\|f-f_k\|_{L^{2}(\mathbb R^n\setminus B_R)} 
		< 3\varepsilon,
	\]
	which implies $S$ is relatively compact in $L^2(\mathbb R^n)$.
\end{proof}

\section{The Kolmogorov–Riesz theorem}
\begin{theorem}[Kolmogorov-Riesz]
	 Let $1 \leq p<\infty .$ A subset $\mathcal{F}$ of $L^{p}\left(\mathbb{R}^{n}\right)$ is totally bounded if, and only if,
	 \begin{itemize}
	 	\item $\mathcal{F}$ is bounded,
	 	\item for every $\varepsilon>0$ there is some $R$ so that, for every $f \in \mathcal{F}$
	 	$$
	 	\int_{|x|>R}|f(x)|^{p} d x<\varepsilon^{p}
	 	$$
	 	\item for every $\varepsilon>0$ there is some $\rho>0$ so that, for every $f \in \mathcal{F}$ and $y \in \mathbb{R}^{n}$ with $|y|<\rho$
	 	$$
	 	\int_{\mathbb{R}^{n}}|f(x+y)-f(x)|^{p} d x<\varepsilon^{p}
	 	$$	
	 \end{itemize}
\end{theorem}

\begin{proof}
	sufficient. 
	see \cite{Hanche-Olsen2010}.
	necessary.
	for every $\varepsilon>0$, 
	there exists $\{f_\varepsilon^1, f_\varepsilon^2, \cdots, f_\varepsilon^k\}\subset\mathcal{F}$ such that 
	for any $f\in\mathcal{F}$, 
	there exists some $f_\varepsilon^i$, $1\leq i \leq k$ such that 
	\[
	\|f-f_\varepsilon^i\|_p < \varepsilon.
	\]
	Let $\varepsilon=1$, then $\|f\|_p\leq1+\sup_i\|f_1^i\|_p$.
	For any fixed $f_\varepsilon^i$, there $R_\varepsilon^i$ and $\rho_\varepsilon^i$ such that the last two assertions hold, respectively. An application of the triangle inequality extends the conclusion of $f_\varepsilon^i$ to any $f\in\mathcal{F}$.
\end{proof}
\section{Vitali convergence theorem}

\begin{theorem}
Let $\left(f_{n}\right)_{n \in \mathbb{N}} \subseteq L^{p}(X, \tau, \mu), f \in L^{p}(X, \tau, \mu),$ with $1 \leq p<\infty .$ Then, $f_{n} \rightarrow f$ in $L^{p}$ if and only if we have
\begin{itemize}
	\item $f_{n}$ converge in measure to $f$.
	\item For every $\varepsilon>0$ there exists a measurable set $E_{\varepsilon}$ with $\mu\left(E_{\varepsilon}\right)<\infty$ such that for every $G \in \tau$ disjoint from $E_{\varepsilon}$ we have, for every $n \in \mathbb{N}$, $\int_{G}\left|f_{n}\right|^{p} d \mu<\varepsilon^{p}$
	\item  For every $\varepsilon>0$ there exists $\delta(\varepsilon)>0$ such that, if $E \in \tau$ and $\mu(E)<\delta(\varepsilon)$ then, for every $n \in \mathbb{N}$ we have $\int_{E}\left|f_{n}\right|^{p} d \mu<\varepsilon^{p}$	
\end{itemize}
\end{theorem}

\begin{proof}
	sufficient: as $ f\in L^p(X, \tau, \mu) $,
	
	as $ f_n \rightarrow f $ in $ L^p $, for any $ m\in N^\star $, we have $$\frac{1}{m^p} \mu(\{|f_n-f|>\frac{1}{m}\}) \leqslant \|f_n-f\|_{L^p}^p\rightarrow 0, n\rightarrow\infty,$$
	which entails $ f_n $ converge in measure to $ f $.
	
	denote $ B(r) = \{|x|\leqslant r\} $, then for any $ \varepsilon>0 $, there exists a $R_0>0$ such that \[ \|f\|_{L^p(X\backslash B(R_0))} < \frac{\varepsilon}{2},\]
	meanwhile, there exists a positive integer $ N $, such that for any $ n>N $, we have 
	\[ \|f_n-f\|_{L^p(X\backslash B(R_0))} < \frac{\varepsilon}{2};\]
	there exist $ R_1, R_2, \cdots, R_N > 0 $, such that 
	\[ \|f_i\|_{L^p(X\backslash B(R_i))} < \frac{\varepsilon}{2}, i = 1, 2, \cdots, N.\]
	now, we can conclude that for any $ \varepsilon>0 $, there exists a measure set $ E_\varepsilon = B(R)$ with $ R = \max\limits_{i=0}^N R_i,\mu(E_\varepsilon)<\infty $ such that for every $ G\in\tau $ disjoint from $ E_\varepsilon $, we have, for every $ n\in \mathbb{N} $, $ \|f_n\|_{L^p(G)}<\varepsilon $.
	
	the third one can be verified by the similar manner used in the above. by the absolute continuity of lebesgue integral, we assert that for any $ \varepsilon>0, $ there exists $ \delta_0>0 $ such that, if $ \mu(E)<\delta_0, $ we have $ \|f\|_{L^p(E)} < \dfrac\varepsilon2; $ there exists a positive integer $ N $ such that, for any $ n>N $, we have $ \|f_n-f\|_{L^p(E)} < \dfrac{\varepsilon}{2};$ there exist $ \delta_1 > 0  $ such that, if $ \mu(E)< \delta_1, $ 
	\[ \|f_i\|_{L^p(E)} < \frac{\varepsilon}{2}, i = 1, 2, \cdots, N.\]
	now, we can conclude that for  every $\varepsilon>0$ there exists $\delta=\min\{\delta_0, \delta_1\}>0$ such that, if $E \in \tau$ and $\mu(E)<\delta$ then, for every $n \in \mathbb{N}$ we have $\int_{E}\left|f_{n}\right|^{p} d \mu<\varepsilon^{p}$	
	
	necessary: for any $ \varepsilon>0 $, there exist $ \delta(\varepsilon)>0,$ a measurable set $ E_\varepsilon $ with $ \mu(E_\varepsilon)<\infty, $ $N\in\mathbb{N}$ such that, 
	\[ \mu(X_n) < \delta(\varepsilon) \text{for any } n > N \text{ with } X_n:=\{|f_n-f|>\dfrac{\varepsilon}{\mu(E_\varepsilon)^{1+\frac{1}{p}}}\},\]
	thus we have
	\begin{equation*}
		\begin{split}
			\|f_n-f\|_{L^p(X)} &= \|f_n-f\|_{L^p(E_\varepsilon\cap X)} + \|f_n-f\|_{X\backslash L^p(E_\varepsilon)}\\
			&= \|f_n-f\|_{L^p(E_\varepsilon\cap X_n)} + \|f_n-f\|_{L^p(E_\varepsilon\backslash X_n)} +
			\|f_n-f\|_{X\backslash L^p(E_\varepsilon)}\\
			&< 3\varepsilon, \text{for any } n>N,
		\end{split}
	\end{equation*}
\end{proof}

\section{\texorpdfstring{$W^{1,1}$}{W1,1} embeds into \texorpdfstring{$L^\infty$}{bounded space}}
\begin{lemma}
	If $f\in W^{1,1}(\mathbb{R})$, then $|f|\leq \|f\|_{W^{1,1}}$.
	Particularly, $\lim_{\xi\to\infty}f(\xi) = 0$.
\end{lemma}
\begin{proof}
	By density, without loss of generality,
	we may assume that $f\in C^1(\mathbb{R})$, then
	\[
	f(b) = f(a) + \int_a^b f'(\xi)\dd \xi,\quad (a,b)\subset(a, a+1)\subset\mathbb{R},
	\]
	and hence
	\[
	f(b) \leq f(a) + \int_a^{a+1}|f'(\xi)|\dd\xi,
	\]
	and integrating over $(a, a+1)$,
	\begin{align*}
		|f(b)| &\leq \int_a^{a+1}f(\xi)\dd\xi + \int_a^{a+2}|f'(\xi)|\dd\xi,
	\end{align*}
	which implies the desired conclusion.
\end{proof}

\begin{lemma}
	Assume that the nonnegative function $f\in L^1_{\loc}(\mathbb{R}^+)$ complies with 
	\begin{equation}
		\label{eq: int01f to 0}
		\int_t^{t+1}f(\sigma)\dd\sigma \to 0,\quad \text{as } t\to\infty.
	\end{equation}
	Then 
	\begin{equation}
		\label{eq: f to 0}
		\lim_{t\to\infty} f(t) = 0,
	\end{equation}
	if and only if for any $\varepsilon > 0$, 
	there exist $M>0$ and $\delta > 0$ such that 
	for all $t>s>M$ and $t-s<\delta$, 
	it has 
	\begin{equation}
		\label{eq: f(t)-f(s) < epsilon}
		f(t) - f(s) < \varepsilon.
	\end{equation}
\end{lemma}

\begin{proof}
	If \eqref{eq: f to 0} holds, then for any $\varepsilon > 0$, 
	there exist $M>0$ such that 
	for all $t>M$,  
	it has 
	\begin{equation*}
		f(t) < \varepsilon/2
	\end{equation*}
	and thereby 
	\begin{equation*}
		|f(t) - f(s)| < |f(t)| + |f(s)| < \varepsilon,
	\end{equation*}
	for all $s,t>M$.

	If \eqref{eq: f(t)-f(s) < epsilon} holds, then for any $\varepsilon > 0$, 
	there exist $M>0$ and $\delta \in (0,1)$ such that 
	for all $t>s>M$ and $t-s<\delta$, 
	it has 
	\begin{equation*}
		f(t) - f(s) < \varepsilon,
	\end{equation*}
	and thereby 
	\begin{equation*}
		\delta f(t) 
		< \int_{t-\delta}^t f(s)\dd s + \delta\varepsilon
		\leq\int_{t-1}^t f(s)\dd s + \delta\varepsilon,
	\end{equation*}
	for all $t>M+1$,
	by integrating the preceding inequality over $(t-\delta, t)$ with respect to $s$.
	It follows from \eqref{eq: int01f to 0} that \eqref{eq: f to 0} holds.
\end{proof}

\section{Admissible Relative Entropies and Their Generating
	Functions}

\begin{definition}
	 Let $J$ be either $\mathbb{R}$ or $\mathbb{R}^{+}:=(0, \infty) .$ Let $\psi \in C(\bar{J}) \cap C^{4}(J)$ satisfy the conditions
	$$
	\begin{array}{l}
	\psi(1)=0 \\
	\psi^{\prime \prime} \geq 0, \quad \psi^{\prime \prime} \neq 0 \quad \text { on } J \\
	\left(\psi^{\prime \prime \prime}\right)^{2} \leq \frac{1}{2} \psi^{\prime \prime} \psi^{I V} \quad \text { on } J \quad (\text{or } (\dfrac1{\varphi''})''\leqslant0 \quad \text { on } J)
	\end{array}
	$$
	Let $\rho_{1} \in L^{1}\left(\mathbb{R}^{n}\right), \rho_{2} \in L_{+}^{1}\left(\mathbb{R}^{n}\right)$ with $\int \rho_{1} d x=\int \rho_{2} d x=1$ and $\rho_{1} / \rho_{2} \in$
	$\bar{J} \rho_{2}(d x)-$ a.e. Then
	$$
	e_{\psi}\left(\rho_{1} \mid \rho_{2}\right):=\int_{\mathbb{R}^{n}} \psi\left(\frac{\rho_{1}}{\rho_{2}}\right) \rho_{2}(d x)
	$$
	is called an admissible relative entropy (of $\rho_{1}$ with respect to $\rho_{2}$ ) with generating function $\psi$.
\end{definition}

if \(\varphi\) generates a admissible relative entropy, then its normalization $\tilde{\psi}(\sigma)=\psi(\sigma)-\psi^{\prime}(1)(\sigma-1)$ generates the same admissible relative entropy.

\begin{lemma}\label{entropy_on_R}
	For $J=\mathbb{R}$ all admissible entropies are generated by $\varphi(\sigma)=\alpha(\sigma-1)^{2}, \quad \sigma \geq 0 ; \alpha>0 .$
\end{lemma}

\begin{lemma}
 Let $\psi$ generate an admissible relative entropy with $J=\mathbb{R}^{+} .$ Then there exists a logarithmic-type function 
 \begin{equation}\label{logarithmic-type-entropy-generator}
 \chi(\sigma)=\alpha(\sigma+\beta) \ln \frac{\sigma+\beta}{1+\beta}-\alpha(\sigma-1), \quad \sigma>0 ; \alpha>0, \beta \geq 0\end{equation}
  and a quadratic function
  \begin{equation}\varphi(\sigma)=\alpha(\sigma-1)^{2}, \quad \sigma \geq 0 ; \alpha>0\end{equation}
   such that
$$
\chi(\sigma) \leq \psi(\sigma) \leq \varphi(\sigma), \quad \sigma \in J
$$
and hence
$$
0 \leq e_{\chi}\left(\rho_{1} \mid \rho_{2}\right) \leq e_{\psi}\left(\rho_{1} \mid \rho_{2}\right) \leq e_{\varphi}\left(\rho_{1} \mid \rho_{2}\right)
$$
\(\chi\) and $\varphi$ both satisfy the conditions of admissible relative entropy and thus generate, respectively, an admissible sub- and super-entropy for $e_{\psi}$	
\end{lemma}

\begin{proof}
	since $J=\mathbb{R}^{+},$ the function $g$ from the proof of Lemma \ref{entropy_on_R} satisfies
	\begin{equation}\label{gproperty}
	g>0, \quad g^{\prime} \geq 0, \quad g^{\prime \prime} \leq 0 \quad \text { on } J	
	\end{equation}

	
	Now denote the derivatives of the given function $\psi$ by
\begin{equation}\label{varphiproperty}
	\psi(1)=0, \psi^{\prime}(1)=0, \psi^{\prime \prime}(1)=: \mu_{2}>0, \psi^{\prime \prime \prime}(1)=: \mu_{3} \leq 0
\end{equation}
	From \eqref{gproperty} we readily get the estimate
	$$
	\left.\begin{array}{ll}
	\sigma \mu_{2}^{-1}, & 0<\sigma<1 \\
	\mu_{2}^{-1}, & \sigma>1
	\end{array}\right\} \leq g(\sigma) \leq \gamma \sigma+\delta, \sigma>0
	$$
	with $\gamma:=-\mu_{3} \mu_{2}^{-2} \geq 0, \delta:=\left(\mu_{2}+\mu_{3}\right) \mu_{2}^{-2} \geq 0 .$ Integrating the corresponding
	estimate for $\psi^{\prime \prime}=\frac{1}{g}$
	$$
	(\gamma \sigma+\delta)^{-1} \leq \psi^{\prime \prime}(\sigma) \leq\left\{\begin{array}{ll}
	\mu_{2} / \sigma, & 0<\sigma<1 \\
	\mu_{2}, & \sigma>1
	\end{array}\right.
	$$
	we obtain with \eqref{varphiproperty} the upper bound for $\psi:$
\begin{equation}\label{varphiprimeprimeproperty}
	\psi(\sigma) \leq\left\{\begin{array}{cl}
\mu_{2}(\sigma \ln \sigma-\sigma+1), & 0<\sigma<1 \\
\frac{\mu_{2}}{2}(\sigma-1)^{2}, & \sigma>1
\end{array}\right\}
\end{equation}
	$$
	\leq \mu_{2}(\sigma-1)^{2}=: \varphi(\sigma), \quad \sigma>0
	$$
	To derive the lower bound of $\psi$ one integrates \eqref{varphiprimeprimeproperty} twice to show $\chi(\sigma) \leq$ $\psi(\sigma) .$ For $\gamma>0$ the function $\chi(\sigma)$ is given by \eqref{logarithmic-type-entropy-generator} with $\alpha=\frac{1}{\gamma}, \beta=\frac{\delta}{\gamma}$
	If $\gamma=0$ we set
	$$
	\chi(\sigma)=\frac{\mu_{2}}{2}(\sigma-1)^{2}.
	$$
\end{proof}

\begin{theorem}[Csiszár-Kullback inequality]
	Let $\psi:(0, \infty) \rightarrow \mathbb{R}$ be bounded below, convex on $(0, \infty)$ strictly convex at 1 with $\psi(1)=0 .$ Then there exists a function $\mathrm{W}_{\psi}: \mathbb{R} \rightarrow[0, \infty)$
	such that
	1. W $_{\psi}$ is increasing.
	2. $\mathbf{W}_{\psi}(0)=0$
	3. W $_{\psi}$ is continuous at 0.
	4. For all non-negative $u \in L^{1}(\Omega, \mathscr{S}, \mu)$ (where $(\Omega, \mathscr{S}, \mu)$ is a probability space $)$ with $\int_{\Omega} u d \mu=1$ the Csiszár-Kullback inequality holds:
	$$
	\|u-1\|_{L^{1}(d \mu)} \leqslant \mathbf{W}_{\psi}\left(e_{\psi}(u)\right)
	$$
	where $e_{\psi}(u):=\int_{\Omega} \psi(u) d \mu$ is the entropy of u (relative to the state 1) generated by the function $\psi$.
\end{theorem}

For the generating function $\psi(\sigma)=\sigma \ln \sigma-\sigma+1,$ e.g.,
$\mathrm{W}_{\psi}$ can be chosen such that
$$
\mathrm{W}_{\psi}(c) \leqslant \sqrt{2 c}
$$
that is 
\begin{equation}
\frac{1}{2}\left\|\rho_{1}-\rho_{2}\right\|_{L^{1}\left(\mathbb{R}^{n}\right)}^{2} \leq e\left(\rho_{1} \mid \rho_{2}\right)=\int_{\mathbb{R}^n}\ln\dfrac{\rho_1}{\rho_2}\rho_{1}\mathrm{d}x.
\end{equation}

notice that $u\ln u \geqslant u - 1 + \dfrac{1}{2}(u-1)^2, \forall u\in(0,1),$ 
\begin{equation}
\begin{split}
\left\|\rho_{1}-\rho_{2}\right\|_{L^{1}\left(\mathbb{R}^{n}\right)}^{2} &=
4 (\int_{\rho_{1}<\rho_{2}}\left|\rho_{1}-\rho_{2}\right| d x)^2\\
&= 4 (\int_{\rho_{1}<\rho_{2}}\left|\frac{\rho_{1}}{\rho_{2}}-1\right|\rho_{2} d x)^2\\
&\leqslant
4 \int_{\rho_{1}<\rho_{2}}\left|\frac{\rho_{1}}{\rho_{2}}-1\right|^2\rho_{2} d x\int_{\rho_{1}<\rho_{2}}\rho_{2}\mathrm{d}x\\
&\leqslant 4 \int_{\rho_{1}<\rho_{2}}\left|\frac{\rho_{1}}{\rho_{2}}-1\right|^2\rho_{2} d x\\
&\leqslant 2 \int_{\rho_{1}<\rho_{2}}(\dfrac{\rho_{1}}{\rho_{2}} \ln \dfrac{\rho_{1}}{\rho_{2}} - \dfrac{\rho_{1}}{\rho_{2}} + 1) \rho_{2} \mathrm{d}x\\
&\leqslant 2 \int_{\mathbb{R}^n} (\dfrac{\rho_{1}}{\rho_{2}} \ln \dfrac{\rho_{1}}{\rho_{2}} - \dfrac{\rho_{1}}{\rho_{2}} + 1) \rho_{2} \mathrm{d}x\\
&= 2 \int_{\mathbb{R}^n} \rho_{1} \ln \dfrac{\rho_{1}}{\rho_{2}} \mathrm{d}x\\
\end{split}
\end{equation}

\section{convex sobolev inequation}
The probably easiest equation for which the Bakry-Émery method leads to non-trivial results is the heat equation
\begin{equation}\label{heat-equation}
	\partial_{t} u=u_{x x}, \quad u(0 ; x)=\hat{u}(x)
\end{equation}

on the interval [0,1] with homogeneous Neumann boundary conditions.

\begin{theorem}
	 Assume that $\phi: \mathbb{R}_{\geq 0} \rightarrow \mathbb{R}_{\geq 0}$ is convex and s.t. $\left(\phi^{\prime \prime}\right)^{-1 / 2}$ is concave, and let $\psi$ be such that
	$$
	\psi^{\prime}(s)^{2}=\phi^{\prime \prime}(s)
	$$
	Then the following convex Sobolev inequality \eqref{convexsobolevinequation}
\begin{equation}\label{convexsobolevinequation}
	\int_{0}^{1} \phi(\hat{u}) d x-\phi\left(\int_{0}^{1} \hat{u} d x\right) \leq \frac{1}{2 \pi^{2}} \int_{0}^{1} \psi(\hat{u})_{x}^{2} d x
\end{equation}
	holds for all smooth, positive functions $\hat{u}$ on $[0,1] .$
\end{theorem}

\begin{proof}
	Let us start from the special case of \eqref{convexsobolevinequation} with $\phi(s)=\frac{1}{2} s^{2}$ and $\psi(s)=s$
	$$
	\int_{0}^{1} \hat{u}^{2} d x-\left(\int_{0}^{1} \hat{u} d x\right)^{2} \leq \frac{1}{\pi^{2}} \int_{0}^{1} \hat{u}_{x}^{2} d x
	$$
	This is the Poincaré inequality which we shall not prove again. Instead, we shall now generalize it to other convex functions $\phi: \mathbb{R}_{\geq 0} \rightarrow \mathbb{R}_{\geq 0}$
	
	Define the l.h.s of \eqref{heat-equation} as $H_{\phi}[u],$ and let $u$ be the unique solution to \eqref{heat-equation} . One finds
	$$
	D_{\phi}[u]:=-\frac{d}{d t} H_{\phi}[u]=-\int \phi^{\prime}(u) u_{t} d x=-\int \phi^{\prime}(u) u_{x x} d x=\int \phi^{\prime \prime}(u) u_{x}^{2} d x=\int \psi(u)_{x}^{2} d x
	$$
	as first time derivative and
	$$
	\begin{aligned}
		R_{\phi}[u] &:=-\frac{1}{2} \frac{d}{d t} D_{\phi}[u]=-\int \psi(u)_{x} \psi(u)_{x t} d x=\int \psi(u)_{x x} \psi^{\prime}(u) u_{x x} d x \\
		&=\int \psi^{\prime}(u)^{2} u_{x x}^{2} d x+\int \psi^{\prime \prime}(u) \psi^{\prime}(u) u_{x}^{2} u_{x x} d x
	\end{aligned}
	$$
	as second. The crucial step is to relate $R_{\phi}[u]$ to $D_{\phi}[u] .$ To this end, the expression
	$$
	\begin{aligned}
		0 &=\frac{1}{3} \int\left(\psi^{\prime}(u) \psi^{\prime \prime}(u) u_{x}^{3}\right)_{x} d x \\
		&=\int \psi^{\prime \prime}(u) \psi^{\prime}(u) u_{x}^{2} u_{x x} d x+\frac{1}{3} \int\left(\psi^{\prime \prime}(u)^{2}+\psi^{\prime}(u) \psi^{\prime \prime \prime}(u)\right) u_{x}^{4} d x
	\end{aligned}
	$$
	is added to $R_{\phi}[u],$ obviously without changing the value of the latter. Hence
	$$
	R_{\phi}[u]=\int \psi^{\prime}(u)^{2} u_{x x}^{2} d x+2 \int \psi^{\prime \prime}(u) \psi^{\prime}(u) u_{x x} u_{x}^{2} d x+\frac{1}{3} \int\left(\psi^{\prime \prime}(u)^{2}+\psi^{\prime}(u) \psi^{\prime \prime \prime}(u)\right) u_{x}^{4} d x
	$$
	On the other hand, one has
	$$
	0 \leq(\psi(u))_{x x}^{2}=\psi^{\prime}(u)^{2} u_{x x}^{2}+2 \psi^{\prime \prime}(u) \psi^{\prime}(u) u_{x x} u_{x}^{2}+\psi^{\prime \prime}(u)^{2} u_{x}^{4}
	$$
	In combination with Poincaré's inequality, one concludes
	\begin{equation}\label{entropy-production-prime-entropy-production-inequation}
	D_{\phi}[u] \leq \frac{1}{\pi^{2}} R_{\phi}[u]
	\end{equation}
	provided that
\begin{equation}\label{psi-prime-reverse-concavity}
		\frac{1}{3}\left(\left(\psi^{\prime \prime}\right)^{2}+\psi^{\prime} \psi^{\prime \prime \prime}\right) \geq\left(\psi^{\prime \prime}\right)^{2}
\end{equation}
	since $\psi^{\prime}=\left(\phi^{\prime \prime}\right)^{1 / 2}>0,$ it is easy to see that \eqref{psi-prime-reverse-concavity} is equivalent to the concavity of $\left(\psi^{\prime}\right)^{-1}=$ $\left(\phi^{\prime \prime}\right)^{-1 / 2}$
	To finish the argument, rewrite \eqref{entropy-production-prime-entropy-production-inequation} as
	$$
	-\frac{d}{d t} H_{\phi}[u(t)] \leq-\frac{1}{2 \pi^{2}} \frac{d}{d t} D_{\phi}[u(t)]
	$$
	and integrate both sides from $t=+\infty$ to $t=0 .$ This yields
\begin{equation}\label{entropy-entropy-production-inequation}
		H_{\phi}\left[u_{0}\right]-\lim _{t \rightarrow+\infty} H_{\phi}[u(t)] \leq \frac{1}{2 \pi^{2}}\left(D_{\phi}\left[u_{0}\right]-\lim _{t \rightarrow \infty} D_{\phi}[u(t)]\right)
\end{equation}
	By standard theory, the solution $u(t)$ to \eqref{heat-equation} converges to the homogeneous steady state $u_{\infty} \equiv$ $\int_{0}^{1} \hat{u}(x) d x$ in $C^{\infty},$ implying that $D_{\phi}[u(t)] \rightarrow 0$ and $H_{\phi}[u(t)] \rightarrow H_{\phi}\left[u_{\infty}\right]$ as $t \rightarrow \infty .$ Substituting
	these limits, \eqref{entropy-entropy-production-inequation} becomes \eqref{convexsobolevinequation} . $\square$

\end{proof}
	For example, $\phi(s)=s \log s$ with $\psi(s)=2 s^{1 / 2}$ is a possible choice, leading to a logarithmic Sobolev inequality,
\begin{equation}
	0 \leq \int_{0}^{1} \hat{u} \log \hat{u} d x-\left(\int_{0}^{1} \hat{u} d x\right) \log \left(\int_{0}^{1} \hat{u} d x\right) \leq \frac{2}{\pi^{2}} \int_{0}^{1} \sqrt{\hat{u}}_{x}^{2} d x\label{logarithmic-sobolev-inequation}
\end{equation}


Moreover, the pairs $\phi(s)=s^{\alpha} /(\alpha-1)$ and $\psi(s)=2 s^{\alpha / 2} / \sqrt{\alpha}$ are also allowed, when $1<\alpha<2$. These yield Beckner's interpolation inequalities,
\begin{equation}\label{beckner's-interpolation-inequation}
	0 \leq \frac{1}{\alpha-1}\left[\int_{0}^{1} \hat{u}^{\alpha} d x-\left(\int_{0}^{1} \hat{u} d x\right)^{\alpha}\right] \leq \frac{2}{\alpha \pi^{2}} \int_{0}^{1}\left(\sqrt{\hat{u}^{\alpha}}\right)_{x}^{2} d x
\end{equation}
Notice that from \eqref{beckner's-interpolation-inequation}, one obtains both the Poincare inequality for $\alpha \nearrow 2,$ as well as the logarithmic Sobolev inequality \eqref{logarithmic-sobolev-inequation} for $\alpha \searrow 1$ as limit cases.

\section{operator semigroup}

\begin{definition}
	
 We call a linear operator A in a Banach space $X$ a sectorial operator if it is a closed densely defined operator such that, for some $\phi $ in $(0, \pi / 2)$ and some $M \geq 1$ and real $a$, the
	Sector
	$$
	S_{a, \phi}=\{\lambda|\phi \leq| \arg (\lambda-a) \mid \leq \pi, \quad \lambda \neq a\}
	$$
	is in the resolvent set of $A$ and
	$$
	(\lambda-A)^{-1} \| \leq M /|\lambda-a| \quad \text { for all } \quad \lambda \in S_{a, \phi}
	$$
\end{definition}
\begin{definition}
	
	A \textbf{strongly continuous semigroup} on a Banach space $X$ is a map $T: \mathbb{R}_{+} \rightarrow L(X)$ such that
	\begin{itemize}
		\item [1.] $T(0)=I,$ (identity operator on $X$ )
		\item [2.] $\forall t, s \geq 0: T(t+s)=T(t) T(s)$
		\item [3.] $\forall x_{0} \in X:\left\|T(t) x_{0}-x_{0}\right\| \rightarrow 0,$ as $t \downarrow 0$
	\end{itemize}
	The first two axioms are algebraic, and state that $T$ is a representation of the semigroup ( $\left.\mathbb{R}_{+},+\right)$; the last is topological, and states that the map $T$ is continuous in the strong operator topology.

\end{definition}

The infinitesimal generator $A$ of a strongly continuous semigroup $T$ is defined by
$$
A x=\lim _{t \downarrow 0} \frac{1}{t}(T(t)-I) x
$$
whenever the limit exists. The domain of $A, D(A)$, is the set of $x \in X$ for which this limit does exist; $D(A)$ is a linear subspace and $A$ is linear on this domain.  The operator $A$ is closed, although not necessarily bounded, and the domain is dense in $X.$

The strongly continuous semigroup $T$ with generator $A$ is often denoted by the symbol $e^{A t}$. This notation is compatible with the notation for matrix exponential, and for functions of an operator defined via functional calculus (for example, via the spectral theorem).




\begin{proposition}
	if $T$ is a strongly continuous semigroup with generator $A$, we have 
	
	\begin{itemize}
		\item [I] $A$ is closed;
		\item [II] $D(A)$ is densely in $X$; 
		\item [III] if $x\in D(A)$, then $AT(t)x = T(t)Ax;$
		\item [IV] if $x\in D(A)$, then $A\int_0^tT(s)x\mathrm{d}s=\int_0^tAT(s)x\mathrm{d}s= T(t)x - x;$
	\end{itemize}
\end{proposition}


\begin{proof}
	\begin{itemize}
		\item [I] if $x_n\in D(A), y\in X$ such that $Ax_n\rightarrow y$ and $x_n\rightarrow x,$ we have
	$$|\frac{1}{t}(T(t)-I)x-y| \leqslant |\frac{1}{t}(T(t)-I)(x - x_n)| + |\frac{1}{t}(T(t)-I)x_n-\dfrac{1}{t}\int_0^tT(s)y\mathrm{d}s| + |\dfrac{1}{t}\int_0^tT(s)y\mathrm{d}s-y|$$
	and $$\frac{1}{t}(T(t)-I) x_n = \dfrac{1}{t}\int_0^tT(s)Ax_n\mathrm{d}s\rightarrow \dfrac{1}{t}\int_0^tT(s)y\mathrm{d}s$$
	
	\item [II] select $x\in X$, define \begin{equation*}
		x_\lambda = \dfrac{1}{\lambda}\int_0^\lambda T(t)x\mathrm{d}t,
	\end{equation*}
obviously, $x_\lambda\rightarrow x,$ as $\lambda\rightarrow0.$ noticing that,
\begin{equation}
	\begin{split}
		\lim\limits_{t\downarrow0}\dfrac{1}{t}(T(t)-I)x_\lambda &= \lim\limits_{t\downarrow0}\dfrac{1}{t\lambda}\int_0^\lambda [T(t+s)-T(s)]x\mathrm{d}s\\
		&= \lim\limits_{t\downarrow0}\dfrac{1}{t\lambda}(\int_\lambda^{\lambda + t}T(s)x\mathrm{d}s -
		\int_0^t T(s)x\mathrm{d}s)\\
		&= \lim\limits_{t\downarrow0}\dfrac{T(\lambda)- I}{t\lambda}\int_0^t T(s)x\mathrm{d}s\\
		&= \dfrac{T(\lambda)x- x}\lambda,\\
	\end{split}
\end{equation}
thus $Ax_\lambda = \dfrac{T(\lambda)x- x}\lambda,$ $x_\lambda \in D(A).$
\item [III] \begin{equation}
	AT(t)x = \lim\limits_{s\downarrow0}\dfrac{T(s)T(t)x - T(t)x}{s}= T(t)\lim\limits_{s\downarrow0}\dfrac{T(s)x - x}{s} = T(t)Ax.
\end{equation}
\item [IV] from II, we have $\lambda Ax_\lambda = T(\lambda)x - x,$
\end{itemize}
\end{proof}

\begin{theorem}
	If $A$ is a sectorial operator, then $-A$ is the infinitesimal generator of an analytic semigroup $\left\{e^{-t A}\right\}_{t \geq 0},$ where
	$$
	e^{-A t}=\frac{1}{2 \pi i} \int_{\Gamma}(\lambda+A)^{-1} e^{\lambda t} d \lambda
	$$
	where $\Gamma$ is a contour in $\rho(-\mathrm{A}) \quad$ with arg $\lambda \rightarrow \pm \theta$ as $|\lambda| \rightarrow \infty$ for
	some $\theta \quad$ in $\quad(\pi / 2, \pi)$
	Further e $^{-A t}$ can be continued analytically into a sector
	$\{t \neq 0:|\arg t|<\varepsilon\} \quad$ containing the positive real axis, and if
	$\operatorname{Re} \sigma(\mathrm{A})>\mathrm{a}, \quad$ i. e. $\quad$ if $\quad \operatorname{Re} \lambda>\mathrm{a} \quad$ when ever $\quad \lambda \in \sigma(\mathrm{A}),$ then for $t>0$
	$$
	\left\|e^{-A t}\right\| \leq C e^{-a t},\left\|A e^{-A t}\right\| \leq \frac{C}{t} e^{-a t}
	$$
	for some constant $C$.
	
	Finally $\frac{d}{d t} e^{-A t}=-A e^{-A t}$ for $t>0$
	\end{theorem}

\begin{proof}
\begin{equation}
	\|e^{-At}\| \leqslant \frac{1}{2\pi}\int_\Gamma\frac{M}{|\lambda - a|}e^{Re \lambda t}\mathrm{d}\lambda\\
\end{equation}
\end{proof}

Let us regard $\gamma-\Delta$ as a closed operator in a Banach space. So define the
$$
A_{p}=\gamma-\Delta, \quad D\left(A_{p}\right)=\left\{u \in W^{2, p}(\Omega) ; \frac{\partial u}{\partial n}=0 \quad \text { on } \partial \Omega\right\}
$$

The operator $A_{p}$ is sectorial in $L^{p}(\Omega)$ and $\sigma(A) \subset\left\{z \in C ; \mathscr{R}(z)>\gamma_{0}\right\}$ for a positive number $\gamma_{0},$ where $\sigma(A)$ is the spectrum of $A_{p} .$ Then for $\beta \geqq 0$ the fractional powers $A_{p}^{\beta}$ of $A_{p}$ are defined, and the domain $X_{p}^{\beta}=D\left(A_{p}^{\beta}\right)$ is a Banach space under the norm $\|u\|_{X_{p}^{\beta}}=\left\|A_{p}^{\beta} u\right\|_{p} . \quad$ since $A_{p}$ is sectorial in $L^{p}(\Omega)$ the operator $-A_{p}$ generates the analytic semigroup $\left\{T_{p}(t)\right\} .$ 

\begin{definition}
	suppose $A$ is a sectorial operator and $Re(\sigma(A))>0,$ then for any $\alpha>0$, define 
	\begin{equation}
		A^{-\alpha} = \frac{1}{\Gamma(\alpha)}\int_0^{\infty}t^{\alpha-1}e^{-At}\mathrm{d}t.
	\end{equation}
\end{definition}

Lemma ( [10] ). Suppose $\Omega \subset R^{n}$ a bounded domain with smooth boundary. Then for $0 \leqq \beta \leqq 1$, the following holds:
$$
\begin{array}{ll}
	X_{p}^{\beta} \subset W^{k, q}(\Omega) & \text { when } k-n / q<2 \beta-n / p, \quad q \geqq p \\
	X_{p}^{\beta} \subset C^{\nu}(\Omega) & \text { when } 0 \leqq v<2 \beta-n / p
\end{array}
$$
and the inclusion is continuous.

\begin{example}
estimate \(0=\Delta v - v + u\) subjected to homogeneous Neumann boundary conditions.
if \(u\in L_1\), then 
\[  
    \|v\|_q<c(q) \quad\text{for each }q\in[1,\frac{n}{n-2})
\]
and
\[
    \|\nabla v\|_p<c(p)\quad\text{for each }p\in[1,\frac{n}{n-1}).
\]
\end{example}
\begin{proof}
  Using Laplace transform of operator semigroup, 
  \[
    v = R(1)u=(I-\Delta)^{-1} u = \int_0^\infty e^{(\Delta-1)t}u\dd t
  \]
  we estimate (based on heat kernel estimates and Young's inequality for convolution) as done in \cite[Lemma 1.3]{Winkler2010}
  \begin{align*}
    \|v\|_q &\leq\int_0^\infty \|e^{(\Delta-1)t}\|_q\dd t\\
    &\leq c\|u\|_1\int_0^\infty (1+t^{-\frac{n}{2}(1-\frac1q)})e^{-t}\dd t. 
  \end{align*}
  and
  \begin{align*}
    \|\nabla v\|_p & \leq \int_0^\infty\|\nabla e^{(\Delta-1)t}u\|_p\dd t\\
    &\leq c\|u\|_1\int_0^\infty (1+t^{-\frac{1}{2} - \frac{n}{2}(1-\frac{1}{p})})e^{-t}\dd t    
  \end{align*}
\end{proof}

\section{\texorpdfstring{$W^{1,1}$}{} embedding coefficient}
let $\Omega(t) = \{x\in\mathrm{R}^n : |u(x)| > t\},$ 
\begin{equation}
	\chi_E(x)=
	\begin{cases}
		0,& x\not\in E,\\
		1,& x\in E,
	\end{cases}
\end{equation}
and $w_n$ is the volume of unit ball in $\mathrm{R}^n$, then
\begin{equation}
	\begin{split}
		\|u\|_{L^{\frac{n}{n-1}}(\mathrm{R}^n)} &= \left(\int_{\mathrm{R}^n}|u(x)|^\frac{n}{n-1}\right)^{\frac{n-1}{n}}\\
		&= \left(\int_{\mathrm{R}^n}\left(\int_{0}^{\infty}\chi_{\Omega(t)}\mathrm{d}t\right)^{\frac{n}{n-1}}\right)\\
		&\leqslant \int_0^\infty\left(\int_{\mathrm{R}^n}\chi_{\Omega(t)}^{\frac{n}{n-1}}\mathrm{d}x\right)^{\frac{n-1}{n}}\mathrm{d}t\\
		&= \int_0^{\infty}|\Omega(t)|^{\frac{n-1}{n}}\mathrm{d}t\\
		&\leqslant n^{-1}w_n^{-\frac{1}{n}}\int_0^{\infty}|\partial\Omega(t)|\mathrm{d}t\\
		&= n^{-1}w_n^{-\frac{1}{n}} \int_{\mathrm{R}^n}|\nabla u|.
	\end{split}
\end{equation}


\section{a isomorphism to Laplace}
\begin{theorem}
Suppose $\Omega$ is a bounded, connected open set with smooth boundary, then
\begin{equation}
D := \left\{\phi\in W^{2,2}(\Omega)\mid \frac{\partial \phi}{\partial \boldsymbol{n}}\mid_{\partial\Omega} = 0, \int_\Omega\phi=0\right\}
\cong V :=\left\{f\in L^2(\Omega)\mid \int_\Omega f = 0\right\}
\end{equation}
\end{theorem}
\begin{proof}
if $u\in H^1(\Omega)$ is a weak solution to the Neumann boundary problem 
\begin{equation*}
\left\{
\begin{array}{ll}
-\Delta u =f, & \text{ in } \Omega\\
\frac{\partial u}{\partial \boldsymbol{n}} = 0, & \text{ on } \partial\Omega
\end{array}\right.
\end{equation*}
is to say, 
\begin{equation*}
\int_\Omega\nabla u\cdot\nabla v\mathrm{d}x = \int_\Omega fv\mathrm{d}x,\text{ for all } v\in H^1(\Omega).
\end{equation*}
Define $-\Delta u\in D \mapsto f\in V$. It will be clear that $-\Delta$ is a isomorphic map between $D$ and $V$.
By Lax-Milgram Theorem, for arbitrary $f\in V$, there exists a $u\in H^1(\Omega)$ solving the Poisson equation. In fact, elliptic regularity implies $u\in H^2(\Omega)$, and hence 
$-\Delta$ is surjective; Obviously, the Laplace equation has only zero solution in $D$, and thus $-\Delta$ is injective. A direct norm inequality deduces $-\Delta$ is bounded, and consequently, $(-\Delta)^{-1}$ exists and is a continuous linear functional invoked by Banach inverse mapping theorem.
\end{proof}
\begin{example}
a direct consequence is $\|u\|_{H^2}\leq C\|\Delta u\|_{L^2}$ for all $u\in D$.
\end{example}

\section{an equivalent norm}

\begin{lemma}
	%if $\|f\|_{L^2(\partial\Omega)} + \|\nabla f\|_{L^2(\Omega)} < C$, then $f\in H^1(\Omega)$ is bounded. precisely, 
	there exists $C>0$ such that 
	\[
	\|f\|_{L^2(\Omega)} \leq C\|f\|_{L^2(\partial\Omega)} + C\|\nabla f\|_{L^2(\Omega)}.
	\]
\end{lemma}

\begin{proof}
	By absurdom, for each $n\in \mathbb{N}^+$, there exists $f_n$ such that
	\[
	\|f_n\|_{L^{2}(\Omega)} > n (\|f_n\|_{L^2(\partial\Omega)} + \|\nabla f_n\|_{L^2(\Omega)}).
	\]
	Writing
	\[
	\tilde{f}_n = \frac{f_n}{\|f_n\|_{L^2(\Omega)}},
	\]
	we have 
	\[
	\|\tilde f_n\|_{L^{2}(\Omega)} = 1,
	\quad \|\tilde f_n\|_{L^2(\partial\Omega)} + \|\nabla \tilde f_n\|_{L^2(\Omega)} \to 0.
	\]
	So $\tilde f_n\in H^1(\Omega)$ implies there exists $\tilde f\in L^2(\Omega)\cap L^2(\partial\Omega)$ such that extracting a subsequence if necessary,
	\[
	\tilde f_n\to \tilde f \quad\text{in } L^2(\Omega)\cap L^2(\partial\Omega),
	\]
	and thus 
	\[
	\|\tilde f\|_{L^2(\Omega)} = 1,
	\quad \|\tilde f\|_{L^2(\partial\Omega)} = 0,
	\quad \|\nabla\tilde f\|_{L^2(\Omega)} = 0.
	\]
	The last identity follows from the definition of weak derivative, that is,
	for each $g\in C^\infty_0(\Omega)$,
	\begin{align*}
		\int_\Omega \tilde f\partial_i g \leftarrow \int_\Omega \tilde f_n\partial_i g 
		= - \int_\Omega \partial_i \tilde f_n g \to 0. 
	\end{align*}
	We end up with $\tilde f\in H^1_0(\Omega)$ and $\|\tilde f\|_{L^2(\Omega)} = 0$ by Poincar\'e inequality,
	which is absurd.
\end{proof}


\section{global existence of a chemotaxis system with indirect signal production}
\begin{align}\label{sys: ks with isp}
  u_t &= \nabla\cdot(\nabla u - u\nabla w) \\
  \varepsilon v_t &= u - v \\
  w_t &= \Delta w - w + v
\end{align}

testing the first equation in \eqref{sys: ks with isp} by $pu^{p-1}$ ($p>1$), integrating by parts and substituting $\Delta w$ by the third equation, we get
\begin{align*}
    \frac{\dd}{\dd t}\int u^p &= p\int u^{p-1}\nabla\cdot(\nabla u - u\nabla w)\\
    &= -p(p-1)\int u^{p-2}|\nabla u|^2 + p(p-1)\int u^{p-1}\nabla w\cdot \nabla u\\
    &= -\frac{4(p-1)}{p}\int |\nabla u^{\frac{p}2}|^2 - (p-1)\int u^p\Delta w\\
    &= -\frac{4(p-1)}{p}\int |\nabla u^{\frac{p}2}|^2
    - (p-1)\int u^pw - (p-1)\int u^pw_t + (p-1)\int u^pv\\
    &\leq \frac{4(p-1)}{p}\int |\nabla u^{\frac{p}2}|^2
     + \delta\int u^{p+1} + c(\delta,p)\int w_t^{p+1} + c(\delta,p)\int v^{p+1},
\end{align*}
with some $\delta>0$.
Using Gagliardo-Nirenberg interpolation inequality
\[
    \|u\|_{p+1}^{p+1} = \|u^{\frac{p}2}\|_{\frac{2(p+1)}{p}}^{\frac{2(p+1)}{p}}
    \leq c_{gn}(\|\nabla u^{\frac{p}{2}}\|^2_2\cdot\|u^{\frac{p}{2}}|_{\frac2p}^{\frac2p}
     + \|u^{\frac{p}{2}}\|_{\frac2p}^{\frac{2(p+1)}{p}})
     \leq c_m(\|\nabla u^{\frac{p}2}\|_2^2 + 1),
\]
and putting $\delta c_m < \frac{p-1}{p}$, testing the second equation by $(p+1)v^p$ and integrating by parts,
\begin{equation*}
  \varepsilon\frac{\dd}{\dd t}\int v^{p+1} = (p+1)\int v^p(u-v) \leq \delta\int u^{p+1} + c(\delta, p)\int v^{p+1}
\end{equation*}
we obtain 
\begin{align*}
  \frac{\dd}{\dd t}\int u^p + \varepsilon v^{p+1} &\leq 
  -\frac{2(p-1)}{p}\int |\nabla u^{\frac{p}2}|^2
  + c(p)\int |w_t|^{p+1} + c(p)\int v^{p+1}.
\end{align*}
Denote $\mathcal{F}(t) = \int u^p + \varepsilon v^{p+1}$, then 
\[
    \mathcal{F}'(t) \leq c\mathcal{F} + \|w_t\|_{p+1}^{p+1}
\]
and thus using Gronwall's lemma and parabolic $L_p$ estimates (or Amann's Sobolev maximal regularity theory[Amann 1995, Theorem 4.10.7 and Remark 4.10.9), this yields 
\begin{align*}
  \mathcal{F} &\leq \mathcal{F}(0)e^{ct} + c\int_0^te^{c(t-s)}\|w_t\|_{p+1}^{p+1}\dd s\\
  &\leq \mathcal{F}(0)e^{ct} + c\int_0^t\|w_t\|_{p+1}^{p+1}\dd s\\
  &\leq F(0)e^{ct} + ce^{ct}(\|w_0\|_\star^{p+1} + \int_0^t\|v\|_{p+1}^{p+1}\dd s)\\
  &\leq ce^{ct}\int_0^t\mathcal{F}\dd s + F(0)e^{ct} + ce^{ct}\|w_0\|_\star^{p+1},
\end{align*}
and 
\[
    \int_0^t\mathcal{F}\dd s\leq c(t),
\]
and consequently, \(\mathcal{F}(t)\leq c(t)\).
Here,
\begin{align*}
  \star &= (L^{p+1},D(\Delta-1))_{1-\frac{1}{p+1},p+1}\\
  &\cong (L^{p+1}, W^2_{p+1,\mathcal{B}})_{1-\frac{1}{p+1},p+1}\\
  &\cong B_{p,p,\mathcal{B}}^{\frac{2p}{p+1}}\\
  &\cong\begin{cases}
          W^{\frac{2p}{p+1}}_{p+1,\mathcal{B}}, & \mbox{if } p>1 \\
          H^1_{2,\mathcal{B}}\cong W^1_{2,\mathcal{B}}, & p=1.
        \end{cases}
\end{align*}

\textbf{a Lyapunov functional}

testing the first equation in \eqref{sys: ks with isp} by $\ln u -w$ and integrating by parts, we get
\[
    \frac{\dd}{\dd t}\int u(\ln u - w)
    = \int u_t(\ln u - w) + \int u_t - \int uw_t
    = -\int u|\nabla (\ln u - w)|^2 - \int uw_t.
\]
substituting $u$ and $v_t$ by the last two equations, respectively, we calculate by integration by parts
\begin{align*}
    \int uw_t &= \int (\varepsilon v_t + u)w_t\\
    &= \int \varepsilon(w_{tt} - \Delta w_t + w_t)w_t 
    + \int (w_t - \Delta w + w)w_t\\
    &= \frac{\varepsilon}{2}\frac{\dd}{\dd t}\int w_t^2 
    + \varepsilon\int |\nabla w_t|^2
    + (1+\varepsilon) \int w_t^2
    + \frac{1}{2}\frac{\dd}{\dd t}\int w^2
    + \varepsilon\int |\nabla w_t|^2.
\end{align*}
finally, we have 
\[
    \mathcal{E}' = -\mathcal{D},
\]
where
\[
    \mathcal{E} = \int u\ln u - \int uw + \frac{1}{2}\int |\nabla w|^2 
    + \frac12\int w^2 + \frac{\varepsilon}{2}\int w_t^2
\]
and 
\[
    \mathcal{D} = \int u|\nabla(\ln u - w)|^2 + (1 + \varepsilon)\int w_t^2 
    + \varepsilon\int |\nabla w_t|^2.
\]

Using Morse-Trudinger inequality, one can verify that if $M=\int u_0\in(0,4\pi)$, then there exists $c(M)>0$ such that \(\mathcal{F}\geq -c(M)\) and
\[
    \sup_{t>0}(\|u\ln u\|_1 + \|w\|_{W^1_2} + \|w_t\|_2) + \int_0^\infty\|w_t\|_{W^1_2}^2\dd t < \infty.
\]

\textbf{Uniformly boundedness under the subcritical condition.}

similarly, we test the first equation in \eqref{sys: ks with isp} by $pu^{p-1}$, integrate by parts and substitute $\Delta w$ by the third equation,
\begin{align*}
  \frac{\dd}{\dd t}\int u^p &= -p(p-1)\int u^{p-2}|\nabla u|^2 
  + p(p-1)\int u^{p-1}\nabla u\cdot\nabla w\\
  &= -\frac{4(p-1)}{p}\int |\nabla u^{\frac{p}2}|^2 - (p-1)\int u^p\Delta w\\
  &= -\frac{4(p-1)}{p}\int |\nabla u^{\frac{p}2}|^2 
  - (p-1)\int u^p w - (p-1)\int u^pw_t + (p-1)\int u^p v.
\end{align*} 
Using H\"older inequality, Sobolev imbedding inequality and Young's inequality, we estimate
\begin{align*}
    \int u^pw_t &\leq\left(\int u^{\frac{4p}3}\right)^{\frac34}\|w_t\|_4\\
    &\leq c_S\left(\int u^{\frac{4p}3}\right)^{\frac34}\|w_t\|_{W^1_2}^{\frac12}\|w_t\|_2^{\frac12}\\
    &\leq \int u^{\frac{4p}{3}} + c\|w_t\|_{W^1_2}^2.
\end{align*}
Testing the second equation by $(p+1)v^p$, we have
\[
    \varepsilon\frac{\dd}{\dd t}\int v^{p+1} = (p+1)\int uv^p - (p+1)\int v^{p+1}.
\]
we gather above estimates and obtain 
\begin{align*}
    \mathcal{F}' + \mathcal{F} &\leq
    - \frac{4(p-1)}{p}\int\|\nabla u^{\frac{p}{2}}\|^2
    + \int u^{\frac{4p}3} + c\int u^{p+1} + c\|w_t\|_{W^1_2}^2 + c.
\end{align*} 
Noting a logarithm-type variant of Gagliardo-Nirenberg interpolation inequality,
\[
    \int u^{p+1} = \|u^{\frac{p}{2}}\|_{\frac{2(p+1)}{p}}^{\frac{2(p+1)}{2}}
    \leq c_{GN}\|\nabla u^{\frac{p}{2}}\|_2^2\cdot
    \|(u\ln u)^{\frac{p}{2}}\|_{\frac{2}{p}}^{\frac2p}
    + c_{GN}\|u^{\frac{p}{2}}\|_{\frac2p}^{\frac{2(p+1)}{p}},
\] 
putting $p=3$ and use the Gronwall's inequality, we have
\begin{align*}
    \int u^3 + \varepsilon\int v^4 &\leq
    e^{-t}(\int u^3 + \varepsilon v^4)(0)
    + c(\int_0^te^{s-t}\|w_t\|_{W^1_2}^2\dd s + 1 - e^{-t})\\
    &\leq (\int u^3 + \varepsilon v^4)(0)
    + c(\int_0^t\|w_t\|_{W^1_2}^2\dd s + 1).
\end{align*}

\section{\texorpdfstring{$L^2$}{L2} theory}
\label{sec: L2 theory}

consider \cite{Winkler2023}
\begin{equation}
	v_t = \Delta v - v + f,
\end{equation}
with boundary conditions
\begin{equation*}
	\frac{\partial v}{\partial\nu} = 0,
\end{equation*}
where 
\begin{equation*}
	f\in L^\infty((0,T); L^1(\Omega)),
\end{equation*}
using \[
\nabla v \cdot \nabla \Delta v=\Delta\left(\frac{1}{2}|\nabla v|^2\right)-\left|D^2 v\right|^2,
\]
we get
\begin{align*}
	\frac{\dd}{\dd t}\int|\nabla v|^p 
	&= p\int |\nabla v|^{p-2}\nabla v\cdot \nabla v_t 
	= p\int |\nabla v|^{p-2} (\nabla \Delta v - \nabla v + \nabla f) \\
	&= \frac12\int |\nabla v|^{p-2}\Delta (|\nabla v|^2) - p\int |\nabla v|^{p-2}|D^2v|^2 
		-p\int |\nabla v|^p + p\int |\nabla v|^{p-2}\nabla v \nabla f,
\end{align*}
by additional assumption of convex domain 
or a pointwise estimate of boundary trace alongwith trace imbedding lemma and Sobolev imbedding inequality,
we can control the first term of right hand, and estimate
\begin{align*}
	p\int |\nabla v|^{p-2}\nabla v \cdot \nabla f 
	&= - p\int \nabla\cdot(|\nabla v|^{p-2}\nabla v) f\\
	&= -p(p-2)\int |\nabla v|^{p-4}\nabla v\cdot (D^2v\nabla v) f 
		- p \int |\nabla v|^{p-2}\Delta v f\\
	&\leq c(p,n)\int |\nabla v|^{p-2} |D^2v| |f|\\
	&\leq c(p,n)\left(\int |\nabla v|^{p-2}|D^2v|^2\right)^{\frac12}
	\left(\int |\nabla v|^{p-2}f^2\right)^{\frac12}\\
	&= c(p,n) \|\nabla w\|_{L^2} \left(\int |\nabla v|^{p-2}f^2\right)^{\frac12},
\end{align*}
here, 
\[
	w = |\nabla v|^{\frac{p}{2}},	
\]
let $q\in(2,p)$ to be specified below, we estimate
\begin{align*}
	\left(\int |\nabla v|^{p-2}f^2\right)^{\frac12}
	&\leq \left(\int |\nabla v|^{\frac{(p-2)q}{q-2}}\right)^{\frac{q-2}{2q}}
	\left(\int f^q\right)^{\frac{1}{q}}\\
	&= \|w\|_{L^{\frac{2(p-2)q}{p(q-2)}}}^{\frac{p-2}{p}}
		\|f\|_{L^q},
\end{align*}
using G-N inequality
\[
	\|u\|_{L^{\frac{2(p-2)q}{p(q-2)}}}
	\leq C_g \|u\|_{W^{1,2}}^\theta\|u\|_{L^2}^{1-\theta},\quad u\in W^{1,2}, 
\]
with
\[
	\frac{p(q-2)}{2(p-2)q} = \theta\left(\frac12-\frac1n\right) + \frac{1-\theta}{2}.
\]
i.e.,
\[
	\theta = \frac{n(p-q)}{(p-2)q},
\]
we get
\begin{equation*}
	p\int |\nabla v|^{p-2}\nabla v \cdot \nabla f 
	\leq c(p,n) \|w\|_{W^{1,2}}^{1+\frac{n(p-q)}{pq}}
		\|w\|_{L^2}^{\frac{p-2}{p} - \frac{n(p-q)}{pq}}
		\|f\|_{L^q}.
\end{equation*}
Fixing 
\[
	q = \frac{(n+2)p}{n+p},
\]
which is such that
\[
	\alpha := \frac{2}{1-\frac{n(p-q)}{pq}} = q,\quad \theta = \frac{n}{n+2},
\]
and 
\begin{equation}\label{eq: gn inequality Lpw12L2}
	\|u\|_{L^{\frac{2(n+2)}{n}}} \leq C_g \|u\|_{W^{1,2}}^{\frac{n}{n+2}}\|u\|_{L^2}^{\frac{2}{n+2}},
\end{equation}
using Young inequality, we get
\begin{align*}
	p\int |\nabla v|^{p-2}\nabla v \cdot \nabla f 
	&\leq \varepsilon \|w\|_{W^{1,2}}^2
		\|w\|_{L^2}^{\left(\frac{p-2}{p} - \frac{n(p-q)}{pq}\right)\alpha}
		\|f\|_{L^q}^\alpha\\
	&= \varepsilon \|w\|_{W^{1,2}}^2
	+ c(\varepsilon) \|w\|_{L^2}^{\frac{2(p-2)}{n+p}}
	\|f\|_{L^q}^q,
\end{align*}
and therefore
\begin{equation*}
	\mathcal{F}' + \frac1C F \leq C \mathcal{F}^{\frac{p-2}{n+p}} \|f\|_q^q, 
\end{equation*} 
where
\[
	\mathcal{F} := \int w^2.
\]
So we obtain by a direct application of ODE comparable method,
\[
	\mathcal{F}^{\frac{n+2}{n+p}} \leq C \int_0^T\|f\|_q^q,
\]
and consequently,
\begin{align*}
	\sup_{(0,T)}\int |\nabla v|^p + \int_0^T\int |\nabla v|^{p-2}|D^2 v|^2 
	\leq C \left(\int_0^T\int |f|^{\frac{(n+2)p}{n+p}}\right)^{\frac{n+p}{n+2}}
\end{align*}
and by \eqref{eq: gn inequality Lpw12L2},
\begin{equation*}
	\left(\int_0^T\int |\nabla v|^{\frac{(n+2)p}{n}}\right)^{\frac{n}{n+2}} 
	\leq C \left(\int_0^T\int |f|^{\frac{(n+2)p}{n+p}}\right)^{\frac{n+p}{n+2}}. 
\end{equation*}

\section{quadratic logistic dampening prevents chemotactic collapse}

consider 
\begin{equation}
	\begin{cases}
		u_t = \Delta u - \nabla\cdot(u\nabla v) + \mu u (1-u), \\
		v_t = \Delta v - v + u,\\
		\partial_\nu u = \partial_\nu v = 0,
	\end{cases}
\end{equation}
in a bounded domain $\Omega\subset\mathbb{R}^2$ with smooth boundary.
It is obvious that
\[
	\int_t^{t+\tau}\int u^2 < C,
\]
for any $t\in(0, T-\tau)$ with $\tau = \min\{1, T/2\}$.
By conclusions in Section~\ref{sec: L2 theory}, 
we have 
\begin{align*}
	\sup_{(0,T)}\|\nabla v\|_{L^2} 
	+ \sup_{(0,T-\tau)}\int_t^{t+\tau}\int |D^2v|^2 
	+ \sup_{(0, T-\tau)}\int_t^{t+\tau}\int |\nabla v|^4 < C.
\end{align*}


\subsection{method 1: direct \texorpdfstring{$L_p$}{Lp} estimates}
calculate
\begin{align*}
	\frac{\dd}{\dd t}\int u^p 
	&= p\int u^{p-1}\nabla\cdot(\nabla u - u\nabla v) + \mu p \int u^p(1-u)\\
	&= - p(p-1)\int u^{p-2}|\nabla u|^2 
		+ p(p-1) \int u^{p-1}\nabla u\cdot\nabla v 
		+ \mu p\int u^p(1-u)
\end{align*}
estimate
\begin{align*}
	p(p-1) \int u^{p-1}\nabla u\cdot\nabla v
	&= (p-1)\int \nabla u^p\cdot\nabla v
	 = -(p-1)\int u^p\Delta v\\
	&\leq (p-1)\left(\int u^{2p}\right)^{\frac12}\left(\int |\Delta v|^2\right)^\frac{1}{2}
	 = (p-1)\|\Delta v\|_2\cdot\|u^{\frac{p}{2}}\|_4^2\\
	&\leq c(p) \|\Delta v\|_2\cdot \|u^{\frac{p}{2}}\|_{W^{1,2}}\cdot\|u^{\frac{p}{2}}\|_2,
\end{align*}
and
\begin{align*}
	\|u^{\frac{p}{2}}\|_2 
	\leq C\|u^{\frac{p}{2}}\|_{W^{1,2}}^{\frac{p-1}{p}}\|u^{\frac{p}{2}}\|_{\frac2p}^{\frac1p},
\end{align*}
which implies 
\begin{align*}
	\|u^{\frac{p}{2}}\|_{W^{1,2}}^2 \geq c(m,p) \|u^{\frac{p}{2}}\|_2^{\frac{2p}{p-1}}.
\end{align*}
let 
\[
	\mathcal{F}(u) := \int u^p + c,
\]
we have
\begin{align*}
	\mathcal{F}' + \frac1C\mathcal{F}^\frac{p}{p-1} 
	\leq \mathcal{F} \int |\Delta v|^2 + C,
\end{align*}
let
\[
	\mathcal{G} := \ln\mathcal{F},
\]
we have 
\begin{equation*}
	\mathcal{G}' + \frac{\mathcal{G}}{C} \leq \int|\Delta v|^2 + C,
\end{equation*}
which implies the uniform-in-time boundedness of $\|u\|_p$.

\subsection{method 2: \texorpdfstring{$L\log L$}{LlogL}-type estimate}
Let 
\[
	\phi(\xi) := \int_0^\xi \ln^\alpha(1+\eta) \dd\eta,\quad\alpha \in (0,2],
\]
then
\begin{align*}
	\frac{\dd}{\dd t}\int\phi(u)
	&= \int \nabla\cdot(\nabla u - u\nabla v) \ln^\alpha(1+u) + \mu\int u(1-u)\ln^\alpha(1+u)\\
	&= - \int \frac{\alpha|\nabla u|^2\ln^{\alpha-1}(1+u)}{1+u}
		+ \int \frac{\alpha u}{1+u} \ln^{\alpha-1}(1+u)\nabla u\cdot\nabla v  + \mu\int u(1-u)\ln^\alpha(1+u)\\
	&\leq - \alpha \int \frac{|\nabla u|^2\ln^{\alpha-1}(1+u)}{1+u}
		+ \alpha\left(\int \frac{|\nabla u|^2\ln^{\alpha-1}}{1+u}\right)^{\frac12}
			\left(\int \frac{|\nabla v|^2u^2\ln^{\alpha-1}(1+u)}{1+u}\right)^{\frac12}\\
	&\quad + \mu\int u(1-u)\ln^\alpha(1+u)\\
	&\leq \frac{\mu}{2}\int u^2\ln^{2\alpha-2}(1+u) + c(\alpha, \mu)\int|\nabla v|^4 
		+ \mu\int u(1-u)\ln^\alpha(1+u)\\
	&\leq c(\alpha, \mu)\int|\nabla v|^4 - \int\phi(u) + c(\alpha, \mu),
\end{align*}
which implies
\begin{align*}
	\int u\ln^\alpha(1+u) \leq C\int \phi(u) + C \leq C.
\end{align*}
combining with a logarithmic variant of G-N inequality, one can get 
\[
	\|u\|_p + \|\nabla v\|_{2p} < C(p), \quad p>1,
\]
which implies uniform-in-time boundedness of $u$ 
by well-established parabolic regularity theory in the context of chemotaxis models.

or

denote 
\[
	\psi(\xi) := \int_0^\xi \frac{\alpha \eta}{1+\eta}\ln^{\alpha-1}(1+\eta),
\]
then 
\begin{align*}
	\int \frac{\alpha u}{1+u} \ln^{\alpha-1}(1+u)\nabla u\cdot\nabla v
	&= - \int \psi(u) \Delta v\\
	&\leq \varepsilon\int \psi^2(u) + c(\varepsilon)\int |\Delta v|^2\\
	&\leq \varepsilon \int u^2\ln^{2\alpha -2}(1+u) + c(\varepsilon, n) \int |D^2 v|^2,
\end{align*}


\subsection{is the title of this section right when \texorpdfstring{$n\geq3$}{n>=3}?}

\begin{align*}
	\frac2n\frac{\dd}{\dd t}\int u^{\frac n2} 
	&= \int u^{\frac n2-1}\nabla\cdot(\nabla u - u\nabla v) + \mu\int u^{\frac n2} - \mu\int u^{\frac n2+1}\\
	&= -\frac{n-2}{2}\int u^{\frac n2-2}|\nabla u|^2 
		+ \frac{n-2}{2}\int u^{\frac n2-1}\nabla u\cdot\nabla v + \mu\int u^{\frac n2} - \mu\int u^{\frac n2+1}
\end{align*}
\begin{align*}
	\frac{n-2}{2}\int u^{\frac n2-1}\nabla u\cdot\nabla v
	&\leq c(n) \left(\int u^{\frac n2-2}|\nabla u|^2\right)^{\frac12}
		\left(\int u^{\frac n2}|\nabla v|^2\right)^{\frac12}\\
	&\leq c(n) \left(\int u^{\frac n2-2}|\nabla u|^2\right)^{\frac12}
		\left(\int u^{\frac{n+2}{2}}\right)^{\frac{n}{2(n+2)}}
		\left(\int |\nabla v|^{n+2}\right)^{\frac1{n+2}}\\
	&\leq c\|\nabla u^{\frac n4}\|_2
		\cdot \|u^{\frac{n+2}{2}}\|_1^{\frac{n}{2(n+2)}}
		\cdot \||\nabla v|^{\frac n2}\|_2^{\frac{4}{(n+2)n}}
		\cdot \||\nabla v|^{\frac n2}\|_{W^{1,2}}^{\frac{2}{n+2}},
\end{align*}

\begin{align*}
	\frac{\dd}{\dd t}\int |\nabla v|^n 
	= \frac{n}{2} \int |\nabla v|^{n-2}\Delta |\nabla v|^2 - n\int |\nabla v|^{n-2}|D^2v|^2
		- n\int |\nabla v|^n 
		+ n\int |\nabla v|^{n-2}\nabla v\cdot\nabla u,
\end{align*}
\begin{align*}
	\frac{\dd}{\dd t}\int |\nabla v|^n 
		+ n\int |\nabla v|^n
	\leq - n\int |\nabla v|^{n-2}|D^2v|^2
		+ n\int |\nabla v|^{n-2}\nabla v\cdot\nabla u,
\end{align*}
let 
\[
	w=|\nabla v|^{\frac n2},
\]
\begin{align*}
	n\int |\nabla v|^{n-2}\nabla v\cdot\nabla u
	&= -n\int \nabla\cdot(|\nabla v|^{n-2}\nabla v)u \\
	&= -n\int((n-2)|\nabla v|^{n-4}\nabla v \cdot (D^2v\cdot\nabla v) + |\nabla v|^{n-2}\Delta v) u\\
	&\leq c(n)\int |\nabla v|^{n-2}|D^2v|u\\
	&\leq \|\nabla|\nabla v|^{\frac{n}{2}}\|_2\left(\int |\nabla v|^{n-2}u^2\right)^{\frac12}\\
	&\leq c(n)\|\nabla w\|_2\left(\int w^{\frac{2(n-2)}{n}}u^2\right)^{\frac12}\\
	&\leq c(n)\|\nabla w\|_2 \left(\int w^{\frac{2(n+2)}{n}}\right)^{\frac{n-2}{2(n+2)}}
		\left(\int u^{\frac{n+2}{2}}\right)^{\frac{2}{n+2}}\\
	&= c(n)\|\nabla w\|_2 
		\cdot \|w\|_{\frac{2(n+2)}{n}}^{\frac{n-2}{n}}
		\cdot \|u\|_{\frac{n+2}{2}}\\
	&\leq c(n)\|w\|_{W^{1,2}}^{\frac{2n}{n+2}}
		\cdot \|w\|_2^{\frac{2(n-2)}{n(n+2)}}
		\cdot \|u\|_{\frac{n+2}{2}}
\end{align*}
\subsection{一个问题}
\begin{align*}
	\|u\|_{n/2} < C \Rightarrow \|u\|_\infty < C ?
\end{align*}

let 
\[
	\phi(\xi):= \int_0^\xi \eta^{\frac{n}{2}-1}\ln^{\alpha}(e+\eta), \quad\alpha>1,
\]
then
\[
	\frac{\xi^{\frac{n}{2}}\ln^{\alpha}(e+\xi)}{C} - C 
	\leq \phi(\xi) 
	\leq \xi^{\frac{n}{2}}\ln^{\alpha}(e+\xi),
\]
calculate
\begin{align*}
	\frac{\dd}{\dd t}\int \phi(u) 
	&= \int u^{\frac{n}{2} - 1}\ln^{\alpha}(e+u) \nabla\cdot(\nabla u - u\nabla v)
		+ \mu\int u^{\frac{n}{2}}(1-u)\ln^{\alpha}(e+u)\\
	&= -\frac{n-2}{2}\int |\nabla u|^2u^{\frac{n}{2}-2}\ln^\alpha(e+u)
		- \alpha\int\frac{u^{\frac{n}{2}-1}\ln^{\alpha-1}(e+u)}{e+u}|\nabla u|^2\\
		&\quad + \frac{n-2}{2}\int u^{\frac{n}{2}-1}\ln^\alpha(e+u)\nabla u\cdot\nabla v 
			+ \alpha\int \frac{u^{\frac{n}{2}}\ln^{\alpha-1}(e+u)}{e+u} \nabla u \cdot \nabla v\\
		&\quad + \mu\int u^{\frac{n}{2}}(1-u)\ln^{\alpha}(e+u),
\end{align*}
Noting
\begin{align*}
	\|u\|_{n/2} < C 
	&\Rightarrow \sup_{(0,T-\tau)}\int_t^{t+\tau} \int u^{\frac{n+2}{2}} < C \\
	&\Rightarrow \|\nabla v\|_n 
		+ \sup_{(0,T-\tau)}\int_t^{t+\tau}\int |\nabla v|^{n+2}
		+ \sup_{(0,T-\tau)}\int_t^{t+\tau}\int  |\nabla v|^{n-2}|D^2v|^2 < C,
\end{align*}

\subsection{疑惑哪里错了}
\begin{equation}
	\begin{cases}
		u_t = \Delta u - \nabla\cdot(u\nabla v) + \mu u - \mu u^\alpha,\\
		0 = \Delta v - \overline u + u,
	\end{cases}
\end{equation}
let 
\[
	\phi(t) = \int_0^R ur^{2n-1}\dd r,
\]
and
\[
	m(t) = \int_0^R ur^{n-1}\dd r,
\]
we calculate
\begin{align*}
	\phi'(t) &=
	\int_0^R(u_rr^{n-1})_rr^n - (uv_rr^{n-1})_rr^n\dd r 
		+ \mu\int_0^R u r^{2n-1}\dd r - \mu \int_0^Ru^\alpha r^{2n-1}\dd r,
\end{align*}
estimate
\begin{align*}
	\int_0^R(u_rr^{n-1})_rr^n 
		&= - n\int_0^Ru_rr^{2n-2}\dd r\\
		&= -nu(R) + 2n(n-1)\int_0^Rur^{2n-3}\dd r\\
		&\leq 2n(n-1)\left(\int_0^Rur^{n-1}\dd r\right)^{\frac{2}{n}}
			\cdot \left(\int_0^Rur^{2n-1}\dd r\right)^{\frac{n-2}{n}}.
\end{align*}
Using 
\[
	(v_rr^{n-1})_r -\overline{u}r^{n-1} + ur^{n-1} = 0,
\]
we get
\begin{align*}
	- \int_0^R (uv_rr^{n-1})_rr^n\dd r 
		&= n\int_0^Rur^{n-1}\cdot v_rr^{n-1}\dd r \\
		&= n\int_0^R ur^{n-1} \int_0^r(\overline{u}\eta^{n-1} - u\eta^{n-1})\dd\eta\dd r\\
		&= - \frac{n}2\left(\int_0^Rur^{n-1}\dd r\right)^2 + \overline{u}\int_0^R ur^{2n-1}\dd r, 
\end{align*}
and therefore
\begin{align*}
	\phi'(t) \leq \Phi(t, \phi) := 2n(n-1)m^{\frac2n}\cdot \phi^{\frac{n-2}{n}} - \frac{nm^2}2 + (\overline{u} + \mu)\phi,
\end{align*}

while
\[
	m' = \mu m - \mu m^\alpha,
\]
we have 
\[
	m\in\left[\min\{m(0),1\}, \max\{m(0),1\}\right],
\]
and thus for any $m(0)>0$, there exists $\varepsilon>0$ 
such that for any $\xi\in(0,\varepsilon)$,
$\Phi(t,\xi) < 0$,
that is for any initial data satisfying $\phi(0) < \varepsilon$, 
there exists $t_0\in(0,\infty)$ such that 
\[
	\phi(t_0) = 0,
\]
which implies such solution is not globally well-posed.

\section{local existence}
\begin{align*}
	u_t &= \Delta u - \nabla\cdot(u\nabla v),\\
	v_t &= \Delta v - v + u,
\end{align*}

let 
\[
	X := C^0(\overline\Omega\times[0,T])\times C^0([0,T]; W^{1,p}(\Omega)), \quad p>n,
\]
define
\begin{align*}
	\Psi(u,v) :=
		\left(e^{\Delta t}u_0 + \int_0^te^{\Delta(s-t)}\nabla\cdot(u\nabla v)\dd s, 
		e^{(\Delta-1) t}v_0 + \int_0^te^{(\Delta-1)(s-t)}u\dd s\right),
		\quad (u,v)\in X,
\end{align*}
with 
\[
	(u_0, v_0) \in C^0(\overline{\Omega})\times W^{1,p}.
\]
Here, $\Omega$ is a bounded domain with smooth boundary $\partial\Omega$ in $\mathbb{R}^n$, and $e^{\Delta t}$ denotes the semigroup generated by $-\Delta$ in $L^p$ with domain 
\[
	W^{2,p}_N := \{f\in W^{2,p}: \partial_\nu f = 0\text{ on }\partial\Omega\}.
\]
we estimate 
\begin{align*}
	\|\Psi_1\|_{\infty} &= \left\| e^{\Delta t}u_0 + \int_0^te^{\Delta(s-t)}\nabla\cdot(u\nabla v)\dd s\right\|_\infty\\
	&\leq \|e^{\Delta t}u_0\|_\infty 
		+ \left\|  \int_0^te^{\Delta(s-t)}\nabla\cdot(u\nabla v)\dd s\right\|_\infty\\
	&\leq \|u_0\|_\infty 
		+ \int_0^t \| e^{\Delta(s-t)}\nabla\cdot(u\nabla v)\dd s\|_\infty\dd s\\
	&\leq \|u_0\|_\infty 
		+ C\int_0^t \left(1+(t-s)^{-\frac{1}{2}-\frac{n}{2p}}\|u\nabla v\|_p\right)\dd s\\
	&\leq \|u_0\|_\infty 
		+ C\|u\|_{X_1} \|v\|_{X_2} \int_0^t\left(1+s^{-\frac{1}{2} - \frac{n}{2p}}\right)\dd s,
\end{align*}
and 
\begin{align*}
	\|\Psi_2\|_{W^{1,p}} &= \left\| e^{(\Delta-1) t}v_0 + \int_0^te^{(\Delta-1)(s-t)}u\dd s\right\|_{W^{1,p}}\\
	&\leq \| e^{(\Delta-1) t}v_0 \|_{W^{1,p}}
		+ \left\| \int_0^te^{(\Delta-1)(s-t)}u\dd s\right\|_{W^{1,p}}\\
	&\leq C\|v_0\|_{W^{1,p}} 
		+ C\|u\|_{X_1}\int_0^t\left(1+s^{1/2}\right)e^{-s}\dd s.
\end{align*}
Similarly, we get
\begin{align*}
	\|\Psi_1(u_1,v_1) - \Psi_1(u_2,v_2)\|_\infty
	&= \left\|\int_0^te^{\Delta(s-t)}\nabla\cdot(u_1\nabla v_1 - u_2\nabla v_2)\dd s\right\|_\infty\\
	&\leq \left\|\int_0^te^{\Delta(s-t)}\nabla\cdot(u_1\nabla (v_1 -  v_2))\dd s\right\|_\infty
		+ \left\|\int_0^te^{\Delta(s-t)}\nabla\cdot((u_1 - u_2)\nabla v_2)\dd s\right\|_\infty\\
	&\leq C(\|u_1\|_{X_1} \|v_1-v_2\|_{X_2} + \|u_1-u_2\|_{X_1} \|v_2\|_{X_2}) 
		\int_0^t\left(1+s^{-\frac{1}{2} - \frac{n}{2p}}\right)\dd s
\end{align*}
and
\begin{align*}
	\|\Psi_2(u_1,v_1) - \Psi_2(u_2,v_2)\|_{W^{1,p}}
	&\leq C\|u_1-u_2\|_{X_1}\int_0^t\left(1+s^{1/2}\right)e^{-s}\dd s.
\end{align*}

define the topology
\[
	(u,v)_X := \|u\|_{X_1} + \|v\|_{X_2},
\]
suppose
\[
	(u_0,v_0)_X = \|u_0\|_\infty + \|v_0\|_{W^{1,p}} < M,
\]
then we can choose $T>0$ small enough such that 
$\psi$ has a fixed point on the set 
\[
	S := \{ (u,v)\in X: (u,v)_X \leq M + C + 1\}.
\]
To see this, we can find that 
$\Psi$ is a onto contracting map, i.e.,
\begin{align*}
	\|\Psi(u_1,v_1) - \Psi(u_2,v_2)\|_X 
	&\leq C(M+C+1)(u_1-u_2,v_1-v_2)_X\int_0^t\left(1+s^{-\frac{1}{2}-\frac{n}{2p}}\right)\dd s,
\end{align*}
and 
\begin{align*}
	\|\Psi\|_X \leq (C+1)(u_0,v_0)_X + C(M+C+1)^2\int_0^t\left(1+s^{-\frac{1}{2}-\frac{n}{2p}}\right)\dd s,
\end{align*}
provided that $T$ is less than a given number dependent on $M$.

By Banach fixed-point theorem, the integral equations
\[
	\Psi(u,v) = (u,v)
\]
has a solution $(u,v)\in X$, which can be extended to its maximal existence time $T_m\in(0,\infty]$.
Here, if $T_m < \infty$, then
\[
	\limsup_{t\nearrow T_m} (u,v)_X = \infty.
\]

\section{comparable criterion of elliptic equation under Neumann boundary conditions}

\begin{lemma}
	Let $u\in H^1(\Omega)$ be the weak solution of the following elliptic equation
	\begin{equation*}
		\begin{cases}
			-\Delta u = u = f, & \Omega,\\
			\frac{\partial u}{\partial \nu} = 0, & \partial\Omega,
		\end{cases}
	\end{equation*} 
	i.e.,
	\begin{align*}
		\int \nabla u\cdot\nabla \phi + \int u\phi = \int f\phi,\quad \phi\in H^1(\Omega),
	\end{align*}
	where $f\in L^2(\Omega)$.
	If $f\geq0$ in the sense of distribution,
	then $u\geq0$ a.e..
\end{lemma}
\begin{proof}
	let $\underline{u} = \min\{u,0\}$, 
	set $\phi = \underline{u}$, 
	then
	\begin{align*}
		\int \nabla u\cdot\nabla\phi + \int_\Omega u\phi 
		= \int|\nabla\phi|^2 + \int_\Omega\phi^2 
		= \int f\phi \leq 0, 
	\end{align*}
	which implies $\phi = 0$ a.e., i.e., $u\geq0$ a.e..
\end{proof}

\section{Helly's compactness theorem}
\begin{theorem}
	单调有界序列存在子列点点收敛.
\end{theorem}
证明思路:
\begin{itemize}
	\item 先构造在有理点处逐点收敛的子序列
	\begin{itemize}
		\item 利用有界性逐点迭代构造
		\item 取对角线序列
	\end{itemize}
	\item 利用单调性扩充定义
	\begin{itemize}
		\item 任取无理点, 若该点处的有理点左右极限相等, 则可将该点扩充到定义域中,
		\item 如若不然, 由单调函数的不连续点至多可数, 
		再利用第一步的技巧, 通过取对角线子列的办法, 
		把剩下的所有无理点扩充到定义中, 得到点点收敛的子序列.
	\end{itemize}
\end{itemize}

\section{BMO space}
\begin{theorem}
	\label{thm: bmo estimates}
	Let $u\in W^{1,1}(\Omega)$ where $\Omega$ is convex, and suppose there exists a constant $K$ such that
	\[
		\int_{\Omega\cap B_R}|Du|\dd x \leq KR^{n-1}\quad\text{for all balls }B_R.
	\]
	Then there exist positive constants $\sigma_0$ and $C$ depending only on $n$ such that
	\[
		\int_\Omega \exp\left(\frac{\sigma}{K}|u-u_\Omega|\right)\dd x \leq C(diam \Omega)^n,
	\]
	where $\sigma = \sigma_0|\Omega|(diam \Omega)^{-n}$.
\end{theorem}
\begin{proof}
	We first show for $\Omega$ be convex and $u\in W^{1,1}(\Omega)$, 
	\begin{equation}
		\label{eq: estimate local oscillation by gradient}
		|u(x) - u_S| \leq \frac{d^n}{n|S|}\int_\Omega|x-y|^{1-n}|Du(y)|\dd y\quad\text{a.e. }\Omega,
	\end{equation}
	where 
	\[
		u_S = \frac{1}{|S|}\int_S u\dd x,\quad d = diam \Omega,
	\]
	and $S$ is any measurable subset of $\Omega$.
	By dense argument, it is enough to establish \eqref{eq: estimate local oscillation by gradient} for $u\in C^1(\Omega)$.
	We then have for $x,y\in\Omega$,
	\[
		u(x) - u(y) = - \int_0^{|x-y|} D_ru(x + r\omega)\dd r, \quad \omega = \frac{y-x}{|y-x|}.
	\]
	Integrating with respect to $y$ over $S$, we obtain
	\[
		|S|(u(x) - u_S) = - \int_S\dd y \int_0^{|x-y|}D_r u(x + r\omega)\dd r.
	\]
	Writing 
	\begin{equation*}
		V(x) = 
		\begin{cases}
			|D_ru(x)|, & x\in\Omega,\\
			0, & x\not\in\Omega,
		\end{cases}
	\end{equation*}
	we thus have 
	\begin{align*}
		|u(x) - u_S| 
		&\leq \frac{1}{|S|}\int_{|x-y|<d}\dd y \int_0^\infty V(x+r\omega)\dd r\\
		&= \frac{1}{|S|}\int_0^\infty\int_{|\omega|=1}\int_0^dV(x+r\omega)\rho^{n-1}\dd\rho\dd\omega\dd r\\
		&= \frac{d^n}{n|S|} \int_0^\infty\int_{|\omega|=1}V(x+r\omega)\dd\omega\dd r\\
		&= \frac{d^n}{n|S|} \int_\Omega |x-y|^{1-n}|Du(y)|\dd y.
	\end{align*}
	Using 
	\[
		|x-y|^{1-n} = |x-y|^{(1/q-n)/q} \cdot |x-y|^{(1-1/q)(1+1/q-n)},
	\]
	we have by H\"older inequality
	\begin{align*}
		\int_\Omega |x-y|^{1-n}|Du(y)|\dd y
			&\leq \left(\int_\Omega|x-y|^{1/q-n}|Du(y)|\dd y\right)^{1/q}
				\cdot \left(\int_\Omega |x-y|^{1/q+1-n}|Du(y)|\dd y\right)^{1-1/q}.
	\end{align*}
	Extend $f$ to be zero outside $\Omega$ and write
	\[
		v(\rho) = \int_{B_\rho(x)}|Du(y)|\dd y.
	\]
	Then 
	\begin{align*}
		\int_\Omega|x-y|^{1+1/q-n}|Du(y)|\dd y 
			&= \int_\Omega \rho^{1+1/q-n}|Du(y)|\dd y,\quad \rho = |x-y|\\
			&= \int_0^d\rho^{1+1/q-n}v'(\rho)\dd\rho,\quad d=diam \Omega\\
			&= d^{1+1/q-n}v(d) + (n-1-1/q)\int_0^d\rho^{1/q-n}v(\rho)\dd\rho\\
			&\leq Kd^{1/q} + (n-1-1/q)\int_0^d\rho^{1/q-1}\dd\rho\\
			&\leq q(n-1)Kd^{1/q},
	\end{align*}
	For, choose $R_0>0$ so that $|\Omega| = |B_{R_0}(x)| = \omega_nR^n$. Then
	\begin{align*}
		\int_\Omega |x-y|^{1/q-n}\dd y 
			&\leq \int_{B_{R_0}(x)}|x-y|^{1/q-n}\dd y\\
			&= q\omega_nR_0^{1/q} = q\omega_n^{1-1/(qn)}|\Omega|^{1/(qn)}.
	\end{align*}
	Then it follows that %% by adapting the usual proof of the Young inequality for convolutions
	\begin{align*}
		\int_\Omega\int_\Omega|x-y|^{1/q-n}|Du(y)|\dd y \dd x 
		 &\leq \int_\Omega\int_\Omega |x-y|^{1/q-n}\dd x |Du(y)|\dd y\\
		 &\leq q\omega_nd^{1/q}\|Du\|_1.
	\end{align*}
	Hence 
	\begin{align*}
		\left\|\int_\Omega |x-y|^{1-n}|Du(y)|\dd y\right\|_q^q
			&\leq \int_\Omega\int_\Omega|x-y|^{1/q-n}|Du(y)|\dd y
			\cdot \left(\int_\Omega |x-y|^{1/q+1-n}|Du(y)|\dd y\right)^{q-1}\dd x\\
			&\leq q^q(n-1)^{q-1}K^{q-1}d\omega_n\|Du\|_1.
	\end{align*}
	Consequently,
	\begin{align*}
		\int_\Omega\sum_{m=0}^N \frac{\sigma_0^m|u-u_\Omega|^m}{m!K^m}
			&\leq d\omega_n\|Du\|_1 \sum_{m=0}^N \frac{\sigma_0^md^{nm}m^m(n-1)^{m-1}K^{m-1}}{n^m|\Omega|^mm!K^m}
			\leq C,
	\end{align*}
	if $\sigma_0>|\Omega|/(nd^n)$.
\end{proof}

\section{Moser's Iteration for elliptic equation}
\begin{theorem}[\cite[Theorem~8.15]{Gilbarg2001}]
	Let $u$ be a $W^{1,2}$ solution of 
	\[
		-\Delta u = \partial_if^i + g\quad\text{in }\Omega
	\]
	satisfying $u=0$ on $\partial\Omega$,
	in the sense of
	\begin{equation}\label{eq: formula of weak solutions}
		\int_\Omega \nabla u\cdot\nabla v = - \int_\Omega f^i\partial_iv + \int_\Omega gv 
		\quad\text{for all }v\in C^1_0(\Omega).
	\end{equation}
	Suppose that $f^i\in L^q(\Omega)$, $i=1,2,\cdots,n$, $g\in L^{q/2}(\Omega)$ for some $q>n$.
	Then 
	\[
		\sup_\Omega u \leq C\|u\|_2 + Ck,
	\]
	where $k=\lambda^{-1}(\|\mathbf{f}\|_q + \|g\|_{q/2})$ 
	and $C=C(n,q,|\Omega|)$.
\end{theorem}
\begin{proof}
	For $\beta\geq1$ and $N>k$, let us define a function $H\in C^1([k,\infty))$ by setting
	$H(z) = z^\beta - k^\beta$ for $z\in[k,N]$ and taking $H$ to be linear for $z\geq N$. 
	Let us next set $w=u+k$ and take
	\begin{equation*}
		v = G(w) = \int_k^w|H'(s)|^2\dd s 
	\end{equation*} 
	in the integral equality\eqref{eq: formula of weak solutions}. 
	By the chain rule, $v$ is a legitimate test function in \eqref{eq: formula of weak solutions}
	and on substitution we obtain,
	\begin{align*}
		\int_\Omega |\nabla w|^2 G'(w) &= - \int_\Omega f^i\partial_iwG'(w) 
			+ \int_\Omega G(w)g\\
			&\leq \frac{1}{2}\int_\Omega |\nabla w|^2G'(w) + \int_\Omega |\mathbf{f}|^2G'(w)
				+ \int_\Omega gwG'(w), 
	\end{align*}
	since $G(s)\leq sG'(s)$ and $Du = Dw$ when $v = G(w) >0$. Hence, we obtain,
	\[
		\int_\Omega |\nabla w|^2 G'(w) \leq 6\int_\Omega b G'(w)w^2,
	\]
	with 
	\[
		b = |f|^2/k^2 + |g|/k,
	\]
	that is,
	\[
		\int_\Omega |DH(w)|^2\dd x \leq 6\int_\Omega b|H'(w)w|^2\dd x.
	\]
	Since $H(w)\in W^{1,2}_0(\Omega)$, we may apply the Sobolev inequality and the H\"older inequality to obtain 
	\begin{align*}
		\|H(w)\|_{2n/(n-2)}&\leq C\left(\int_\Omega b(H'(w)w)^2\dd x\right)^{1/2}\\
		&\leq C\|b\|_{q/2}^{1/2} \|H'(w)w\|_{2q(q-2)}.
	\end{align*}
	To proceed further, we recall the definition of $H$ and let $N\to\infty$ in the estimate above.
	It follows then, for any $\beta\geq1$, 
	that the inclusion $w\in L^{2\beta q/(q-2)}(\Omega)$ implies the stronger inclusion, $w\in L^{2\beta n/(n-2)}(\Omega)$, 
	and moreover, setting $q^\ast = 2q/(q-2)$, $\chi=n(q-2)/q(n-2)>1$, we obtain
	\[
		\|w\|_{\beta\chi q^\ast} \leq (C\beta)^{1/\beta}\|w\|_{\beta q^\ast}.
	\] 
	The result is now obtained by iteration of the estimate above. 
	Namely, by induction, we may assume $w\in\cap_{1\leq p<\infty}L^p(\Omega)$. 
	Let us take $\beta = \chi^m$, $m=0,1,2,\cdots$, so that by the estimate above,
	\begin{align*}
		\|w\|_{\chi^N q^\ast} &\leq \Pi_0^{N-1}(C\chi^{m})^{\chi^{-m}}\|w\|_{q^\ast}\\
			&\leq C^{\sigma}\chi^\tau\|w\|_{q^\ast},\quad \sigma = \sum_0^{N-1}\chi^{-m},
				\quad \tau = \sum_0^{N-1}m\chi^{-m}\\
			&\leq C\|w\|_{q^\ast},
	\end{align*}
	where $C=C(n,q,|\Omega|)$. Letting $N\to\infty$, we therefore obtain
	\[
		\sup_\Omega w \leq C\|w\|_{q^\ast},
	\]
	whence by the interpolation inequality we have
	\[
		\sup_\Omega w \leq C\|w\|_2.
	\]
	The desired estimate follows from the definition $w = u + k$.
\end{proof}

The above technique of iteration of $L^p$ norms was introduced by Moser~\cite{Moser1960}.
The proof may also be effected by other choices of test functions.

\subsection{local properties of weak solutions}

Broadly speaking, the scheme of the joint proof follows the Moser iteration method (see \cite{Moser1961}) introduced in the previous section combined with the John-Nirenberg result, which is employed to bridge a vital gap in the iteration scheme.

\begin{lemma}
	if $u$ is a $W^{1,2}(\Omega)$ solution of equation \eqref{eq: formula of weak solutions} in $\Omega$,
	for any ball $B_{2R}(y)\subset\Omega$ and $p>1$,
	\[
		\sup_{B_R(y)} u \leq CR^{-n/p}\|u\|_{p, B_{2R}(y)} + Ck(R),
	\]
	where $C=C(n, \Lambda/\lambda, R, q, p)$.
\end{lemma}
\begin{proof}
	We assume initially that $R=1$ and $k>0$. The general case is later recovered through a simple coordinate transformation: $x\mapsto x/R$, and by letting $k$ tend to zero.
	Let us define, for $\beta\neq0$ and non-negative $\eta\in C^1_0(B_4)$, the test function
	\[
		v = \eta^2\overline{u}^\beta, \quad(\overline{u} = u + k).
	\]
	By the chain and product rules, $v$ is a valid test function in \eqref{eq: formula of weak solutions} 
	and also
	\[
		Dv = 2\eta D\eta \overline{u}^\beta + \beta\eta^2\overline{u}^{\beta-1}Du,
	\]
	so that by substitution into \eqref{eq: formula of weak solutions} we obtain
	\begin{align*}
		\beta\int_\Omega \eta^2 \overline{u}^{\beta-1}|Du|^2
			+ 2\int_\Omega \eta D\eta\cdot Du \overline{u}^\beta 
			= - \beta\int_\Omega \eta^2f^iu_i\overline{u}^{\beta-1}
			- 2\int_\Omega \eta \eta_i\cdot f^i \overline{u}^\beta
			+ \int_\Omega \eta^2\overline{u}^\beta g. 
	\end{align*}
	We can estimate, for any $0<\varepsilon\leq1$,
	\begin{align*}
		- \int_\Omega \eta^2f^iu_i\overline{u}^{\beta-1}
		&\leq \frac{\varepsilon}{2}\int_\Omega \eta^2 \overline{u}^{\beta-1}|Du|^2 
			+ \frac1\varepsilon\int_\Omega \eta^2 \overline{u}^{\beta-1}|\mathbf{f}|^2,
	\end{align*}
	By choosing $\varepsilon = \min\{1,|\beta|<4\}$, we then obtain from estimates above
	\begin{equation*}
		\int_\Omega \eta^2\overline{u}^{\beta-1}|Du|^2\dd X
		\leq C(\beta)\int_\Omega (b\eta^2 + |D\eta|^2)\overline{u}^{\beta +1}\dd x,
	\end{equation*}
	where $C(|\beta|)$ is bounded if $|\beta|$ is bounded away from zero.
	It is now convenient to introduce a function $w$ defined by
	\begin{equation*}
		w = 
		\begin{cases}
			\overline{u}^{(\beta+1)/2}, & \text{if } \beta\not=-1,\\
			\log\overline{u}, & \text{if }\beta = -1.
		\end{cases}
	\end{equation*}
	Letting $\gamma = \beta + 1$, we may rewrite the estimate above
	\begin{equation}
		\label{eq: local gradient estimates}
		\int_\Omega |\eta Dw|^2 \dd x \leq
		\begin{cases}
			C(|\beta|)\gamma^2\int_\Omega (b\eta^2 + |D\eta|^2)w^2\dd x, & \text{if }\beta\neq-1,\\
			C\int_\Omega (b\eta^2 + |D\eta|^2)\dd x, & \text{if }\beta = -1.
		\end{cases}
	\end{equation}
	The desired iteration process can now be developed from the first part of the estimate above.
	For from the Sobolev inequality we have
	\[
		\|\eta w\|^2_{2n/(n-2)} \leq C\int_\Omega (|\eta Dw|^2 + |wD\eta|^2)\dd x.
	\]
	Using the H\"older inequality followed by the interpolation inequality, we obtain, for any $\varepsilon>0$,
	\begin{align*}
		\int_\Omega b(\eta w)^2\dd x 
			&\leq \|b\|_{q/2}\|\eta w\|_{2q/(q-2)}^2\\
			&\leq \|b\|_{q/2}(\varepsilon\|\eta w\|_{2n/(n-2)} + \varepsilon^{-\sigma}\|\eta w\|_2)^2
	\end{align*}
	where $\sigma = n/(q-n)$. 
	Hence, by substitution into the last second estimate and appropriate choice of $\varepsilon$, 
	we obtain
	\begin{equation*}
		\|\eta w\|_{2n/(n-2)} \leq C(1+|\gamma|)^{\sigma+1}\|(\eta + |D\eta|)w\|_2,
	\end{equation*}
	where $C=C(n,q,|\beta|)$ is bounded when $|\beta|$ is bounded away from zero.
	It is now desirable to specify the cut-off function $\eta$ more precisely. 
	Let $r_1$, $r_2$ be such that $1\leq r_1 < r_2 \leq 3$ and set $\eta\equiv 1$ in $B_{r_1}$,
	$\eta\equiv0$ in $\Omega-B_{r_2}$ with $D\eta\leq 2/(r_2-r_1)$.
	Writing $\chi=n/(n-2)$ we then have from the estimate above
	\[
		\|w\|_{L^{2\chi}(B_{r_1})} \leq \frac{C(1+|\gamma|)^{\sigma + 1}}{r_2-r_1}\|w\|_{L^2(B_{r_2})}.
	\]
	For $r<4$ and $p\neq0$, let us now introduce the quantities
	\begin{equation}
		\Phi(p,r) = \left(\int_{B_r}|\overline{u}|^p\right)^{1/p}.
	\end{equation}
	Then we have
	\[
		\Phi(\infty, r) = \lim_{p\to\infty} \Phi(p,r) = \sup_{B_r}\overline{u},
	\]
	and 
	\[
		\Phi(-\infty, r) = \lim_{p\to -\infty}\Phi(p,r) = \inf_{B_r}\overline{u}.
	\]
	From inequality above, we now obtain
	\begin{align*}
		\Phi(\chi\gamma, r_1) &\leq \left(\frac{C(1+|\gamma|)^{\sigma+1}}{r_2-r_1}\right)^{2/|\gamma|}\Phi(\gamma,r_2)\quad \text{if } \gamma > 0,\\
		\Phi(\gamma, r_2) &\leq \left(\frac{C(1+|\gamma|)^{\sigma+1}}{r_2-r_1}\right)^{2/|\gamma|}\Phi(\chi\gamma,r_1)\quad \text{if } \gamma < 0.
	\end{align*}
	These inequalities can now be iterated to yield the desired estimates. 
	For example, taking $p>1$, we set $\gamma = \gamma_m = \chi^mp$ and $r_m = 1 + 2^{-m}$, $m = 0, 1, \cdots$,
	so that, 
	\begin{align*}
		\Phi(\chi^mp,1) &\leq (C\chi)^{2(1+\sigma)\sum m\chi^{-m}}\Phi(p, 2)\\
			& = C\Phi(p,2), \quad C = C(n,q,p).	
	\end{align*}
	Consequently, letting $m$ tend to infinity, we have
	\[
		\sup_{B_1}\overline{u} \leq C\|\overline{u}\|_{L^p(B_2)},
	\]
	and, by means of the transformation: $x\mapsto x/R$, the desired estimate is established.
\end{proof}

\begin{lemma}
	if $u$ is a $W^{1,2}(\Omega)$ solution of equation \eqref{eq: formula of weak solutions} in $\Omega$,
	non-negative in a ball $B_{4R}(y)\subset\Omega$ and $1\leq p<n/(n-2)$, 
	\[
		R^{-np}\|u\|_{L^p(B_{2R}(y))} \leq C\inf_{B_R(y)}u + Ck(R),
	\]
	where $C=C(n, \Lambda/\lambda, R, q, p)$.
\end{lemma}
\begin{proof}
	When $\beta<0$ and $\gamma<1$, we may prove in a similar manner, for any $p$, $p_0$ such that 
	$0<p_0 < p < \chi$,
	\begin{align*}
		&\Phi(p,2)\leq C\Phi(p_0,3)\\
		&\Phi(-p_0, 3) \leq C\Phi(-\infty, 1),\quad C = C(q,p,p_0).
	\end{align*}
	The conclusion of this lemma will thus follow if we can show that, for some $p_0 > 0$,
	\[
		\Phi(p_0, 3) \leq C\Phi(-p_0, 3).
	\]
	In order to establish the estimate above, we turn to the second of the estimates \eqref{eq: local gradient estimates}.
	Let $B_{2r}$ be any ball of radius $2r$, lying in $B_4(=B_4(y))$, and choose the cut-off function $\eta$
	so that $\eta\equiv1$ in $B_r$, $\eta\equiv0$ in $\Omega-B_4$ and $|D\eta\leq2/r$.
	From \eqref{eq: local gradient estimates}, with the aid of the H\"older inequality, we then obtain
	\begin{align*}
		\int_{B_r}|Dw|\dd x &\leq Cr^{n/2}\left(\int_{B_r}|Dw|^2\dd x\right)^{1/2}\\
			&\leq Cr^{n-1},\quad C=C(n).
	\end{align*}
	Hence, by Theorem~\ref{thm: bmo estimates}, there exists a constant $p_0>0$ depending on $n$ such that,
	for 
	\[
		w_0 = \frac{1}{|B_3|}\int_{B_3}w\dd x,
	\]
	we have 
	\[
		\int_{B_3}e^{p_0|w-w_0|}\dd x \leq C(n, |\Omega|),
	\]
	and thus
	\[
		\int_{B_3}e^{p_0w}\dd x \int_{B_3}e^{-p_0w}\dd x 
			\leq C e^{p_0w_0}e^{-p_0w_0} = C.
	\]
	Recalling the definition of $w$, we obtain the desired estimate. 
	The full result then follows by means of the transformation: $x\mapsto x/R$ and by letting $k$ tend to zero.
\end{proof}

\subsection{Harnack inequality}

By combining two lemmas above, we obtain the full Harnack inequality.

\begin{theorem}
	Let $u\in W^{1,2}(\Omega)$ satisfy $u\geq0$ in $\Omega$ and 
	\[
		-\Delta u = \partial_if^i + g,\quad \text{in }\Omega.
	\]
	Then for any ball $B_{4R}(y)\subset\Omega$, 
	\[
		\sup_{B_R(y)}u \leq C\inf_{B_R(y)}u,
	\]
	where $C=C(n,|\Omega|,R)$.
\end{theorem}

\subsection{H\"older estimate}

\begin{theorem}
	Suppose that $f^i\in L^q(\Omega)$, $i=1, \cdot, n$, $g\in L^{q/2}(\Omega)$ for some $q>n$.
	Then if $u$ is a $W^{1,2}$ solution of the equation
	\[
		-\Delta u = D_if^i + g,\quad\text{in }\Omega,
	\]
	it follows that $u$ is locally H\"older continuous in $\Omega$, 
	and for any ball $B_0=B_{R_0}(y)\subset\Omega$ and $R\leq R_0$, we have
	\[
		osc_{B_R(y)}u\leq CR^{\alpha}(R_0^{-\alpha}\sup_{B_0}|u| + k),
	\]
	where $C=C(n,q,R_0)$ and $\alpha=\alpha(n,q,R_0)$ are positive constants, 
	and $k=\|f\|_q + \|g\|_{q/2}$. 
\end{theorem}

\begin{proof}
	We may assume without loss of generality that $R\leq R_0/4$.
	Let us write $M_0=\sup_{B_0}|u|$, $M_4 = \sup_{B_{4R}}u$, $m_4 = \inf_{B_{4R}}u$,
	$M_1=\sup_{B_R}u$, $m_1 = \inf_{B_R}u$.
	Then we have 
	\begin{gather*}
		-\Delta(M_4-u) = -D_if^i - g,\\
		-\Delta(u-m_4) = D_if^i + g.
	\end{gather*}
	Hence, if we set
	\begin{align*}
		K(R) &= R^\delta\|f\|_q + R^{2\delta}\|g\|_{q/2},\\
		\delta &= 1-n/q 
	\end{align*}
	and apply the weak Harnack inequality with $p=1$ to the functions 
	$M_4-u$, $u-m_4$ in $B_{4R}$, we obtain
	\begin{gather*}
		R^{-n}\int_{B_{2R}}(M_4-u)\dd x \leq C(M_4 - M_1 + K(R)),\\
		R^{-n}\int_{B_{2R}}(u-m_4)\dd x \leq C(m_1-m_4) + K(R).
	\end{gather*}
	Hence by addition,
	\[
		M_4 - m_4 \leq C(M_4 - m_4 + m_1 - M_1 + K(R)),
	\]
	so that, writing 
	\[
		\omega(R) = osc_{B_R} u = M_1 - m_1,
	\]
	we have 
	\[
		\omega(R) \leq \gamma\omega(4R) + K(R),
	\]
	where $\gamma = 1- C^{-1}$, $C = C(n,R_0,q)$.
	The following simple lemma then implies the desired result.
\end{proof}

\begin{lemma}
	Let $\omega$ be a non-decreasing function on an interval $(0,R_0]$ satisfying,
	for all $R\leq R_0$, the inequality 
	\[
		\omega(\tau R) \leq \gamma \omega(R) + \sigma(R),
	\]
	where $\sigma$ is also non-decreasing and $0<\gamma, \tau < 1$.
	Then, for any $\mu\in(0,1)$ and $R\leq R_0$,
	we have 
	\[
		\omega(R)\leq C\left(\left(\frac{R}{R_0}\right)^\alpha \omega(R_0) + \sigma(R^\mu R_0^{1-\mu})\right),
	\]
	where $C=C(\gamma, \tau)$ and $\alpha=\alpha(\gamma, \tau, \mu)$ are positive constants.
\end{lemma}

\begin{proof}
	Let us fix initially some number $R_1\leq R_0$.
	Then for any $R\leq R_1$ we have
	\[
		\omega(\tau R) \leq \gamma \omega(R) + \sigma(R_1),
	\]
	since $\sigma$ is non-decreasing. 
	We now iterate this inequality to get,
	for any positive integer $m$,
	\begin{align*}
		\omega(\tau^mR_1)
			&\leq \gamma^m\omega(R_1) + \sigma(R_1)\sum_{i=0}^{m-1}\gamma^i\\
			&\leq \gamma^m\omega(R_0) + \frac{\sigma(R_1)}{1-\gamma}.
	\end{align*}	
	For any $R\leq R_1$, we can choose $m$ such that
	\[
		\tau^mR_1<R\leq\tau^{m-1}R_1.
	\]
	Hence
	\begin{align*}
		\omega(R) &\leq \omega(\tau^{m-1}R_1)\\
			&\leq \gamma^{m-1}\omega(R_0) + \frac{\sigma(R_1)}{1-\gamma}\\
			&\leq \frac{1}{\gamma}\left(\frac{R}{R_1}\right)^{\log\gamma/\log\tau}\omega(R_0)
				+ \frac{\sigma(R_1)}{1-\gamma}.
	\end{align*}
	Now let $R_1=R_0^{1-\mu}R^{\mu}$ so that we have from the preceding
	\[
		\omega(R) \leq \frac{1}{\gamma}\left(\frac{R}{R_0}\right)^{(1-\mu)\log\gamma/\log\tau}\omega(R_0)
		+ \frac{\sigma(R_0^{1-\mu}R^\mu)}{1-\gamma}.
	\]
\end{proof}


\section{Moser's iteration for parabolic equation}

\begin{lemma}
	suppose $u\in C^0(\overline{\Omega}\times[0,T))\cap C^{2,1}(\Omega\times[0,T))$ satisfy 
\begin{equation*}
	\begin{cases}
		u_t \leq \Delta u + \nabla\cdot F, & \Omega,\\
		\nabla u\cdot n \leq 0, \quad F\cdot n \leq 0, & \partial\Omega,
	\end{cases}
\end{equation*}
if 
\[
	f\in L^\infty((0,T); L^q(\Omega)),\quad q>n,
\]
then
$u\in L^\infty(\Omega\times(0,T))$.
\end{lemma}
\begin{proof}
	\begin{align*}
		\frac{\dd}{\dd t}\int u^p &+ p(p-1)\int u^{p-1}|\nabla u|^2 
		\leq - p(p-2)\int u^{p-2}\nabla u\cdot F\\
		&\leq p(p-1)\left(\int u^{\frac{q(p-2)}{q-1}}|\nabla u|^{\frac{q}{q-1}}\right)^{\frac{q-1}{q}}
			\left(\int F^q\right)^{1/q}\\
		&= p(p-1)\left(\int \left(u^{\frac{p-2}{2}}|\nabla u|\right)^{\frac{q}{q-1}}
			\cdot u^{\frac{q(p-2)}{2(q-1)}}\right)^{\frac{q-1}{q}}\|F\|_q\\
		&\leq p(p-1)\left(\int u^{p-2}|\nabla u|^2\right)^{\frac12}
			\cdot\left(\int u^{\frac{q(p-2)}{q-2}}\right)^{\frac{q-2}{2q}}\|F\|_q,
	\end{align*}
	\textbf{integrability.} using
	\begin{align*}
		\left(\int u^{\frac{q(p-2)}{q-2}}\right)^{\frac{q-2}{2q}}
		&= \|u^{\frac{p}{2}}\|_{\frac{2q(p-2)}{p(q-2)}}^{\frac{p-2}{p}}\\
		&\leq C\|\nabla u^{\frac{p}{2}}\|_2^{\frac{(p-2)\theta}{p}}\cdot\|u^{\frac{p}{2}}\|_2^{\frac{(1-\theta)(p-2)}{p}} + C\|u^{\frac{p}{2}}\|_2^{\frac{p-2}{p}},
	\end{align*}
	with 
	\[
		\frac{p(q-2)}{2q(p-2)} = \theta\left(\frac12-\frac1n\right) + 1-\theta,
	\]
	noting $p>q>n$ ensures that
	\[
		0<\gamma := \frac{(p-2)\alpha}{p} = \frac{n}{q} - \frac{n}{p} < 1,
	\]
	we have 
	\begin{align*}
		\frac{\dd}{\dd t}\int u^p &+ \frac{4(p-1)}{p}\int |\nabla u^{\frac{p}{2}}|^2
		\leq Cp(p-1)\left(\int u^{p-2}|\nabla u|^2\right)^{\frac12+\frac\gamma2}\cdot\|u^{\frac{p}{2}}\|_2^{\frac{p-2}{p} - \gamma}\cdot\|F\|_q \\
		&\quad +Cp(p-1) \left(\int u^{p-2}|\nabla u|^2\right)^{\frac12}\cdot\|u^{\frac{p}{2}}\|_2^{\frac{p-2}{p}}\cdot\|F\|_q \\
		&\leq \frac{2(p-1)}{p}\int |\nabla u^{\frac{p}{2}}|^2 
			+ C(F)p^2\|u^{\frac{p}{2}}\|_2^{\frac{2(p-2)}{p(1-\gamma)}-\frac{2\gamma}{1-\gamma}}
			+ C(F)p^2\|u^{\frac{p}{2}}\|_2^{\frac{2(p-2)}{p}},
	\end{align*}
	and obtain $\|u\|_p\leq C(p)$.

	\textbf{boundedness.}
	\begin{align*}
		\left(\int u^{\frac{q(p-2)}{q-2}}\right)^{\frac{q-2}{2q}}
		&= \|u^{\frac{p}{2}}\|_{\frac{2q(p-2)}{p(q-2)}}^{\frac{p-2}{p}}\\
		&\leq C\|\nabla u^{\frac{p}{2}}\|_2^{\frac{(p-2)\alpha}{p}}\cdot\|u^{\frac{p}{2}}\|_1^{\frac{(1-\alpha)(p-2)}{p}} + C\|u^{\frac{p}{2}}\|_1^{\frac{p-2}{p}},
	\end{align*}
	with 
	\[
		\frac{p(q-2)}{2q(p-2)} = \alpha\left(\frac12-\frac1n\right) + 1-\alpha,
	\]
	noting $q>n$ ensures that
	\[
		\kappa := \frac{(p-2)\alpha}{p} = \frac{\frac{p-2}{p}-\frac{q-2}{2q}}{\frac12+\frac1n} <\gamma < 1,
		\quad p > 1,
	\]
	we get
	\begin{align*}
		\frac{\dd}{\dd t}\int u^p &+ \frac{4(p-1)}{p}\int |\nabla u^{\frac{p}{2}}|^2
		\leq Cp(p-1)\left(\int u^{p-2}|\nabla u|^2\right)^{\frac12+\frac\kappa2}\cdot\|u^{\frac{p}{2}}\|_1^{\frac{p-2}{p} - \kappa}\cdot\|F\|_q \\
		&\quad +Cp(p-1) \left(\int u^{p-2}|\nabla u|^2\right)^{\frac12}\cdot\|u^{\frac{p}{2}}\|_1^{\frac{p-2}{p}}\cdot\|F\|_q \\
		&\leq \frac{2(p-1)}{p}\int |\nabla u^{\frac{p}{2}}|^2 
			+ C(F)p^2\|u^{\frac{p}{2}}\|_1^{\frac{2(p-2)}{p(1-\kappa)}-\frac{2\kappa}{1-\kappa}}
			+ C(F)p^2\|u^{\frac{p}{2}}\|_1^{\frac{2(p-2)}{p}}.
	\end{align*}
	Using 
	\[
		\|f\|_2\leq C\|\nabla f\|_2^{\frac{n}{n+2}}\cdot\|f\|_1^{\frac{2}{n+2}} + C\|f\|_1,\quad f\in W^{1,2}(\Omega),
	\]
	we arrive 
	\begin{align*}
		\frac{\dd}{\dd t}\int u^p &+ \int  u^{p}
			\leq Cp^2 \|u^{\frac{p}{2}}\|_1^{\frac{2(p-2)}{p(1-\kappa)}-\frac{\kappa}{1-\kappa}}
				+ Cp^2\|u^{\frac{p}{2}}\|_1^{\frac{2(p-2)}{p}}
				+ C\|u^{\frac{p}{2}}\|_1^2,
	\end{align*}
	Let
	\[
		M_j(t) := \int u^{2^j} + 1, \quad j\geq1,
	\]
	noting 
	\[
		\frac{2(p-2)}{p(1-\kappa)}-\frac{\kappa}{1-\kappa} < 2,\quad p>2,
	\]
	we have 
	\[
		M_j' + M_j \leq C4^jM_{j-1}^2,
	\]
	which implies 
	\begin{align*}
		N_j := \sup_{t\in(0,T)}M_j \leq \max\left\{ M_j(0), C4^jM_{j-1}^2\right\},
	\end{align*}
	and hence
	\[
		N_j \leq \max\left\{ M_j(0), C4^jN_{j-1}^2\right\}.
	\]
	If there exist countably many $j$ such that $N_j \leq M_j(0)$, 
	then $\|u\|_\infty \leq \|u_0\|_\infty + 1$.
	On the other hand, there exists $j_0$ such that for any $j>j_0$, 
	\[
		N_j \leq C4^jN_{j-1}^2,
	\]
	it is no harm to assume that $j_0=1$ since we can appropriately enlarge the value of $C$.
	Therefore,
	\[
		N_j\leq C^{\sum_{i=1}^j2^{i-1}}4^{\sum_{i=1}^{j}(j+1-i)2^{i-1}}N_1^{2^{j-1}}
			\leq C^{2^j}4^{2^{j+1}}N_1^{2^{j-1}},
	\]
	and 
	\[
		\|u\|_\infty\leq\limsup_{j\to\infty}N_j^{2^{-j}}
			\leq\limsup_{j\to\infty}C4^2N_1^{2^{-1}}=16C\sqrt{N_1}.
	\]
\end{proof}

\section{De Giorgi iteration}

\begin{lemma}
	suppose $u\in C^0(\overline{\Omega}\times[0,T))\cap C^{2,1}(\Omega\times[0,T))$ satisfy 
\begin{equation*}
	\begin{cases}
		u_t \leq \Delta u + \nabla\cdot F, & \Omega,\\
		\nabla u\cdot n \leq 0, \quad F\cdot n \leq 0, & \partial\Omega,
	\end{cases}
\end{equation*}
if 
\[
	f\in L^p(\Omega)\times(0,T)),\quad p>n+2,
\]
then
$u\in L^\infty(\Omega\times(0,T))$.
\end{lemma}

\begin{proof}
	\begin{align*}
		-\int(u-k)_+^2(0) + \int (u-k)_+^2
		&= \iint (u-k)_+u_t = \iint (u-k)_+\Delta u + \iint(u-k)_+\nabla\cdot F\\
		&\leq - \iint |\nabla(u-k)_+|^2 - \iint \nabla(u-k)_+\cdot F\\
		&\leq - \iint |\nabla(u-k)_+|^2 
			- \left(\iint |\nabla(u-k)_+|^2\right)^{\frac12}
			\cdot\left(\iint |F|^p\right)^{\frac1p}
			\cdot |\{u\geq k\}|^{\frac{1}{2}-\frac1p},
	\end{align*}
using 
\[
	\|f\|_{\frac{2(n+2)}{n}}\leq C\|\nabla f\|_2^{\frac{n}{n+2}}\cdot\|f\|_2^{\frac{2}{n+2}} 
		+ C\|f\|_2,	\quad f\in W^{1,2},
\]
we have
\begin{align*}
	\left(\iint (u-k)_+^{\frac{2(n+2)}{n}}\right)^{\frac{n}{n+2}}
	\leq \int(u-k)_+^2(0) + \|F\|_p^2\cdot|\{u\geq k\}|^{1-\frac2p}.
\end{align*}
fix $l$ sufficiently large such that 
\[
	\int (u-l)_+^2(0) = 0,
\]
and noting that for any $h>k$,
\[
	\left(\iint (u-k)_+^{\frac{2(n+2)}{n}}\right)^{\frac{n}{n+2}}
	\geq (h-k)^2|\{u\geq h\}|^{\frac{n}{n+2}},
\]
denote 
\[
	\psi(\eta) = |\{u\geq \eta\}|,
\]
we get
\[
	\psi(h) \leq \frac{C(F)\psi(k)^{\frac{n+2}{n}\cdot\frac{p-2}{p}}}{(h-k)^{\frac{2(n+2)}{n}}},
	\quad h>k>l.
\]
Here, $p>n+2$ guarantees that 
\[
	\frac{n+2}{n}\cdot\frac{p-2}{p} > 1,
\]
which implies the desired assertion by the following lemma.
\end{proof}

\begin{lemma}
	if there exist $\varepsilon>0$, $p>0$, $c>0$ and $s_0>0$ such that
	for each $b>a>s_0$, the non-increasing function $f:\mathbb{R}^+\mapsto\mathbb{R}^+$ has the following property,
	\[
		f(b) \leq \frac{cf(a)^{1+\varepsilon}}{(b-a)^p},
	\]
	then there exists $d>0$ such that $f(s_0 + d) = 0$.
\end{lemma}
\begin{proof}
	let
	\[
		b_k = s_0+d-\frac{d}{2^k},
	\]
	then 
	\begin{align*}
		f(b_k) &\leq \frac{cf^{1+\varepsilon}(b_{k-1})}{\left(\frac{d}{2^k}\right)^p}
			= \frac{c}{d^p}2^{pk}f^{1+\varepsilon}(b_{k-1})\\
			&\leq  \frac{c}{d^p}2^{pk}\left(\frac{c}{d^p}2^{p(k-1)}f^{1+\varepsilon}(b_{k-2})\right)^{1+\varepsilon}\\
			&= \left(\frac{c}{d^p}\right)\cdot \left(\frac{c}{d^p}\right)^{1+\varepsilon}
				\cdot 2^{pk + p(k-1)(1+\varepsilon)}f^{(1+\varepsilon)^2}(b_{k-2})\\
			&= \left(\frac{c}{d^p}\right)^{\sum_{j=0}^{k-1}(1+\varepsilon)^j}
				\cdot 2^{p\sum_{j=0}^k(k-j)(1+\varepsilon)^j}
				\cdot f^{(1+\varepsilon)^k}(b_0)\\
			&\leq \left(\frac{c}{d^p}\right)^{\frac{(1+\varepsilon)^k-1}{\varepsilon}}
				\cdot 2^{\frac{p(1+\varepsilon)^k}{\varepsilon^2}-\frac{1+\varepsilon}{\varepsilon^2}-\frac{k}{\varepsilon}}
				\cdot f^{(1+\varepsilon)^k}(b_0)\\
			&= \left(\left(\frac{c}{d^p}\right)^{\frac{1}{\varepsilon}}
				\cdot 2^{\frac{p}{\varepsilon^2}}
				\cdot f(b_0)\right)^{(1+\varepsilon)^k},
	\end{align*}
	which implies $f(s_0 + d) = 0$, if $d$ is chosen so large that 
	\[
		d > c^{\frac{1}{p}} 
			\cdot 2^{\frac{1}{\varepsilon}}
			\cdot f^{\frac{\varepsilon}{p}}(b_0).
	\] 
\end{proof}

\section{porous medium diffusion}
the chemotaxis model with porous medium diffusion ($m>1$)
\begin{equation}
	\begin{cases}
		u_t = \Delta u^m - \nabla \cdot ( u\nabla v), & (x,t)\in\Omega\times(0,T),\\
		\varepsilon v_t = \Delta v - \alpha v + u, & (x,t)\in\Omega\times(0,T),
	\end{cases}
\end{equation}
has a free energy functional
\begin{equation}
	\mathcal{F}_\varepsilon(u,v)(t) :=  \int_\Omega \frac{u^m-1}{m-1} - \int_\Omega uv 
		+ \frac{1}{2}\int_\Omega |\nabla v|^2 + \frac{\alpha}{2}\int_\Omega v^2,
\end{equation}
which satisfies 
\begin{equation}
	\frac{\dd}{\dd t}\mathcal{F}_\varepsilon = - \mathcal{D}_\varepsilon.
\end{equation}
Here, 
\begin{equation*}
	\mathcal{D}_\varepsilon(u,v) 
		= - \int_\Omega u\left|\nabla\left(\frac{m}{m-1}u^{m-1} - v\right)\right|^2
			- \varepsilon\int_\Omega v_t^2.
\end{equation*}

When $\varepsilon = 0$, due to
\[
	\int_\Omega |\nabla v|^2 + \alpha\int_\Omega v^2 = \int_\Omega uv,
\]
we have
\begin{equation}
	\mathcal{F}_0(u) = \int_\Omega \frac{u^m-1}{m-1} - \frac{1}{2}\int_\Omega uv 
		= \int_\Omega\frac{u^m-1}{m-1} - \frac{1}{2}\int_\Omega u(-\Delta + 1)^{-1}u.
\end{equation}
Thanks to properties of the green function \cite{Grueter1982},
we estimate
\begin{align*}
	\int_\Omega u(-\Delta+1)^{-1}u 
		&\leq C\int_\Omega u\cdot |x|^{n-2}*u\\
		&\leq C\|u\|_m\cdot \||x|^{2-n}*u\|_{\frac{m}{m-1}}\\
		&\leq C\|u\|_m \cdot \||x|^{2-n}\|_r\cdot\|u\|_s,\quad 1+\frac{m-1}{m} = \frac{1}{r} + \frac{1}{s},
			\quad (2-n)r + n-1 > -1,\\
		&\leq C\|u\|_m\cdot \||x|^{2-n}\|_r\cdot\|u\|_m^{\theta}\cdot\|u\|_1^{1-\theta},
			\quad \frac{1}{s} = \frac{\theta}{m} + \frac{1-\theta}{1}, \theta = \frac{s-1}{s}\frac{m}{m-1}\\
		&\leq C\|u\|_m^{1+\theta}\cdot \||x|^{2-n}\|_r\cdot\|u\|_1^{1-\theta}.
\end{align*}
If 
\[
	m > 2-\frac2n,
\]
we can choose appropriately $r$ such that 
\begin{align*}
	1+\theta 
		&= 1+\frac{s-1}{s}\cdot\frac{m}{m-1}\\
		&= 1 + \left(\frac1r-\frac{m-1}{m}\right)\cdot\frac{m}{m-1}\\
		&< 1 + \left(\frac{n-2}{n} - \frac{m-1}{m}\right)\cdot\frac{m}{m-1}\\
		&= \frac{n-2}{n}\frac{m}{m-1}\leq m,
\end{align*} 
and hence $\mathcal{F}_0(u) \geq - C(\lambda)$ with $\lambda := \int_\Omega u$.

\textbf{method 2.}
\begin{align*}
	\int_\Omega uv &\leq \|u\|_{\frac{2n}{n+2}}\cdot\|v\|_{\frac{2n}{n-2}}\\
		&\leq C\|u\|_{\frac{2n}{n+2}}\cdot\|v\|_{W^{1,2}}\\
		&\leq C\|u\|_m^\theta\cdot\|u\|_1^{1-\theta}\cdot\|v\|_{W^{1,2}},
			\quad \frac{n+2}{2n} = \frac{\theta}{m} + \frac{1-\theta}{1},\\
		&\leq \frac{1}{2}\|v\|_{W^{1,2}}^2 + C(\lambda)\|u\|_m^{2\theta}.
\end{align*}
If 
	\[
		m\geq 2-\frac2n,
	\]
we have 
\[
	2\theta = \frac{n-2}{n}\cdot\frac{m}{m-1}\leq m,
\]
and hence $\mathcal{F}_0(u) \geq - C(\lambda)$ with $\lambda$ appropriately small 
when $m=2-2/n$. 

Particularly, when $m=2-2/n$,
\[
	\int_\Omega uv \leq C\|u\|_m^m \|u\|_1^{2-m},
\]
we end up with 
\begin{equation}
	C_\ast := \sup_{f\neq0}\left\{\frac{\int_\Omega f(-\Delta+1)^{-1}f}{\|f\|_m^m\|f\|_1^{2-m}}: f\in L^m(\Omega)\cap L^1(\Omega) = L^m(\Omega)\right\} < \infty.
\end{equation}

Suppose $\Omega = B_R := \{x\in\mathbb{R}^n: |x|<R\}$. 
Then if $(u,v)$ is a solution of the stationary system
\begin{equation}
	\begin{cases}
		0 = \Delta u^m - \nabla\cdot(u\nabla v), & x\in B_R\\
		0 = \Delta v - \alpha v + u, & x\in B_R.
	\end{cases}
\end{equation}
Then $(\tilde{u}, \tilde{v}) = (R^nu(Rx), R^{n-2}v(Rx))$ is a solution of the following problem
\begin{equation*}
	\begin{cases}
		0 = \Delta \tilde{u}^m - \nabla\cdot(\tilde u\nabla \tilde v), & x\in B_1,\\
		0 = \Delta \tilde v - R^2\alpha \tilde v + \tilde u, & x\in B_1.
	\end{cases}
\end{equation*}

for $f\in L^p(\Omega)$, Schwarz Symmetrization~\cite{Kesavan2006}, or, the spherically symmetric and decreasing rearrangement is the function $f^*: \Omega \mapsto \mathbb{R}$ defined by 
\[
	f^\ast(x) = f^\sharp (V_n |x|^n),\quad x\in\Omega,
\]
where
\begin{equation*}
	f^\sharp(s) := 
	\begin{cases}
		\esssup(f),& s=0,\\
		\inf\{t:\mu_f(t) <s\}, & s>0,
	\end{cases}
	\quad \mu_f(t) = |\{|f|>t\}| = |\{x\in\Omega: f(x)>t\}|.
\end{equation*}
Here, $V_n$ denotes the volume of the unit ball in $\mathbb{R}^n$,
\[
	V_n=\frac{\pi^{\frac{n}{2}}}{\Gamma\left(\frac{n}{2}+1\right)},
\]
where $\Gamma(s)$ is the usual Gamma function.

Then for $f\in L^m(\Omega)$, 
\[
	\|f\|_1 = \|f^*\|_1, \quad \|f\|_m = \|f^*\|_m.
\]
Writing
\[
	\tilde G_\alpha(x) = G_\alpha(x)\chi_\Omega,\quad \tilde f = f\chi_\Omega, 
\]
where $G_\alpha$ is the usual green function of $-\Delta + \alpha I$ associated with homogeneous Neumann boundary conditions.
Then by Resiz rearrangement properties \cite[Lemma~2.1]{Lieb1983}
\begin{align*}
	\int_\Omega f(-\Delta + \alpha I)^{-1}f 
		&= \int_\Omega f \cdot G_\alpha * f\\
		&= \iint_{\mathbb{R}^n\times\mathbb{R}^n} \tilde{f}(x)\cdot\tilde{G}_\alpha(x-y)\tilde{f}(y)\\
		&\leq \iint_{\mathbb{R}^n\times\mathbb{R}^n} \tilde{f}^\ast(x)\cdot\tilde{G}^\ast_\alpha(x-y)\tilde{f}^\ast(y)\\
		&= \int_\Omega f^\ast\cdot G_\alpha^\ast*f^\ast.
\end{align*}

Since $G_\alpha\in C^{\infty}_{\mathrm{loc}}(\overline{\Omega}\setminus\{0\})$ solves
\begin{align*}
	-\Delta G_\alpha + \alpha G_\alpha &= \delta_0, \quad x\in\Omega,\\
	\frac{\partial G_\alpha}{\partial n} &= 0,\quad x\in\partial\Omega,
\end{align*}
we have $G_\alpha$ is positive and radially symmetric function, 
and integrating over $\Omega\setminus B_s$, for $s\in(0,1)$, get 
\begin{align*}
	0 &= \int_{\Omega\setminus B_s} \delta_0\dd x 
		= - \int_{\Omega\setminus B_s} \Delta G_\alpha + \alpha\int_{\Omega\setminus B_s} G_\alpha\\
		&= -\left(\int_{\partial\Omega}\frac{\partial G_\alpha}{\partial n}\dd S 
			- \int_{\partial B_s}\frac{\partial G_\alpha}{\partial n}\dd S\right) 
			+ \alpha\int_{\Omega\setminus B_s} G_\alpha\\
		&= \omega_n G_\alpha'(s)s^{n-1} + \alpha\int_{\Omega\setminus B_s} G_\alpha,
\end{align*}  
which implies $G_\alpha$ is radially decreasing, and therefore $G_\alpha = G_\alpha^\ast$.
We end up with 
\[
\int_\Omega f(-\Delta + \alpha I)^{-1}f\leq \int_\Omega f^\ast(-\Delta +\alpha I)^{-1}f^\ast.
\]
Define 
\[
\Lambda(f) := \frac{\int_\Omega f_j(-\Delta + \alpha I)^{-1}f_j}{\|f_j\|_m^m\cdot\|f_j\|_1^{2-m}},
	\quad f\in L^m(\Omega),
\]
and consider a maximizing sequence $\{f_j\}$ in $L^m$, that is
\[
\Lambda(f_j) \to C_\ast.
\]
So we may assume without loss of generality,
$\{f_j\}$ is a family of nonnegative, radially symmetric and non-increasing functions such that $\|f_j\|_m=1$. 
Then for any $R\in(0,1)$,
\begin{align*}
	f_j(R) \leq \left(\frac{\int_{B_R}f_j^m}{|B_R|}\right)^{1/m}
		\leq CR^{-n/m}.
\end{align*}
Relying on Helly's compactness method, there exists a subsequence (still denoted by $f_j$) 
and a measurable function $f$ such that 
\[
f_j \to f, \quad \text{p.p. } x\in\Omega.
\]
By Fatou lemma, $f\in L^m(\Omega)$ and $\|f\|_m\leq\liminf\|f_j\|_m = 1$. 
Moreover, $\|f_j-f\|_1 \to 0$.
Using $f\in L^{2n/(n+2)}$ and Hardy-Littlewood-Sobolev inequality, by Lebesgue dominated convergence theorem,
we have 
\[
\int_\Omega f_j(-\Delta + \alpha I)^{-1}f_j \to \int_\Omega f (-\Delta + \alpha I)^{-1}f,
\]
and thus 
\[
C_\ast = \lim_{j\to\infty}\Lambda(f_j) \leq 
\]
\emph{how to rule out $\|f\|_1 = 0$?}


% cSpell:ignore ager, pdfkeywords, pdfsubject, poho, Amann
% cSpell:enableCompoundWords
