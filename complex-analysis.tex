\chapter{复变函数}

\section{复变函数}
\begin{exercise}
\hfill\\
$\displaystyle|\frac{z-1}{z+1}|<1$表示点$z$的轨迹和图形是什么?它是不是区域?

解答,令$z=a+bi$,带入原不等式化简可得$0<4a[(a+1)^2+b^2]$,这等价于$a>0$,或者说$Rez>0$,此即是说原不等式表示的区域是z平面的右半平面。

证明:z平面上的圆周可以写成$Az\overline{z}+\beta\overline{z}+\overline{\beta}z+C=0$,其中$A>0,C$为实数,$A\neq0$,$\beta$为复数,且$\beta>AC$。
\end{exercise}
\begin{exercise}
\hfill\\
z平面上的圆周可以写成$a\overline{z}+\overline{a}z=C$(a是非零复常数,C是实常数)。

\[
\begin{aligned}
Az\overline{z}+\beta\overline{z}+\overline{\beta}z+c=0&\iff z\overline{z}+\frac{\beta}A\overline{z}+\frac{\overline{\beta}}Az+\frac CA=0\\
&\iff(z+\frac{\beta}A)(\overline{z}+\frac{\overline{\beta}}A)=\frac{\beta\overline{\beta}}{A^2}-\frac CA\\
&\iff(z+\frac{\beta}A)(\overline{z+\frac{\beta}A})=\frac{|\beta|^2-AC}{A^2}\\
&\iff|z+\frac{\beta}A|=\frac{\sqrt{|\beta|^2-AC}}A
\end{aligned}
\]
故原方程表示以$z=-\frac{\beta}A$为圆心以$\frac{\sqrt{|\beta|^2-AC}}A$为半径的圆。

一来,在扩充z平面上直线可表示为半径无限大的直线,此即证明了另一个问题。或者设$z=x+yi;a=\alpha+\beta i,x,y,\alpha,\beta\in R$,由$a\overline{z}+\overline{a}z=C$知,$\alpha x+\beta y=\frac C2$,而由a非零知,$\alpha\beta\neq0$,即$\alpha x+\beta y=\frac C2$表示为xoy平面的一条直线,即$a\overline{z}+\overline{a}z=C$表示为z平面的一条直线;另一方面,设z平面上的直线在xoy平面为$mx+ny=t$(m,n,t均为实数),若取$a=m+ni$则有$a\overline{z}+\overline{a}z=2(mx+ny)=2t$,再取$C=2t$即有$a\overline{z}+\overline{a}z=C$。这就完成了证明。


\end{exercise}







\begin{exercise}
\hfill\\
试证方程$z+e^{-z}=a(a>1)$在$Rez>0$内只有一个根,且为实根。

证明,考虑以$R$为半径原点为圆心的右半圆与虚轴所形成的闭曲线$\Gamma_R$。令$z=i\lambda,\lambda\in R$。代入原方程得:$i\lambda+\cos\lambda-i\sin\lambda=a$,即$\cos\lambda=a>1$,这就是说原方程在虚轴上无解。从而记$f(z)=z+e^{-z}-a$,有$f(z)$在$\Gamma_R$内解析,在$\Gamma_R$内连续且不为零。由幅角原理,当$R$足够大使得$\Gamma_R$包含$f(z)$的零点时,$f(z)$在$\Gamma_R$上的幅角变化应为$2\pi n$,其中$n$为$\Gamma_R$包含零点个数。当$R\rightarrow\infty$时,$n$表示原方程在虚轴右侧的所有零点个数,易知$n<+\infty$。考虑到对称性沿半圆弧的幅角变化等于沿虚轴的幅角变化,且$f(z)=z(1+g(z))$,其中$(1+\frac{e^{-z}-a}{z})$;我们有
\[
\begin{aligned}
\Delta_{\Gamma_R}argf(z)&=\Delta_{\Gamma_R}argz(1+g(z))\\
&=\Delta_{\Gamma_R}argz+\Delta_{\Gamma_R}arg(1+g(z)),
\end{aligned}
\]
其中$g(z)$在$R\rightarrow\infty$时$g(z)$沿$\Gamma_R$
一致趋于零。由此知\[\lim_{R\rightarrow\infty}\Delta_{\Gamma_R}arg(1+g(z))=0.\]
这样一来$\lim_{R\rightarrow\infty}\Delta_{\Gamma_R}f(z)=\pi$,这即是说$f(z)$
在右边平面仅有一解。如若该解$z_0$不在实轴上则方程两边取逆易见$\overline{z_0}$也是原方程的解。
与仅有一解矛盾。从而原方程在右半平面仅有一解且在实轴上。

\end{exercise}



