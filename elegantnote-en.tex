% !TeX spellcheck = en_US
% !TeX encoding = UTF-8
%!TeX program = xelatex


\documentclass[en,hazy,screen,blue,14pt]{elegantnote}

%%%%%%%%%%%%%%%%%%%%%%%%%%%%%%%%%%%%%%%
%%%%%%%%%%%%% my own configurations %%%%%%%%%%%%%%%%%%%%%%%%%
\usepackage{amssymb,amsmath,amsthm} %% $\leqslant%
\usepackage{mathrsfs} %% $\mathscr{text}$
\usepackage{pxfonts}


\newcounter{dummy}
\numberwithin{dummy}{section}
\newcommand{\Lref}[1]{Lemma~\ref{#1}}
\newtheorem{exercise}{Exercise}[dummy]
%% add bib resource in .cls file
%\addbibresource{mynote.bib}


\newcommand{\set}[1]{\left\{#1\right\}}

\newcommand{\dd}{\;\mathrm{d}}
\newcommand{\chapter}[1]{}

\DeclareMathOperator*{\esssup}{ess\,sup}



%%%%%%%%%%%%%%%%%%%%%%%%%%%%%%%%%%%%%%%%%%%%%%%%%%%%%%%%%%%%%


\title{Note}

\author{mao}
\institute{seu}

\version{0.1}
\date{\today}

\begin{document}
\maketitle
% logo
\centerline{\includegraphics[width=0.2\textwidth]{example-image-duck}}


\section{background}

Painter and Hillen \cite{Painter2002} introduced volume-filling effect into chemotaxis models, 
in the form of two component quasilinear parabolic system.

Cieslak and Winkler \cite{Cieslak2008} showed that $q=2/N$ is the critical exponent for finite-time blowup of the following simplified chemotaxis model,
\begin{equation}
	\label{sys: cieslak-winkler}
	\begin{cases}
		u_t = \nabla\cdot(D(u)\nabla u - u\nabla v), &x\in\Omega, t>0,\\
		0 = \Delta v - M + u, &x\in\Omega, t>0,
	\end{cases}
\end{equation}
where 
\[
D(u) = (1+u)^{q-1},
\]
$\Omega = B_R$ and $M$ denotes the average mass of initial data $u_0$. 
Finite-time blowup was detected by constructing a finite-time gradient-explosion subsolution to the scalar parabolic problem transformed form the original model via the partial mass function.
This results can be extended to chemotaxis models with nonlinear sensitivity \cite{Winkler2010a}.


\section{Porous medium diffusion}
the chemotaxis model with proous medium diffusion ($m>1$)
\begin{equation}
	\begin{cases}
		u_t = \Delta u^m - \nabla \cdot ( u\nabla v), & (x,t)\in\Omega\times(0,T),\\
		\varepsilon v_t = \Delta v - \alpha v + u, & (x,t)\in\Omega\times(0,T),
	\end{cases}
\end{equation}
has a free energy functional
\begin{equation}
	\mathcal{F}_\varepsilon(u,v)(t) :=  \int_\Omega \frac{u^m-1}{m-1} - \int_\Omega uv 
		+ \frac{1}{2}\int_\Omega |\nabla v|^2 + \frac{\alpha}{2}\int_\Omega v^2,
\end{equation}
which satisfies 
\begin{equation}
	\frac{\dd}{\dd t}\mathcal{F}_\varepsilon = - \mathcal{D}_\varepsilon.
\end{equation}
Here, 
\begin{equation*}
	\mathcal{D}_\varepsilon(u,v) 
		= - \int_\Omega u\left|\nabla\left(\frac{m}{m-1}u^{m-1} - v\right)\right|^2
			- \varepsilon\int_\Omega v_t^2.
\end{equation*}

When $\varepsilon = 0$, due to
\[
	\int_\Omega |\nabla v|^2 + \alpha\int_\Omega v^2 = \int_\Omega uv,
\]
we have
\begin{equation}
	\mathcal{F}_0(u) = \int_\Omega \frac{u^m-1}{m-1} - \frac{1}{2}\int_\Omega uv 
		= \int_\Omega\frac{u^m-1}{m-1} - \frac{1}{2}\int_\Omega u(-\Delta + 1)^{-1}u.
\end{equation}
Thanks to properties of the green function \cite{Grueter1982},
we estimate
\begin{align*}
	\int_\Omega u(-\Delta+1)^{-1}u 
		&\leq C\int_\Omega u\cdot |x|^{n-2}*u\\
		&\leq C\|u\|_m\cdot \||x|^{2-n}*u\|_{\frac{m}{m-1}}\\
		&\leq C\|u\|_m \cdot \||x|^{2-n}\|_r\cdot\|u\|_s,\quad 1+\frac{m-1}{m} = \frac{1}{r} + \frac{1}{s},
			\quad (2-n)r + n-1 > -1,\\
		&\leq C\|u\|_m\cdot \||x|^{2-n}\|_r\cdot\|u\|_m^{\theta}\cdot\|u\|_1^{1-\theta},
			\quad \frac{1}{s} = \frac{\theta}{m} + \frac{1-\theta}{1}, \theta = \frac{s-1}{s}\frac{m}{m-1}\\
		&\leq C\|u\|_m^{1+\theta}\cdot \||x|^{2-n}\|_r\cdot\|u\|_1^{1-\theta}.
\end{align*}
If 
\[
	m > 2-\frac2n,
\]
we can choose appropriately $r$ such that 
\begin{align*}
	1+\theta 
		&= 1+\frac{s-1}{s}\cdot\frac{m}{m-1}\\
		&= 1 + \left(\frac1r-\frac{m-1}{m}\right)\cdot\frac{m}{m-1}\\
		&< 1 + \left(\frac{n-2}{n} - \frac{m-1}{m}\right)\cdot\frac{m}{m-1}\\
		&= \frac{n-2}{n}\frac{m}{m-1}\leq m,
\end{align*} 
and hence $\mathcal{F}_0(u) \geq - C(\lambda)$ with $\lambda := \int_\Omega u$.

\textbf{method 2.}
\begin{align*}
	\int_\Omega uv &\leq \|u\|_{\frac{2n}{n+2}}\cdot\|v\|_{\frac{2n}{n-2}}\\
		&\leq C\|u\|_{\frac{2n}{n+2}}\cdot\|v\|_{W^{1,2}}\\
		&\leq C\|u\|_m^\theta\cdot\|u\|_1^{1-\theta}\cdot\|v\|_{W^{1,2}},
			\quad \frac{n+2}{2n} = \frac{\theta}{m} + \frac{1-\theta}{1},\\
		&\leq \frac{1}{2}\|v\|_{W^{1,2}}^2 + C(\lambda)\|u\|_m^{2\theta}.
\end{align*}
If 
	\[
		m\geq 2-\frac2n,
	\]
we have 
\[
	2\theta = \frac{n-2}{n}\cdot\frac{m}{m-1}\leq m,
\]
and hence $\mathcal{F}_0(u) \geq - C(\lambda)$ with $\lambda$ appropriately small 
when $m=2-2/n$. 

Particularly, when $m=2-2/n$,
\[
	\int_\Omega uv \leq C\|u\|_m^m \|u\|_1^{2-m},
\]
we end up with 
\begin{equation}
	C_\ast := \sup_{f\neq0}\left\{\frac{\int_\Omega f(-\Delta+1)^{-1}f}{\|f\|_m^m\|f\|_1^{2-m}}: f\in L^m(\Omega)\cap L^1(\Omega) = L^m(\Omega)\right\} < \infty.
\end{equation}

Suppose $\Omega = B_R := \{x\in\mathbb{R}^n: |x|<R\}$. 
Then if $(u,v)$ is a solution of the stationary system
\begin{equation}
	\begin{cases}
		0 = \Delta u^m - \nabla\cdot(u\nabla v), & x\in B_R\\
		0 = \Delta v - \alpha v + u, & x\in B_R.
	\end{cases}
\end{equation}
Then $(\tilde{u}, \tilde{v}) = (R^nu(Rx), R^{n-2}v(Rx))$ is a solution of the following problem
\begin{equation*}
	\begin{cases}
		0 = \Delta \tilde{u}^m - \nabla\cdot(\tilde u\nabla \tilde v), & x\in B_1,\\
		0 = \Delta \tilde v - R^2\alpha \tilde v + \tilde u, & x\in B_1.
	\end{cases}
\end{equation*}

for $f\in L^p(\Omega)$, Schwarz Symmetrization~\cite{Kesavan2006}, or, the spherically symmetric and decreasing rearrangement is the function $f^*: \Omega \mapsto \mathbb{R}$ defined by 
\[
	f^\ast(x) = f^\sharp (V_n |x|^n),\quad x\in\Omega,
\]
where
\begin{equation*}
	f^\sharp(s) := 
	\begin{cases}
		\esssup(f),& s=0,\\
		\inf\{t:\mu_f(t) <s\}, & s>0,
	\end{cases}
	\quad \mu_f(t) = |\{|f|>t\}| = |\{x\in\Omega: f(x)>t\}|.
\end{equation*}
Here, $V_n$ denotes the volume of the unit ball in $\mathbb{R}^n$,
\[
	V_n=\frac{\pi^{\frac{n}{2}}}{\Gamma\left(\frac{n}{2}+1\right)},
\]
where $\Gamma(s)$ is the usual Gamma function.

Then for $f\in L^m(\Omega)$, 
\[
	\|f\|_1 = \|f^*\|_1, \quad \|f\|_m = \|f^*\|_m.
\]
Writing
\[
	\tilde G_\alpha(x) = G_\alpha(x)\chi_\Omega,\quad \tilde f = f\chi_\Omega, 
\]
where $G_\alpha$ is the usual green function of $-\Delta + \alpha I$ associated with homogeneous Neumann boundary conditions.
Then by Resiz rearrangement properties \cite[Lemma~2.1]{Lieb1983}
\begin{align*}
	\int_\Omega f(-\Delta + \alpha I)^{-1}f 
		&= \int_\Omega f \cdot G_\alpha * f\\
		&= \iint_{\mathbb{R}^n\times\mathbb{R}^n} \tilde{f}(x)\cdot\tilde{G}_\alpha(x-y)\tilde{f}(y)\\
		&\leq \iint_{\mathbb{R}^n\times\mathbb{R}^n} \tilde{f}^\ast(x)\cdot\tilde{G}^\ast_\alpha(x-y)\tilde{f}^\ast(y)\\
		&= \int_\Omega f^\ast\cdot G_\alpha^\ast*f^\ast.
\end{align*}

Since $G_\alpha\in C^{\infty}_{\mathrm{loc}}(\overline{\Omega}\setminus\{0\})$ solves
\begin{align*}
	-\Delta G_\alpha + \alpha G_\alpha &= \delta_0, \quad x\in\Omega,\\
	\frac{\partial G_\alpha}{\partial n} &= 0,\quad x\in\partial\Omega,
\end{align*}
we have $G_\alpha$ is positive and radially symmetric function, 
and integrating over $\Omega\setminus B_s$, for $s\in(0,1)$, get 
\begin{align*}
	0 &= \int_{\Omega\setminus B_s} \delta_0\dd x 
		= - \int_{\Omega\setminus B_s} \Delta G_\alpha + \alpha\int_{\Omega\setminus B_s} G_\alpha\\
		&= -\left(\int_{\partial\Omega}\frac{\partial G_\alpha}{\partial n}\dd S 
			- \int_{\partial B_s}\frac{\partial G_\alpha}{\partial n}\dd S\right) 
			+ \alpha\int_{\Omega\setminus B_s} G_\alpha\\
		&= \omega_n G_\alpha'(s)s^{n-1} + \alpha\int_{\Omega\setminus B_s} G_\alpha,
\end{align*}  
which implies $G_\alpha$ is radially decreasing, and therefore $G_\alpha = G_\alpha^\ast$.
We end up with 
\[
\int_\Omega f(-\Delta + \alpha I)^{-1}f\leq \int_\Omega f^\ast(-\Delta +\alpha I)^{-1}f^\ast.
\]
Define 
\[
\Lambda(f) := \frac{\int_\Omega f_j(-\Delta + \alpha I)^{-1}f_j}{\|f_j\|_m^m\cdot\|f_j\|_1^{2-m}},
	\quad f\in L^m(\Omega),
\]
and consider a maximising sequence $\{f_j\}$ in $L^m$, that is
\[
\Lambda(f_j) \to C_\ast.
\]
So we may assume without loss of generality,
$\{f_j\}$ is a family of nonnegative, radially symmetric and non-increasing functions such that $\|f_j\|_m=1$. 
Then for any $R\in(0,1)$,
\begin{align*}
	f_j(R) \leq \left(\frac{\int_{B_R}f_j^m}{|B_R|}\right)^{1/m}
		\leq CR^{-n/m}.
\end{align*}
Relying on Helly's compactness method, there exists a subsequence (still denoted by $f_j$) 
and a measurable function $f$ such that 
\[
f_j \to f, \quad \text{p.p. } x\in\Omega.
\]
By Fatou lemma, $f\in L^m(\Omega)$ and $\|f\|_m\leq\liminf\|f_j\|_m = 1$. 
Moreover, $\|f_j-f\|_1 \to 0$.
Using $f\in L^{2n/(n+2)}$ and Hardy-Littlewood-Sobolev inequality, by Lebesgue dominated convergence theorem,
we have 
\[
\int_\Omega f_j(-\Delta + \alpha I)^{-1}f_j \to \int_\Omega f (-\Delta + \alpha I)^{-1}f,
\]
and thus 
\[
C_\ast = \lim_{j\to\infty}\Lambda(f_j) \leq 
\]
\emph{how to rule out $\|f\|_1 = 0$?}





\section{quadratic logistic dampening prevents chemotactic collapse}

consider 
\begin{equation}
	\begin{cases}
		u_t = \Delta u - \nabla\cdot(u\nabla v) + \mu u (1-u), \\
		v_t = \Delta v - v + u,\\
		\partial_\nu u = \partial_\nu v = 0,
	\end{cases}
\end{equation}
in a bounded domain $\Omega\subset\mathbb{R}^2$ with smooth boundary.
It is obvious that
\[
	\int_t^{t+\tau}\int u^2 < C,
\]
for any $t\in(0, T-\tau)$ with $\tau = \min\{1, T/2\}$.
By conclusions in Section~\ref{sec: L2 theory}, 
we have 
\begin{align*}
	\sup_{(0,T)}\|\nabla v\|_{L^2} 
	+ \sup_{(0,T-\tau)}\int_t^{t+\tau}\int |D^2v|^2 
	+ \sup_{(0, T-\tau)}\int_t^{t+\tau}\int |\nabla v|^4 < C.
\end{align*}


\subsection{method 1: direct \texorpdfstring{$L_p$}{Lp} estimates}
calculate
\begin{align*}
	\frac{\dd}{\dd t}\int u^p 
	&= p\int u^{p-1}\nabla\cdot(\nabla u - u\nabla v) + \mu p \int u^p(1-u)\\
	&= - p(p-1)\int u^{p-2}|\nabla u|^2 
		+ p(p-1) \int u^{p-1}\nabla u\cdot\nabla v 
		+ \mu p\int u^p(1-u)
\end{align*}
estimate
\begin{align*}
	p(p-1) \int u^{p-1}\nabla u\cdot\nabla v
	&= (p-1)\int \nabla u^p\cdot\nabla v
	 = -(p-1)\int u^p\Delta v\\
	&\leq (p-1)\left(\int u^{2p}\right)^{\frac12}\left(\int |\Delta v|^2\right)^\frac{1}{2}
	 = (p-1)\|\Delta v\|_2\cdot\|u^{\frac{p}{2}}\|_4^2\\
	&\leq c(p) \|\Delta v\|_2\cdot \|u^{\frac{p}{2}}\|_{W^{1,2}}\cdot\|u^{\frac{p}{2}}\|_2,
\end{align*}
and
\begin{align*}
	\|u^{\frac{p}{2}}\|_2 
	\leq C\|u^{\frac{p}{2}}\|_{W^{1,2}}^{\frac{p-1}{p}}\|u^{\frac{p}{2}}\|_{\frac2p}^{\frac1p},
\end{align*}
which implies 
\begin{align*}
	\|u^{\frac{p}{2}}\|_{W^{1,2}}^2 \geq c(m,p) \|u^{\frac{p}{2}}\|_2^{\frac{2p}{p-1}}.
\end{align*}
let 
\[
	\mathcal{F}(u) := \int u^p + c,
\]
we have
\begin{align*}
	\mathcal{F}' + \frac1C\mathcal{F}^\frac{p}{p-1} 
	\leq \mathcal{F} \int |\Delta v|^2 + C,
\end{align*}
let
\[
	\mathcal{G} := \ln\mathcal{F},
\]
we have 
\begin{equation*}
	\mathcal{G}' + \frac{\mathcal{G}}{C} \leq \int|\Delta v|^2 + C,
\end{equation*}
which implies the uniform-in-time boundedness of $\|u\|_p$.

\subsection{method 2: \texorpdfstring{$L\log L$}{LlogL}-type estimate}
Let 
\[
	\phi(\xi) := \int_0^\xi \ln^\alpha(1+\eta) \dd\eta,\quad\alpha \in (0,2],
\]
then
\begin{align*}
	\frac{\dd}{\dd t}\int\phi(u)
	&= \int \nabla\cdot(\nabla u - u\nabla v) \ln^\alpha(1+u) + \mu\int u(1-u)\ln^\alpha(1+u)\\
	&= - \int \frac{\alpha|\nabla u|^2\ln^{\alpha-1}(1+u)}{1+u}
		+ \int \frac{\alpha u}{1+u} \ln^{\alpha-1}(1+u)\nabla u\cdot\nabla v  + \mu\int u(1-u)\ln^\alpha(1+u)\\
	&\leq - \alpha \int \frac{|\nabla u|^2\ln^{\alpha-1}(1+u)}{1+u}
		+ \alpha\left(\int \frac{|\nabla u|^2\ln^{\alpha-1}}{1+u}\right)^{\frac12}
			\left(\int \frac{|\nabla v|^2u^2\ln^{\alpha-1}(1+u)}{1+u}\right)^{\frac12}\\
	&\quad + \mu\int u(1-u)\ln^\alpha(1+u)\\
	&\leq \frac{\mu}{2}\int u^2\ln^{2\alpha-2}(1+u) + c(\alpha, \mu)\int|\nabla v|^4 
		+ \mu\int u(1-u)\ln^\alpha(1+u)\\
	&\leq c(\alpha, \mu)\int|\nabla v|^4 - \int\phi(u) + c(\alpha, \mu),
\end{align*}
which implies
\begin{align*}
	\int u\ln^\alpha(1+u) \leq C\int \phi(u) + C \leq C.
\end{align*}
combining with a logarithmic variant of G-N inequality, one can get 
\[
	\|u\|_p + \|\nabla v\|_{2p} < C(p), \quad p>1,
\]
which implies uniform-in-time boundedness of $u$ 
by well-established parabolic regularity theory in the context of chemotaxis models.

or

denote 
\[
	\psi(\xi) := \int_0^\xi \frac{\alpha \eta}{1+\eta}\ln^{\alpha-1}(1+\eta),
\]
then 
\begin{align*}
	\int \frac{\alpha u}{1+u} \ln^{\alpha-1}(1+u)\nabla u\cdot\nabla v
	&= - \int \psi(u) \Delta v\\
	&\leq \varepsilon\int \psi^2(u) + c(\varepsilon)\int |\Delta v|^2\\
	&\leq \varepsilon \int u^2\ln^{2\alpha -2}(1+u) + c(\varepsilon, n) \int |D^2 v|^2,
\end{align*}


\subsection{is the title of this section right when \texorpdfstring{$n\geq3$}{n>=3}?}

\begin{align*}
	\frac2n\frac{\dd}{\dd t}\int u^{\frac n2} 
	&= \int u^{\frac n2-1}\nabla\cdot(\nabla u - u\nabla v) + \mu\int u^{\frac n2} - \mu\int u^{\frac n2+1}\\
	&= -\frac{n-2}{2}\int u^{\frac n2-2}|\nabla u|^2 
		+ \frac{n-2}{2}\int u^{\frac n2-1}\nabla u\cdot\nabla v + \mu\int u^{\frac n2} - \mu\int u^{\frac n2+1}
\end{align*}
\begin{align*}
	\frac{n-2}{2}\int u^{\frac n2-1}\nabla u\cdot\nabla v
	&\leq c(n) \left(\int u^{\frac n2-2}|\nabla u|^2\right)^{\frac12}
		\left(\int u^{\frac n2}|\nabla v|^2\right)^{\frac12}\\
	&\leq c(n) \left(\int u^{\frac n2-2}|\nabla u|^2\right)^{\frac12}
		\left(\int u^{\frac{n+2}{2}}\right)^{\frac{n}{2(n+2)}}
		\left(\int |\nabla v|^{n+2}\right)^{\frac1{n+2}}\\
	&\leq c\|\nabla u^{\frac n4}\|_2
		\cdot \|u^{\frac{n+2}{2}}\|_1^{\frac{n}{2(n+2)}}
		\cdot \||\nabla v|^{\frac n2}\|_2^{\frac{4}{(n+2)n}}
		\cdot \||\nabla v|^{\frac n2}\|_{W^{1,2}}^{\frac{2}{n+2}},
\end{align*}

\begin{align*}
	\frac{\dd}{\dd t}\int |\nabla v|^n 
	= \frac{n}{2} \int |\nabla v|^{n-2}\Delta |\nabla v|^2 - n\int |\nabla v|^{n-2}|D^2v|^2
		- n\int |\nabla v|^n 
		+ n\int |\nabla v|^{n-2}\nabla v\cdot\nabla u,
\end{align*}
\begin{align*}
	\frac{\dd}{\dd t}\int |\nabla v|^n 
		+ n\int |\nabla v|^n
	\leq - n\int |\nabla v|^{n-2}|D^2v|^2
		+ n\int |\nabla v|^{n-2}\nabla v\cdot\nabla u,
\end{align*}
let 
\[
	w=|\nabla v|^{\frac n2},
\]
\begin{align*}
	n\int |\nabla v|^{n-2}\nabla v\cdot\nabla u
	&= -n\int \nabla\cdot(|\nabla v|^{n-2}\nabla v)u \\
	&= -n\int((n-2)|\nabla v|^{n-4}\nabla v \cdot (D^2v\cdot\nabla v) + |\nabla v|^{n-2}\Delta v) u\\
	&\leq c(n)\int |\nabla v|^{n-2}|D^2v|u\\
	&\leq \|\nabla|\nabla v|^{\frac{n}{2}}\|_2\left(\int |\nabla v|^{n-2}u^2\right)^{\frac12}\\
	&\leq c(n)\|\nabla w\|_2\left(\int w^{\frac{2(n-2)}{n}}u^2\right)^{\frac12}\\
	&\leq c(n)\|\nabla w\|_2 \left(\int w^{\frac{2(n+2)}{n}}\right)^{\frac{n-2}{2(n+2)}}
		\left(\int u^{\frac{n+2}{2}}\right)^{\frac{2}{n+2}}\\
	&= c(n)\|\nabla w\|_2 
		\cdot \|w\|_{\frac{2(n+2)}{n}}^{\frac{n-2}{n}}
		\cdot \|u\|_{\frac{n+2}{2}}\\
	&\leq c(n)\|w\|_{W^{1,2}}^{\frac{2n}{n+2}}
		\cdot \|w\|_2^{\frac{2(n-2)}{n(n+2)}}
		\cdot \|u\|_{\frac{n+2}{2}}
\end{align*}
\subsection{a question}
\begin{align*}
	\|u\|_{n/2} < C \Rightarrow \|u\|_\infty < C ?
\end{align*}

let 
\[
	\phi(\xi):= \int_0^\xi \eta^{\frac{n}{2}-1}\ln^{\alpha}(e+\eta), \quad\alpha>1,
\]
then
\[
	\frac{\xi^{\frac{n}{2}}\ln^{\alpha}(e+\xi)}{C} - C 
	\leq \phi(\xi) 
	\leq \xi^{\frac{n}{2}}\ln^{\alpha}(e+\xi),
\]
calculate
\begin{align*}
	\frac{\dd}{\dd t}\int \phi(u) 
	&= \int u^{\frac{n}{2} - 1}\ln^{\alpha}(e+u) \nabla\cdot(\nabla u - u\nabla v)
		+ \mu\int u^{\frac{n}{2}}(1-u)\ln^{\alpha}(e+u)\\
	&= -\frac{n-2}{2}\int |\nabla u|^2u^{\frac{n}{2}-2}\ln^\alpha(e+u)
		- \alpha\int\frac{u^{\frac{n}{2}-1}\ln^{\alpha-1}(e+u)}{e+u}|\nabla u|^2\\
		&\quad + \frac{n-2}{2}\int u^{\frac{n}{2}-1}\ln^\alpha(e+u)\nabla u\cdot\nabla v 
			+ \alpha\int \frac{u^{\frac{n}{2}}\ln^{\alpha-1}(e+u)}{e+u} \nabla u \cdot \nabla v\\
		&\quad + \mu\int u^{\frac{n}{2}}(1-u)\ln^{\alpha}(e+u),
\end{align*}
Noting
\begin{align*}
	\|u\|_{n/2} < C 
	&\Rightarrow \sup_{(0,T-\tau)}\int_t^{t+\tau} \int u^{\frac{n+2}{2}} < C \\
	&\Rightarrow \|\nabla v\|_n 
		+ \sup_{(0,T-\tau)}\int_t^{t+\tau}\int |\nabla v|^{n+2}
		+ \sup_{(0,T-\tau)}\int_t^{t+\tau}\int  |\nabla v|^{n-2}|D^2v|^2 < C,
\end{align*}

\subsection{where is wrong?}
\begin{equation}
	\begin{cases}
		u_t = \Delta u - \nabla\cdot(u\nabla v) + \mu u - \mu u^\alpha,\\
		0 = \Delta v - \overline u + u,
	\end{cases}
\end{equation}
let 
\[
	\phi(t) = \int_0^R ur^{2n-1}\dd r,
\]
and
\[
	m(t) = \int_0^R ur^{n-1}\dd r,
\]
we calculate
\begin{align*}
	\phi'(t) &=
	\int_0^R(u_rr^{n-1})_rr^n - (uv_rr^{n-1})_rr^n\dd r 
		+ \mu\int_0^R u r^{2n-1}\dd r - \mu \int_0^Ru^\alpha r^{2n-1}\dd r,
\end{align*}
estimate
\begin{align*}
	\int_0^R(u_rr^{n-1})_rr^n 
		&= - n\int_0^Ru_rr^{2n-2}\dd r\\
		&= -nu(R) + 2n(n-1)\int_0^Rur^{2n-3}\dd r\\
		&\leq 2n(n-1)\left(\int_0^Rur^{n-1}\dd r\right)^{\frac{2}{n}}
			\cdot \left(\int_0^Rur^{2n-1}\dd r\right)^{\frac{n-2}{n}}.
\end{align*}
Using 
\[
	(v_rr^{n-1})_r -\overline{u}r^{n-1} + ur^{n-1} = 0,
\]
we get
\begin{align*}
	- \int_0^R (uv_rr^{n-1})_rr^n\dd r 
		&= n\int_0^Rur^{n-1}\cdot v_rr^{n-1}\dd r \\
		&= n\int_0^R ur^{n-1} \int_0^r(\overline{u}\eta^{n-1} - u\eta^{n-1})\dd\eta\dd r\\
		&= - \frac{n}2\left(\int_0^Rur^{n-1}\dd r\right)^2 + \overline{u}\int_0^R ur^{2n-1}\dd r, 
\end{align*}
and therefore
\begin{align*}
	\phi'(t) \leq \Phi(t, \phi) := 2n(n-1)m^{\frac2n}\cdot \phi^{\frac{n-2}{n}} - \frac{nm^2}2 + (\overline{u} + \mu)\phi,
\end{align*}

while
\[
	m' = \mu m - \mu m^\alpha,
\]
we have 
\[
	m\in\left[\min\{m(0),1\}, \max\{m(0),1\}\right],
\]
and thus for any $m(0)>0$, there exists $\varepsilon>0$ 
such that for any $\xi\in(0,\varepsilon)$,
$\Phi(t,\xi) < 0$,
that is for any initial data satisfying $\phi(0) < \varepsilon$, 
there exists $t_0\in(0,\infty)$ such that 
\[
	\phi(t_0) = 0,
\]
which implies such solution is not globally well-posed.



%\chapter{数学分析}


%-=-=-=-=-=-=-=-=-=-=-=-=-=-=-=-=-=-=-=-=-=-=-=-=
%	SECTION:
%-=-=-=-=-=-=-=-=-=-=-=-=-=-=-=-=-=-=-=-=-=-=-=-=

\section{集合与映射}\index{Set ! function}

\begin{example}\label{20151012-190708}
\hfill \\
设$f(x)$对任意的$x\in R$有$f(x)=f(x^2)$,且$f(x)$在$x=0$和$x=1$ 处连续。证明$f(x)$在$R$上为常数。(上海交通大学,2003)


steps
  $f(x)$在$x=0$处连续$\Leftrightarrow\forall\varepsilon>0,\exists\delta>0,\forall x\in(-\delta,\delta),|f(x)-f(0)|<\varepsilon$。
  
  $\forall x\in(-1,1),|f(x)-f(0)|=|f(x^2)-f(0)|\cdots=|f(x^{2n}-f(0)|$
  即对上述的$\varepsilon>0$,$\exists N=\max\{1,[\log_x\frac{\delta}2]\}$,$\forall n>N,x^{2n}<\delta$,从而$|f(x)-f(0)|=|f(x^{2n}-f(0)|<\varepsilon$成立。
  
  由$\varepsilon$的任意性,有$\forall x\in(-1,1),f(x)\equiv f(0).(|f(x)-f(0)|\leq 0)$
  
  又$f(x)$在$x=1$处连续$\Leftrightarrow\forall\varepsilon>0,\exists\delta>0,\forall x\in(1-\delta,1+\delta),|f(x)-f(1)|<\varepsilon$成立。
  而$\forall x>1,|f(x)-f(1)|=|f(x^{\frac12})-f(1)|=\cdots|f(x^{\frac1{2n}})-f(1)|$,
  即对上述的$\varepsilon>0,\exists N=\max\{1,[\frac1{2\log_x(1+\delta)}]\},\forall n>N,0<x^{\frac 1{2n}}-1<1+\delta,|f(x)-f(1)|=|f(x^{\frac1{2n}})-f(1)|<\varepsilon$。
  
  由$\varepsilon$的任意性,$0\leq|f(x)-f(1)|\leq 0$,从而$\forall x>1,f(x)\equiv f(1)$。
  
  易知$\forall x<-1,f(x)=f(x^2)=f(1)$,又$f(0)=\lim_{x\rightarrow1^-}f(x)=f(1)=f(-1)$,从而$\forall x\in R,f(x)\equiv f(0)$。

%\qdepend

%\qdependlist
\end{example}
\hfill\\

\section{数列极限}
  \begin{example}
  \hfill\\
   Cauchy收敛准则,叙述并证明。(浙江大学,2003)

  Cauchy收敛准则:数列$\{a_n\}$收敛的充分必要条件是$$\forall\varepsilon>0,\exists N,\forall m,n>N,|a_m-a_n|<\varepsilon.$$
  
  必要性:数列$\{a_n\}$收敛,则$\exists a$使得$\forall\varepsilon>0,\exists N,\forall n>N$,有
  $|a_n-a|<\frac{\varepsilon}2$,从而$\forall m,n>N,|a_m-a_n|\leq|a_m-a|+|a_n-a|<\varepsilon.$
  
  充分性:若$\forall\varepsilon>0,\exists N,\forall m,n>N,|a_m-a_n|<\varepsilon.$则当$\varepsilon=1$时,$\exists N_1$,固定$m+N_1+1,\forall n>N_1,|a_n-a_m|<\varepsilon$。从而数列
  $\{a_n\}$必有界,由抽子列定理,存在子列$\{a_{n_j}\}$收敛到$a$。
  
  于是$\forall\varepsilon>0,\exists N_2,\forall n_j>N_2,|a_{n_j}-a|<\varepsilon.$
  
  $|a_n-a|=|a_n-a_{n_j}+a_{n_j}-a|\leq|a_n-a_{n_j}|+|a_{n_j}-a|\leq\varepsilon+\varepsilon=2\varepsilon.$
  故$\{a_n\}$收敛。

\end{example}

\begin{example}
\hfill\\

 任意给定$x\in R$,令$x_1=\cos x,x_{n+1}=\cos x_n$,证明数列收敛。
 

  $\because x_1\in[-1,1],\therefore x_2\in[-1,1]$归纳易知,$\forall n\in N^+,0\leq|x_n|\leq1$又
  \[
  \begin{aligned}
  |x_{n+1}-x_n|&=|\cos x_n-\cos x_{n-1}|\\
  &=2|\sin^2\frac{x_n}2-\sin^2\frac{x_{n-1}}2|\\
  &=2|\sin\frac{x_n}2-\sin\frac{x_{n-1}}2||\sin\frac{x_n}2+\sin\frac{x_{n-1}}2|\\
  &\leq2|\sin\frac{x_n}2-\sin\frac{x_{n-1}}2|(|\sin\frac{x_n}2|+|\sin\frac{x_{n-1}}2|)\\
  &<2|\sin\frac{x_n}2-\sin\frac{x_{n-1}}2|(\sin\frac12+\sin\frac12)\\
  &=2|\sin\frac{x_n}2-\sin\frac{x_{n-1}}2|\cdot\delta,\verb+其中+\delta=2\sin\frac12\in(0,1)
  \end{aligned}
\]
由拉格朗日中值定理可证:$|\sin\frac{x_n}2-\sin\frac{x_{n-1}}2|\leq\frac12|x_n-x_{n-1}|$。于是$|x_{n+1}-x_n|<\delta|x_n-x_{n-1}|$其中$\delta\in(0,1)$。易推:$|x_{n+1}-x_n|<\delta^n|x_1-x|$于是对$0<n<m$有
\[
\begin{aligned}
|x_m-x_n|&=|x_m-x_{m-1}+x_{m-1}\cdots+x_{n+1}-x_n|\\
&\leq\sum_{i=n+1}^m|x_i-x_{i-1}|\\
&<|x_1-x|\sum_{i=n+1}^m\delta^{i-1}\\
&=|x_1-x|\frac{\delta^n(1-\delta^{m-n}}{1-\delta}\\
&<|x_1-x|\frac{\delta^n}{1-\delta}\rightarrow0,n\rightarrow\infty.
\end{aligned}
\]
即$\forall\epsilon>0,\exists N\in N^+,\forall m,n>N,\verb+有+|x_m-x_n|<\epsilon$,由Cauchy收敛准则知:$\{x_n\}$收敛。

\end{example}
\begin{example}
\hfill\\
设$\phi(x)$与$f(x)$是区间$[a,b]$上的正值连续函数,求证:$$\displaystyle\lim_{n\rightarrow\infty}\frac{\int_a^b\phi(x)f^{n+1}(x)dx}{\int_a^b\phi(x)f^n(x)dx}=\max_{a\leq x\leq b}f(x).$$

设$I_n=\int_a^b\phi(x)f^n(x)\mathrm{d}x$。应用Schwarz不等式,得
$$I_n^2\leq\int_a^b\phi(x)f^{n-1}(x)\mathrm{d}x\int_a^b\phi(x)f^{n+1}(x)\mathrm{d}x=I_{n-1}I_{n+1}.$$
故$$\frac{I_{n+1}}{I_n}\geq\frac{I_n}{I_{n-1}}.$$
因此数列$\{\frac{I_{n+1}}{I_n}\}$是递增的,又
$$\frac{I_{n+1}}{I_n}\leq\max_{a\leq x\leq b}f(x)\frac{I_n}{I_n}=\max_{a\leq x\leq b}f(x),$$
所以$\{\frac{I_{n+1}}{I_n}\}$有界。于是$\lim_{n\rightarrow\infty}\frac{I_{n+1}}{I_n}$存在,且
$$\lim_{n\rightarrow\infty}\frac{I_{n+1}}{I_n}=\lim_{n\rightarrow\infty}\sqrt[n]{I_n}.$$
而$$\lim_{n\rightarrow\infty}\sqrt[n]{I_n}=\max_{a\leq x\leq b}f(x).$$
这就完成了证明。
\end{example}

\begin{example}
\hfill\\
试证$\sum1/p$发散,其中$p$遍历一切质数。

提示:给定$N$,设$p_1,p_2,\cdots,p_k$是至少能整除一个不大于$N$的正整数。那么$$\sum_{n=1}^N\frac{1}{n}\leq\prod_{i=1}^k(1+\frac{1}{p_i}+\frac{1}{p_i^2}+\cdots)=\prod^k(1-\frac{1}{p_i})^{-1}\leq\exp\sum_{i=1}^k\frac{2}{p_i}.$$
最后这个不等式能成立,是因为当$0\leq x\leq \frac{1}{2}$时,$(1-x)^{-1}\leq e^{2x}$。显然当$N\rightarrow+\infty$时,$k\to\infty$,从而
$$\lim_{N\to\infty}\exp\sum_{i=1}^{N(k)}\frac{2}{p_i}\geq\lim_{n\to\infty}\sum_{n=1}^N\frac{1}{n}=+\infty.$$
于是$\sum\frac{2}{p_i}\to\infty$,$k\to\infty$。得证。
\end{example}

 \section{函数极限与连续函数}

\begin{theorem}[Stone-Weierstrass theorem]
  Suppose $f\in C([0,1])$. Then there exists a sequence $\{P_n\}$ of polynomials such that 
  $P_n$ converges to $f$ uniformly $x\in[0,1]$ as $n\to\infty$.  
\end{theorem}

\begin{proof}
  Without loss of generality, we may assume that $f(0) = f(1) = 0$ and extend $f$ to $\mathbb{R}$ such that 
  $f(x)=0$ for $x\not\in[0,1]$.
  Let $g_n(x) = c_n(1-x^2)^n$ with 
  \[
  c_n = \left(\int_0^1(1-x^2)^n\dd x\right)^{-1}.
  \]
  Let 
  \[
  P_n(x) = \int_0^1f(t)g_n(x-t)\dd t.
  \]
  It is clear that $P_n$ is a polynomial, as we need.
\end{proof}

 \begin{example}
  \hfill\\
  
  若函数$f(x)$在$[0,+\infty)$上连续,$\displaystyle\lim_{x\rightarrow\infty}f(x)=A$存在,则$f(x)$在$[0,+\infty)$上一致连续。(中国人民大学,2001)
 
  $\lim_{x\rightarrow\infty}f(x)=A\Leftrightarrow$对任意的$\varepsilon>0,\exists M>0,\forall x_1,x_2>M,|f(x_1)-f(x_2)|<\varepsilon$。
  
  因为$f(x)$在$[0,M+1]$上连续,从而在$[0,M+1]$上一致连续,即存在$0<\delta<1$,对任意的$x_1,x_2\in[0,M+1]$,当$|x_1-x_2|<\delta$时,有$|f(x_1)-f(x_2)|<\varepsilon$。因此,对任意的$x_1,x_2\in[0,+\infty)$,当$|x_1-x_2|<\delta$时,有$|f(x_1)-f(x_2)|<\varepsilon$。
  
  即$f(x)$在$[0,+\infty)|<\varepsilon$。
  \end{example}
  \begin{example}
  \hfill\\
  证明:函数$f(x)=\sqrt{x}lnx$在$[1,+\infty)$上一致连续。(北京大学,2001)
  
  steps
  对$f(x)$求导可得
  \[f'(x)=\frac{\ln x+2}{2\sqrt x},\]
  \[f''(x)=-\frac{\ln x}{4x\sqrt x.}\]
  当$x\in[1,+\infty)$时,$f'(x)>0,f''(x)<0$,因此,$f'(x)$在$[1,+\infty)$上单调递减趋于0,又$\lim_{x\rightarrow1}f'(x)=1$,意味着$f'(x)$在$[1,+\infty)$有界,不妨假设$|f'(x)|\leq M$。
  
  对任意的$\varepsilon>0,\exists\delta=\frac{\delta}M,$对任意的$x_1,x_2\in[1,+\infty)$,当$|x_1,x_2|<\delta$时,有$|f(x_1)-f(x_2)|=|f'(\xi)(x_1-x_2)|<M\cdot\frac{\varepsilon}{M}=\varepsilon.$其中$\xi\in[x_1,x_2]$,即函数$f(x)=\sqrt{x}\ln x$在$[1,+\infty)$上一致收敛。
  \end{example}

\begin{remark}
  札记:可以用Cauchy收敛准则和微分中值定理证明函数一致连续性
\end{remark}

\begin{example}
\hfill\\
  设$f(x)$在$[a,b]$上连续,对$x\in[a,b]$定义$\displaystyle m(x)=\inf_{a\leq t\leq x}f(t)$。证明:$m(x)$在$[a,b]$上连续。(大连理工大学,2004)
  
  
  首先,因为闭区间上的连续函数必有最大值和最小值,故$m(x)$是良定义的。
  
  其次,$\forall a\leq x_1<x_2\leq b$,
  \[
  \begin{aligned}
  m(x_2)&=\inf_{a\leq t\leq x_1}f(t)\\
  &=\min\{\inf_{a\leq t\leq x_1}f(t)+\inf_{x_1\leq t\leq x_2}f(t)\}\\
  &\leq\min_{a\leq t\leq x_1}f(t)\\
  &=m(x_1)\\
  \end{aligned}
  \]
  故$m(x)$在$[a,b]$上单调递减,$\forall x_1,x_2\in[a,b],|x_1-x_2|<2\delta$,因为$f(x)$在$[a,b]$上一致连续,故$\forall\varepsilon>0,\exists\delta<0,|f(x_1)-f(x_2)|<\varepsilon.$从而$\forall x\in[a,b]$,
  \[
  \begin{aligned}
  m(x+)&=\lim_{\delta\rightarrow0+}m(x+\delta)\\
  &=\lim_{\delta\rightarrow0+}\frac{m(x)+\inf_{x\leq t\leq x+\delta}f(t)-|m(x)-\inf_{x\leq t\leq x+\delta}f(\delta)|}{2}\\
  &=\frac{m(x)+f(x)-|f(x)-m(x)|}{2}\\
  &=m(x)\\
  \end{aligned}
  \]
  对$m(x-)$,若$m(x)<m(x-\delta)$,则$f(\eta)=m(x)\leq m(x-\delta)\leq f(x-\delta),(\eta\in(x-\delta,x))$
  因为$f(x)$在$[a,b]$上一致连续,故$\forall\varepsilon>0,\exists\delta>0,x_1,x_2\in[a,b],|x_1-x_2|<\delta,|f(x_1)-f(x_2)|<\varepsilon$。从而对上述$\varepsilon>0$,有上述$\delta>0$,$$|m(x)-m(x-\delta)|\leq|f(\eta)-f(x-\delta)|<\varepsilon,(\eta\in(x-\delta,x).$$
  由此$m(x)$在$x$处左连续,右连续。故$m(x)$在$[a,b]$上连续。
  \end{example}
  
  \begin{example}
  \hfill\\
  如果$f$是$(0,\infty)$上的正值函数,合于
  \begin{enumerate}
  \item[a] $f(x+1)=xf(x)$;
  \item[b] $f(1)=1$;
  \item[c] $\log f$是凸的;
  \end{enumerate}
  那么$f(x)=\Gamma(x)$。其中$$\Gamma(x)=\int_0^{\infty}t^{x-1}e^{-t}\mathrm{d}t.$$
  
  因为$\Gamma$函数满足以上三点,那么只需证明$x>0$时$f(x)$是由以上三点决定的函数就行了。根据$a$只需考虑$x\in(0,1).$
  
  令$\varphi =\log f$,那么$\varphi(x+1)=\varphi(x)+\log x(0<x<\infty).$ $\varphi(1)=0$,并且$\varphi$是凸的,假定$0<x<1$,而$n$是正整数,由上式$\varphi(n+1)=\log (n!).$现在考虑一下$\varphi$在$[n,n+1]$,$[n+1,n+1+x]$,$[n+1,n+2]$三个区间上的差商,既然$\varphi$是凸的,那么$$\log n\leq\frac{\varphi(n+1+x)-\varphi(n+1)}{x}\leq\log(n+1).$$
  $$\varphi(n+1+x)=\varphi(x)+\log[x(x+1)\cdots(x+n)].$$
  所以$$0\leq\varphi(x)-\log[\frac{n!n^x}{x(n+1)\cdots(x+n)}]\leq x\log(1+\frac{1}{n}).$$
  最末的式子当$n\to\infty$时趋于零。从而确定了$\varphi$。
  \end{example}
 
\begin{example}
定义区间$(a,b)$上的实连续函数满足:
$$\forall x,y \in(a,b),f(\frac{x+y}{2})\leq\frac{1}{2}(f(x)+f(y)).$$
求证:$f$是凸函数。其中,所谓凸函数是指
$$f(\lambda x+(1-\lambda)y)\leq\lambda f(x)+(1-\lambda)f(y),\forall x,y\in(a,b),\lambda\in(0,1).$$

\textbf{引理:}若非空集合$S\subset(0,1)\cap\mathit{Q}$满足:若$\frac{1}{2}\in S$;若$p\in S$,则$\frac{p}{1+p},\frac{1}{1+p}\in S.$则$S=(0,1)\cap\mathit{Q}$。

引理的证明只需注意到,$(0,1)$的有理数可用真分数表示,而全体真分数具有$\frac{q}{p}$形式,其中$p<q\in\mathrm{N}^*$。于是我们可以关于分母$p=N$作第二数学归纳法证明上述引理。

我们想用上述引理说明,当$\lambda\in S$时,凸函数定义中的不等式成立。又因为$S$在$(0,1)$中稠密,以及$f$的连续性,可以说明$\forall\lambda\in(0,1)$,凸函数定义中的不等式成立。从而证毕。

现在我们只需说明,若$p\in S$,则$\frac{p}{1+p},\frac{1}{1+p}\in S$即可。

\begin{align*}
f(\frac{p}{1+p}x+\frac{1}{p+1}y)&=f(p(\frac{1}{1+p}x+\frac{p}{1+p}y)+(1-p)y)\\
&\leq pf(\frac{1}{1+p}x+\frac{p}{1+p}y)+(1-p)f(y)\\
&\leq p^2f(\frac{p}{1+p}x+\frac{1}{p+1}y)+p(1-p)f(x)+(1-p)f(y)\\
\end{align*}
于是
$$f(\frac{p}{1+p}x+\frac{1}{p+1}y)\leq \frac{p}{1+p}f(x)+\frac{1}{p+1}f(y),$$
即$\frac{p}{1+p},\frac{1}{1+p}\in S.$证毕。
\end{example} 
 
\begin{proposition}
	Let $I\in\mathbb{R}$ be an open interval. Suppose that $f: I\mapsto\mathbb{R}$ is a convex function,
	i.e.,
	\[
		f(\lambda a + (1-\lambda)b) \leq \lambda f(a) + (1-\lambda)f(b),\quad a,b\in I, \quad \lambda\in(0,1).
	\]
	Then $f\in C^{0,1}_{\mathrm{loc}}(I,\mathbb R)$.
\end{proposition}

\begin{proof}
	Fix $a\in I$.
	If $a<b$, then we have by putting $\lambda\nearrow1$
	\[
		\limsup_{\lambda\nearrow1}f(\lambda a + (1-\lambda)b) = \limsup_{c\searrow a}f(c)\leq f(a),
	\]
	if $a>b$, then we have similarly
	\[
		\limsup_{\lambda\nearrow1}f(\lambda a + (1-\lambda)b) = \limsup_{c\nearrow a}f(c)\leq f(a).
	\]
	Particularly, using
	\begin{align*}
		f(a) &= f(\lambda(a-\varepsilon(1-\lambda)) + (1-\lambda)(a+\varepsilon\lambda))\\
			&\leq \lambda f(a-\varepsilon(1-\lambda)) + (1-\lambda)f(a+\varepsilon\lambda),
			\quad 0<\varepsilon\ll 1,
	\end{align*}
	we have by putting $\lambda\nearrow1$,
	\[
		f(a)\leq \liminf_{\lambda\nearrow1}\lambda f(a-\varepsilon(1-\lambda)) 
			+ \limsup_{\lambda\nearrow1}(1-\lambda)f(a+\varepsilon\lambda)
			= \liminf_{c\nearrow a}f(c),
	\]
	and by putting $\lambda\searrow0$,
	\[
		f(a)\leq \limsup_{\lambda\searrow0}\lambda f(a-\varepsilon(1-\lambda)) 
			+ \liminf_{\lambda\searrow0}(1-\lambda)f(a+\varepsilon\lambda)
			= \liminf_{c\searrow a}f(c).
	\]	
	Four estimates above imply the continuity of $f$ at $a$.

	Using 
	\[
	\frac{f(\lambda a + (1-\lambda)b) - f(a)}{\lambda a + (1-\lambda)b - a} \leq \frac{f(b) - f(a)}{b-a},\quad a < b,
	\]
	we have
	\[
	\limsup_{c\searrow a}\frac{f(c)-f(a)}{c-a} 
	= \limsup_{\lambda\nearrow 1}\frac{f(\lambda a + (1-\lambda)b) - f(a)}{\lambda a + (1-\lambda)b - a} 
	\leq \frac{f(b) - f(a)}{b-a},
	\]
	and hence 
	\[
	\limsup_{c\searrow a}\frac{f(c)-f(a)}{c-a} 
	\leq \liminf_{b\searrow a}\frac{f(b)-f(a)}{b-a},
	\]
	which implies $f'(a+)$ exists and satisfies 
	\[
	f'(a+)\leq \frac{f(b) - f(a)}{b-a},\quad b>a.
	\]
	If $a>c=b$, we similarly obtain $f'(a-)$ exists and satisfies 
	\[
	f'(a-)\geq \frac{f(c) - f(a)}{c-a},\quad c<a.
	\]
	Noting 
	\begin{align*}
	\frac{f(b) - f(a)}{b-a} - \frac{f(a) - f(c)}{a-c}
	&= \frac{f(b)(a-c) + f(c)(b-a) - f(a)(b-c)}{(b-a)(a-c)}\\
	&= \frac{(b-c)}{(b-a)(a-c)}\left(f(b)\frac{a-c}{b-c} + f(c)\frac{b-a}{b-c} - f(a) \right)\geq 0,
	\end{align*}
	we have $f'(a-)\leq f'(a+)$.
	Using 
	\[
	\frac{f(b) - f(\lambda a + (1-\lambda)b)}{b - \lambda a - (1-\lambda)b} \geq \frac{f(b) - f(a)}{b-a},\quad b>a,
	\]
	we obtain
	\[
	f'(b-) = \lim_{\lambda\searrow0}\frac{f(b) - f(\lambda a + (1-\lambda)b)}{b - \lambda a - (1-\lambda)b}\geq \frac{f(b) - f(a)}{b-a}\geq f'(a+).
	\]
	So $g(a) := f'(a-)$, $a\in I$ is a nondecreasing functions 
	which has at most countably many  discontinuous points.
	At its continuous point $p\in I$,
	\begin{align*}
		f'(p-) = g(p) = \lim_{a\searrow p}g(a) 
		= \lim_{a\searrow p} f'(a-) \geq \lim_{a\searrow p}f'(p+) = f'(p+),
	\end{align*} 
	which implies that $f'$ exists almost everywhere.
	Moreover, for any compact set $K\subset I$, 
	there exists an open interval $(c,b)$ such that $K\subset(c,b)$ and $\overline{(c,b)}\subset I$. 
	For any $a\in K$, if $f'(a)$ exists, 
	then  
	\[
	\frac{f(a) - f(c)}{a - c} 
	\leq f'(a) \leq \frac{f(b) - f(a)}{b - a}.
	\] 
	Therefore $f\in W^{1,\infty}_{\mathrm{loc}}(I) = C^{0,1}_{\mathrm{loc}}(I)$.
\end{proof}



\section{微分}
  \begin{example}
   已知$f(x)$可导,$f(x)+f'(x)\rightarrow a,x\rightarrow+\infty$,则$x\rightarrow+\infty$有$f(x)\rightarrow a,f'(x)\rightarrow0$. (清华大学, 2006)
  \end{example} 

  \begin{proof}
    Without loss of generality, we may assume that $a=0$, otherwise we consider $g=f-a$.
  For $\varepsilon>0$, there exists $R>0$ such that 
  \[
  |f(x)+f'(x)| < \varepsilon,\quad x>R.
  \]
  Then 
  \[
  -\varepsilon e^x < \left(e^xf\right)' < \varepsilon e^x,\quad x>R,
  \]
  which implies by integrating over $(R, x)$,
  \[
  -\varepsilon (e^x - e^R) < e^xf(x) - e^Rf(R) < \varepsilon(e^x-e^R),\quad x>R,
  \]
  thus,
  \[
  e^{R-x}f(R) - \varepsilon(1-e^{R-x}) < f(x) < e^{R-x}f(R) + \varepsilon(1-e^{R-x}), \quad x>R,
  \]
  and we get 
  \[
  \limsup_{x\to\infty}|f(x)| < \varepsilon.
  \]
  This concludes the proof.
  \end{proof}



  \begin{example}
  \hfill\\
   设$f(x)$,$g(x)$在$[a,b]$上连续,$(a,b)$内可导,$g'(x)\neq0$。证明:$\exists\xi\in(a,b)$,使$\frac{f(a)-f(\xi)}{g(\xi)-g(b)}=\frac{f'(\xi)}{g(\xi)}$(武汉大学,2004)
  
  记$F(x)=f(a)g(x)+g(b)f(x)-f(x)g(x)$,
  则$F(x)$在$[a,b]$上连续,在$(a,b)$内可导,又$F(a)=f(a)g(b)=F(b)$,从而由Largange中值定理:
  $\exists\xi\in(a,b),$$$\frac{F(a)-F(b)}{a-b}=F'(\xi)=0$$,即$$f(a)g'(\xi)+g(b)f'(\xi)-f'(\xi)g(\xi)-f(\xi)g'(\xi)=0.$$
  
  即$\frac{f(a)-f(\xi)}{g(\xi)-g(b)}=\frac{f'(\xi)}{g'(\xi)}$。
  
  考虑到$g'(x)\neq 0,$从而$g(\xi)\neq g(b)$,因为不然,若$g(\xi)=g(b),\xi\in(a,b)$,由微分中值定理,$\exists\eta\in(\xi,b),g'(\eta)=0$与$g'(x)\neq0$矛盾,故上式是有意义的。
  \end{example}
  \begin{example}
  \hfill\\
   设$f(x)$在$(a,b)$上可微,$f'(x)$在$(a,b)$上单调。求证:$f'(x)$在$(a,b)$上连续。(北京航空航天大学,2004)
  
  不妨设$f(x)$单调递增,从而$\forall c\in(a,b)$,$f'(x)$在$(a,c)$上单调递增有上界,$f(x)\leq f(c)$。由确界原理,$\exists S_c=\sup_{x\in(a,c)}f'(x)$,由确界定义与左极限定义易知:$f(c-)=S_c$。同理$f(c+)=I_c=\inf_{x\in(c,b)}f'(x)$。
  故单调的导函数不存在第二类间断点。
  
  又$f'(c-)\leq f'(c+)$,由达布引理,导函数具有介值性,若$f'(c-)<f'(c+)$,$\forall\eta\in[f'(c-),f'(c+)],\exists\xi\in[c-\varepsilon,c+\varepsilon],f'(\xi)=\eta.$而这样的$\xi$时不存在的,因为$f'(x)$单调。从而$f'(c-)=f'(c+)$,即导函数连续。
  \end{example}
  \begin{exercise}
  \hfill\\
 设$f(x)$在$[a,b](a<b)$上处处有正的二阶导数,证明存在常数$0<m\leq M$,使$\frac m4(x-a)^2\leq f(x)-2f(\frac{x+a}2)+f(a)\leq \frac M4(x-a)^2,x\in(a,b]$。(南京航空航天大学,2003) 
  
  设$$F(x)=
  \begin{cases}
  \frac{f(x)-2f(\frac{x+a}{2})+f(a)}{(x-a)^2},&x>a,\\
  \frac{f''(a)}{4},&x=a,\\
  \end{cases}
  $$
  则
  \begin{align*}
  F(a+)&=\lim_{x\rightarrow a+}\frac{f(x)-2f(\frac{x+a}{2})+f(a)}{(x-a)^2}\\
  &=\lim_{x\rightarrow a+}\frac{f'(x)-f'(\frac{x+a}{2}}{2(x-a)}\\
  &=\lim_{x\rightarrow a+}\frac{f''(x)}{2}-\frac{f''(\frac{x+a}{2})}{4}\\
  &=\frac{f''(a)}{4}\\
  &=F(a),\\
  \end{align*}
  从而$F(x)$在$[a,b]$上连续。
  
  下面证明$F(x)$在$[a,b]$上恒正。
 因为
 \begin{align*}
 &f(x)-2f(\frac{x+a}{2})+f(a)\\
 &=f'(\xi)(x-\frac{x+a}{2})-f'(\eta)(\frac{x+a}{2}-a)\\
 &=[f'(\xi)-f'(\eta)]\frac{x-a}{2}\\
 &=\frac{f''(\phi)}2(x-a)(\xi-\eta)\quad(x>a),\\
 \end{align*}
 其中$\xi\in(\frac{x+a}{2},x)$,$\eta\in(a,\frac{x+a}{2})$,$\phi\in(\eta,\xi)$,又$f''(x)$在$[a,b]$上恒正。
 故$F(x)$在$[a,b]$上恒正。$F(x)$在$[a,b]$上连续故闭存在最大值$\frac{M}{4}$和最小值$\frac{m}{4}$。这就完成了证明。  
  \end{exercise}  
  
  \hfill\\

  \section{微分中值定理及其应用}
  \begin{example}
  \hfill\\
  设$f(x)\in[0,1]$上可导,$f(1)=2\int xf(x)dx$。求证:存在$\xi\in(0,1)$,使得$f'(\xi)=-\frac{f(\xi)}{\xi}$。(清华大学,2003)
  
  
  反设$\forall\xi\in(0,1),f'(\xi)+\frac{f(\xi)}{\xi}=\frac{\xi f'(\xi)+f(\xi)}{\xi}=0$均不成立。
  
  从而$\xi f'(\xi)+f(\xi)$恒正或恒负(Darboux引理,导函数介值性)。不妨设
  $$\forall\xi\in(0,1),\xi f'(\xi)+f(\xi)>0.$$
  从而$f(x)x$在$(0,1)$上严格单调递增。从而
  \[
  \begin{aligned}
  f(1)&=2\int_0^{1/2}xf(x)dx\\
  &=\int_0^{1/2}xf(x)dx+\int_0^{1/2}xf(x)dx\\
  &<\int_0^{1/2}xf(x)dx+\int_{1/2}^1xf(x)dx\\
  &=\int_0^1xf(x)dx\\
  &<\int_0^1f(1)dx\\
  &=f(1).
  \end{aligned}
  \]
  矛盾。
  
  故$\exists\xi\xi(0,1),f'(\xi)\xi+f(\xi)=0$。
  \end{example}
  \begin{example}
  \hfill\\
  $x^y+y^x>1,x,y>0$.(中国科学院,2007)
  
  
  考虑$f(x)=(1+x)^r-rx-1,(x>0,r>1)$,
  
  $$f'(x)=r(1+x)^{r-1}-r>r(1+x)^0-r=0.$$
  故$f(x)>f(0)=0,$,得$(1+x)^r>rx+1.$
  $$(1+\frac{x}{y})^{\frac1x}>\frac1y+1>\frac1y,x=\frac xy,r=\frac1x.$$
  $$\Longrightarrow 1+\frac xy>\frac{1}{y^x}$$
  $$\Longrightarrow y^x>\frac{y}{x+y}$$
  同理$x^y>\frac{x}{x+y},$
  
  故$x^y+y^x>1,x,y\in(0,1).$ 
  \end{example}
  \begin{example}
  \hfill\\
   若$-\infty<a<b<c<+\infty,f(x)$在$[a,c]$上连续,且$f(x)$在$(a,c)$内二阶可导,求证:存在$\xi\in(a,c)$,使得(四川大学,2005)$$\frac{f(a)}{(a-b)(a-c)}+\frac{f(b)}{(b-c)(b-a)}+\frac{f(c)}{(c-a)(c-b)}=\frac12f^{''}(\xi)$$。
\end{example}
\begin{example}
\hfill\\
  已知$f(x)$在$[a,b]$上单调增加,$f(a)\geq a,f(b)\leq b$,求证:$\exists\xi\in[a,b]$,使得$f(\xi)=\xi$。(武汉大学,2006)
  
  
  反设$\forall\xi\in[a,b],f(\xi)\neq\xi,$则$\forall\xi\in[a,b],f(\xi)<\xi$或$f(\xi)>\xi$。
  记$A=\{\xi:f(\xi)>\xi\},B=\{\xi:f(\xi)>\xi\},$则由$f(a)>a,f(b)<b$,有$a\in A,b\in B$。于是$A,B$不空且$A\cup B=[a,b],A\cap B=\emptyset$。因为$B$有界,故$\eta=\inf B$存在且$\eta\in[a,b]$。
  
  若$\eta=a$,则存在子列$\{x_n\}\subset[a,b]$,且当$x_n\rightarrow a_+,n\rightarrow\infty$时有$f(x_n)<x_n$,从而对$\varepsilon=\frac{f(a)-a}{2},\exists N,\forall n>N,f(x_n)<x_n<a+\varepsilon<f(a)$与$f(x)$在$[a,b]$上单调增加矛盾,从而$\eta>a$,于是$\forall x\in[a,\eta),f(x)>x.$
  
  若$\eta\in A$即$f(\eta)>\eta$,同上矛盾。
  
  则$\eta\in B$,即$f(\eta)<\eta$,而$f(\eta-)=\lim_{x\rightarrow\eta-}f(x)\geq\eta>f(\eta)$,与$f(x)$单调增加矛盾。故$\eta\not\in B,\eta\not\in A$,则$f(\eta)=\eta$。
  \end{example}
  \begin{example}
  \hfill\\
   设$f(x)$在$[0,1]$上连续可微,且$f'(0)\neq0$,若对$\forall x\in(0,1),\theta(x)$满足$\int_0^xf(t)dt=f(\theta(x))x$,求证:$\displaystyle \lim_{x\rightarrow 0^+}\frac{\theta(x)}x=\frac12$.
   
     令$$F(x)=\int_0^xf(t)\mathrm{d}t,$$
  则有$F''(0)=f(0)\neq0$,用泰勒公式得
  $$F(x)=F(0)+f'(0)x+\frac{1}{2}F''(0)x^2+o(x^2)$$
  及
  $$F'(\theta)=F'(0)+F''(0)+o(\theta),$$
  由$F(0)=0$,$F(x)=f(\theta(x))x=F'(\theta)x$联立得
  $$F'(0)x+\frac{1}{2}F''(0)x^2+o(x^2)=x(F'(0)+F''(0)\theta+o(\theta)),$$
  消去$F'(0)x$并同时除$F''(0)x^2$得:
  $$\frac{1}{2}+o(1)=\frac{\theta}{x}+o(\frac{\theta}{x}=\frac{\theta}{x}(1+o(1)),x\rightarrow0+.$$
  即
  $$\frac{\theta}{x}=\frac{\frac{1}{2}+o(1)}{1+o(1)},x\rightarrow0+.$$
  即
  $$\lim_{x\rightarrow0+}\frac{\theta}{x}=\frac{1}{2}.$$
  \end{example}
  
  
 \hfill\\
 
  
  \section{定积分}
  \begin{example}
  \hfill\\
  设在任意的有限区间$[0,A]$上$f(x)$Riemann可积,且$\displaystyle\lim_{x\rightarrow+\infty}f(x)=0$,求证:$\displaystyle\lim_{t\rightarrow+\infty}\frac 1t\int_0^t|f(x)|dx=0$(清华大学,2006)
  
  
  因为$\lim_{x\rightarrow\infty}f(x)=0,$所以$$\forall\varepsilon>0,\exists A>0,\forall x>A,|f(x)|<\frac{\varepsilon}{2},$$
  又$f(x)$在$[0,A]$上Riemann可积,从而$f(x)$在$[0,A]$上有界,即$$\exists M>0,\forall x\in[0,A],|f(x)|<M.$$
  于是对上述$\varepsilon>0,\exists T=\max\{\frac{2MA}{\varepsilon},A\},\forall t>T,$
  $$
  \begin{aligned}
  \frac{1}{t}\int_0^t|f(x)|dx&=\frac1t\int_0^A|f(x)|dx+\frac1t\int_A^t|f(x)|dx\\
  &<\frac{MA}{t}+\frac{t-A}{t}\cdot\frac{\varepsilon}{2}\\
  &<\frac{\varepsilon}{2}+\frac{\varepsilon}{2}=\varepsilon\\
  \end{aligned}
  $$
  从而$$\lim_{t\rightarrow\infty}\frac{1}{t}\int_0^t|f(x)|dx=0.$$
  \end{example}
  \begin{example}
  \hfill\\
  $\int_0^{2\pi}\frac x{1+cos^2x}dx$(浙江大学,2004)
  
  
  \begin{equation}
  \begin{aligned}
    &\int_0^{2\pi }\frac{x dx}{1+\cos^2x}\\&=\int_0^{\pi}\frac{x dx}{1+\cos^2x}+\int_0^{\pi}\frac{x+\pi dx}{1+\cos^(x+\pi)}\\
    &=2\int_0^{\pi}\frac{xdx}{1+\cos^2x}+\int_0^{\pi}\frac{\pi dx}{1+\cos^2x}\\
    &=2[\int_0^{\frac{\pi}{2}}\frac{xdx}{1+\cos^2x}+\int_0^{\frac{\pi}{2}}\frac{x+\frac{\pi}{2}dx}{1+\cos^2(x+\frac{\pi}{2})}]+\\
    &\int_0^{\frac{\pi}{2}}\frac{\pi dx}{1+\cos^2x}+\int_0^{\frac{\pi}{2}}\frac{\pi dx}{1+\cos^2}\\
    &=2[\int_{-\frac{\pi}{2}}^0\frac{x+\frac{\pi}2dx}{1+\cos^2(x+\frac{\pi}{2})}+\int_0^{\frac{\pi}{2}}\frac{xdx}{1+\sin^x}+\frac{\pi}{2}\int_0^{\frac{\pi}{2}}\frac{dx}{1+\sin^x}]\\
    &+\pi[\int_0^{\frac{\pi}{2}}\frac{dx}{1+\cos^2x}+\int_0^{\frac{\pi}{2}}\frac{dx}{1+\sin^2x}]\\
    &=2[\int_{-\frac{\pi}{2}}^0\frac{xdx}{1+\sin^2x}+\frac{\pi}{2}\int_{-\frac{\pi}{2}}^0\frac{dx}{1+\sin^2x}\\
    &+\int_0^{\frac{\pi}{2}}\frac{xdx}{1+\sin^2x}+\frac{\pi}{2}\int_0^{\frac{\pi}{2}}\frac{dx}{1+\sin^2x}]\\
    &+2\pi\int_0^{\frac{\pi}{2}}\frac{dx}{1+\cos^2x}\\
    &=2[\int_{-\frac{\pi}{2}}^0\frac{xdx}{1+\sin^2x}+\frac{\pi}{2}\int_{\frac{\pi}{2}}^0\frac{d(-x)}{1+\sin^2x}\\
    &+\int_0^{\frac{\pi}{2}}\frac{xdx}{1+\sin^2x}+\frac{\pi}{2}\int_0^{\frac{\pi}{2}}\frac{dx}{1+\sin^2x}]\\
    &+2\pi\int_0^{\frac{\pi}{2}}\frac{dx}{2\cos^2x+\sin^2x}\\
    &=2\pi\int_0^{\frac{\pi}{2}}\frac{d\tan x}{\tan^2x+2}\\
    &=\sqrt{2}\pi(\arctan(\frac{\tan x}{\sqrt{2}}))|_0^{\frac{\pi}{2}}\\
    &=\frac{\pi^2}{\sqrt{2}}.\\
  \end{aligned}
  \end{equation}
  \end{example}
  \begin{example}
  \hfill\\
  若$f(x)$在$(0,+\infty)$上可微,$\displaystyle\lim_{x\rightarrow+\infty}\frac{f(x)}x=0$,求证:$(0,+\infty)$内存在一个单调数列${\xi_n}$,使得$\displaystyle\lim_{n\rightarrow\infty}\xi_n=+\infty$且$\displaystyle\lim_{n\rightarrow\infty}f'(\xi_n)=0$。 (浙江师范大学,2005)
  
  
  $\lim_{x\rightarrow\infty}\frac{f(x)}{x}\Longrightarrow\forall\varepsilon>0,\exists N,\forall m,n>N$,有$$|\frac{f(m)}{m}-\frac{f(n)}{n}|<\frac{\varepsilon}{2},|\frac{f(n)}n|<\frac{\varepsilon}{2}.$$
  
  于是取$m=2n$,有$|\frac{f(2n)-f(n)}{n}|=|\frac{f(2n)}{2n}-\frac{f(n)}{n}+\frac{f(2n)}{2n}|<\frac{\varepsilon}{2}+\frac{\varepsilon}{2}=\varepsilon.$
  
  取$\{\xi_n\}$,满足$f'(\xi_n)=\frac{f(2^{n+1})-f(2^n)}{2^n},\xi_n\in(2^n,2^{n+1}).$
  
  显见$\xi_n\rightarrow\infty,n\rightarrow\infty.$对上述$\varepsilon>0,\exists N_1>\max\{\log_2N,1\}$时有$|f'(\xi_n)|<\varepsilon$,从而$f'(\xi_n)\rightarrow0,(n\rightarrow\infty)$。
  \end{example}
  \begin{example}
  \hfill\\
  设函数$f(x)$在$[a,b]$上连续且严格递增,证明下式成立$$\int_a^bf(x)dx=bf(b)-af(a)-\int_{f(a)}^{f(b)}f^{-1}(x)dx.$$(河海大学,2006)

对$a,b$中插入$n$个分点$$a=x_0<x_1<x_2<\cdots<x_{n-1}<x_n=b,$$
从而对应$x=f^{-1}(y)$也有$n$个分点:
$$f(a)=f(x_0)<f(x_1)<f(x_2)<\cdots<f(x_{n-1})<f(x_n)=f(b).$$
记$\Delta_n=\max\{x_i-x_{i-1}:i=1,2,\cdots,n\},$因为$f(x)$连续,故
$$\Delta_n\rightarrow0,\max\{f(x_i)-f(x_{i-1}):i=1,2,\cdots,n\}\rightarrow0,n\rightarrow\infty.$$
从而
\begin{align*}
\int_a^bf(x)\mathrm{d}x+\int_{f(a)}^{f(b)}f^{-1}(x)\mathrm{d}x&=\lim_{\Delta_n\rightarrow0}\sum_{i=1}^nf(x_i)(x_i-x_{i-1})\\
&\qquad+\lim_{\Delta_n\rightarrow0}\sum_{i=1}^nx_{i-1}[f(x_i)-f(x_{i-1})]\\
&=\lim_{\Delta_n\rightarrow0}\sum_{i=1}^nf(x_i)x_i-f(x_{i-1})x_{i-1}\\
&=f(b)b-f(a)a.\\
\end{align*}
\end{example}
\begin{example}
\hfill\\
计算积分$\int_0^1\frac{\ln(1+x)}{1+x^2}dx$。(南京航空航天大学,2005)
  
  \begin{align*}
  \int_0^1\frac{\ln(1+x)}{1+x^2}dx&=\int_0^{\frac{\pi}{4}}\frac{\ln(1+\tan t)}{1+\tan^2t}d\tan t\\
  &=\int_0^{\frac{\pi}{4}}\ln(1+\tan t)dt\\
  &=\int_0^{\frac{\pi}{4}}\ln(\sin t+\cos t)dt-\int_0^{\frac{\pi}{4}}\ln\cos tdt\\
  &=\frac{\pi}{8}\ln2+\int_0^{\frac{\pi}{4}}\ln\cos(t-\frac{\pi}{4})-\int_0^{\frac{\pi}{4}}\ln\cos tdt\\
  &=\frac{\pi}{8}\ln2.\\
  \end{align*}
\end{example}
\begin{example}
  \hfill\\
   利用可积条件证明:函数$f(x)=\frac1x-[\frac1x],x\neq0;f(x)=0,x=0$在$[0,1]$上可积。(南京师范大学,2006)注:此题函数为分段函数,标准书写格式为大括号括起来书写。
   
     $0\leq f(x)\leq1$且$f(x)$在$[0,1]$上的不连续点为$$x=\frac{1}{2},\frac{1}{3},\cdots,\frac{1}{n},\cdots$$与$x=0$。$\forall\varepsilon>0$,取定$m>\frac{2}{\varepsilon}$,$f(x)$在区间$[\frac{1}{m},1]$上只有有限个不连续点,因此,$f(x)$在$[\frac{1}{m},1]$上可积,即存在$[\frac{1}{m},1]$的一个划分$P$,使得
  $$\sum_{i=1}^nw_i\Delta x_i<\frac{\varepsilon}{2},$$
  将$P$的分点和$0$合在一起,作为$[0,1]$的划分$P'$,则
  $$\sum_{i=1}^{n+1}w_i'\Delta x_i'=\sum_{i=1}^nw_i\Delta x_i+w_i'\Delta x_i'<\frac{\varepsilon}{2}+\frac{\varepsilon}{2}=\varepsilon.$$
  因此,$f(x)$在$[0,1]$上可积。
\end{example}
\begin{example}
\hfill\\
 设$f(x)\in C^1[0,1]$,试证明:$$\displaystyle\lim_{n\rightarrow\infty}n[\int_0^1f(t)dt-\frac1n\sum_{k=1}^{n-1}f(\frac kn)]=\frac{f(1)-f(0)}2.$$(南京大学,2005)
 
 由泰勒公式可得
$$f(t)=f(\frac{k}{n})+f'(\xi_k)(t-\frac{k}{n}),$$
其中$\xi_k$介于$t$和$\frac{k}{n}$之间,$k=0,1,2,\cdots,n-1$,则
\begin{align*}
\int_0^1f(t)\mathrm{d}t-\frac{1}{n}\sum_{k=0}^{n-1}f(\frac{k}{n})
&=\sum_{k=0}^{n-1}\int_{\frac{k}{n}}^{\frac{k+1}{n}}[f(t)-f(\frac{k}{n})\mathrm{d}t\\
&=\sum_{k=0}^{n-1}\int_{\frac{k}{n}}^{\frac{k+1}{n}}[f'(\xi_k)(t-\frac{k}{n})\mathrm{d}t\\
&=\sum_{k=0}^{n-1}\frac{1}{2n^2}f'(\xi_k)
\end{align*}
由于$f(t)\in C^1[0,1]$,进一步可得
$$f(1)-f(0)=\int_0^1f'(t)\mathrm{d}t=\lim_{n\rightarrow\infty}\sum_{k=0}^{n-1}\frac{1}{n}f'(\xi_k).$$

这样就完成了证明。
\end{example}
\begin{example}
\hfill\\
 求$\displaystyle\lim_{n\rightarrow\infty}\sum_{k=1}^n\sin\frac k{n^2}$.

  \begin{equation*}
  \begin{aligned}
  e^{\int_0^1\ln f(x)dx}&=\displaystyle e^{\lim_{n\rightarrow\infty}\frac1n\sum_{i=1}^n\ln f(\frac kn)}\\
  &=e^{\lim_{n\rightarrow\infty}[\ln(\prod_{i=1}^nf(\frac kn))]^{\frac1n}}\\
  &=\lim_{n\rightarrow\infty}[\prod_{i=1}^nf(\frac kn)]^{\frac1n}\\
  &\leq\lim_{n\rightarrow\infty}\frac1n\sum_{i=1}^nf(\frac kn)\\
  &=\int_0^1f(x)dx\\
  \end{aligned}
  \end{equation*}
%  $\displaystyle e^{\int_0^1\ln f(x)dx}=\displaystyle e^{\lim_{n\rightarrow\infty}\frac1n\sum_{i=1}^n\ln f(\frac kn)}=e^{\lim_{n\rightarrow\infty}[\ln(\prod_{i=1}^nf(\frac kn))]^{\frac1n}}=\lim_{n\rightarrow\infty}[\prod_{i=1}^nf(\frac kn)]^{\frac1n}\leq\lim_{n\rightarrow\infty}\frac1n\sum_{i=1}^nf(\frac kn)=\int_0^1f(x)dx$.
\end{example}
\begin{example}
\hfill\\
 设$f(x)$在$[a,b]$上可积,求证:$$\displaystyle\lim_{p\rightarrow+\infty}\int_a^bf(x)\sin pxdx=0,\lim_{p\rightarrow+\infty}\int_a^bf(x)\cos psdx=0.$$
 
 对任意有界区间$[\alpha,\beta]$有$$|\int_{\alpha}^{\beta}\sin px\mathrm{d}x|=|\frac{\cos p\alpha-\cos p\beta}{p}|\leq\frac{2}{p}.$$
设在$[a,b]$上,$|f(x)|\leq M$,任给$\varepsilon>0$,存在$[a,b]$的分割$T$:
$$a=x_0<x_1<x_2<\cdots<x_{n-1}<x_n=b,$$
使得
$$S(T,f)-s(T,f)<\frac{\varepsilon}{2},$$
其中$S(T,f)$与$s(T,f)$分别代表$f(x)$关于$T$的大和和小和,于是当$p\geq\frac{4nM}{\varepsilon}$,有
\begin{align*}
|\int_a^bf(x)\sin px\mathrm{d}x|&=
|\sum_{k=1}^n\int_{x_{k-1}}^{x_k}(f(x_k)+f(x)-f(x_k))\sin px\mathrm{d}x|\\
&\leq\sum_{k=1}^n(|f(x_k)||\int_{x_{k-1}}^{x_k}\sin px\mathrm{d}x|+\int_{x_{k-1}}^{x_k}|f(x)-f(x_k)||\sin px|\mathrm{d}x)\\
&<(S(T,f)-s(T,f))+\frac{2Mn}{p}\\
&<\varepsilon.\\
\end{align*}
同理可证后者极限等式成立。
\end{example}
\begin{example}
\hfill\\
 设$f(x)$在$[A,B]$上连续,$\phi(x)$在$[a,b]$上可积,而且$\phi([a,b])\subset[A,B]$,求证:$f(\phi(x))$在$[a,b]$上可积。
 
   设$|f(n)|\leq M$,$\forall u\in[A,B]$。由$f(u)$在$[A,B]$上连续可知,$\forall\varepsilon>0$,$\exists\delta>0$,当$u',u''\in[A,B]$且$|u'-u''|<\delta$时,有
  $$|f'(u')-f(u'')|<\frac{\varepsilon}{2(b-a)}.$$
  由$\phi(x)$在$[a,b]$上可积可知,$\exists\eta>0$对一切分割
  $$T:a=x_0<x_1<x_2<\cdots<x_n=b,$$
  只要$\|T\|<\eta$时,有
  $$\sum_{k=1}^nw_k(\phi)\Delta x_k<\frac{\delta\varepsilon}{4M},$$
  其中$$w_k(\phi)=\sup_{x_k\leq x\leq x_{k+1}}\phi(x)-\inf_{x_k\leq x\leq x_{k+1}}\phi(x).$$
  将 $\sum\limits_{k=1}^nw_k(\phi)\Delta x_k$分成$\sum\limits'$与$\sum\limits''$在$\sum\limits'$中有$w_k(\phi)\geq\delta$,在$\sum\limits''$中有$w_k(\phi)<\delta$,则
  $$\delta\sum'\Delta x_k=\sum'\delta\Delta x_k\leq\sum_{k=1}^nw_k(\phi)\Delta x_k\leq\frac{\delta\varepsilon}{4M}.$$
  即$$\sum'\Delta x_k<\frac{\varepsilon}{4M}.$$
  综上所述,可得
  \begin{align*}
  \sum_{k=1}^nw_k(f(\phi))\Delta x_k&=\sum'w_k(f(\phi))\Delta x_k+\sum''w_k(f(\phi))\Delta x_k\\
  &\leq2M\sum'\Delta x_k+\frac{\varepsilon\sum''\Delta x_k}{2(b-a)}\\
  &<\varepsilon.\\
  \end{align*}
  这就证得了$f(\phi(x))$在$[a,b]$上可积。
\end{example}  
 \hfill\\
 
    \section{反常积分}
    \begin{example}
    \hfill\\
    $\int_0^{\frac {\pi}2}\ln\sin xdx$
    
    
    \begin{equation}
  \begin{aligned}
  &\int_0^{\frac{\pi}{2}}\ln\sin xdx\\
  &=\int_0^{\frac{\pi}{2}}(\ln2+\ln\cos\frac{x}{2}+\ln\sin\frac{x}{2})dx\\
  &=2\int_0^{\frac{\pi}{4}}(\ln2+\ln\cos x+\ln\sin x)dx\\
  &=\frac{\pi\ln2}{2}+2\int_0^{\frac{\pi}{4}}\ln\cos xdx+2\int_0^{\frac{\pi}{4}}\sin xdx\\
  &=\frac{\pi\ln2}{2}-2\int_0^{-\frac{\pi}{4}}\ln\cos xdx+2\int_0^{\frac{\pi}{4}}\sin xdx\\
  &=\frac{\pi\ln2}{2}-2\int_{\frac{\pi}{2}}^{\frac{\pi}{4}}\ln\cos(x-\frac{\pi}{2})d(x-\frac{\pi}{2})+2\int_0^{\frac{\pi}{4}}\sin xdx\\
  &=\frac{\pi\ln2}{2}+2\int_0^{\frac{\pi}{2}}\ln\sin xdx\\
  &=-\frac{\pi\ln2}{2}\\
  \end{aligned}
  \end{equation}
    
    \end{example}
  
  
    \begin{example}
    \hfill\\
    设$a>0$,函数$f(x)$在$[0,a]$上连续可微,证明:$|f(0)|\leq\frac1a\int_0^a|f(x)|dx+\int_0^a|f'(x)|dx$
    
    
    $f(x)$在$[0,a]$上连续,由积分第一中值定理:
  
  存在$\xi\in(0,a)$使$\int_0^af(x)dx=f(\xi)(a-0)=f(\xi)a.$
  
  由于$f(0)=f(\xi)-\int_0^{\xi}f'(x)dx$,
  
  可得
  \begin{equation}
  \begin{aligned}
  |f(0)|&=|f(\xi)-\int_0^{\xi}f'(x)dx|\\
  &\leq|f(\xi)|+|\int_0^{\xi}f'(x)dx|\\
  &\leq\frac{1}{a}|\int_0^af(x)dx|+|\int_0^{\xi}f'(x)dx|\\
  &\leq\frac{0}{a}|f(x)|dx+\int_0^a|f'(x)|dx\\
  \end{aligned}
  \end{equation}
    
    \end{example}
  
  
     \begin{example}
    \hfill\\
    证明:$\int_0^1\frac{\ln\frac1x}{1-x}dx=\frac{\pi^2}6$。(复旦大学,2001)
    
    
    \begin{equation}
  \begin{aligned}
  \int_0^1\frac{\ln\frac1x}{1-x}dx&=-\int_0^1\frac{\ln(1-x)}{x}dx\\
  &=\int_0^1\sum_{n=1}^{\infty}\frac{x^{n-1}}{n}dx\\
  &=\int_0^1\sum_{n=0}^{\infty}\frac{x^n}{n+1}dx\\
  &=\sum_{n=0}^{\infty}\int_0^1\frac{x^n}{n+1}dx\\
  &=\sum_{n=0}^{\infty}\frac{1}{(n+1)^2}\\
  &=\sum_{n=1}^{\infty}\frac{1}{n^2}\\  
  &=\frac{\pi^2}6
  \end{aligned}
  幂级数在收敛域内可交换积分与求和次序。
  \end{equation}
    
    \end{example}
  
    \begin{example}
    \hfill\\
    计算$\int_0^{+\infty}\frac{\sin x}xdx$(浙江大学,2005)
  考虑二重积分$$\int_0^{+\infty}\int_0^{+\infty}e^{-xy}\sin xdxdy,$$
    
    
    $\int_0^{+\infty}e^{-xy}\sin xdx$关于$y$在$(0,+\infty)$上内闭一致收敛。从而
  \begin{equation}
  \begin{aligned}
  \int_0^{+\infty}\int_0^{+\infty}e^{-xy}\sin xdxdy&=\int_0^{+\infty}\sin xdx\int_0^{+\infty}e^{-xy}dy\\
  &=\int_0^{+\infty}\sin xdx(-\frac{e^{-xy}}{x}|_0^{+\infty}\\
  &=\int_0^{+\infty}\frac{\sin x}xdx,\\
  \end{aligned}
  \end{equation}
  又
  \begin{equation}
  \begin{aligned}
  &\int_0^{+\infty}\int_0^{+\infty}e^{-xy}\sin xdxdy\\
  &=\int_0^{+\infty}dy\int_0^{+\infty}e^{-xy}\sin xdx\\
  &=\int_0^{+\infty}dy[(-\frac{e^{-xy}}{y}\sin x)|_0^{+\infty}+\int_0^{+\infty}\frac{e^{-xy}\cos x}{y}dx]\\
  &=\int_0^{+\infty}dy[(-\frac{e^{-xy}\sin x}{y}|_0^{+\infty}-\int_0^{+\infty}\frac{e^{-xy}\sin x}{y}]\\
  &=\int_0^{+\infty}dy[\frac{1}{y^2}-\frac{1}{y^2}\int_0^{+\infty}e^{-xy}\sin xdx]\\
  &=\int_0^{+\infty}\frac{dy}{1+y^2}\\
  &=\frac{\pi}{2}
  \end{aligned}
  \end{equation}
    
    \end{example}  
    \begin{example}
    \hfill\\
    讨论不同$p$对$f(x)$在$[1,+\infty)$上积分的敛散性,$$\displaystyle f(x)=\ln(1+\frac{\sin x}{x^p}).$$(北京大学,2007)
    
    
    \begin{align*}
  \overline{\lim_{x\rightarrow\infty}}x^p|\ln(1+\frac{\sin x}{x^p})|
  &=\lim_{x\rightarrow\infty}x^p\ln(1-\frac{1}{x^p})^{-1}\\
  &=\lim_{x\rightarrow\infty}x^p\ln(1+\frac{1}{x^p-1})\\
  &=\left\{\begin{array}{ll}1,&p>0\\\ln2,&p=0\\0,&p<0.\end{array}\right.  
  \end{align*}

  从而当$p>1$时,$f(x)$绝对收敛;当$p\leq0$时,$f(x)$发散;当$p=1$时,$f(x)=\ln(1+\frac{\sin x}{x})$,考虑到

  \begin{align*}
  &\int_{2n\pi}^{2(n+1)\pi}\ln(1+\frac{\sin x}{x})dx\\&=\int_{2n\pi}^{(2n+1)\pi}\ln(1+\frac{\sin x}{x})dx+\int_{(2n+1)\pi}^{2(n+1)\pi}\ln(1+\frac{\sin x}{x}dx\\
  &=\int_{2n\pi}^{(2n+1)\pi}\ln(1+\frac{\sin x}{x})dx+\int_{2n\pi}^{(2n+1)\pi}\ln(1-\frac{\sin x}{x+\pi})dx\\
  &=\int_{2n\pi}^{(2n+1)\pi}\ln(1+\frac{\pi\sin x}{x(x+\pi)}-\frac{\sin^2x}{(x+\pi)x})dx\\
  &=\int_{2n\pi}^{(2n+1)\pi}\ln(1+\frac{(\pi-1)\sin x}{x(x+\pi)})dx\\
  &\leq\int_{2n\pi}^{(2n+1)\pi}\ln(1+\frac{\pi}{x(x+\pi)})dx\\
  &<\int_{2n\pi}^{(2n+1)\pi}(\frac{1}{x}-\frac{1}{x+\pi}dx\\
  &=\ln\frac{(2n+1)^2}{(2n+1)(2n)}\\
  &=\ln\frac{4n^2+4n+1}{4n^2+4n}\\
  &<\frac{1}{4n^2}\\
  \end{align*}

  
  从而$0<\int_{2\pi}^{2n\pi}\ln(1+\frac{\sin x}{x})dx<\frac{\pi^2}{24}$。
  
  而$\forall A>0,\exists n\in N_+$,有$A\in[2n\pi,2(n+1)\pi]$,从而!!!!!!
  
  
  
  !!
  
  @@
  
  未完成!!
    
    \end{example}  
\hfill\\
    
 \section{数项级数}
     \begin{example}
    \hfill\\
    利用数项级数$\displaystyle\sum_{n=1}^{\infty}\frac 1{n^2}=\frac{\pi^2}6$计算积分$I=\int_0^1\frac{\ln(x+1)}xdx$。(三峡大学,2006;厦门大学,2000)
    
    
     \begin{equation}
  \begin{aligned}
  I&=\int_0^1\frac{\ln(1+x)}{x}dx\\
  &=\int_0^1\sum_{n=1}^{\infty}\frac{(-1)^{n+1}x^{n-1}}{n}dx\\
  &=\sum_{n=1}^{\infty}\int_0^1\frac{(-1)^{n+1}x^{n-1}}{n}dx\\
  &=\sum_{n=1}^{\infty}\frac{(-1)^{n+1}}{n^2}dx\\
  &=\sum_{n=1}^{\infty}\frac{1}{(2n-1)^2}-\sum_{n=1}^{\infty}\frac{1}{(2n)^2}\\
  &=\sum_{n=1}^{\infty}\frac{1}{n^2}-\frac{1}{2}\sum_{n=1}^{\infty}\frac{1}{n^2}\\
  &=\frac{1}{2}\sum_{n=1}^{\infty}\frac{1}{n^2}=\frac{\pi^2}{12}.
  \end{aligned}
  \end{equation} 
    
    \end{example} 
    \begin{example}
    \hfill\\
    设$\displaystyle a_n>0,n=1,2,\cdot\cdot\cdot\lim_{n\rightarrow\infty}n(\frac{a_n}{a_{n-1}}-1)=c>0$。证明:$\displaystyle\sum_{n=1}^{\infty}(-1)^{n+1}a_n$收敛。(大连理工大学,2004)  
    
    
   由$\displaystyle a_n>0,n=1,2,\cdot\cdot\cdot\lim_{n\rightarrow\infty}n(\frac{a_n}{a_{n-1}}-1)=c>0$可知,当$n$充分大时,有$a_n>a_{n+1}$。取$a>0,b>0$使得$c>b>a>0$,当$n$充分大时,可得$$\frac{a_n}{a_{n+1}}>1+\frac{b}{n}>(1+\frac{1}{n})^{\alpha}=\frac{(n+1)^{\alpha}}{n^{\alpha}},$$
  从而得到$(n+1)^{\alpha}a_{n+1}<n^{\alpha}a_n$,即数列$\{n^{\alpha}a_n\}$对较大的$n$单调递减。从而存在$M>0$,使得$n^{\alpha}a_n\leq M$,即$$0<a_n<\frac{M}{n^{\alpha}}.$$
  由夹逼定理可知数列$\{a_n\}$趋于0,$\sum_{n=1}^{\infty}(-1)^{n+1}a_n$时Lebniz级数,因此$\sum_{n=1}^{\infty}(-1)^{n+1}a_n$收敛。   
    
    \end{example}  
    \begin{example}
    \hfill\\
  利用幂级数展开以及公式$\displaystyle\sum_{n=1}^{\infty}\frac1{n^2}=\frac{\pi^2}6$计算$\int_0^1\frac{\ln x}{1-x^2}dx$。    
    
    
  \begin{equation}
  \begin{aligned}
  \int_0^1\frac{\ln x}{1-x^2}dx&=\int_0^1\sum_{n=0}^{\infty}x^{2n}\ln xdx\\
  &=\sum_{n=0}^{\infty}\int_0^1x^{2n}\ln xdx\\
  &=\sum_{n=0}^{\infty}((\frac{x^{2n+1}}{2n+1}\ln x)|_0^1-\int_0^1\frac{x^{2n}}{2n+1}dx)\\
  &=\sum_{n=0}^{\infty}-\frac{1}{(2n+1)^2}\\
  &=\sum_{n=1}^{\infty}-\frac{1}{(2n-1)^2}-\sum_{n=1}^{\infty}-\frac{1}{(2n)^2}+\sum_{n=1}^{\infty}\frac{1}{n^2}+\frac{1}{(2n)^2}+\sum_{n=1}^{\infty}\frac{1}{n^2}\\
  &=-\sum_{n=1}^{\infty}\frac{1}{n^2}+\frac{1}{4}\sum_{n=1}^{\infty}\frac{1}{n^2}\\
  &=-\frac{\pi^2}6+\frac{\pi^2}{24}\\
  &=-\frac{\pi^2}{12}\\
  \end{aligned}
  \end{equation}   
    
    \end{example}  
    \begin{example}
    \hfill\\
  $\displaystyle\sum_{n=0}^{\infty}q^n\cos n\theta(|q|<1)$.    
    
    
  \begin{align*}
  \sum_{n=0}^{\infty}(qe^{i\theta})^n&=\frac{1}{1-qe^{i\theta}}=\frac{1-qe^{-i\theta}}{1-2q\cos\theta+q^2}\\
  &=\frac{1-q\cos\theta}{1-2q\cos\theta+q^2}+i\frac{q\sin\theta}{1-2q\cos\theta+q^2}\\
  &=\sum_{n=0}^{\infty}q^n\cos n\theta+i\sum_{n=0}^{\infty}q^n\sin n\theta\\
  \end{align*}    
    
    \end{example}  
  
\begin{example}
\[
\sum_{n=0}^{\infty}(1+\frac12+\cdot\cdot\cdot\frac1n)x^n
=\left(\sum_{n=0}^{\infty}x^n\right)\left(\sum_{n=1}^{\infty}\frac{x^n}n\right)
=\frac1{1-x}\ln\frac1{1-x}
\]
\end{example} 

  
\section{函数项级数}
\begin{exercise}
\hfill\\
  设函数列${a_n(x)}$在$[a,b]$上可导,且存在$M>0$,使对任意正整数$n$和$x\in[a,b]$,有$\displaystyle|\sum_{k=1}^na_k(x)|\leq M$ 成立,证明:如果级数$\displaystyle\sum_{n=1}^{\infty}a_n(x)$在$[a,b]$收敛,则必一致收敛。(大连理工大学,2006)
  
  对区间$[a,b]$作如下分割:
  $$a=x_0<x_1<\cdots x_{m-1}<x_m=b,$$
  使得$[a,b]$被分割成$m$个小区间$\Delta_i=[x_{i-1},x_i],i=1,2,\cdots m,$且$\Delta_i=x_i-x_{i-1}<\frac{\varepsilon}{2M}$。
  因为级数$\sum_{n=1}^{\infty}a_n(x)$在$[a,b]$上收敛,所以对$\Delta_i$上任意一点$\overline{x}_i$,存在$N_i>0$,使得$\forall m,n>N_i$,有$$|\sum_n^ma_n(\overline{x}_i)|<\frac{\varepsilon}{2}(\overline{x}_i\in\Delta_i).$$
  对函数$\sum_n^ma_n(x)$应用微分中值定理可知:
  
  对任意$x\in\Delta_i$,存在$\phi$介于$x$与$\overline{x}_i$之间,有$$|\sum_n^ma_n(x)-\sum_n^ma_n(\overline{x}_i|=|\sum_n^ma'_n(\phi)||x-\overline{x}_i|<M\cdot\frac{\varepsilon}{2M}=\frac{\varepsilon}{2}.$$
  
  于是$$|\sum_n^ma_n(x)|\leq|\sum_n^ma_n(x)-\sum_n^ma_n(\overline{x}_i|+|\sum_n^ma_n(\overline{x}_i|<\varepsilon.$$
  即$\forall\varepsilon>0,\exists N=\max\{N_1,N_2,\cdots N_m\},\forall m,n>N,\forall x\in[a,b],$$$|\sum_n^ma_n(x)|<\varepsilon.$$
  即$\sum_n^ma_n(x)$在$[a,b]$上一致收敛。
  \end{exercise}  
  \begin{exercise}
  \hfill\\
    设对每个自然数n,$f_n(x)$为$[a,b]$上的单调函数,${f_n(x)}$收敛于连续函数$f(x)$,求证${f_n(x)}$必在$[a,b]$上一致收敛。
  
  对$x_i\in(a,b)$,$\forall\varepsilon>0$,$\exists\delta>0$,$\forall x\in B(x_i,\delta)$,
  $$|f(x)-f(x_i)|<\frac{\varepsilon}3.$$
  $\exists N>0$,$\forall n>N$,
  $$|f_n(x_i)-f(x_i)|<\frac{\varepsilon}{3},$$
  $$|f_n(x_i-\frac{\delta}{2})-f(x_i-\frac{\delta}{2})|<\frac{\varepsilon}{3},$$
  $$|f_n(x_i+\frac{\delta}{2})-f(x_i+\frac{\delta}{2})|<\frac{\varepsilon}{3},$$
  于是$\forall x\in B(x_i,\frac{\delta}{2}),$
  \begin{align*}
  |f_n(x)-f(x)|&=|f_n(x)-f(x_i)+f(x_i)-f(x)|\\
  &\leq|f_n(x)-f(x_i)|+|f(x_i)-f(x)|\\
  &\leq\max\{|f_n(x_i-\frac{\delta}{2})-f(x_i)|,|f_n(x_i+\frac{\delta}{2})-f(x_i)|\}+\frac{\varepsilon}{3}\\
  &\leq\varepsilon.\\
  \end{align*}
  即对$x_i\in(a,b)$,$\forall\varepsilon>0,$ $\exists\delta_i>0,N_i>0$,$\forall n>N_i$,$x\in B(x_i,\frac{\delta}{2})$,
  $$|f_n(x)-f(x)|<\varepsilon.$$
  显然$\cup_{x_i\in(a,b)}B(x_i,\frac{\delta_i}{2})$是$[a,b]$的一个开覆盖。于是必有有限子覆盖不妨记为$\{B(x_i,\frac{\delta_i}{2}):i=1,2,\cdots,m\}$覆盖$[a,b]$。于是取$N=\max\{N_1,N_2,\cdots,N_m\}$。有$\forall\varepsilon>0,\exists N>0,\forall n>N,\forall x\in[a,b],|f_n(x)-f(x)|<\varepsilon.$
  \end{exercise}
  \hfill\\
    \section{Euclid空间上的极限与连续}
    \begin{exercise}
    \hfill\\
  设$f(x)$是$R^n$上的连续函数,满足$lim_{|x|\rightarrow+\infty}f(x)=+\infty$,其中$x=(x_1,x_2,\cdot\cdot\cdot,x_n),|x|=(x_1^2+x_2^2+\cdot\cdot\cdot x_n^2)^{\frac12}$。证明一定存在$x_0\in R^n$使得$f(x_0)=\inf_{x\in R^n}f(x)$。(兰州大学,2006)
	
	取$x'=(x_1',x_2',\cdots x_n'\in R^n$,因为$\lim_{|x|\rightarrow+\infty}f(x)=+\infty$,从而$\exists R>0,\forall|x|>R,f(x)>f(x')$。于是在紧集$\{x||x|\leq R\}$上,必有最小值$f(x_0)=\inf_{|x|\leq R}f(x)$。显然$\{x||x|\leq R|\}$上的最小值就是$R^n上$的最小值。
	\end{exercise}
	\begin{exercise}
	\hfill\\
  设$f$是$[a,b]\times[a,b]$上的二元连续函数,定义$$g(x)=\max\{f(x,y)|y\in[a,b]\}$$。证明:$g$在$[a,b]$上连续。(南京理工大学,2005)
	
	对固定的$x_0\in[a,b],g(x_0,y)$对$y\in[a,b]$上的最大值是存在且唯一的。因为$g(x_0,y)$对$y$是闭区间上的连续函数。从而$g(x)$是良定义的。
	
	因为$f$是区间$[a,b]\times[a,b]$上的二元连续函数,从而$f$在$[a,b]\times[a,b]$上一致连续。从而存在$\delta>0,\forall(x_1,y_1),(x_2,y_2)\in[a,b]\times[a,b]$,只要$|x_1-x_2|<\delta,|y_1-y_2|<delta,|f(x_1,y_1)-f(x_2,y_2)|<\frac{\varepsilon}{2}$成立。
	
	从而对上述$\varepsilon>0$,有$\forall|x-x_0|<\delta,x\in[a,b]$,有$|f(x,y)-f(x_0,y)|<\varepsilon$。
	
	对$g(x_0)=\max\{f(x_0,y):y\in[a,b]\}$,不妨令$g(x_0)=f(x_0,y_0),y_0\in[a,b]$。于是对$g(x)=\max\{f(x,y):y\in[a,b]\}=f(x_1,y_1),y_1\in[a,b]$。从而
	$$
	\begin{aligned}
	|g(x)-g(x_0)|&=|f(x,y_1)-f(x_0,y_0)|\\
	&\leq|f(x,y_1)-f(x_0,y_1)|+|f(x_0,y_1)-f(x_0,y_0)|\\
	&\leq3\varepsilon.\\
	\end{aligned}
	$$
	于是$g(x)$在$[a,b]$上连续。
	\end{exercise}
	
	\begin{exercise}
	\hfill\\
  设n元函数$f$在$R^n$上具有连续偏导数,证明对于任意的$x=(x_1,x_2,\cdot\cdot\cdot x_n),y=(y_1,y_2,\cdot\cdot\cdot y_n)\in R^n$成立下述Hadamard公式:
  $$\displaystyle f(\bar{y})-f(\bar{x})=\sum_{i=1}^n\int_0^1(y_i-x_i)\frac{\partial f}{\partial y_i}(\bar{x}+t(\bar{y}-\bar{x}))dt$$
	
设$$F(t)=f(\overline{x}+t(\overline{y}-\overline{x})),$$
则$$f(\overline{y})-f(\overline{x})=F(1)-F(0)=\int_0^1F'(t)\mathrm{d}t.$$
由于
\begin{align*}
F'(t)&=\sum_{i=1}^n\frac{\partial f}{\partial x_i}(\overline{x}+t(\overline{y}-\overline{x}))\frac{\partial(x_i-t(y_i-x_i))}{\partial t}\\
&=\sum_{i=1}^n(y_i-x_i)\frac{\partial f(\overline{x}+t(\overline{y}-\overline{x}))}{\partial x_i}.\\
\end{align*}
所以Hadamard公式得证。
	\end{exercise}
	\begin{exercise}
	\hfill\\
	  设函数$z=f(x,y)$在全平面上有定义,具有连续的偏导数,且满足方程$xf_x(x,y)+yf_y(x,y)=0$,证明:$f(x,y)$为常数。
	
	作极坐标变换:
	$$
	\begin{cases}
	x=r\cos\theta,&r\geq0\\
	y=r\sin\theta,&\theta\in[0,2\pi]\\
	\end{cases}	
	$$
	于是由$$f_r=f_x\cos\theta+f_y\sin\theta$$可知,$$rf_r=0,$$
	即$$f(r\cos\theta,r\sin\theta)=F(\theta).$$
	又因为$z=f(x,y)$在$(0,0)$处连续。
	从而$$f(0,0)=\lim_{r\rightarrow0}f(r\cos\theta,r\sin\theta)=\lim_{r\rightarrow0}F(\theta)=F(\theta).$$
	这就是说$$f(x,y)=f(0,0).$$
	\end{exercise}
	\hfill\\
  \section{多元函数的微分学}
  \begin{exercise}
  \hfill\\
    设函数$f(x)$在$[a,b]$上可导,$f'(x)$在$[a,b]$上单调下降,且$f'(x)\geq m>0.$求证:$|\int_a^b\cos f(x)dx|\leq\frac 2m$。

因为$f'(x)>0$,所以$f(x)$严格单调递增,设$f(a)=A$,$f(b)=B$,则$f(x)$的反函数$x=\phi(t)$在$[A,B]$上业单调递增,且可导。同时有$$0<\phi(t)=\frac{1}{f'(x)}\leq\frac{1}{m}.$$
但因为$f'(x)$在$[a,b]$上单调下降,所以$\phi'(x)$在$[A,B]$上严格单调递增,从而$\phi'(t)$若有间断点,只能是第一类间断点,但是因为$\phi(t)$在$[A,B]$上可导,所以$\phi'(t)$在$[A,B]$上不可能有第一类间断点,从而$\phi'(t)$在$[A,B]$上连续。于是
$$\int_a^b\cos f(x)\mathrm{d}x=\int_A^B\cos t\cdot\phi'(t)\mathrm{d}t,$$
根据积分第二中值定理,$\exists\xi\in[A,B]$,使得
$$\int_A^B\cos t\cdot\phi'(t)\mathrm{d}t=\phi'(B)\int_{\xi}^B\cos t\mathrm{d}t=\phi'(B)(\sin B-\sin\xi),$$
联立上式可得:
$$|\int_a^b\cos f(x)\mathrm{d}x|\leq2|\phi'(B)|\leq\frac{2}{m}.$$
\end{exercise}
\hfill\\
  \section{重积分}
  \begin{exercise}
  \hfill\\
    计算$\displaystyle\lim_{n\rightarrow\infty}\sum_{j=1}^{2n}\sum_{i=1}^n\frac2{n^2}[\frac{2i+j}n]$,这里$[x]$为不超过$x$的最大整数。(清华大学,2007)
  
  \begin{align*}
  \lim_{n\rightarrow\infty}\sum_{j=1}^{2n}\sum_{i=1}^n\frac{2}{n^2}[\frac{2i+j}{n}]&=4\lim_{n\rightarrow\infty}\frac{1}{n}\frac{1}{2n}\sum_{i=1}^{2n}\sum_{i=1}^n[\frac{2}{n}i+2\frac{j}{2n}]\\
  &=4\iint_D[2x+2y]dxdy,\\
  \end{align*}
  其中$D=\{(x,y):0\leq x\leq1,0\leq y\leq1\}$,因此
  $$\lim_{n\rightarrow\infty}\sum_{j=1}^{2n}\sum_{i=1}^n\frac{2}{n^2}[\frac{2i+j}{n}]=4\iint_D[2x+2y]dxdy=6.$$
  \end{exercise}
  
  \begin{exercise}
  \hfill\\
   设$f(t)$连续,试证$\displaystyle\iiint_{x^2+y^2+z^2\leq1}f(ax+by+cz)dxdydz=\pi\int_{-1}^1(1-u^2)f(ku)du$,其中$k=\sqrt{a^2+b^2+c^2}>0$。(南京大学,2007) 
  
   作旋转变换使平面$ax+by+cz=0$变成$Ox'y'z'$空间中的坐标平面$z'=0$。则
  \begin{align*}
  \iiint_{x^2+y^2+z^2\leq 1}f(ax+by+cz)dxdydz&=\iiint_{x'^2+y'^2+z'^2\leq1}f(kz')dx'dy'dz'\\
  &=\int_{-1}^1f(kz')\iint_{x'^2+y'^2\leq1-z'^2}dx'dy'\\
  &=\pi\int_{-1}^1(1-z'^2)f(kz')dz'\\
  &=\pi\int_{-1}^1(1-u^2)f(ku)du.\\
  \end{align*} 
  \end{exercise}
  
  \begin{exercise}
  \hfill\\
  求$\int\int_S\frac{ds}z$,其中$S$是球面$x^2+y^2+z^2=a^2$被平面$z=h(0<h<a)$截得的球冠部分。(华东师范大学,2007)  
  
   曲面$S$的方程为$z=\sqrt{a^2-x^2-y^2}$,定义域$D$为圆域$x^2+y^2\leq a^2-h^2$。由于$$z_x=\frac{-x}{\sqrt{a^2-x^2-y^2}},z_y=\frac{-y}{\sqrt{a^2-x^2-y^2}},$$
  从而得到$\sqrt{1+z_x^2+z_y^2}=\frac{a}{\sqrt{a^2-x^2-y^2}}$。因此
  \begin{align*}
  \iint_S\frac{dS}{z}&=\iint_D\frac{adxdy}{a^2-x^2-y^2}\\
  &=\int_0^{2\pi}d\theta\int_0^{\sqrt{a^2-h^2}}\frac{ar}{a^2-r^2}dr\\
  &=2\pi a\int_0^{\sqrt{a^2-h^2}}\frac{ardr}{a^2-r^2}\\
  &=-\pi a\ln(a^2-r^2)|_0^{\sqrt{a^2-h^2}}\\
  &=2a\pi\ln\frac{a}{h}.\\
  \end{align*} 
  \end{exercise}
  \begin{exercise}
  \hfill\\
   $I=\iint_{x^2+y^2+z^2=R^2}\frac{ds}{\sqrt{x^2+y^2+(z-h)^2}}$,其中$h\neq R$(浙江大学,2002)
   
  
  令
  \[
	\begin{cases}
	x=R\cos\phi\sin\theta,&\\
	y=R\sin\phi\sin\theta,&0\leq\theta\leq\pi,0\leq\phi\leq2\pi,\\
	z=R\cos\theta,\\
	\end{cases}  
  \]
则$$E=x_{\theta}^2+y_{\theta}^2+z_{\theta}^2=R^2,$$
$$F=x_{\theta}x_{\phi}+y_{\theta}y_{\phi}+z_{\theta}z_{\phi}=0,$$
$$G=x_{\phi}^2+y_{\phi}^2+z_{\phi}^2=R^2\sin^2\theta.$$

因此
\begin{align*}
I&=\iint\limits_{x^2+y^2+z^2=R^2}\frac{dS}{\sqrt{x^2+y^2+(z-h)^2}}\\
&=\iint\limits_{\substack{0\leq\theta\leq\pi\\0\leq\phi\leq\leq2\pi}}f(x(\theta,\phi),y(\theta,\phi),z(\theta,\phi))\sqrt{EG-F^2}\mathrm{d}\theta\mathrm{d}\phi\\
&=R^2\int_0^{2\pi}\mathrm{d}\phi\int_0^{\pi}\frac{\sin\theta\mathrm{d}\theta}{\sqrt{R^2-2Rh\cos\theta+h^2}}\\
&=4\pi R.\\
\end{align*}
 
  \end{exercise}
\hfill\\
\section{曲线积分、曲面积分与场论}  
  
  \begin{exercise}
  \hfill\\
  计算$\int_Lx^2ds$,其中$L$是球面$x^2+y^2+z^2=1$与平面$x+y+z=0$的交线。(北京大学,2005)
   
  
   由坐标对称性知:
  $$\int_Lx^2\mathrm{d}S=\int_Ly^2\mathrm{d}S=\int_Lz^2\mathrm{d}S,$$
  从而$$\int_Lx^2\mathrm{d}S=\frac{1}{3}\int_L(x^2+y^2+z^2)\mathrm{d}S=\frac{1}{3}\int_L\mathrm{d}S=\frac{2\pi}{3}.$$
 
  \end{exercise}
  \begin{exercise}
  \hfill\\
   设$n$是平面区域$D$的正向边界线$C$的外法线,则$$\int_C\frac{\partial u}{\partial n}ds=\int\int_D(\frac{\partial^2u}{\partial x^2}+\frac{\partial^2u}{\partial y^2})dxdy.$$(中国科学院,2006) 
  
记$n=(\cos\alpha,\cos\beta)$,设切线方向为$l=(\cos x,\cos y)$。则
$$\cos x=-\cos\beta,\cos y=\cos\alpha.$$
从而
\begin{align*}
\int_L\frac{\partial u}{\partial n}dS&=\int_C(\frac{\partial u}{\partial x}\cos\alpha+\frac{\partial u}{\partial y}\cos\beta)ds\\
&=\int_L\frac{\partial u}{\partial x}\cos y-\frac{\partial u}{\partial y}\cos x\mathrm{d}S\\
&=\int_L\frac{\partial u}{\partial x}dy-\frac{\partial u}{\partial y}dx\\
&=\iint_D(\frac{\partial^2u}{\partial x^2}+\frac{\partial^2u}{\partial y^2})dxdy.\\
\end{align*}  
  \end{exercise}

  \begin{exercise}
  \hfill\\
设区域$\Omega$由分片光滑封闭曲面$\sum$所围成,$u(x,y,z)$在$\overline{\Omega}$上具有二阶连续偏导数,且在$\overline{\Omega}$上调和,即满足
$$\frac{\partial^2u}{\partial x^2}+\frac{\partial^2u}{\partial y^2}+\frac{\partial^2u}{\partial z^2}=0.$$  
  
  证明:$\iint_{\Sigma}\frac{\partial u}{\partial n}ds=0$,其中$n$为$\Sigma$的单位外法向量。
  
  设$(x_0,y_0,z_0)\in\Omega$为一定点,证明
  $$u(x_0,y_0,z_0)=\frac1{4\pi}\int\int_{\Sigma}(u\frac{\cos(r,n)}{r^2}+\frac1r\frac{\partial u}{\partial n})dS,$$
  其中$r=(x-x_0,y-y_0,z-z_0),r=|r|.$  
  
   设$n=(\cos\alpha,\cos\beta,\cos\gamma)$,则
  $$\frac{\partial f}{\partial n}=(\frac{\partial f}{\partial x}\cos\alpha,\frac{\partial f}{\partial y}\cos\beta,\frac{\partial f}{\partial z}\cos\gamma).$$
  \begin{align*}
    \iint_{\Sigma}\frac{\partial u}{\partial n}ds&=\iint_{\Sigma}\frac{\partial f}{\partial x}dydz+\frac{\partial f}{\partial y}dzdx+\frac{\partial f}{\partial z}dxdy\\
    &=\iiint_V(\frac{\partial^2u}{\partial x^2}+\frac{\partial^2u}{\partial y^2}+\frac{\partial^2z}{\partial z^2})dxdydz.
  \end{align*} 
  
  记
$$S=\frac1{4\pi}\int\int_{\Sigma}(u\frac{\cos(r,n)}{r^2}+\frac1r\frac{\partial u}{\partial n})dS,$$
$$\sum':=\{(x,y,z):|r|=\delta,\delta\in(0,1)\}.$$
则
\begin{align*}
S&=\frac{1}{4\pi}\iint_{\sum}(u\frac{(x-x_0)\cos\alpha+(y-y_0)\cos\beta+(z-z_0)\cos\gamma}{r^3}+\frac{1}{r}\frac{\partial u}{\partial \textbf{n}})\mathrm{d}S\\
&=\frac{1}{4\pi}\iint_{\sum}(\frac{u(x-x_0)}{r^3}+\frac{1}{r}\frac{\partial u}{\partial x})\mathrm{d}y\mathrm{d}z
+(\frac{u(y-y_0)}{r^3}+\frac{1}{r}\frac{\partial u}{\partial y})\mathrm{d}z\mathrm{d}x\\
&+(\frac{u(z-z_0)}{r^3}+\frac{1}{r}\frac{\partial u}{\partial z})\mathrm{d}y\mathrm{d}z\\
&=\iiint_{\overline{\Omega/\sum'}}\frac{1}{r}(\frac{\partial^2u}{\partial x^2}+\frac{\partial^2u}{\partial y^2}+\frac{\partial^2u}{\partial z^2})\mathrm{d}x\mathrm{d}y\mathrm{d}z\\
&+\frac{1}{4\pi}\iint_{\sum'}(\frac{u\cos(r,n)}{r^2}+\frac{1}{r}\frac{\partial u}{\partial\textbf{n}})\mathrm{d}S\\
&=\frac{1}{4\pi}\iint_{\sum'}(\frac{u\cos(r,n)}{r^2}+\frac{1}{r}\frac{\partial u}{\partial\textbf{n}})\mathrm{d}S\\
&=\frac{1}{4\pi}\iint_{\sum'}\frac{u}{\delta^2}\mathrm{d}S.\\
\end{align*}
由$u$的连续性,令$\delta\rightarrow0$,可得$S=u(x_0,y_0,z_0)$。
  \end{exercise}
 \hfill\\
 
   \section{含参变量积分}
  \begin{exercise}
  \hfill\\
 $\displaystyle\int_0^{\pi}\ln(1-2a\cos x+a^2)dx$(华东师范大学,2003)  
  
   记$I(a)=\int_0^{\pi}\ln(1-2a\cos x+a^2)\mathrm{d}x$,则$I(0)=0$,$$I'(a)=\int_0^{\pi}\frac{-2\cos x+2a}{1-2a\cos x+a^2}\mathrm{d}x$$。
  作变换$t=\tan\frac{x}{2}$,则$$\mathrm{d}x=\frac{2\mathrm{d}t}{1+t^2},\cos x=\frac{1-t^2}{1+t^2},\sin x=\frac{2t}{1+t^2}.$$
  于是
  \begin{align*}
  I'(a)&=4\int_0^{\infty}\frac{t^2-1+a+at^2}{1+t^2-2a(1-t^2)+a^2(1+t^2)}\frac{\mathrm{d}t}{1+t^2}\\
  &=\frac{2}{a}\int_0^{\infty}\frac{\mathrm{d}t}{1+t^2}+2(a-\frac{1}{a})\int_0^{\infty}\frac{\mathrm{d}t}{(1-a)^2+(1+a)^2t^2}\\
  &=\frac{2}{a}\int_0^{\infty}\frac{\mathrm{d}t}{1+t^2}-\frac{2}{a}\int_0^{\infty}\frac{\mathrm{d}(\frac{1+a}{1-a}t)}{1+(\frac{1+a}{1-a})^2t^2}\\
  &=0\quad(|a|<1).
  \end{align*}
  从而$I(a)=0.$
 
  \end{exercise}

  \begin{exercise}
  \hfill\\
   设$f(x)$是$[-1,1]$上的连续函数,则$\displaystyle\lim_{y\rightarrow0^+}\int_{-1}^{1}\frac{yf(x)}{x^2+y^2}dx=\pi f(0)$。(中国科学院,2003) 
  
 因为$\forall\delta>0$,
$$\lim_{y\rightarrow0+}\int_{-1}^{-\delta}\frac{yf(x)}{x^2+y^2}\mathrm{d}x=\int_{-1}^{-\delta}\lim_{y\rightarrow0+}\frac{yf(x)}{x^2+y^2}\mathrm{d}x=0,$$
$$\lim_{y\rightarrow0+}\int^{1}_{\delta}\frac{yf(x)}{x^2+y^2}\mathrm{d}x=\int^{1}_{\delta}\lim_{y\rightarrow0+}\frac{yf(x)}{x^2+y^2}\mathrm{d}x=0,$$
从而
$$\lim_{y\rightarrow0+}\int_{-1}^{1}\frac{yf(x)}{x^2+y^2}\mathrm{d}x=\lim_{y\rightarrow0+}\int_{-\delta}^{\delta}\frac{yf(x)}{x^2+y^2}\mathrm{d}x.$$
又$\forall\varepsilon>0,\exists\delta>0,\forall x\in(-\delta,\delta),f(0)-\varepsilon<f(x)<f(0)+\varepsilon,$从而
\begin{align*}
\lim_{y\rightarrow0+}\int_{-\delta}^{\delta}\frac{yf(x)}{x^2+y^2}\mathrm{d}x&<(f(0)+\varepsilon)\lim_{y\rightarrow0+}\int_{-\delta}^{\delta}\frac{y\mathrm{d}x}{x^2+y^2}\\
&=(f(0)+\varepsilon)\lim_{y\rightarrow0+}\arctan\frac{\delta}{y}-\arctan\frac{-\delta}{y}\\
&=(f(0)+\varepsilon)\pi.\\
\end{align*}
同理$$\lim_{y\rightarrow0+}\int_{-\delta}^{\delta}\frac{yf(x)}{x^2+y^2}\mathrm{d}x>(f(0)-\varepsilon)\pi.$$
于是由$\varepsilon$的任意性,可得所求结果。 
  \end{exercise}

  \begin{exercise}
  \hfill\\
   设$0<r<1,x\in R$,求证:$\displaystyle\frac{1-r^2}{1-2r\cos x+r^2}=1+2\sum_{n=1}^{\infty}r^n\cos nx$; 
  

\begin{align*}
\frac{1-r^2}{1-2r\cos x+r^2}&=-1+\frac{2-2r\cos x}{1-2r\cos x+r^2}\\
&=-1+\frac{2-2r\cos x}{(1-r\cos x)^2+r^2\sin^2x}\\
&=-1+\frac{2-2r\cos x}{(1-r\cos x-i\sin x)(1-r\cos x+i\sin x)}\\
&=-1+\frac{1}{1-r\cos x-i\sin x}+\frac{1}{1-r\cos x+i\sin x}\\
&=-1+\frac{1}{1-re^{ix}}+\frac{1}{1-re^{-ix}}\\
&=-1+\sum_{n=0}^{\infty}(re^{ix})^n+\sum_{n=0}^{\infty}(re^{-ix})^n\\
&=-1+\sum_{n=0}^{\infty}r^n(e^{inx}+e^{-inx})\\
&=-1+2\sum_{n=0}^{\infty}r^n\cos nx\\
&=1+2\sum_{n=1}^{\infty}r^n\cos nx.\\
\end{align*}

  \end{exercise}
\begin{exercise}
\hfill\\
  设二元函数$f(x,y)$为$[a,b]\times[c,+\infty)$上的连续非负函数,$I(x)=\int_c^{+\infty}f(x,y)dy$在$[a,b]$上连续。证明:$I(x)$在$[a,b]$上一致连续。(南京师范大学,2004,2008)

设$x_0\in[a,b]$,由于$\int_c^{+\infty}f(x_0,y)\mathrm{d}y$收敛,因此对任意$\varepsilon>0$,存在$M_{x_0}>c$,对任意$M>M_{x_0}$,有$\int_M^{+\infty}f(x_0,y)\mathrm{d}y<\frac{\varepsilon}{4}.$

$f(x,y)$在$[a,b]\times[c,M]$上一致连续,从而存在$\delta>0,$对任意$x\in(x_0-\delta_1,x_0+\delta_1)\cap[a,b]$,有$$|f(x_0,y)-f(x,y)|<\frac{\varepsilon}{4(M_{x_0}-c)}.$$
$I(x)=\int_c^{\infty}f(x,y)\mathrm{d}y$在$[a,b]$上一致连续,对上述$\varepsilon>0$,存在$\delta_2>0$,对任意$x\in(x_0-\delta_2,x_0+\delta_2)\cap[a,b]$,有
$$|I(x)-I(x_0)|=|\int_c^{\infty}f(x,y)\mathrm{d}y-\int_c^{\infty}f(x_0,y)\mathrm{d}y|<\frac{\varepsilon}{2}.$$
取$\delta=\min\{\delta_1,\delta_2\}$,对任意$x_0\in[a,b]$,$x\in(x_0-\delta,x_0+\delta)\cap[a,b]$,有
\begin{align*}
\int_M^{\infty}f(x,y)\mathrm{d}y&=\int_c^{\infty}f(x,y)\mathrm{d}y-\int_c^Mf(x,y)\mathrm{d}y\\
&=|\int_c^{\infty}f(x,y)\mathrm{d}y-\int_c^{\infty}f(x_0,y)\mathrm{d}y\\
&+\int_c^M[f(x_0,y)-f(x,y)]\mathrm{d}y+\int_M^{\infty}f(x_0,y)\mathrm{d}y|\\
&<|\int_c^{\infty}f(x,y)\mathrm{d}y-\int_c^{\infty}f(x_0,y)\mathrm{d}y|\\
&+|\int_c^M[f(x_0,y)-f(x,y)\mathrm{d}y|+|\int_M^{\infty}f(x_0,y)\mathrm{d}y|\\
&<\frac{\varepsilon}{2}+(M-c)\frac{\varepsilon}{4(M_{x_0}-c)}+\frac{\varepsilon}{4}<\varepsilon.\\
\end{align*}
从而得到$[a,b]$的一个开覆盖$\{(x_0-\delta,x_0+\delta)|x\in[a,b]\}$,有有限开覆盖定理,存在其中有限个开区间$(x_i-\delta,x_i+\delta),i=1,2,\cdots,n$覆盖$[a,b]$。不妨令$M'=\max\{M_{x_1},M_{x_2},\cdots,M_{x_n}\}$,则对任意的$M>M'$,有$\int_M^{\infty}f(x,y)\mathrm{d}y<\frac{\varepsilon}{4}$。因此$I(x)$在$[a,b]$上一致收敛。
\end{exercise}
  \hfill\\
  \section{傅里叶级数}
  \begin{exercise}
  \hfill\\
  设$\psi(x)$在$[0,+\infty)$上连续且单调,$$\displaystyle\lim_{x\rightarrow+\infty}\psi(x)=0$$,证明:$$\displaystyle\lim_{p\rightarrow+\infty}\int_0^{+\infty}\psi(x)\sin pxdx=0.$$  
  
 因为$\lim_{x\rightarrow+\infty}\phi(x)=0$,所以存在$N>0$,使得$x\geq N$,$|\phi(x)|<1$。利用积分第二中值定理可得
\begin{align*}
|\int_N^A\phi(x)\sin px\mathrm{d}x|&=|\phi(N)\int_N^{\xi}\sin px\mathrm{d}x+\phi(A)\int_{\xi}^A\sin px\mathrm{d}x|\\
&<|\int_N^{\xi}\sin px\mathrm{d}x|+|\int_{\xi}^A\sin px\mathrm{d}x|\\
&\leq\frac{4}{p}(\forall A>N),
\end{align*}
因此$|\int_N^{+\infty}\phi(x)\sin px\mathrm{d}x|\leq\frac{4}{p}.$从而
$$\lim_{p\rightarrow+\infty}\int_N^{+\infty}\phi(x)\sin px\mathrm{d}x=0.$$
而由Riemann引理
$$\lim_{p\rightarrow+\infty}\int_0^N\phi(x)\sin px\mathrm{d}x=0.$$
这就得到了我们所要求的结果。 
  \end{exercise}

  \begin{exercise}
  \hfill\\
 三角级数$\displaystyle\frac{a_0}2+\sum_{n=1}^{\infty}(a_n\cos x+b_n\sin nx)$是某个在$[-\pi,\pi]$上可积且绝对可积函数的Fourier级数的必要条件是$\displaystyle\sum_{n=1}^{\infty}\frac{b_n}n$收敛。  
  
  设$f(x)\sim\frac{a_0}{2}+\sum_{n=1}^{\infty}(a_n\cos x+b_n\sin x)$。
  令$F(x)=\int_c^x[f(t)-\frac{a_0}{2}\mathrm{d}t.$(仅考虑$f(x)$只有有限个间断点的情况)
  $F(x)$是周期为$2\pi$的连续函数,在$f(x)$的连续点,成立$F'(x)=f(x)-\frac{a_0}{2}$。在$f(x)$的第一类间断点,$F(x)$的两个单侧导数$F'_{\pm}(x)=f(x\pm)-\frac{a_0}{2}$都存在。由Dini-Lipschitz判别法,$F(x)$可展开为收敛的Fourier级数$$F(x)=\frac{A_0}{2}+\sum_{n=1}^{\infty}(A_n\cos nx+B_n\sin nx).$$
  利用分步积分法
  \begin{align*}
  A_n&=\frac{1}{\pi}\int_{-\pi}^{\pi}F(x)\cos nx\mathrm{d}x\\
  &=[\frac{1}{\pi}\frac{\sin nx}{n}F(x)]|_{-\pi}^{\pi}-\frac{1}{n\pi}\int_{-\pi}^{\pi}f(x)\sin nx\mathrm{d}x\\
  &=-\frac{1}{n\pi}\int_{-\pi}^{\pi}(f(x)-\frac{a_0}{2})\sin nx\mathrm{d}x\\
  &=-\frac{b_n}{n}.
  \end{align*}
  类似可得:$B_n=\frac{a_n}{n}$。于是$$F(x)=\frac{A_0}{2}+\sum_{n=1}^{\infty}(-\frac{b_n}{n}\cos nx+\frac{a_n}{n}\sin nx).$$
  令$x=c$,有
  $$0=\frac{A_0}{2}+\sum_{n=1}^{\infty}(-\frac{b_n}{n}\cos nc+\frac{a_n}{n}\sin nc).$$
  从而
  \begin{align*}
  F(x)&=\int_c^x[f(t)-\frac{a_0}{2}]\mathrm{d}t\\
  &=\sum_{n=1}^{\infty}(a_n\frac{\sin nx-\sin nc}{n}+b_n\frac{-\cos nx+\cos nc}{n})\\
  &=\sum_{n=1}^{\infty}\int_c^x(a_n\cos nt+b_n\sin nt)dt.\\
  \end{align*}
  在此,令$x=0$,得到$F(0)=\frac{A_0}{2}-\sum_{n=1}^{\infty}\frac{b_n}{n}$,说明了级数$\sum_{n=1}^{\infty}\frac{b_n}{n}$收敛。  
  \end{exercise}
  \begin{exercise}
  \hfill\\
  
  
  
  \end{exercise}
  
   \begin{exercise}
  \hfill\\
  
  
  
  \end{exercise}
  
  \begin{exercise}
  \hfill\\
  
  
  
  \end{exercise}
  
  \begin{exercise}
  \hfill\\
  
  
  
  \end{exercise}
  
  
  \begin{exercise}
  \hfill\\
  
  
  
  \end{exercise} 
\begin{example}[id:20151012-201647] \label{20151012-201647}\index{Example!20151012-201647} \hfill \\
\%\%\%\%\%\%\%\%\%\%\%\%\%\%\%



\%\%\%\%\%\%\%\%\%\%\%\%\%\%\%


example \ref{20151012-190708}-20151012-190708

\end{example}



























%-=-=-= DEFINITION
\begin{definition}[Natural Numbers]\index{Number System! Natural Numbers}

\[
\mathbb{N}=\set{0, 1, 2, 3 \ldots}
\]

\end{definition}

\begin{remark}
It is not uncommon for zero to be excluded from the natural numbers.  In fact, some exclude zero from the natural numbers and then describe the set of natural numbers that include zero the whole numbers. \\

\[
\mathbb{W}=\set{0, 1, 2, 3, \ldots}
\]

For the purposes of these notes, zero will be included within the set of natural numbers.
\end{remark}

%-=-=-= DEFINITION 



%\input{elegantNote-manual.tex}




\section{\texorpdfstring{$L^2$}{L2} theory}
\label{sec: L2 theory}

consider \cite{Winkler2023}
\begin{equation}
	v_t = \Delta v - v + f,
\end{equation}
with boundary conditions
\begin{equation*}
	\frac{\partial v}{\partial\nu} = 0,
\end{equation*}
where 
\begin{equation*}
	f\in L^\infty((0,T); L^1(\Omega)),
\end{equation*}
using \[
\nabla v \cdot \nabla \Delta v=\Delta\left(\frac{1}{2}|\nabla v|^2\right)-\left|D^2 v\right|^2,
\]
we get
\begin{align*}
	\frac{\dd}{\dd t}\int|\nabla v|^p 
	&= p\int |\nabla v|^{p-2}\nabla v\cdot \nabla v_t 
	= p\int |\nabla v|^{p-2} (\nabla \Delta v - \nabla v + \nabla f) \\
	&= \frac12\int |\nabla v|^{p-2}\Delta (|\nabla v|^2) - p\int |\nabla v|^{p-2}|D^2v|^2 
		-p\int |\nabla v|^p + p\int |\nabla v|^{p-2}\nabla v \nabla f,
\end{align*}
by additional assumption of convex domain 
or a pointwise estimate of boundary trace alongwith trace imbedding lemma and Sobolev imbedding inequality,
we can control the first term of right hand, and estimate
\begin{align*}
	p\int |\nabla v|^{p-2}\nabla v \cdot \nabla f 
	&= - p\int \nabla\cdot(|\nabla v|^{p-2}\nabla v) f\\
	&= -p(p-2)\int |\nabla v|^{p-4}\nabla v\cdot (D^2v\nabla v) f 
		- p \int |\nabla v|^{p-2}\Delta v f\\
	&\leq c(p,n)\int |\nabla v|^{p-2} |D^2v| |f|\\
	&\leq c(p,n)\left(\int |\nabla v|^{p-2}|D^2v|^2\right)^{\frac12}
	\left(\int |\nabla v|^{p-2}f^2\right)^{\frac12}\\
	&= c(p,n) \|\nabla w\|_{L^2} \left(\int |\nabla v|^{p-2}f^2\right)^{\frac12},
\end{align*}
here, 
\[
	w = |\nabla v|^{\frac{p}{2}},	
\]
let $q\in(2,p)$ to be specified below, we estimate
\begin{align*}
	\left(\int |\nabla v|^{p-2}f^2\right)^{\frac12}
	&\leq \left(\int |\nabla v|^{\frac{(p-2)q}{q-2}}\right)^{\frac{q-2}{2q}}
	\left(\int f^q\right)^{\frac{1}{q}}\\
	&= \|w\|_{L^{\frac{2(p-2)q}{p(q-2)}}}^{\frac{p-2}{p}}
		\|f\|_{L^q},
\end{align*}
using G-N inequality
\[
	\|u\|_{L^{\frac{2(p-2)q}{p(q-2)}}}
	\leq C_g \|u\|_{W^{1,2}}^\theta\|u\|_{L^2}^{1-\theta},\quad u\in W^{1,2}, 
\]
with
\[
	\frac{p(q-2)}{2(p-2)q} = \theta\left(\frac12-\frac1n\right) + \frac{1-\theta}{2}.
\]
i.e.,
\[
	\theta = \frac{n(p-q)}{(p-2)q},
\]
we get
\begin{equation*}
	p\int |\nabla v|^{p-2}\nabla v \cdot \nabla f 
	\leq c(p,n) \|w\|_{W^{1,2}}^{1+\frac{n(p-q)}{pq}}
		\|w\|_{L^2}^{\frac{p-2}{p} - \frac{n(p-q)}{pq}}
		\|f\|_{L^q}.
\end{equation*}
Fixing 
\[
	q = \frac{(n+2)p}{n+p},
\]
which is such that
\[
	\alpha := \frac{2}{1-\frac{n(p-q)}{pq}} = q,\quad \theta = \frac{n}{n+2},
\]
and 
\begin{equation}\label{eq: gn inequality Lpw12L2}
	\|u\|_{L^{\frac{2(n+2)}{n}}} \leq C_g \|u\|_{W^{1,2}}^{\frac{n}{n+2}}\|u\|_{L^2}^{\frac{2}{n+2}},
\end{equation}
using Young inequality, we get
\begin{align*}
	p\int |\nabla v|^{p-2}\nabla v \cdot \nabla f 
	&\leq \varepsilon \|w\|_{W^{1,2}}^2
		\|w\|_{L^2}^{\left(\frac{p-2}{p} - \frac{n(p-q)}{pq}\right)\alpha}
		\|f\|_{L^q}^\alpha\\
	&= \varepsilon \|w\|_{W^{1,2}}^2
	+ c(\varepsilon) \|w\|_{L^2}^{\frac{2(p-2)}{n+p}}
	\|f\|_{L^q}^q,
\end{align*}
and therefore
\begin{equation*}
	\mathcal{F}' + \frac1C F \leq C \mathcal{F}^{\frac{p-2}{n+p}} \|f\|_q^q, 
\end{equation*} 
where
\[
	\mathcal{F} := \int w^2.
\]
So we obtain by a direct application of ODE comparable method,
\[
	\mathcal{F}^{\frac{n+2}{n+p}} \leq C \int_0^T\|f\|_q^q,
\]
and consequently,
\begin{align*}
	\sup_{(0,T)}\int |\nabla v|^p + \int_0^T\int |\nabla v|^{p-2}|D^2 v|^2 
	\leq C \left(\int_0^T\int |f|^{\frac{(n+2)p}{n+p}}\right)^{\frac{n+p}{n+2}}
\end{align*}
and by \eqref{eq: gn inequality Lpw12L2},
\begin{equation*}
	\left(\int_0^T\int |\nabla v|^{\frac{(n+2)p}{n}}\right)^{\frac{n}{n+2}} 
	\leq C \left(\int_0^T\int |f|^{\frac{(n+2)p}{n+p}}\right)^{\frac{n+p}{n+2}}. 
\end{equation*}






\printbibliography%[heading=bibintoc, title=\ebibname]

\end{document}
