\chapter{泛函分析}
\section{度量空间}
\subsection{压缩映像原理}
\begin{exercise}
\hfill\\
证明完备空间的闭子集是一个完备的子空间,而任一度量空间中的完备子空间必是闭子集。

设距离空间$(\mathscr{X},\rho)$是一个完备的距离空间,闭集$U\subset\mathscr{X}$。任取闭集$U$中的基本列$\{x_n\}\subset U$,则$\{x_n\}$也是定义在$(\mathscr{X},\rho)$中的基本列,从而存在$x\in\mathscr{X}$,使得$\lim_{n\rightarrow\infty}x_n=x$。又因为$U$是闭集,所以$x\in U$。这就是说距离空间$(U,\rho)$是完备的。

设$(U,\rho)$是度量空间$(\mathscr{X},\rho)$的一个完备子空间。则$\forall\{x_n\}\subset U$,若$x_n\rightarrow x$,则易知$\{x_n\}$是距离空间$(U,\rho)$的基本列,从而存在$x'\in U$使得$$\lim_{n\rightarrow\infty}x_n=x'.$$
于是$\rho(x',x)=\lim_{n\rightarrow\infty}\rho(x_n,x)=0$,从而$x'=x\in U$。于是$U$是闭子集。
\end{exercise}

\begin{exercise}
\hfill\\
(Newton)设$f$是定义在$[a,b]$上的二次连续可微的实值函数,$x_0\in(a,b)$使得$f(x_0)=0$,$f'(x_0)\neq0$。求证存在$x_0$的临域$U(x_0)$使得$\forall x_0\in U(x_0)$,迭代序列
$$x_{n+1}=x_n-\frac{f(x_n)}{f‘(x_n)}\quad(n=0,1,2,\cdots)$$是收敛的,并且
$$\lim_{n\rightarrow\infty}x_n=x_0.$$

\begin{align*}
\shortintertext{记}
 Tx&=x-\frac{f(x)}{f'(x)},\\
\shortintertext{则}
T'x&=\frac{f(x)f''(x)}{f'^2(x)}\rightarrow0,x\rightarrow x_0.
\end{align*}
于是存在$\delta>0$,使得$T$在$(x_0-\delta,x_0+\delta)$上是压缩的。

又因为$f'(x_0)\neq0$,于是存在$h>0$,使得$\forall x\in(x_0-h,x_0+h)$,$f'(x)>0$。于是
$$|Tx-x_0|=|x-x_0-\frac{f(x)-f(x_0)}{f'(x)}|=|x-x_0||1-\frac{f'(\xi)}{f'(x)}|<|x-x_0|,$$
$\forall x\in(x_0-h,x_0+h)$,其中$\xi$介于$x$与$x_0$之间。这就是说$T$是在$(x_0-h,x_0+h)$到上的映射。

于是只要取$\tau=\frac{1}{2}\min\{\delta,h\}$,就有$T$是在$[x_0-\tau,x_0+\tau]$上的压缩映射。于是由压缩映象原理知,存在$x_0$的一个临域,使得迭代序列$\{x_n\}$收敛。收敛到$x_0$是显然的。
\end{exercise}
\begin{exercise}\label{13}
\hfill\\
设$M$是$(\mathbb{R}^n,\rho)$中的有界闭集,映射$T:M\rightarrow M$满足:
$$\rho(Tx,Ty)<\rho(x,y)(\forall x,y\in M,x\neq y).$$
求证$T$在$M$中存在唯一的不动点。

首先,如果$$\rho(x_n,x_0)\rightarrow0\quad(n\rightarrow\infty)$$,则$$\rho(Tx_n,Tx_0)<\rho(x_n,x_0)\rightarrow0.\quad(n\rightarrow\infty)$$
这就是说$T$是一个连续映射。又距离函数$\rho(x,y)$是连续的,从而映射
$$f(x)=\rho(Tx,x),x\in M$$是$M\mapsto\mathbb{R}^+\cup\{0\}$上的连续映射。又$M\subset\mathbb{R}^n$是有界闭集,从而$M$是紧集,于是存在$x_0\in M$,使得$\forall x\in M$,$f(x_0)\leq f(x)$。如果$Tx_0\neq x_0$,则
$$f(x_0)=\rho(Tx_0,x_0)>\rho(T(Tx_0),Tx_0)=f(Tx_0).$$
这就与$x_0$的定义矛盾。于是$x_0$是不动点。假设$x'\neq x_0$也是$T$的一个不动点,则
$$\rho(x',x_0)=\rho(Tx',Tx_0)<\rho(x',x_0),$$
矛盾。于是不动点是唯一的。
\end{exercise}

\begin{exercise}
\hfill\\
对于积分方程
$$x(t)-\lambda\int_0^1e^{t-s}x(s)\mathrm{d}s=y(t),$$
其中$y(t)\in C[0,1]$为一给定函数,$\lambda$为常数,$|\lambda|<1$,求证存在唯一解$x(t)\in C[0,1]$。

考虑原积分方程等价于
$$e^{-t}x(t)=\lambda\int_0^1e^{-s}x(s)\mathrm{d}s+e^{-t}y(t),$$
籍此记
$$Tx=\lambda\int_0^1x\mathrm{d}s+e^{-t}y(t),$$
于是
\begin{align*}
\rho(Tx-Ty)&=\max_{t\in[0,1]}|\lambda\int_0^1(x-y)\mathrm{d}s|\\
&\leq|\lambda|\max_{t\in[0,1]}\int_0^1|x-y|\mathrm{d}s\\
&\leq|\lambda|\max_{t\in[0,1]}|x-y|\int_0^1\mathrm{d}s\\
&=|\lambda|\rho(x-y),\\
\end{align*}
即$T$是完备内积空间$(C[0,1],\rho)$映上的压缩映射。于是$T$在$(C[0,1],\rho)$存在唯一不动点。

考虑
$$\int_0^1e^{-t}x(t)dt=\lambda\int_0^1\int_0^1e^{-s}x(s)\mathrm{d}s\mathrm{d}t+\int_0^1e^{-t}y(t)\mathrm{d}t,$$
即
$$\int_0^1e^{-t}x(t)\mathrm{d}t=\frac{1}{1-\lambda}\int_0^1e^{-t}y(t)\mathrm{d}t,$$
所以原积分方程的解为:
$$x(t)=\frac{1}{1-\lambda}\int_0^1e^{t-s}y(s)\mathrm{d}s+y(t).$$
\end{exercise}
\hfill\\
\subsection{完备化}
\begin{exercise}
\hfill\\
令空间$S$为一切实或复数列
$$x=(\xi_1,\xi_2,\cdots,\xi_n,\cdots)$$
组成的集合,在$S$中定义距离为
$$\rho(x,y)=\sum_{k=1}^{\infty}\frac{1}{2^k}\frac{|\xi_k-\eta_k|}{1+|\xi_k-\eta_k|},$$
其中$x=(\xi_1,\xi_2,\cdots,\xi_n,\cdots)$,$y=(\eta_1,\eta_2,\cdots,\eta_n,\cdots)$。求证$S$为一个完备的距离空间。

首先检验$\rho(x,y)$是一个距离。
\begin{enumerate}
\item[i] $\rho(x,y)\geq0$是显然的;进一步,$\rho(x,y)=0$ $\Longleftrightarrow|\xi_k-\eta_k|=0,\forall k\geq0$ $\Longleftrightarrow x=y.$
\item[ii] $\rho(x,y)=\rho(y,x)$是显然的。
\item[iii] 考虑到
\begin{align*}
\frac{|x-y|}{1+|x-y|}&=\frac{|x-z+z-y|}{1+|x-z+z-y|}\\
&\leq\frac{|x-z|+|z-y|}{1+|x-z|+|z-y|}\\
&=\frac{|x-z|}{1+|x-z|+|z-y|}+\frac{|z-y|}{1+|x-z|+|y-z|}\\
&\leq\frac{|x-z|}{1+|x-z|}+\frac{|z-y|}{1+|z-y|}\\
\end{align*}
于是容易得到
$$\rho(x,y)\leq\rho(x,z)+\rho(y,z).$$
\end{enumerate}
这就证实了$\rho(x,y)$是一个距离。接下来证明完备性。
设$\{x^{(n)}\}\subset S$为基本列,即
$$\rho(x^{(m)}-x^{(n)})\rightarrow0,\text{ if }n,m\rightarrow\infty.$$
对固定的$k$,由$$\frac{1}{2^k}\frac{|\xi^{(m)}_k-\xi^{(n)}_k|}{1+|\xi^{(m)}_k-\xi^{(n)}_k|}\leq\rho(x^{(m)}-x^{(n)})\rightarrow0,\text{ if }n,m\rightarrow\infty.$$
可知,$\{\xi_k^{(n)}\}$是Cauchy列。从而存在一个实(复)数$\xi_k$使得
$$\lim_{n\rightarrow\infty}\xi_k^{(n)}=\xi_k.$$
令$m\rightarrow\infty,$
于是我们有$$\rho(x_k^{(n)},x)\rightarrow0,\text{ if }n\rightarrow\infty,$$
其中,$x=(\xi_1,\xi_2,\cdots,\xi_n,\cdots)$。这就是说$x^{(n)}\rightarrow x$,显然$x\in S$。这就完成了完备性的证明。
\end{exercise}

\begin{exercise}
\hfill\\
设$F$是只有有限项不为0的实数列全体,在$F$上引进距离
$$\rho(x,y)=\sup_{k\geq1}|\xi_k-\eta_k|,$$
其中$x=\{\xi_k\}\in F,y=\{\eta_k\}\in F$,求证$(F,\rho)$不完备,并指出它的完备化空间。

要说明$(F,\rho)$不完备,只需给出一个反例。考虑$$x_n(1,\frac{1}{2},\cdots,\frac{1}{n},0,\cdots,0,\cdots),$$
显然$$\rho(x_m,x_n)\leq\min\{\frac{1}{n},\frac{1}{m}\}\rightarrow0,\textbf{ if }m,n\rightarrow0,$$
即$\{x_n\}\subset F$是基本列,然而
$$x_n\rightarrow(1,\frac{1}{2},\cdots\frac{1}{n},\cdots)\not\in F.$$

$F$的完备化空间为
$$\overline{F}=\{\{x_n\}\subset\mathbb{R}:\lim_{n\rightarrow\infty}x_n=0.\}.$$

首先,$$F\subset\overline{F}.$$
因为$\forall\{x_n\}\in F$,有$\lim_{n\rightarrow\infty}x_n=0$,所以$\{x_n\}\in\overline{F}$,即$F\subset\overline{F}$。

其次,$F$在$\overline{F}$中稠密。因为$\forall\{x_n\}\in\overline{F}$,我们总可以定义$X_n=(x_1,x_2,\cdots,x_n,0,0,\cdots)\in F$,$n=1,2,\cdots$。易见$X_n\rightarrow x=(x_1,x_2,\cdots),n\rightarrow\infty$。

最后,我们证明$\overline{F}$是完备的。设$\{x^{(n)}=(\xi_1^{(n)},\xi_2^{(n)},\cdots)\}_{n=1}^{\infty}$是$\overline{F}$中的基本列。则$$\rho(x^{(m)},x^{(n)})=\sup_{k\geq1}|\xi_k^{(m)}-\xi_k^{(n)}|\rightarrow0,n,m\rightarrow0.$$
于是对固定的$k$,有
$$|\xi_k^{(m)}-\xi_k^{(n)}|\leq\rho(x^{(m)},x^{(n)})\rightarrow0,n,m\rightarrow\infty,$$
即$\{\xi_k^{(n)}\}$是Cauchy列。从而存在$\xi_k\in\mathbb{R}$使得$$\lim_{n\rightarrow\infty}\xi_k^{(n)}=\xi_k.$$
记$x=(\xi_1,\xi_2,\cdots)$,则令$m\rightarrow\infty$,有
$$\rho(x^{(n)},x)\rightarrow0,n\rightarrow\infty.$$
下面只需证明$\{\xi_n\}$是Cauchy列即可。
\begin{align*}
|\xi_l-\xi_k|&=|\xi_l-\xi_l^{(n)}+\xi_l^{(n)}-\xi_k^{(n)}+\xi_k^{(n)}-\xi_k|\\
&\leq|\xi_l-\xi_l^{(n)}|+|\xi_l^{(n)}-\xi_k^{(n)}|+|\xi_k^{(n)}-\xi_k|\\
\end{align*}
因为$$\lim_{n\rightarrow\infty}\xi_l^{(n)}=\xi_l,$$$$\lim_{n\rightarrow\infty}\xi_k^{(n)}=\xi_k$$
且$\{\xi_k^{(n)}\}_{k=1}^{\infty}\in\overline{F}$是Cauchy列。所以有
$$|\xi_l-\xi_k|\rightarrow0,k,l\rightarrow\infty.$$
即$x\in\overline{F}$,$\overline{F}$是完备距离空间。

综上,$\overline{F}$是$F$的完备化空间。
\end{exercise}
\hfill\\
\section{列紧性}
\begin{exercise}
\hfill\\
设$(\mathscr{X},\rho)$是度量空间,$M$是$\mathscr{X}$中的列紧集,映射$f$满足
$$\rho(f(x_1),f(x_2))<\rho(x_1,x_2)\quad(\forall x_1,x_2\in\mathscr{X},x_1\neq x_2).$$
求证:$f$在$\mathscr{X}$中存在唯一的不动点。

定义
$$\overline{M}:=\{x\in\mathscr{X}:\exists x_n\in M,n=1,2,\cdots\text{ s.t. }\lim_{n\rightarrow\infty}x_n=x.\}.$$

因为$M$是$\mathscr{X}$中的列紧集,所以上述定义是合理的。易见$M\subset\overline{M}$。下证$\overline{M}$在$\mathscr{X}$上也是列紧的。

如果$I:=\overline{M}/M$是有限集,则结论是显然的。现在假设$I$是至少可数集。任取$I$中点列$\{\xi_i\}_{i=1}^{\infty}$,则对每个$\xi_i$,存在$M$中Cauchy列$\{x_i^{(n)}\}_{n=1}^{\infty}$,有
$$\lim_{n\rightarrow\infty}x_i^{(n)}=\xi_i.$$


对于$\{x_i^{(i)}\}_{i=1}^{\infty}$也是$M$中的点列,从而有收敛子列,不妨设为$\{x_{i_k}^{(i_k)}\}_{k=1}^{\infty}$,于是存在$\xi_0\in\mathscr{X}$,使得
$$\lim_{k\rightarrow\infty}x_{i_k}^{(i_k)}=\xi_0.$$

\begin{align*}
|\xi_{i_m}-\xi_{i_n}|&=|\xi_{i_m}-x_{i_m}^{(i_m)}+x_{i_m}^{(i_m)}-x_{i_n}^{(i_n)}+x_{i_n}^{(i_n)}-\xi_{i_n}|\\
&\leq|\xi_{i_m}-x_{i_m}^{(i_m)}|+|x_{i_m}^{(i_m)}-x_{i_n}^{(i_n)}|+|x_{i_n}^{(i_n)}-\xi_{i_n}|\\
&\rightarrow0,m,n\rightarrow\infty.\\
\end{align*}
于是$\{\xi_{i_k}\}$是Cauchy列,且
$$\lim_{k\rightarrow\infty}\xi_{i_k}=\xi_0.$$
这就是说$I$也是列紧集,从而$\overline{M}$也是列紧集。从以上证明过程还可看出,$\overline{M}$还是闭集。

于是将$f$限制在有界闭集$\overline{M}$上便有:$f_{\overline{M}}=f|_{\overline{M}}:\overline{M}\mapsto M$满足
$$\rho(f_{\overline{M}}(x_1),f_{\overline{M}}(x_2))<\rho(x_1,x_2)\quad(\forall x_1,x_2\in\overline{M},x_1\neq x_2).$$

再利用练习\ref{13}的结果(证明方法),就得到了我们所要的结论。
\end{exercise}
\hfill\\
\section{线性赋范空间}
\begin{exercise}
\hfill\\
设$C(0,1]$表示$(0,1]$上连续且有界的函数$x(t)$全体。$\forall x\in C(0,1]$,令$\|x\|=\sup\limits_{0<t\leq1}|x(t)|$。求证:
\begin{enumerate}
\item[(1)] $\|\cdot\|$是$C(0,1]$空间上的范数。
\item[(2)] $l^{\infty}$与$C(0,1]$空间的一个子空间是等距同构的。
\end{enumerate}

\begin{enumerate}
\item[(1)]
\begin{enumerate}
\item[1.]$\|x\|\geq0\forall x\in C(0,1]$显然成立。特别的
$$\|x\|=0\Leftrightarrow x=0.$$
\item[2.]$\|ax\|=\sup\limits_{0<t\leq1}|ax(t)|=|a|\sup\limits_{0<t\leq1}|x(t)|=|a|\|x\|.$
\item[3.]$\|x+y\|=\sup\limits_{0<t\leq1}|x+y|\leq\sup\limits_{0<t\leq1}(|x|+|y|)=\|x\|+\|y\|.$
\end{enumerate}
\item[(2)] 取$l=(l_1,l_2,\cdots)\in l^{\infty}$,则$\|l\|=\sup\limits_{i\geq1}|l_i|<\infty.$定义$$\phi(l)(x)=\begin{cases}
4l_n-2^{n+1}l_nx,&3\leq2^{n+1}x\leq4,\\
2^{n+1}l_nx-2l_n,&2\leq2^{n+1}x\leq3,\\
\end{cases}$$
易知$\phi(l)(x)\in C(0,1]$。且显然$\phi(l)$是从$l^{\infty}$到$\phi(l^{\infty}\subset C(0,1]$上的一一映射。且有$$\|\phi(l_1)-\phi(l_2)\|=\|\phi(l_1-l_2)\|=\|l_1-l_2\|.$$
现在只需证明$\phi(l)$是连续映射。只需证$\phi(l)$在$l=\theta$处连续。这是显然的因为,对任意的$\varepsilon>0$,存在$\delta=\varepsilon$,只要$\|l\|<\delta$,就有$\|\phi(l)\|=\|l\|<\varepsilon$成立。于是$l^{\infty}$与$\phi(l^{\infty})\subset C(0,1]$同构。
\end{enumerate}
\end{exercise}

\begin{exercise}
\hfill\\
设$\mathscr{X}$是$B^*$空间。求证:$\mathscr{X}$是$B$空间,必须而且仅须
$$\forall \{x_n\}\subset\mathscr{X},\sum_{n=1}^{\infty}\|x_n\|<\infty\Longrightarrow\sum_{n=1}^{\infty}x_n\text{收敛}.$$

$\Leftarrow:$记$$S_n=\sum_{i=1}^{n}x_i,$$
由$$\sum_{n=1}^{\infty}\|x_n\|<\infty,$$我们有
$\forall\varepsilon>0$,$\exists N>0$,使得只要$m>n>N$,就有
$$\sum_{i=n+1}^{m}\|x_i\|<\varepsilon.$$于是对上述$\varepsilon>0$,我们有
$$\|S_n-S_m\|=\|\sum_{i=n+1}^{m}x_i\|\leq\sum_{i=n+1}^{m}\|x_i\|<\varepsilon.$$从而$\{S_n\}$是柯西列。由$\mathscr{X}$是$B$空间知,$S_n$收敛。

$\Rightarrow:$任取柯西列$\{x_n\}\subset\mathscr{X}$,则我们只要证明存在子列$\{x_{n_k}\}$收敛即可。因为$\forall\varepsilon>0$,$\exists N$,使得$m>n>N$,就有$$\|x_m-x_n\|<\varepsilon.$$于是对$\frac{1}{2}$,能找到$N_1$,使得只要$m>n>N_1$,就有$$\|x_m-x_n\|<\frac12,$$不妨取$n_1=N_1+1$。同样的,对$\frac{1}{4}$,能找到$N_2>N_1$,使得只要$m>n>N_2$,就有$$\|x_m-x_n\|<\frac{1}{4},$$不妨取$n_2=N_2+1$。依次进行下去,我们就得到子列$\{x_{n_k}\}$满足
$$\|x_{n_{k+1}}-x_{n_k}\|<\frac{1}{2^k}.$$
显然$$\sum_{k=1}^{\infty}\|x_{n_{k+1}}-x_{n_k}\|<1<\infty.$$于是$$S_n=\sum_{i=1}^{k}x_{n_{k+1}}-x_{n_k}=x_{n_{k+1}}-x_{n_1}$$收敛,即$\{x_{n_k}\}$收敛。
\end{exercise}

\begin{exercise}
\hfill\\
设$\mathscr{X}$是线性赋范空间,函数$\phi:\mathscr{X}\mapsto R^1$称为凸的,如果不等式
\begin{equation}\label{tu}
\phi(\lambda x+(1-\lambda)y)\leq\lambda\phi(x)+(1-\lambda)\phi(y)\quad(\forall 0\leq\lambda\leq1)
\end{equation}
成立。求证凸函数的局部极小值必然是全空间最小值。

设$x_0$是凸函数$f(x)$的一个局部极小值。则存在$\delta>0$,使得$\forall x\in B^0(x_0,\delta)$,有$f(x)\geq f(x_0)$。对任意固定的$y$,当$n$足够大时,总有
$$(1-\frac{1}{n})x_0+\frac{1}{n}y\in B^0(x_0,\delta).$$
于是$$f(x_0)\leq f((1-\frac{1}{n})x_0+\frac{1}{n}y)\leq (1-\frac1n)f(x_0)+\frac{1}{n}f(y),$$
即$f(x_0)\leq f(y)$。得证。
\end{exercise}

\begin{exercise}
\hfill\\
设$\mathscr{X}$是$B^*$空间,$\mathscr{X}_0$是$\mathscr{X}$的线性子空间,假定$\exists c\in(0,1)$,使得
\begin{equation}
\inf_{x\in\mathscr{X}_0}\|y-x\|\leq c\|y\|\quad(\forall y\in\mathscr{X}).
\end{equation}
求证:$\mathscr{X}_0$在$\mathscr{X}$中稠密。

反证,$\mathscr{X}$在$\mathscr{X}$中不稠密,则存在$x\in\mathscr{X}$但$x\not\in\overline{\mathscr{X}_0}$。显然$\overline{\mathscr{X}_0}$也是$\mathscr{X}$的子空间,即是说$\overline{\mathscr{X}_0}$是$\mathscr{X}$的真闭子空间。于是应用F.Riesz引理,$\forall0<\varepsilon<1$,$\exists y\in\mathscr{X}$,使得$\|y\|=1$,并且$$\|y-x\|\geq1-\varepsilon\quad(\forall x\in\overline{\mathscr{X}_0}).$$
于是对$\varepsilon=\frac{1-c}{2}$,$\exists y_0\in\overline{\mathscr{X}}$,使得$\|y_0\|=1$,并且
$$\|y_0-x\|\geq\frac{1+c}{2}>c\quad(\forall x\in\overline{\mathscr{X}_0}).$$
这就与$$\inf_{x\in\mathscr{X}_0}\|y_0-x\|\leq c.$$
矛盾。
\end{exercise}

\begin{exercise}
\hfill\\
设$C_0$表示以0为极限的实数全体,并在$C_0$中赋以范数
$$\|x\|=\max_{n\geq1}|\xi_n\|\quad(\forall x=(\xi_1,\xi_2,\cdots,\xi_n,\cdots)\in C_0).$$
又设$$M\overset{\Delta}{=}\left\{x=\{\xi_n\}_{n=1}^{\infty}\in C_0|\sum_{n=1}^{\infty}\frac{\xi_n}{2^n}=0\right\}.$$
\begin{enumerate}
\item[(1)] 求证:$M$是$C_0$的闭线性子空间。
\item[(2)] 设$x_0=(2,0,0,\cdots)$,求证:
\begin{equation}
\inf_{Z\in M}\|x_0-Z\|=1,
\end{equation}
但$\forall y\in M$有$\|x_0-y\|>1$。

注: 

本题提供一个例子说明:对于无穷维闭线性子空间来说,给定其外一点$x_0$,未必能在其上找到一点$y$适合
$$\|x_0-y\|=\inf_{Z\in M}\|x_0-Z\|.$$
换句话说,给定$x_0\not\in M$,未必能在$M$上找到最佳逼近元。
\end{enumerate}

\begin{enumerate}
\item[(1)] $\forall k,l\in\mathbb{K}$以及$x,y\in M$,易见
$$\sum_{n=1}^{\infty}\frac{k\xi_n+l\eta_n}{2^n}=k\sum_{n=1}^{\infty}\frac{\xi_n}{2^n}+l\sum_{n=1}^{\infty}\frac{\eta_n}{2^n}=0,$$
其中$x=(\xi_1,\xi_2,\cdots,\xi_n,\cdots)$,$y=(\eta_1,\eta_2,\cdots,\eta_n,\cdots)$即$M$是线性的。
下面说明$M$是闭的。
设$\{x_n\}_{n=1}^{\infty}\in M$是柯西列,其中$x_n=(\xi^{(n)}_1,\xi^{(n)}_2,\cdots)$。由范数定义易知,数列$\{\xi_i^{(n)}\}_{n=1}^{\infty}$是柯西列,于是不妨设
$$\lim_{n\to\infty}\xi_i^{(n)}=\xi_i,\forall i\geq1.$$
记$x=(\xi_1,\xi_2,\cdots)$,于是
$$\lim_{n\to\infty}x_n=x.$$
要证$M$闭,则只需证$$\lim_{n\to\infty}\xi_n=0.$$

因为$\{x_n\}$是柯西列,所以$\forall\varepsilon>0,$ $\exists N>0,$只要$m,n>N$,就有
$$|\xi_m^{(i)}-\xi_n^{(i)}|\leq\|x_m-x_n\|<\frac{\varepsilon}{2}.$$
令$i\to\infty$,则有
$$|\xi_m-\xi_n|\leq\frac{\varepsilon}2<\varepsilon.$$
即$\{\xi_n\}$收敛,不妨设
$$\lim_{n\to\infty}\xi_n=\xi.$$
另一方面考虑数列$\{\xi_n^{(n)}\}$,我们有
$$|\xi_m^{(m)}-\xi|\leq|\xi_m^{(m)}-\xi_m|+|\xi_m-\xi|\leq\|x_m-x\|+|\xi_m-\xi|<\varepsilon,$$
即数列$\{\xi_n^{(n)}\}$收敛且收敛到$\xi$。另外取定$i=m$,则令$n\to\infty$,我们有$$|\xi_m^{(m)}-\xi_n^{(m)}|\to|\xi_m^{(m)}-0|=|\xi_m^{(m)}|\leq\frac{\varepsilon}2.$$再令$m\to\infty$,我们就有
$|\xi|\leq\frac{\varepsilon}{2}<\varepsilon.$
这就是说$\xi=0$。证毕。

\item[(2)] 记$Z=(z_1,z_2,\cdots)$,如果存在$i>1$使得$|z_i|>1$,则$\|x_0-Z\|\geq|0-z_i|>1$。否则如果$\forall i>1$,$|z_i|\leq1$,那么要使
$$\sum_{i=1}^{\infty}=0,$$
有$$|\frac{z_1}{2}|=|\sum_{i=2}^{\infty}\frac{z_i}{2^i}|\leq\sum_{i=2}^{\infty}|\frac{z_i}{2^i}|\leq\sum_{i=2}^{\infty}\frac{1}{2^i}=\frac{1}{2},$$即$|z_1|\leq1$。这样也有
$$\|x_0-Z\|\geq1.$$
上述等号成立当且仅当$z_i=1\forall i>1$或$z_i=-1\forall i>1$,而这是不可能的,因为$\{z_i\}$收敛到$0$。故上述不等式严格成立,即$$\|x_0-Z\|>1,\forall Z\in M.$$

取$$x_n=(1-\frac{1}{2^n},-1,-1,\cdots,-1,0,0,\cdots)$$,其中$x_n$中共有$n$项的值为$-1$。易验证,$x_n\in M$,且
$$\|x_0-x_n\|=1+\frac{1}{2^n}\to1.$$
综上,$\inf_{Z\in M}\|x_0-Z\|=1.$
\end{enumerate}
\end{exercise}

\begin{exercise}
\hfill\\
设$\mathscr{X}$是$B^*$空间,$M$是$\mathscr{X}$上的有限维真子空间,求证$\exists y\in\mathscr{X}$,$\|y\|=1$,使得
$$\|y-x\|\geq1\quad(\forall x\in M).$$

取定一个$y_0\in\mathscr{X}\\M$,则在$M$中存在最佳逼近元$x_0\in M$使得$$\|y_0-x_0\|\leq\|y_0-x\|,\forall x\in M.$$因为有限维子空间$M$是闭的,故$\|y_0-x_0\|>0$。取
$$y=\frac{y_0-x_0}{\|y_0-x_0\|},$$
则$\|y\|=1$。且
\begin{align*}
\|y-x\|&=\frac{1}{\|y_0-x_0\|}\|y_0-x_0-\|y_0-x_0\|x\|\\
&\geq\frac{1}{\|y_0-x_0\|}\|y_0-x_0\|\\
&=1.
\end{align*}
\end{exercise}
\hfill\\
\section{凸集与不动点}
\begin{exercise}
\hfill\\
求证在$B$空间中,列紧集的凸包是列紧集。

设$\mathscr{X}$是$B$空间,$M\in\mathscr{X}$是列紧集。则$\forall\varepsilon>0,$存在$$N_{\varepsilon}=\{x_{\varepsilon}^1,x_{\varepsilon}^2,\cdots,x_{\varepsilon}^N\}$$为$M$的有穷$\varepsilon$网。易证$co(N_{\varepsilon})$是$co(M)$的一个$\varepsilon$网。而$co(N_{\varepsilon})$是有限维子空间$span(N_{\varepsilon})$上的有界闭凸集,从而列紧,即对上述$\varepsilon>0$,存在$co(N_{\varepsilon})$的有穷$\varepsilon$网,记为$$coN_{\varepsilon}.$$下面我们要证明$coN_{\varepsilon}$恰好是$co(M)$的有穷$2\varepsilon$网。

因为$\forall x\in co(M)$,$\exists x_1\in co(N_{\varepsilon})$,使得$$\|x-x_1\|<\varepsilon;$$又对$x_1$,存在$x_{\varepsilon}\in coN_{\varepsilon}$,使得$$\|x_1-x_{\varepsilon}\|<\varepsilon.$$
于是,$\forall x\in co(M)$,$\exists x_{\varepsilon}\in coN_{\varepsilon}$,使得
$$\|x-x_{\varepsilon}\|=\|x-x_1+x_1-x_{\varepsilon}\|\leq\|x-x_1\|+\|x_1-x_{\varepsilon}\|<2\varepsilon.$$
这就是说,$coN_\varepsilon$是$co(M)$的一个有穷$\varepsilon$网。由$\varepsilon$的任意性,我们知道,$co(M)$是列紧集。
\end{exercise}

\begin{exercise}
\hfill\\
设$K(x,y)$是$[0,1]\times[0,1]$上的正值连续函数,定义映射
$$(Tu)(x)=\int_0^1K(x,y)u(y)\mathrm{d}y\quad(\forall u\in C[0,1]).$$
求证:存在$\lambda>0$及非负但不恒为零的连续函数$u$,满足
$$Tu=\lambda u.$$

定义
$$C:=\{u(x)\in C[0,1]|\int_0^1u(x)\mathrm{d}x=1,u(x)\geq0\},$$
则$C$是$C[0,1]$的闭凸子集。
定义
$$f(u)=\frac{\int_0^1K(x,y)u(y)\mathrm{d}y}{\int_0^1\int_0^1K(x,y)u(y)\mathrm{d}y\mathrm{d}x}.$$
则$$\int_0^1f(u)\mathrm{d}x=1.$$显然$f(u)\geq0,$
从而$f(u):C\mapsto C.$
下面我们说明$f(C)$列紧。
首先因为$K(x,y)$是正值连续函数,所以存在$m,M>0$使得
$$m\leq f(x,y)\leq M\quad(\forall(x,y)\in[0,1]\times[0,1]),$$
任取$u\in C$则有
$$|f(u)|\leq\frac{M\int_0^1u(y)\mathrm{d}y}{m\int_0^1\int_0^1u(y)\mathrm{d}y\mathrm{d}x}=\frac{M}{m},$$
其次$K(x,y)$在$[0,1]\times[0,1]$上是一致连续的,所以$\forall\varepsilon>0$,$\exists\delta>0$,使得只要$x_1,x_2\in[0,1]$且$|x_1-x_2|<\delta$,就有$$|K(x_1,y)-K(x_2,y)|<m\varepsilon.$$
\begin{align*}
|f(u)(x_1)-f(u)(x_2)|&=\frac{|\int_0^1(K(x_1,y)-K(x_2,y))u(y)\mathrm{d}y|}{\int_0^1\int_0^1K(x,y)u(y)\mathrm{d}y\mathrm{d}x}\\
&\leq\frac{\int_0^1|K(x_1,y)-K(x_2,y)|u(y)\mathrm{d}y}{\int_0^1\int_0^1K(x,y)u(y)\mathrm{d}y\mathrm{d}x}\\
&\leq\frac{m\varepsilon\int_0^1u(y)\mathrm{d}y}{\int_0^1\int_0^1K(x,y)u(y)\mathrm{d}y\mathrm{d}x}\\
&\leq\varepsilon.\\
\end{align*}
以上说明了$C$是一致有界且等度连续的,由Arzela-Ascoli定理知$C$列紧。最后由Schauder不动点定理知,$\exists u_0\in C$使得$$f(u_0)=u_0,$$
即
$$(Tu_0)=\int_0^1K(x,y)u_0(y)\mathrm{d}y=\lambda u_0,$$其中$\lambda=\int_0^1\int_0^1K(x,y)u_0(y)\mathrm{d}y\mathrm{d}x>0.$
\end{exercise}
\hfill\\
\section{内积空间}


\begin{exercise}
\hfill\\
设$\{e_n\}_1^{\infty}$,$\{f_n\}_1^{\infty}$是Hilbert空间$\mathscr{X}$中的两个正交规范集,满足条件
\begin{equation}
\sum_{n=1}^{\infty}\|e_n-f_n\|^2<1.
\end{equation}
求证:$\{e_n\}$和$\{f_n\}$两者中一个完备蕴含另一个完备。

不妨设$\{e_n\}$是完备的,而$\{f_n\}$不完备。于是存在$f\in\mathscr{X}$满足$(f,f)=1$,而$(f,f_n)=0,\forall n\geq1.$
因为$\{e_n\}$完备,所以
$$f=\sum_{i=1}^{\infty}(f,e_i)e_i,$$
%$$f_n=\sum_{i=1}^{\infty}(f_n,e_i)e_i,$$
定义$$\overline{f}=\sum_{i=1}^{\infty}(f,e_i)f_n,$$
则有$$(\overline{f},\overline{f})=\sum_{i=1}^{\infty}(f,e_i)\overline{(f,e_i)}=(f,f)=1,$$
$$(f,\overline{f})=\sum_{i=1}^{\infty}(f,(f,e_i)f_n)=0.$$
于是我们有
$$(f-\overline{f},f-\overline{f})=(f,f)+(\overline{f},\overline{f})=2.$$
另一方面
\begin{align*}
(f-\overline{f},f-\overline{f})&=\|f-\overline{f}\|^2\\
&=\|\sum_{i=1}^{\infty}(f,e_i)(e_i-f_i)\|^2\\
&\leq(\sum_{i=1}^{\infty}\|(f,e_i)(e_i-f_i)\|)^2\\
&=(\sum_{i=1}^{\infty}|(f,e_i)|\|e_i-f_i\|)^2\\
&\leq(\sum_{i=1}^{\infty}(f,e_i)\overline{(f,e_i)})(\sum_{i=1}^{\infty}\|e_i-f_i\|^2)\\
&\leq1.
\end{align*}
这就导出了矛盾。于是如果$\{e_n\}$完备,则必然$\{f_n\}$完备。
\end{exercise}

\begin{exercise}
\hfill\\
设$f(x)\in C^2[a,b]$,满足边界条件:
$$f(a)=f(b)=0,f'(a)=1,f'(b)=0.$$
求证:
$$\int_a^b|f''(x)|^2\mathrm{d}x\geq\frac{4}{b-a}.$$

只需注意到由题目条件定义的三次样条插值函数为
$$g(x)=\frac{(x-a)(x-b)^2}{(b-a)^2},$$
且
$$\int_a^b|g''(x)|^2\mathrm{d}x=\frac{4}{b-a}.$$
\end{exercise}

\begin{exercise}
\hfill\\
设$D$是$\mathbb{C}$中的单位开圆域,$H^2(D)$表示在$D$内满足
$$\iint_D|u(z)|^2\mathrm{d}x\mathrm{d}y<\infty\quad(z=x+iy)$$
的解析函数全体组成的空间。规定内积为
$$(u,v)=\iint_Du(z)\overline{v(z)}\mathrm{d}x\mathrm{d}y.$$
\begin{enumerate}
\item[1] 如果$u(z)$的泰勒展开式是
$$u(z)=\sum_{k=0}^{\infty}b_kz^k,$$
求证:
$$\sum_{k=0}^{\infty}\frac{|b_k|^2}{1+k}<\infty;$$
\item[2] 设$u(z),v(z)\in H^2(D)$,并且
$$u(z)=\sum_{k=0}^{\infty}a_kz^k,\quad v(z)=\sum_{k=0}^{\infty}b_kz^k,$$
求证:
$$(u,v)=\pi\sum_{k=0}^{\infty}\frac{a_k\overline{b_k}}{k+1};$$
\item[3] 设$u(z)\in H^2(D)$,求证:
$$|u(z)|\leq\frac{\|u\|}{\sqrt[2]{\pi}(1-|z|)}\quad(\forall|z|<1);$$
\item[4] 验证$H^2(D)$是Hilbert空间。
\end{enumerate}

\begin{enumerate}
\item[(1)]记 $u_k(z)=z^k,$则
$$
(u_m,u_n)=\iint_Du_m\overline{u_n}\mathrm{d}x\mathrm{d}y=\begin{cases}
0,&m\neq n;\\
\frac{\pi}{k+1},&m=n,\\
\end{cases}
$$
于是
\begin{align*}
(u,u)&=\iint_Du(z)\overline{u(z)}\mathrm{d}x\mathrm{d}y\\
&=\iint_D|u(z)|^2\mathrm{d}x\mathrm{d}y\\
&=\iint_Du(z)\sum_{k=0}^{\infty}\overline{b_k}\overline{z}^k\mathrm{d}x\mathrm{d}y\\
&=\sum_{k=0}^{\infty}\overline{b_k}\iint_Du(z)\overline{z}^k\mathrm{d}x\mathrm{d}y\\
&=\sum_{k=0}^{\infty}\overline{b_k}\iint_D\sum_{l=0}^{\infty}b_lz^l\overline{z}^k\mathrm{d}x\mathrm{d}y\\
&=\sum_{k=0}^{\infty}\overline{b_k}\sum_{l=0}^{\infty}b_l\iint_Dz^l\overline{z}^k\mathrm{d}x\mathrm{d}y\\
&=\sum_{k=0}^{\infty}b_k\overline{b_k}\frac{\pi}{1+k}\\
&=\pi\sum_{k=0}^{\infty}\frac{|b_k|^2}{1+k}\\
&<+\infty.\\
\end{align*}
\item[(2)]类似$(1)$计算可得结论。
\item[(3)]我们先证当$u(z)=u_k(z),k\geq1$时结论成立,
即证$$|z|^{2k}(1-|z|)^2\leq\frac{1}{k+1}.$$
记$f(x)=x^{2k}(1-x)^2,x\in(0,1)$,则
$$0\leq f(x)\leq(\frac{k}{k+1})^{2k}\frac{1}{(k+1)^2}<\frac{1}{k+1}.$$

那么就有
\begin{align*}
|u(z)|&=|\sum_{k=0}^{\infty}b_ku_k(z)|\\
&\leq\sum_{k=0}^{\infty}|b_k||u_k(z)|\\
&\leq\sum_{k=0}^{\infty}|b_k|\frac{\|u_k\|}{\sqrt{\pi}(1-|z|)}\\
&=\frac{1}{\sqrt{\pi}(1-|z|)}\sum_{k=0}^{\infty}\|b_ku_k\|\\
&=\sum_{k=0}^{\infty}|b_k
\end{align*}


未完待续
\end{enumerate}
\end{exercise}

\begin{exercise}
\hfill\\



\end{exercise}



