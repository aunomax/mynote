\chapter{高等代数}
数学的思维方法:观察客观世界的现象,抓住其主要的特征,抽象出概念或建立模型;运用直觉判断,归纳,类比,联想,推理等进行摸索,猜测可能有的规律;然后通过深入分析,逻辑推理和计算进行论证,揭示事物的内在规律,这就是数学思维方式的全过程。

按照“观察-抽象-探索-猜测-论证”的思维方式学习数学是学好数学的正确途径,而且可以培养正确处理工作和生活中遇到的各种问题的能力,从而终身受益。——丘维生。


\begin{exercise}
\hfill
设$n$个方程的$n$元齐次线性方程组的系数矩阵$A$的行列式等于0,并且$A$的$(k,l)$元的代数余子式$A_{kl}=0$。证明:$$\eta=(A_{k1},A_{k2},\cdots,A_{kn})'$$是这个其次线性方程组的一个基础解系。


因为$A$有一个$n-1$阶子式不为零,又$A$的行列式为0,所以$A$的秩为$n-1$.从而此齐次线性方程组的解空间维数为1.由按一行展开公式知,$\eta$是齐次线性方程组的一个解,且是非零解。从而线性无关。因此$\eta$构成了解空间的一组基。得证。
\end{exercise}
\begin{exercise}
\hfill\\
设$A=(a_{ij})$是实数域上的$n$阶矩阵,证明:
如果$a_{ii}>\sum_{j\not=i}|a_{ij}|,i=1,2,\cdots,n$,那么$|A|>0$。

令
$$
B(t)=
\left(
\begin{array}{llll}
a_{11}&a_{12}t&\cdots&a_{1n}t\\
a_{21}t&a_{22}&\cdots&a_{2n}t\\
\vdots&\vdots&\ddots&\vdots\\
a_{n1}t&a_{n2}t&\cdots&a_{nn}\\
\end{array}
\right)
$$
则$|B(t)|$是$t$的多项式,当$t\in(0,1]$时,由已知条件,$B(t)$是主对角占优矩阵,从而$|B(t)|\neq0$。又$|B(0)|=a_{11}a_{12}\cdots a_{nn}>0$。根据连续函数的中间值定理知:$|B(1)|>0$,即$|A|>0$。
\end{exercise}
\hfill\\

\begin{exercise}
    if $x_1, x_2, x_3,\cdots, x_n\in\mathbb R$ and $x_1^2+x_2^2+\cdots x_n^2\neq0$, then 
    \begin{equation*}
        \frac{x_1\cdot x_2 + x_2\cdot x_3 + \cdots + x_{n-1}\cdot x_n}{x_1^2 + x_2^2 + x_3^2 + \cdots + x_n^2} \leq \cos\frac{\pi}{n+1}.
    \end{equation*}
\end{exercise}

\begin{proof}
    left side is equal to a half of $(x_1,x_2,\cdots, x_n)A(x_1,x_2,\cdots, x_n)^T$ with 
    \begin{equation*}
        A = \begin{pmatrix}
            0 & 1 & 0 & \cdots & 0 \\
            1 & 0 & 1 & \cdots & 0 \\
            0 & 1 & 0 & \cdots & 0 \\
            \vdots & \vdots & \vdots & \ddots & 1 \\
            0 & 0 & 0 & 1 & 0
            \end{pmatrix}.
    \end{equation*}
so it suffices to show the most large enginvalue is less than $2\cos\pi/(n+1)$. 
For the equation $Av = \lambda v$,
the corresponding enginfunction $v = (v_1,v_2,\cdots, v_n)$ has relations 
\[
v_{k-1} + v_{k+1} = \lambda v_k \quad (1 \leq k \leq n)
\]
with boundary conditions $v_0=0$ and $v_{n+1} = 0$. As the left hand is less than 1, we have $|\lambda| < 2$. We may assume $v_k = \sin k\theta$, then 
\[
\sin((k-1)\theta) + \sin((k+1)\theta) = \lambda \sin(k\theta)
\]
which reduce to 
\[
\lambda = 2\cos\theta
\]
by triangle inequality.
The boundary condition entails 
\[
\theta = \frac{k\pi}{n+1} \quad (k = 1, 2, \dots, n).
\]
Therefore,
\[
\boxed{\lambda_k = 2\cos\left(\frac{k\pi}{n+1}\right) \quad (k = 1, 2, \dots, n)}
\]
We complete the proof as it is obvious that $\lambda_k \leq \lambda_1$.
\end{proof}